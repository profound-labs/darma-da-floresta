
\chapter{Conselho para estudantes e professores de Darma}

{\itshape
Palestra dada por Ajahn Jayasaro durante um encontro de professores de
Darma.}

Com a permissão da venerável sangha, gostaria de oferecer meus bons
votos e apreciação pela presença de todos aqui. E a reunião hoje é uma
ocasião histórica para mim, pois esta é a primeira vez que eu dou uma
palestra sobre o Darma havendo antes um show de abertura. Várias vezes,
quando era um jovem monge, eu era convidado a palestrar durante sessões
de Darma que ocorriam na Tailândia, que duravam a noite inteira. Vão
das oito da noite até as cinco da manhã. Eu costumava ser convidado por
volta da uma da manhã, quando todos estavam começando a ficar bastante
sonolentos. “Oh, vejam! É um monge estrangeiro, e ele sabe falar
tailandês!”, então todos acordavam. Quando vi a agenda de hoje, notei
que estava diretamente após o almoço e pensei: “Oh, esse é o mesmo tipo
de ideia…”, mas tivemos esse maravilhoso grupo cantando e vi todos
rindo e batendo palmas. Portanto não caiu sobre mim a tarefa de acordar
as pessoas. 

Então, começando com valores e, é claro, tenho que começar com
\textit{sīla}. Eu estava conversando com um venerável hoje cedo, que
uma das razões pela qual eu escolhi tornar-me um budista theravada foi
por causa do ensinamento de \textit{sīla }e da ideia de um
treinamento holístico triplo: \textit{sīla, sam\=adhi, paññ\=a}. E
\textit{sīla} sendo algo que deve emergir de sabedoria e deve vir de
uma vontade, um desejo de limitar suas ações e fala, dentro de limites
aceitos voluntariamente. E a relação desse tipo de esforço com a
purificação da mente. Eu achei que isso era apresentado de forma
extremamente clara na tradição theravada. 

Do que eu já estava bastante cansado, tendo tido algumas
experiências com isso, era de seitas e movimentos nos quais a ideia é
que o professor espiritual alcança, em última estância, um nível que
está além da moralidade e no qual ele não pode ser avaliado sob o mesmo
padrão das pessoas comuns – ele já transcendeu isso. Portanto seria
ilegítimo, incorreto, criticar uma pessoa dessa estatura por agir de
maneiras pelas quais uma pessoa comum seria condenada severamente. Eu
achava que essa era uma atitude muito perigosa para se tomar, perigosa
para comunidades e perigosa para os próprios professores espirituais,
porque mesmo na época do Buda existiam monges que superestimavam suas
realizações. Mesmo tendo o Buda e os grandes \textit{arahants} como
\textit{kaly\=anamittas}. Então, quanto maior é a possibilidade e a
probabilidade de superestimação espiritual nos dias e época atuais? Me
parece que nossa tradição theravada é uma tradição segura e confiável,
porque deixa muito claro à ordem monástica que, mesmo de um monge que
já esteja usando o manto por trinta, quarenta, cinquenta anos, é
esperado que observe todas as mesmas regras que um que tenha sido monge
por apenas um dia, mesmo que ele seja um \textit{arahant }– ainda mais
se ele for um \textit{arahant}. Nós temos essa ideia de praticar para
as gerações futuras, então mesmo que você não tenha que manter um
código moral em especial para sua própria purificação, mesmo que sua
mente esteja completamente livre de impulsos impuros ou corrompidos,
ainda assim nos comportamos dentro desses limites, como um exemplo aos
demais. 

Quando um jovem monge, ou uma jovem monja, ou jovem praticante
começa a se desenvolver no Darma e fica um pouco confiante demais,
então M\=ara ou as impurezas mentais surgem: “Na verdade, eu não
preciso mais fazer isso… eu posso fazer isso e não vejo nenhuma
impureza surgindo, minha mente está completamente indiferente, está
estável e clara. Então, por que não?” Esse é o tipo de pensamento
perigoso que surge em pessoas no caminho espiritual, em particular
quando você ganha um pouco de experiência com meditação. “Na verdade,
eu não preciso fazer isso.” E é por isso que temos esse valor de que,
não importa quão realizado você seja, você faz do mesmo jeito. Você faz
do mesmo jeito, mesmo que você não precise, por compaixão pelos seus
estudantes e pelas gerações futuras. Esses são os valores chave.

Agora, ser um professor espiritual significa que você está sob os
olhos de todos, todos estão olhando para você. Todo ponto bom, todo
ponto ruim, não são apenas exagerados, mas são também distorcidos, e
isso faz parte do carma de ser um professor espiritual. E não adianta
reclamar sobre isso, simplesmente é assim que as coisas são. Há muitas
projeções sobre os líderes, as pessoas projetarão em você ideias sobre
os pais delas. Eu me lembro, tendo há pouco tomado a função de abade de
um monastério, de ter ficado bastante exasperado com um jovem monge
americano e lhe dito: “Ei, eu não sou o seu pai, OK? Não me trate como
o seu pai porque eu não sou!” Então, embora o \textit{upajjh\=aya }ou o
\textit{ajahn }cumpram o papel de um pai, o lado ruim disso são
relacionamentos disfuncionais entre o estudante e figuras-chave na vida
dele, que são muitas vezes projetadas em você, sendo a nova figura de
autoridade, e isso é algo com o qual você tem que estar preparado para
lidar. 

Ontem eu estava falando sobre essa ideia de ser apenas você mesmo e
eu disse: “O que é que isso realmente significa?” Eu não acho que isso
esteja muito claro e acho que podemos ser muito iludidos sobre quem
realmente somos. Mas um valor e ponto de referência mais prático é:
“não tente ser outra pessoa”. É muito comum jovens aspirantes a líderes
espirituais tentarem moldar suas ações, suas condutas, até mesmo o
estilo de falar sobre o Darma etc., sobre um professor que eles
admiram. Mas passa a impressão de algo um pouco falso. Portanto, não
tentem ser uma outra pessoa, mas seguir os princípios de
\textit{sīla, sam\=adhi }e \textit{paññ\=a}. 

E um outro princípio que eu gostaria de expressar é diligência, ou
\textit{appam\=ada}. E particularmente no sentido de que o Buda
reservou o termo \textit{asekha}, \textit{asekha puggala}, “aquele que
se graduou”, “aquele que encerrou o treinamento”, para o
\textit{arahant}. Então, como professor, você deve sempre lembrar a si
mesmo que você é um estudante, você é um “professor-estudante” e você
está ensinando “estudantes-professores”. Porque se você se esquecer que
você também é um estudante, você provavelmente também esquecerá que
seus estudantes também são seus professores. E uma vez que você se
esquece dessas coisas, você pode entrar em modos de relacionamento
muito rígidos e começar a esperar e exigir certas atitudes e certos
tipos de respeito, quando na verdade essas coisas deveriam ser
conquistadas. Você deve estar constantemente treinando a si mesmo e
aprendendo. 

A tendência é que, se a sua prática espiritual atingir um certo tipo
de nível estável, você perderá aquela vontade, ou paixão por mais
desenvolvimento e é muito mais provável que essa armadilha lhe prenda
caso você esteja cercado de pessoas que pensam que você é maravilhoso.
Então, é um perigo muito real, e fique sempre voltando para si mesmo e
sua própria prática. E isso significa encontrar tempo para meditação
formal todos os dias. 

Algumas pessoas ensinam, promovem meditação regularmente, mas quase
não praticam, e uma das razões para isso, eu vou sugerir, é que prática
de meditação não é quantificável, ou é muito difícil dizer: “Hoje eu
coloquei esse tanto de horas e eu consegui esse tanto de produto”,
entende? Há o \textit{input }nesse processo, onde está o
\textit{output}? É difícil dizer. Mas se você passar uma hora em frente
ao computador, ou escrevendo um artigo, ou aconselhando, ou ensinando
às pessoas, no final você tem um senso real de realização. E eu não
estou denegrindo essas coisas, elas são certamente parte de nossas
vidas e responsabilidades, mas o que eu estou apontando é que é um
pouco perigoso, no sentido de que podemos usufruir dessa sensação de
realização, realização tangível, mensurável, ao ponto de ignorar,
negligenciar o muito importante e vital trabalho interior, que não
necessariamente tem um resultado diário óbvio sobre que possamos dizer:
“É, hoje eu consegui isso ou alcancei algo.” Mas, a longo prazo, para
ser um bom professor e um professor criativo, que possa realmente
liderar comunidades, esse trabalho constante não pode ser
negligenciado. 

E o meu próprio professor, Ajahn Chah, muitos de vocês sabem, não
sabia falar uma só palavra de inglês, mas foi capaz de atrair
estudantes de todo o mundo. E muitas vezes os ensinamentos que ele dava
eram maravilhosos e profundos, mas a maioria deles eram ensinamentos
bastante comuns, que nós poderíamos ouvir ou ler em outros lugares. Mas
ouvindo um ensinamento diretamente de Ajahn Chah, o ensinamento mais
básico, que você encontraria num “ABC do Budismo”, lhe impressionava,
alcançava seu coração de uma maneira maravilhosa por razão de quem ele
era, por causa da integridade dele, por causa da prática dele, por
causa da autoridade espiritual dele. Então, comunicação não é só
questão de ter feito o estudo, conhecer seu assunto, ser articulado,
mas é viver isso. É quem você é, e isso não é algo que você consegue
fingir. Sem isso não há uma transmissão de longo prazo realmente
eficiente do Darma. Isso é um chamado para essa diligência e real
devoção a estudar e à prática interior, e a ver que quanto mais
trabalho você põe nisso, mesmo que você não consiga identificar muito
claramente, você tem a fé de que isso de fato afeta a sua habilidade de
modelar os ensinamentos e torná-los reais e de dar às pessoas uma
sensação de esperança de que, se ele consegue fazer, eu também consigo.
Se ela consegue fazer, eu também consigo. Isso é absolutamente vital. 

Um aspecto da disciplina monástica que é muitas vezes negligenciado
é que os estudantes têm o dever de repreender, com grande respeito, os
professores deles se os professores começarem a agir de maneira
excêntrica ou inapropriada. E em sociedades hierárquicas, como aquela
em que vivo em particular, é muito difícil. Psicologicamente, é muito,
extremamente difícil em uma sociedade de não-enfrentamento, como a
sociedade tailandesa, para os membros mais jovens de uma comunidade dar
\textit{feedback} positivo ou repreensão respeitosa para os mais
velhos. E aqui você pode dizer: “Bom, eu sou um professor, mas eu faço
um verdadeiro esforço para estar aberto e convidar as pessoas a apontar
defeitos.” E isso é muito bom, mas você também deveria se perguntar:
quão frequentemente isso acontece? Porque muitas vezes professores
espirituais e líderes de qualquer tipo de situação podem fazer esse
tipo de oferta, mas ninguém acredita de verdade. Isso é dito, mas em
efeito ninguém de fato aceita o convite. Então, tem que haver uma
sensação real de convite, e não apenas as palavras de convite. 

Essa devoção à \textit{sīla}, o senso de humildade que vem de
olhar e ver claramente o trabalho que precisa ser feito… É verdade que
humildade pode facilmente tornar-se do tipo “Uriah Heep”, aqueles entre
vocês que leram Charles Dickens conhecem, a pessoa que é a mais
deficiente em humildade, Uriah Heep, está sempre dizendo: “Você deve
ser humilde, seja humilde” e, é claro, toda a ironia do personagem dele
é que ele não é humilde. Então, só falar em humildade não é o
suficiente, não é algo que você pode assumir como se fosse uma posição.
Humildade vem de olhar com bastante clareza, sem medo e com
neutralidade, para a natureza das corrupções surgindo e desaparecendo
na mente, sem tomar posse daquilo e dizer: “Eu sou uma pessoa ruim, eu
realmente não sirvo para ser professor, eu sou uma farsa, eu ainda
possuo essas corrupções”, mas perceber: “Sim, isso é um trabalho que eu
ainda tenho que realizar e é um trabalho ao qual estou devotado, mas,
no meio tempo, vou fazer o melhor que puder para ajudar as pessoas ao
meu redor.” Esse constante olhar para dentro, e desassociar-se do ego,
como uma prática. 

Outro valor é a compreensão da natureza dos “Darmas
Mundanos”\footnote{Loka dhammas são os oito fenômenos inerentes ao
mundo: elogio e crítica, perda e ganho, fama e anonimato, felicidade e
sofrimento.}, em particular elogio e crítica. Esse é um teste constante
da sua maturidade espiritual. Mas não dizemos: “Você é um professor
espiritual, você deveria ser equânime como uma montanha é para o vento.
As pessoas lhe elogiam ou criticam, você não deveria ser afetado por
essas coisas.” Essa atitude à vida espiritual que diz que você deveria
ser de um certo jeito e que não deveria ser de um certo jeito é
completamente inútil, portanto não se deixe enganar por isso. O que
estamos olhando é: o que é elogio, o que é crítica, qual é o
relacionamento que temos com eles, e começarmos a notar, por exemplo,
que quanto mais nos deleitamos em elogios, mais nos incomodamos com
críticas. Então, se você simplesmente gostaria de viver num ambiente
sem críticas e ser apenas elogiado, você estaria vivendo num paraíso
falso, porque essas duas coisas andam juntas. E a única maneira de não
sentir angústia ou desapontamento quando alguém lhe critica é ver a
natureza dos elogios sendo desse mesmo jeito. É só isso e nada mais:
você não é melhor porque alguém lhe elogiou e não fica pior porque
alguém lhe criticou. Tem uma história boa que vou contar sobre um rei
que é aconselhado por seu sábio ministro: 

“Tem uma pessoa vindo vê-lo hoje e ele é um verdadeiro galanteador.
Ele vai tentar te conquistar, ele vai dizer que você é maravilhoso, ele
vai te elogiar terrivelmente. Não acredite nele, OK?” 

Então o rei diz: “Não, não vou acreditar em nenhum elogio.” 

O homem entra e imediatamente começa: “Oh rei, você é tão
maravilhoso, tão sábio…” e vai dessa forma.

E o rei: “Hum… obrigado…”

O primeiro-ministro então olha para o rei e diz, sussurrando no
ouvido dele: “Eu te disse, eu te disse! Não acredite quando ele te
elogia!” 

O rei diz: “Tudo bem, eu não me esqueci de você. Ele ainda não
começou a me elogiar, tudo que ele disse até agora é verdade!” 

Então, tenha muito cuidado com acreditar no seu próprio currículo,
acreditar na sua própria propaganda, acreditar no papelzinho que o
moderador lê quando ele te introduz. Não acredite em nada disso. Não
acredite nos seus pensamentos, sempre tenha os ensinamentos do Buda
como um lembrete e como um ponto de referência. E o seu
\textit{kaly\=anamitta}: mesmo que você seja um professor, você sempre
precisa de alguém para olhar e referenciar-se. O seu
\textit{kaly\=anamitta }deveria ser um “kaly\=ana-metro”, ser uma
pessoa que metrifica, que mede, que julga a quantidade de
\textit{kaly\=ana dhamma }no seu coração. Não é alguém que lhe dá
elogios o tempo todo. E algo que eu notei em meu professor, Ajahn Chah:
ele era tão firmemente fundado no Darma, que ele não se importava caso
você ficasse com raiva dele. Ele não se importava mesmo que você o
odiasse, porque ele sabia que o que ele estava fazendo era para seu
próprio bem. Como um médico que lhe dá um remédio que você odeia ou um
tratamento que você não quer, pois ele sabe que é o melhor para você.
Então, como professor, não pense que você tem que ser popular o tempo
todo, não meça seu sucesso como professor pelos rostos sorridentes e
“Oh… você é tão maravilhoso.” 

E, estudantes, tenham cuidado com carisma! Existe uma correlação
muito forte entre o número de professores espirituais que traíram seus
estudantes e se envolveram em todo tipo de impropriedades e o carisma
deles. Isso não é dizer que todo líder carismático é corrupto, mas a
maioria dos líderes espirituais corruptos são carismáticos. Então, qual
é o “trio de ouro”? Carisma, um bom vocabulário e um uniforme. Se você
tiver essas três coisas a seu favor, você vai ter seguidores. Tenham
muito cuidado com essas coisas, não deem muita importância a elas. Não
estou dizendo que elas são sinais de que uma pessoa não é confiável,
mas mantenha em mente: “incerto, incerto…” A moralidade e a conduta, ao
invés da propaganda, são realmente o lugar onde podemos testar se este
é um centro, um professor, um líder para quem possamos nos entregar. 

Sim, como professores precisamos ter esse sentido de devoção ao
estudo e à prática. Algo que foi mencionado essa manhã. O Darma, eu
estava pensando… alguém estava falando em espalhar o Darma e eu pensei
“É, o Darma é como manteiga, não é?” Por quê? Porque se a manteiga
estiver muita fria, mesmo sendo manteiga, você não consegue espalhar,
então como é que se espalha manteiga? Primeiro você esquenta, então
você espalha. O Darma, os ensinamentos, podem ser duros e frios, então
eles necessitam desse calor, o calor humano da compaixão em aliança com
a sabedoria do Darma, e estar disposto a abrir mão do seu próprio
conforto, sua zona de conforto, para o bem-estar dos outros. Isso não é
\textit{mah\=ay\=ana} ou theravada, não é um ensinamento confinado a
nenhum grupo em particular. Esse é o principal valor no budismo:
sabedoria e compaixão; essas são as duas asas da águia, como disse um
professor. 

Como professor você precisa ser muito paciente. Não espere que o que
é obvio para você vá ser óbvio para seus estudantes. Se fosse
completamente óbvio para eles, eles seriam os professores e você seria
o aluno. O fato de que não é óbvio é justamente o ponto em questão.
Seja muito paciente e constantemente tente encontrar novas maneiras de
se fazer entender às pessoas. Nunca se sente no alto de um pedestal,
mas, ao mesmo tempo, não dê tanto aos outros ao ponto de negligenciar
seu próprio bem-estar, não entre nisso. Não fique isolado das pessoas
por ter que tentar criar uma certa imagem de professor. Você talvez
sinta que deve ficar distante e que deve esconder seu sofrimento e suas
imperfeições. Esse é um caminho perigoso; sempre tenha bons amigos com
quem possa conversar. Não fique solitário, que é uma das principais
causas para monges largarem a vida monástica: quando ficam solitários
como professores. 

E não entre nesse tipo de pena de si mesmo: “Ninguém me dá valor,
ninguém…” Eu vou encerrar isso com uma anedota. Uma história
verdadeira, aparentemente, de um líder que estava reclamando: “Eu fiz
todas essas coisas, eu fiz isso, eu fiz aquilo e não ganhei nenhum
reconhecimento, nenhum avanço na minha carreira e agora estou sendo mal
tratado e isso está acontecendo. Ninguém veio à frente para me apoiar,
me ajudar, é tão injusto, é uma completa conspiração de silêncio. O que
devo fazer?”, e o amigo disse: “Junte-se a ela!” 

Eu estou para me juntar a essa conspiração, a essa associação de
silêncio, mas, antes de fazê-lo, existe uma antiga tradição poética na
Inglaterra e na Tailândia de pegar palavras chaves e usar a primeira
letra de cada palavra para começar um ensinamento ou uma linha de
poesia, então minha benção para todos vocês é que vocês se desenvolvam
como líderes, com “sábia atenção, ciência clara e
desapego”\footnote{Essa palestra foi proferida, em inglês, durante um
encontro de professores de Darma na Malásia e o nome da conferência era
“Wacana”. Então o trocadilho com “Wise Attention, Clear Awareness and
No Attachment.”}. Obrigado. 
