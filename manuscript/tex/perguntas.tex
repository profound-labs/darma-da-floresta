
\chapter{Perguntas de praticantes brasileiros}

{\itshape
Em entrevista, Ajahn Piak responde a perguntas enviadas por praticantes
brasileiros.}

-- Luang Pó, a primeira pergunta é “Por que a Tailândia, apesar de
ser um país budista, ainda possui violência, ainda possui problemas com
pessoas se comportando mal, como em qualquer outro país?”

-- Eu acho que é normal, em todos os países só há confusão. Não é
“ser país budista” que faz não haver confusão. Se todas as pessoas
agissem de acordo com os ensinamentos do budismo não haveria confusão.
É igual no mundo inteiro. No islamismo, no cristianismo, por que eles
têm confusão? Porque confusão vem das pessoas. No mundo inteiro são as
pessoas que criam confusão, mas o propósito da religião não é gerar
confusão. 

-- Pergunta de uma mulher. Ela pergunta se é errado querer ir morar
sozinha, largar o marido e ir morar sozinha, porque a sensação dela é
que viver num relacionamento é um obstáculo em desenvolver a si, faz
desperdiçar muito tempo em nossa prática, pois temos que viver com
pessoas que não se comportam de forma correta. Está errado? Se ela
quiser largar do marido e ir viver sozinha?

-- Se tivermos essa resolução é bom, mas temos que olhar qual é
nosso objetivo, não é? Se vivemos no mundo, o modo de ser é assim:
quando você cresce você tem que ter um marido ou esposa, tem que ter um
emprego, esse é o modo de viver das pessoas no mundo. Mas se você pensa
que quer praticar para alcançar nirvana, você vai ter que fugir dessas
coisas todas. Mas você vai ter força de vontade suficiente? Porque
algumas pessoas têm fé, mas quando vão praticar não conseguem os
resultados que queriam, ou seja, paz. Se praticarmos mas ainda não
encontramos \textit{samādhi}, ficamos desapontados e voltamos para
onde estávamos antes. Temos que ter fé e ter sabedoria. Se tivermos fé
mas não tivermos sabedoria, não vamos conseguir alcançar o ponto, não
vamos alcançar o objetivo que havíamos determinado. Temos que ter fé e
também sabedoria.

-- Uma pergunta parecida, suponha que não largue o marido…

-- Comece praticando os cinco preceitos. Treine observando os cinco
preceitos ainda junto à família.

-- Ela quer saber se vivendo uma vida conjugal ainda é possível
conseguirmos fazer progresso na nossa prática do Darma.

-- Sim, só depende das nossas ações. Mas na maioria dos casos as
pessoas não conseguem fazer, porque na vida conjugal há muitos fardos,
muitos deveres, não temos liberdade.

-- Mais uma mulher, ela diz que os hormônios femininos – essa é a
sensação dela – fazem as mulheres terem uma libido muito forte. Ela diz
que é difícil aguentar, é difícil lidar com isso. Ela pergunta se há
algum artifício para não se identificar com aquilo, ela quer sair desse
ciclo.

-- Se ela quer sair disso ela tem que observar – se for fazer de
forma correta – tem que observar os oito preceitos\footnote{Um dos oito
preceitos é o voto de celibato.}, quer dizer, praticar
\textit{nekkhamma pāramī}\footnote{A capacidade de renunciar,
principalmente prazeres sensuais.}. Mas será que ela consegue fazer?
Isso é que é importante.

-- A pergunta é “Se desejamos que algo aconteça, esse desejo
consegue realmente ajudar aquilo a acontecer? Por exemplo, nosso
pensamento é capaz de fazer que aquilo aconteça?”

-- Isso é pensar, mas temos que agir, não é só pensar.

-- Se apenas pensar…

-- Não há ação.

-- Se, por exemplo, alguém está doente e nós rezamos, pensamos “que
ele melhore”, desse jeito, ajuda?

-- Ajuda mas pouco. Procurar um médico funciona melhor.

-- A próxima pergunta é: “A energia do Buda ainda envolve o mundo?”
Ela pergunta: “Essa energia consegue nos ver? Está ao nosso redor? Ou
ela desaparece quando o Buda alcança parinirvana? A energia dele
desaparece?”

-- Cinco mil anos após o Buda ter alcançado nirvana ela
desaparecerá. Mas ela ainda está presente nos lugares em que as pessoas
têm fé; nos lugares em que as pessoas não têm fé, ela não existe.

-- Significa que a energia mental do Buda ainda está presente no
mundo?

-- Ainda está.

-- Pergunta: “O Buda e os arahants desaparecem após o parinirvana?
E se for assim, como é possível alguns mestres dizerem que ainda
conseguem conversar com o Buda e com arahants?”

-- Isso eu não respondo, pois perguntou alto demais. Depois que
você tornar-se um arahant venha falar sobre esse assunto. É difícil
explicar, as pessoas não conseguem entender. Alto demais, se você ainda
é uma pessoa comum e pergunta sobre isso, foi alto demais.

-- É verdade que o Buda ensinou para acreditarmos que, após
morrermos, vamos renascer? O Buda ensinou assim de verdade ou não? A
pessoa que perguntou viu num livro que o Buda ensinou a não acreditar
em nada: se não vemos por nós mesmos, não devemos acreditar. Então a
pergunta é a seguinte: “Se quisermos ser budistas, somos obrigados a
acreditar nisso?”

-- Tem dois aspectos. Isso é crença, significa acreditar – nós
acreditamos. No começo temos que ter fé, acreditar. Não acreditar em
absolutamente nada tem nível alto e baixo. O baixo é, na verdade, não
acreditar em nada que qualquer pessoa diga. O alto, na verdade, é o
Buda tendo declarado, ensinado: “praticando assim vai acontecer isso”,
e a pessoa então vai praticar e treinar até conseguir, como
Sāriputta. Em uma ocasião o Buda estava ensinando à sangha sobre
prática, sobre o Darma, e perguntou:

“Sāriputta, acredita em mim?” 

E ele disse “Não acredito.” 

Os demais disseram “Sāriputta não acredita na doutrina do Buda!”

Mas o Buda perguntou “Por que não acredita?”, e Sāriputta
respondeu:

“Antes tenho que ir praticar até conseguir. Quando conseguir, vou
acreditar. Se ainda não consigo fazer, não acredito.” Tem alto e baixo,
se estiver muito baixo não acredita em ninguém. O alto é praticar até
conseguir, e só depois acreditar.

-- Uma pessoa comum é capaz de ver seres de outros mundos, como
anjos, fantasmas e esse tipo de coisa?

-- Se tiver visão divina\footnote{A capacidade de ver que vai além
das pessoas comuns, desenvolvida através da prática de samādhi e
jhāna.} consegue… Se tiver! Eu não quero falar sobre isso, pois, se
falar, é como se estivesse declarando ter poderes psíquicos\footnote{É
uma ofensa à regra monástica declarar possuir esse tipo de capacidade a
um leigo.}. É uma pergunta que não deve ser respondida, mas se a mente
daquela pessoa consegue ver, então vê!

-- E ele pergunta como fazer para desenvolver essa habilidade.

-- Faça sua visão ficar boa. Seus olhos ainda não estão bons, vá
curá-los e vai ver.

-- Como faz para melhorar a visão?

-- Vá praticar.

-- Praticar e desenvolver \textit{samādhi}?

-- É.

-- E como é possível saber se aquilo que vemos, por exemplo, se
estivermos vendo um anjo, como podemos saber se aquilo é de verdade ou
é apenas a nossa imaginação?

-- É difícil responder isso porque ele ainda não sabe. Ele ainda
não sabe o que é o quê. Isso é difícil falar porque explicar certas
coisas… uma pessoa com visão boa explicar algo a uma pessoa cega é
difícil. É melhor você ir curar sua visão, depois venha conversar
comigo!

-- É mais fácil.

-- É mais fácil. Explicar para uma pessoa cega entender é difícil,
essa cor ela não vê, como explicar o céu se ela não vê?

-- A pergunta é “Se uma pessoa é leiga e ensina prática de
meditação, é apropriado que ela cobre? Quando as pessoas vêm aprender
ele cobra e às vezes ainda se envolve amorosamente com as pessoas que
vêm aprender.”

-- Depende da opinião, mas, na verdade, ensinar ou transmitir o
Darma deveria ser feito de graça. Ensinar o Darma como doação. Mas, às
vezes, em alguns lugares, eles querem obter algo e então abrem um curso
de meditação. Aqueles que não têm dinheiro são proibidos de participar.
Às vezes é assim. Mas se for olhar de acordo com a forma correta,
deveríamos ensinar o Darma como doação.

-- E ter relacionamento com as pessoas que vêm estudar, é
apropriado? Se uma moça vem estudar…

-- Budista pode ser homem, mulher, criança, não importa.

-- Digo, ele vai namorar com as alunas.

-- Como?

-- Às vezes se aproveitam da situação em que se é um professor para
namorar com as alunas ou alunos…

-- Isso mostra que, como dizer… você é um professor, mas você ainda
não alcançou o Darma. Ainda não entrou no Darma, você ensina mas não
tem Darma no coração. Você vai ensinando, e enquanto ensina surge
desejo e você logo vai para a cama com aquela pessoa… Você está
ensinando o Darma mas está ensinando para obter dinheiro, fama,
elogios, ou para buscar marido e esposa? Isso não é ensinar o Darma.
Você tem que lembrar que você não é iluminado, mas consegue ensinar
porque leu e memorizou. Os mestres falaram e você escutou, você leu as
escrituras, você estudou, memorizou e portanto é capaz de repetir o que
ouviu, mas você ainda não é Darma. Se você não é Darma ainda há algo
defeituoso em você.

-- Supondo que uma pessoa bebe álcool, mas só um pouco, não o
suficiente para ficar embriagado, não o suficiente para interferir em
sua clareza mental, ele pergunta: “Isso é pecado?” Se ele beber só um
pouco de álcool, sem ficar bêbado, é pecado?

-- Não sei se é pecado ou não, mas é uma quebra dos cinco
preceitos. Esse quer beber de um jeito ou de outro…

-- A religião cristã ensina que as pessoas homossexuais, quer sejam
homens ou mulheres, todas irão para o inferno. É o que ensina a
religião cristã. Ele quer saber se a religião budista ensina a mesma
coisa ou não.

-- Eles se envolvem de maneira correta ou incorreta? Por exemplo,
se duas pessoas são casadas elas podem ter relacionamento sexual, mas
se não forem casadas, se aquela pessoa for esposa de uma outra pessoa,
você então está errado. Seguindo os cinco preceitos você pode ter
esposo ou esposa, não tem problema. Pode fazer sexo, não tem nada de
errado. Mas se sair disso, se você tem uma esposa ou esposo e vai se
envolver com outra pessoa, você está errado.

-- Uma pergunta estranha: quando ele senta em meditação ainda
consegue ouvir outras pessoas conversando sem entender a linguagem
deles. Ouve somente o som. Então quer saber se isso está certo, se
fizer isso há algo de errado? Às vezes ele está trabalhando e gosta de
fazer isso, foca a mente de modo a não ouvir linguagem, ouvir somente o
som. Tem algum problema?

-- Não tem nada demais não entendermos a língua dos outros. Não é
sentando em meditação que você vai conseguir entender todas as línguas.

-- Eles falam a mesma língua…

-- Então é como não estar interessado. Se não está interessado não
tem problema.

-- Mais uma: quando ele foca na respiração ele sente que, se forçar
muito, a mente não se pacifica, não fica parada. Se ele relaxa, larga,
a mente vai pensar em outros assuntos. Ele não sabe como fazer para
encontrar o equilíbrio. Se forçar a mente se excita, se relaxar a mente
se distrai.

-- Tem que ter equilíbrio: feche os olhos de forma equilibrada,
inspire e expire de forma equilibrada. O equilíbrio faz surgir paz.

-- Pergunta: quando senta em meditação, sente que a mente se
pacifica mas somente quando está se sentindo bem; se não está bem ou se
tem um barulho incomodando ou algo que o incomoda… Se for algo que ele
gosta, sente paz, mas se não gosta não consegue suportar, não consegue
ficar parado, a mente foge para outro lugar, ele pede o conselho do
Luang Pó.

-- Procure um lugar onde não haja som para se sentar, assim vai
surgir pouca confusão. Mas não está correto o que ele disse, se ele
ouvir uma música e a mente canta junto com a música, onde vai haver
paz? Se ouvimos algo que gostamos a mente fica pensando naquilo que
gostamos; se não gostamos, ela pensa naquilo que não gostamos.

-- A última é: “Como ter \textit{sati} na hora em que estamos
trabalhando ou conversando com outras pessoas?”

-- Tem que treinar, se esforçar em treinar. Neste momento o que
estamos fazendo? Se a mente escapar, estabelecemos novamente. Escapa,
estabelecemos de novo, tem que fazer assim. Reestabelece um bilhão de
vezes, mil, dez mil, cem mil, um milhão, vai reestabelecendo até
conseguir fazer. Quando ela escapa não é nada de ruim, quando voltamos
a ter \textit{sati} estabelecemos de novo, até conseguir.

-- Ele tem dúvida se praticando assim, tendo \textit{sati} na vida
diária, isso é prática de verdade, dá resultado de verdade fazer isso?

-- Se quiser praticar para valer contemple seu corpo e mente sendo
\textit{anicca, dukkha, anatta}. Tem que contemplar o corpo sendo algo
vazio, \textit{anicca, dukkha, anatta}; não é “meu”, não sou “eu”, não
é estável, permanente. Tem que contemplar bem aqui.

-- É só isso, muito obrigado.

