
\chapter{Problemas do mundo}

{\itshape
Ajahn Chah conversa com um grupo de discípulas leigas. (Devido a
interrupções na gravação, algumas frases estão inacabadas e outras
perderam um pouco seu contexto.)}

-- (…) e também pratique meditação, não precisa ser nada de mais,
apenas não deixe a mente ficar confusa ou irritada, deixe ela se
pacificar.

-- Onde devo me sentar para que haja paz?

-- Com um monte de impurezas mentais desse jeito, será que vai
haver paz!? Fala sério!

-- Qual a forma mais fácil de pacificar a mente?

-- Não importa! Não precisa ser fácil, as causas têm que estar
presentes para que seja fácil. Qual a forma mais fácil de encher a
barriga? Se comer só uma colherada, vai ficar cheia? Falam desse jeito:
“Quero conseguir rápido.” Eu me ordenei e quase morri pelo
\textit{Buddha Sāsana} e ainda não alcancei paz e você quer sentar e
num momento já alcançar paz!? Fala sério! As pessoas têm muito desejo,
então sentam e só dá confusão.

-- Somos muito impacientes.

-- Senta e não tem paz, só tem confusão! O quê? Quem? Sim, viajei
ao ocidente duas vezes, dois meses em cada ocasião. Fiquei muito tempo.

-- E teve quem oferecesse?

-- Oferecesse o quê?

-- Comida…

-- Ôôe, ofereceram! Se não oferecessem como ia ser? Onde eu ia
comer?

-- Mostra que há interesse pelo budismo por lá…

-- Estão interessados, estão começando a se interessar. Estão
começando agora. Os ocidentais vão ao templo e não conversam sobre
outras coisas, vão para praticar meditação e pacificar a mente, então
só há silêncio. Não são iguais aos tailandeses: quem vende cesto vai
conversar sobre cesto, quem tem horta vai conversar sobre horta.
Conversam sobre qualquer coisa, então não têm tempo de pacificar a
mente e aí querem alcançar o mais rápido possível. Vão ao templo
conversar sobre isso e aquilo, voltam para casa e dizem que foram ao
templo. Na verdade não procuram o \textit{Buddha Sāsana} dentro da
própria mente. Já os ocidentais vão ao templo e sentam na almofada em
silêncio, ninguém fala. Pacificam a mente. Já nós aqui vamos conversar
sobre isso, aquilo… arrastam o templo inteiro para dentro da conversa,
depois vão comer.

Muitas cerimônias… Os estrangeiros não estão interessados nisso, não
estão interessados. No país deles não há esse tipo de interesse:
estudar escrituras, ir comer, etc. Eles não estão interessados. Estão
interessados em procurar paz, vão direto ali. Eles não vão ao templo
dessa forma, são só os tailandeses é que vão. O filho ou a filha vai
estudar no exterior e o pai e a mãe vêm visitar, então vão ao templo
para comer, conversar, se divertir e voltam para casa. Não se vê quem
vá para praticar meditação – eu observei. Não vi nenhum tailandês ir
praticar meditação. Só vão conversar.

-- Vão passear.

-- Só isso, não dá em nada. Os estrangeiros não fazem assim, eles
buscam o verdadeiro ensinamento do Buda. Os tailandeses têm o
ensinamento do Buda mas não o conhecem, vão estudar outras coisas. Por
lá tem que ir sentar em meditação, pacificar a mente. Teve uma pessoa
chamada Sudama que foi para os Estados Unidos, ela mora na Rua
Plêngsit.\footnote{Rua de um bairro nobre de Bangkok.} Quando fui lá
ensinar meus discípulos, estava cheio de estrangeiros e essa Sudama,
acho que era cristã, ficou curiosa em saber o que estávamos fazendo e
então veio ver. Ela sabe falar tailandês. Fui ensinar meditação e um
dia ela veio perguntar: “Luang Pó, em Bangkok tem esse tipo de
atividade?” Veja isso, ela não sabe onde se pratica meditação, ela só
veio a conhecer no exterior e não sabia se também havia na Tailândia. É
uma pessoa pobre. Eu dei uma bronca nela… Tem que levar bronca primeiro
para poder começar a praticar. Mostra que nunca foi a um templo, não
conhece essa atividade. Naquela ocasião praticamos nove dias, faz a
mente se pacificar e surgem diversos tipos de sabedoria. Chama-se
Sudama, acho que já voltou para cá. Mora na Rua Plêngsit. Falou que
quando voltasse viria me visitar no monastério, ainda não a vi
aparecer.

Várias pessoas daqui vão viver lá, vão estudar, e só fazem contato
com o budismo no exterior. Só vão entender no exterior, quando moravam
na Tailândia não entendiam nada. Os monges ficam ali, os leigos ficam
aqui, e é assim, cada um na sua e não dá em nada. Algumas pessoas vão
lá e se sentem solitárias, não têm ninguém, olham ao redor e os velhos
são gringos, os jovens são gringos, as crianças são gringas, todo mundo
é gringo, não tem tailandeses. Quando veem um tailandês, como eu, eles
vêm me procurar. É como se viéssemos da mesma vila, todo mundo se
conhece. Eles trazem os filhos para oferecer,\footnote{Tradição comum no
interior da Tailândia. As pessoas pedem que o Ajahn aceite ser padrinho
do filho deles com a ideia de que assim o mérito do Ajahn vai ajudar a
proteger a criança. Isso é feito apenas como cerimônia, o monge não
assume de fato nenhuma responsabilidade pela criança.} oferecem para
quem não tem filho! Eles vêm oferecer – várias pessoas! Eles chamam os
amigos para vir junto. Não dá em nada. A maioria dos tailandeses são
estudantes, só vão entender lá, só lá é que fazem contato com o
budismo. Quando fui, vieram muitos desses aí. Me assustei: “Eh! De onde
veio essa gente!?” Ensinei o método de praticar, os estrangeiros
praticavam, mas os tailandeses não sabiam praticar. Eles sentam e é
como se algo estivesse queimando, então pensei muito sobre isso e
refleti sobre o \textit{Buddha Sāsana}. 

Vão procurar os monges, mas vão por outros motivos. Não deveria ser
assim. Vão procurar amuletos de vários tipos, só confusão! Eu não quero
ouvir essas coisas. Isso não é assunto de monge, não é assunto do
\textit{Buddha Sāsana}. Todos só vão procurar isso. E então não
conhecem o assunto real do \textit{Buddha Sāsana}; se tivéssemos o
\textit{Buddha Sāsana} ia haver toda essa confusão? Eles pegam
somente aquilo que é o oposto, só as coisas de que gostam.
\textit{Sīla} que é o verdadeiro aspecto do \textit{Buddha Sāsana}
capaz de proteger as pessoas, eles não têm… não têm. 

-- Ajahn, alguns têm medo que se a Tailândia não tiver mais
confusão, respeitar o \textit{Buddha Sāsana} e todos praticarem o
Darma, os países estrangeiros virão criar confusão aqui…

-- Incerto.

-- Eles querem roubar nosso país…

-- É incerto, isso são as outras pessoas vindo nos agredir, não dá
para evitar, é assunto deles. Nós não temos como evitar, mas…

-- Se respeitarmos o \textit{Buddha Sāsana} isso deve nos
proteger.

-- Protege a si mesmo, não protege aos outros.

-- Protege o nosso país?

-- Pode ser, mas não é garantido. Até o Buda eles queriam matar!
Não é incrível? Um Buda nasce no mundo e eles queriam matar. É assim o
mundo. Se nasceu no mundo, só vai encontrar perigos e desgraças. É um
tipo de carma, não está além do sofrimento. Neste mundo há violência.
Com o passar do tempo as pessoas que seguem uma religião vão mudando.
Mas conseguir evitar todos os perigos é muito difícil neste mundo. Este
mundo é um perigo.

-- Mas, por exemplo, queremos cuidar do nosso monastério, mas tem
sempre alguém vindo roubar…

-- Vai fazer o quê? Proteja, se não conseguir proteger, entregue
para eles! Não tem como carregar o monastério nas costas, no final a
gente morre. Mundo, isso é coisa do mundo. O \textit{Buddha Sāsana}
se estabelece neste mundo, mas não é capaz de proteger o mundo, isto
está além do âmbito dele. Quando chega a hora, tem que ir de acordo com
sua natureza. Não é possível proteger com garantia. O Buda ensinava que
é impermanente, tudo isso é impermanente. Ainda estando vivos,
procuramos felicidade e tranquilidade, procuramos proteger o país
desses perigos todos. Se não tivermos \textit{sīla,} surge ainda
mais confusão. É assim.

-- Eu ouvi dizer que antigamente o \textit{Buddha Sāsana
}prosperava em Pequim,\footnote{A pessoa refere-se ao fato de que no
passado existia budismo na China, mas depois o país se tornou
comunista. Note que essa gravação ocorreu na época da guerra do Vietnã
e havia um perigo real de a Tailândia também tornar-se um país
comunista.} deve ser verdade…

-- Isso eles dizem, mas é algo que dizem sem saber.

-- Hoje, na China, religiões são proibidas…

-- Às vezes foi porque acabou o combustível. Quando isso passar
talvez as coisas melhorem. Na verdade as pessoas que hoje em dia são
budistas ainda não alcançam o coração do ensinamento do Buda. Como os
que dizem “Oh! Nada vai acontecer, nós temos o \textit{Buddha
Sāsana}, nada vai acontecer conosco”, esse tipo de coisa. Eles
entendem que o \textit{Buddha Sāsana} tem um poder mágico e é capaz
de proteger o mundo inteiro. Não é possível se nós não praticarmos. O
Buda ensinava a praticar desse jeito, se esforçar dessa forma. Não é
para jogar a responsabilidade para que ele tome conta de todos nós.
Fácil demais, não? É assim. Está com medo? Medo por quê? Se acontecer
está bom, se não acontecer também está bom. Que diferença faz? Não se
apegue demais.

Algumas pessoas pensam errado, têm tanto medo dos comunistas que
ficam loucos: “Vou lutar até morrer!”, só confusão. Todas as coisas têm
seu lado bom, por exemplo, uma pessoa que não é budista, vamos ensinar
o budismo e ele não enxerga porque não reflete sobre aquilo. Na verdade
há algo de bom ali, mas ele não enxerga. Como as pessoas ruins que
depredam o mundo, às vezes tem um lado bom ali mas ainda não
enxergamos, temos a mente fixa em outro entendimento daquilo. Há vários
tipos de coisas escondidas. Mas eu digo que bom é ficar quieto. Não
pense muito.

(…) não é possível vencer: quando chega a hora, se degenera. Por
exemplo, nós nascemos, antes de nascer estávamos na barriga da mãe, não
é? Quando saímos da barriga viramos crianças, garoto, garota, etc., até
conseguir andar. No final o cabelo e os dentes caem, dá para evitar?

-- Não dá. Os olhos começam a ficar embaçados…

-- Dá para evitar?

-- Não dá.

-- E vocês gostam disso ou não?

-- Não gostamos.

-- Não gostam, mas não conseguem evitar. Pois é, que fazer? É algo
natural, é melhor aceitarmos. Aceitem, aceitem essa situação, nós temos
que envelhecer. Envelhecer é bom? É! Se não envelhecêssemos não
cresceríamos a este tamanho. Íamos ficar igual a um feto que não
nasceu, se não envelhecêssemos. Graças à força da velhice é que nos
desenvolvemos até este ponto. Temos que saber a causa, velhice tem sua
razão de ser. Se desenvolve e no final se degenera até desaparecer.
Isso é a natureza, o mundo é assim. Isso podemos aceitar, é assim.

(…) Se olharmos o que fizemos no passado, como é que é? O mundo é
assim. O Buda e o Ānanda eram mendigos. Eles saíram em
\textit{pindapāta}, o Ānanda e o Buda. Umas pessoas com
\textit{micchā-ditthi},\footnote{Visão Incorreta, mas nesse caso
específico, significa pessoas que não gostavam do Buda e do ensinamento
dele.} não sabiam de nada, eles tinham a opinião de que esses
monges não serviam para nada a não ser pedir esmolas. Não viam
utilidade nenhuma neles. Pensavam assim e sentiam desdém. Às vezes o
Buda ia em \textit{pindapāta} com o Ānanda e essa gente criticava,
xingava. Às vezes ameaçavam caso eles não fossem embora. O Buda ficava
com ainda mais sabedoria, o Ānanda ficava ainda mais tolo, ficava com
raiva, com vergonha. O Buda via de acordo com o Darma, que as pessoas
são assim mesmo. Duas pessoas: o discípulo pensa de um jeito, o
professor pensa de outro. O discípulo ficava cada vez mais tolo, pois
não aguentava ouvir aquilo. O Buda era como uma pipa soprada pelo
vento, voava ainda mais alto. O Ānanda ficava tremendo como uma pipa
que não aguenta a força do vento.

O Buda ficava de pé bem em frente à casa dos
\textit{micchā-ditthi}, com a tigela à mão, e as pessoas da casa
xingavam “Que monge é esse? Não serve para nada! Eu não vou dar nada,
vá embora!” O Ānanda ficava ainda mais confuso, com raiva e
envergonhado, o Buda ficava com ainda mais sabedoria, ficava ainda mais
tranquilo. Ele pensava: “Neste mundo as pessoas ignorantes são assim.”
O Ānanda não sabia de nada e ficava ainda mais tolo, o Buda sabia das
coisas e ficava ainda mais sábio. 

Quando o Buda disse para o Ānanda: “Vamos embora”, Ānanda
avisou: 

“Eu não aguento essas ofensas.” 

O Buda respondeu: “Ānanda, para onde ir então?”

“Vamos àquela outra casa mais à frente.”

“Ānanda, se naquela casa eles também falarem desse jeito, para
onde vamos?”

“Vamos mais à frente. Vamos sair dessa vila e vamos para a outra.”

“Se naquela vila eles falarem assim, para onde vamos?”

“Vamos à outra vila ainda mais à frente.” O Buda perguntou:

“Se na outra vila eles falarem assim, para onde vamos?”

“Vamos para outro distrito.” O Buda respondeu: 

“Se no outro distrito eles falarem assim, para onde vamos?”

“Vamos àquela outra cidade.”

“Se naquela cidade falarem assim, para onde vamos?”

“Vamos para a outra cidade ainda mais à frente.”

“Quando não houver mais para onde ir, para onde vamos, Ānanda?
Aqui eles falam, lá eles falam, onde vamos conseguir escapar disso?
‘\textit{Natthi loke anindito}’, neste mundo não há quem não
seja criticado.” 

Se nós caminharmos o mundo inteiro e eles nos criticarem, como é que
vai ser? O mundo é assim mesmo. As pessoas que têm
\textit{micchā-ditthi} falam de um jeito, os que têm
\textit{sammā-ditthi} falam de um outro jeito, não tem para onde ir. 

“Ānanda, se vencermos aqui, mais à frente vamos vencer. Se
perdermos nesta cidade, na cidade mais à frente vamos continuar
perdendo. Que fazer?”

O Ānanda não sabia responder. Isso também é assim, onde vai
conseguir escapar? Não apenas indo em \textit{pindapāta}, pedir
esmola, como dizem as pessoas. Mesmo um casal que se casa com
sinceridade, se gostam, se amam, ou mesmo do pai e da mãe ainda temos
que ouvir palavras desagradáveis. Fazer o quê? O mundo é assim.
\textit{Lokavidū} – se nós não conhecermos o mundo, nossa vida neste
mundo será grosseira. Não vamos ter onde morar, aonde formos será
assim. O problema do mundo é assim. Aonde quer que vá, não é possível
evitar que digam que somos bons ou ruins. No mundo é assim, tem que ser
assim. O Buda ensinou ao Ānanda “Nós temos que vencer bem aqui! Se
vencermos aqui, mais à frente vamos vencer, onde quer que formos vamos
vencer. Vamos vencer no mundo inteiro.” Isso é algo sobre o qual o Buda
alertava seus discípulos: quando algo desagradável vem, algo agradável
vem junto com aquilo.

Nem precisa falar das outras pessoas, mesmo nós às vezes não
gostamos do que nós mesmos pensamos. Às vezes pensamos assim, mas
acontece daquele jeito; pensamos daquele jeito, mas acontece assim, aí
não gostamos. Nem precisa falar dos outros, nós mesmos às vezes somos
assim. Por isso precisamos nos esforçar para refletir e ver que o mundo
é assim, não é possível fugir desse tipo de fala: o Buda tinha que
ouvir essas coisas, os \textit{sāvakas} tinham que ouvir essas
coisas. Portanto temos que começar a entender que o mundo é assim,
temos que aguentar. Aguentar. Aguentar todo tipo de coisa. Aos poucos
ensine sua mente. Como se diz: cabeça fria. Cabeça fria… Vá fazendo e
aos poucos se desapegando. Se entendermos essas coisas seremos capazes
de vencer esses estados mentais.

Nasce sabedoria pensando assim, fazendo assim. Só isso. Não dá para
proibi-los: “Não me critique! Não me critique!”, não dá, o mundo é
assim. Viver no mundo é assunto mundano. Se entendemos, podemos dizer
que isso nos ensina a ser uma boa pessoa; se pensarmos bem, tudo isso
nos ensina a refletir, só há Darma ali. Não pense: “Que ele não venha
me criticar, cada um no seu canto!” Não pense dessa forma. Queremos que
eles pensem de um jeito, mas as pessoas não pensam daquele jeito. E aí,
como é que fica? Queremos que eles pensem assim e eles não pensam
assim. Onde vai acabar isso? Não queremos que eles falem desse jeito e
eles falam exatamente daquele jeito! Onde vai acabar isso? Aquela
pessoa não para e nós não paramos, onde vai acabar isso? Não vai
acabar. É assim. O correto é: se ele não parar, nós paramos. Não é? É
assim que se resolve. Se ele não parar, nós paramos. Se nós paramos,
ele não corre para lugar algum, ele estava correndo por uma razão.
Quando paramos de correr atrás dele, ele também para, se volta e
reflete sobre nós e sobre si mesmo. As pessoas que têm sabedoria
pensam: “Ôôô, estou errado! Ele está tranquilo, ele consegue parar.” O
problema acaba bem ali. Hoje em dia é a mesma coisa, o mundo é assim.

Quem não conhece esse Darma sofre continuamente, não sabe que o
mundo é assim. Portanto, nós que temos sabedoria temos que contemplar
dessa forma, temos que fugir através da sabedoria. Onde as pessoas não
nos critiquem, não existe. Elas nos criticam uma vez, é como se
afiássemos nossa faca uma vez. Elas nos criticam com frequência, é como
se afiássemos a faca com frequência, não é? Se afiamos a faca com
frequência, está sempre afiada, não é? É assim. Todas as coisas que
elas dizem serem boas, serem ruins, são ótimos ensinamentos, só que nós
não usamos para refletir. Se nos agrada, gostamos; se não nos agrada,
não gostamos, desse jeito. Desse jeito não dá, o mundo é assim. Por
isso as pessoas sofrem o tempo todo – pois não pensam dessa forma. Vai
aumentando, aumentando, e sofre com frequência – aí nasce sabedoria com
frequência. Sofre com frequência, tem sabedoria com frequência, tem
sabedoria com frequência e então surgem benefícios com frequência. Isso
é resultado do nosso pensamento correto.

(…) deve ser porque estamos indo para casa que esfria, se nós
pensarmos que não vamos mais embora deve esquentar. 

-- O senhor ainda não enterrou as \textit{nimitas}?\footnote{Neste
caso específico, nimitas significam pedras que demarcam os limites do
bôt (uma espécie de templo). É tradição que essas pedras (em geral
esféricas) sejam enterradas sob o solo.}

-- Ainda não, ainda não está pronto, a estátua principal do Buda
ainda não veio. Vem de Bangkok.

-- E essa que está ali?

-- Aquela é a secundária, a principal é do mesmo tipo, mas maior,
mais alta, por volta de quatro metros. Hã? Ajahn Kitbóbut\footnote{A
palavra ‘ajahn’ significa ‘professor’ mas também é utilizada na
Tailândia de forma parecida com ‘doutor’ no Brasil, ou seja, qualquer
pessoa que possua um certo status é chamado de ‘ajahn’ tal como no
Brasil qualquer pessoa que possua uma posição de certo respeito é às
vezes chamada de ‘doutor’.} é o artesão. Se não estiver bonita ele não
vai trazer. Nós não fizemos um contrato escrito, fizemos um acordo
verbal para agosto e agosto já passou… Encomendamos essa estátua sem
fazer contrato. Vários milhares de bahts e não há contrato. Ele não
gosta de fazer contrato, mas se não houver contrato como é que vai ser?
Por volta de 300.000 ou 400.000 bahts. Mandamos fazer a estátua e não
há contrato, não há documento. As pessoas comuns não fazem isso, as
pessoas hoje em dia não aceitam. Se fosse para fazer um contrato, Ajahn
Kitbóbut disse: “Eu vou fazer essa estátua, mas eu não sou empregado.
Eu não sou empregado, não quero pagamento, vou oferecer minhas
habilidades ao monastério de forma satisfatória, assim é mais correto.”

Sendo assim, como fazer? Eu falei “Vamos fazer assim:” – eu reuni a
comunidade leiga – “esses 400.000 bahts para construir a estátua, se
for para vocês conversarem com o artesão, a estátua não vai sair, pois
um não confia no outro. Façamos dessa forma, se for para construir, vai
ter que ser assim; acreditem em mim. Certo ou errado, dessa vez abram
mão. Esses 400.000 bahts vocês deem para mim, não se envolvam. Se eu
jogar no rio, vocês não digam nada, se eu usar para abrir uma loja,
vocês não digam nada, já não é mais assunto de vocês, é assunto meu.
Deem para mim e vocês abram mão do direito a dar opinião. Não fiquem
pensando ‘Como é que vai ser?’ Não pensem, fiquem tranquilos. Eu me
decidi a construir essa estátua, a ter uma estátua. Se conseguir,
ótimo, se não conseguir, ótimo. Que ninguém fique chateado. Se for
assim vamos conseguir fazer. Não é necessário fazer contrato, vamos
tratar em termos de bondade, honra, mérito de ser gente. Se o Ajahn
Kitbóbut não morrer, tiver senso de dever, tiver apreço pela honra
dele, já está bom. Ele nasceu nesta vida, se quiser jogar isso fora,
que jogue! Eu também jogo fora esses 400.000 bahts, sem remorso. Nem
quero saber para onde vai. Se não for assim, não vai ser possível
construir essa estátua.”

Feita a forma, veio folhear a ouro aqui em frente ao monastério, na
rotatória. Encheu de gente, folheamos a ouro bem aqui, as pessoas ao
redor. Terminado, botou a estátua no caminhão e foi embora. As pessoas
disseram: “Eh… por que levou embora?” Eu disse: “Para onde for, não
importa! Que leve embora! Deixe estar. Qual o problema? Se ele pode
fazer assim, nós também podemos. Nós já abrimos mão, então deixem ir.”
Aconteceu que ele disse que ia trazer de volta em agosto, agosto já
passou e ainda não veio. Os leigos vieram me pressionar, eu falei:
“Calma, calma, se sumir, que suma! Não tenham medo, nós já abrimos
mão.” E aí fui ajustar a estátua, fui a primeira vez e achei que já
estava bom, mas não gostei, ajustei de novo. Na segunda, achei que ia
ficar bom, mas ainda não gostei e mandei fazer de novo. Ajustei três
vezes. Um dia ele veio ouvir Darma e disse: “Não tenho coragem de
prometer. Na verdade não falta muito. Eu vim dizer ao Luang Pó que eu
vou terminar essa estátua antes do ano novo com certeza, mas já prometi
várias vezes e então não tenho coragem de prometer de novo.”

-- Ele tem reputação, não?

-- Com uma reputação daquele tamanho, se ele quiser jogar fora o
Wat Pah Pong, que jogue! Eu também jogo fora os 400.000 bahts. O Darma
tem preço? O Darma não tem preço. Dizer que não tem preço não é o mesmo
que dizer que não vale nada. O Darma não tem preço. É assim. Já as
\textit{nimitas,} colocamos no lugar e um rio de gente veio folhear
ouro,\footnote{No caso anterior, da estátua do Buda, significa de fato
folhear a ouro – o artesão usa ouro derretido para cobrir a imagem. Já
no caso nas nimitas citado aqui, significa que as pessoas vão trazer
pequenas folhas de papel metálico de cor dourada (só em casos muito
raros são de fato folhas de ouro) e, usando um pouco de cera de abelha,
colam esse metal à nimita (às vezes também às estátuas do Buda).
Algumas pessoas fazem isso como uma forma de reverência, mas a maioria
faz porque acha que como resultado desse ato elas vão ficar ricas.} e
aí fechei, quando fui para o exterior eu fechei. Se forem fazer quando
não estou, vira uma bagunça. Eles não podem mais folhear a ouro, mas
não dão ouvidos, ainda vêm jogar coisas dentro.\footnote{As nimitas são
enterradas sob o solo, e o significado aqui é que as nimitas já foram
postas nos buracos, mas ainda não foram enterradas, e as pessoas se
aproveitam para jogar objetos dentro do buraco como forma de reverência
(na maioria dos casos jogam moedas ou pequenas imagens do Buda).}
Antigamente vinha um rio de gente folhear ouro. Então eu fechei.
Enquanto isso o Ajahn Kitbóbut continua trabalhando e eu vou esperar.
Agora só falta ele trazer de volta. Trazer de volta e construir a base.
Eu disse para juntar o preço da base com o preço da estátua, ele então
disse que se juntar com o preço da estátua ele não faz. “Eu trabalho
mas não sou empregado, se for para fazer como oferta, eu faço.” Bom,
não? Falou assim, bom, não? Então está bom! Eu sei que ele vai fazer um
bom trabalho.

Não tem perigo. Se não fosse assim, eu não teria oferecido minha
vida ao \textit{Buddha Sāsana}. Monge joga fora várias coisas, joga
fora tudo. Coisas que no mundo as pessoas desejam, acham divertido, os
monges jogam tudo fora. Não é? Se não jogar fora, não consegue virar
monge. Como vai conseguir? Eu já joguei fora minha vida, por que não
haveria de jogar fora 300.000 ou 400.000 bahts, algo que não faz parte
do meu corpo? Se tiver que dar, dou, se fizer, fez – já conversamos. Se
ele fizer algo errado, quem está errado é ele. Ele está errado, nós
estamos certos. Se foi ele quem fez errado, para que vamos esquentar a
cabeça? Temos que esquentar quando somos nós que fazemos errado. Se a
outra pessoa foi quem fez errado, para que vamos esquentar? A gente
costuma ser assim, mas na verdade deveria ser a pessoa que fez errado
que deveria esquentar. Ela faz errado e nós é que esquentamos, é muita
burrice. Entende? Fique de olho. Quem é dono de casa, dona de casa,
fique de olho: quando os outros fazem algo errado, somos nós que
sofremos. Pense direito. Quando acontecer, vai lembrar: “É como o Luang
Pó falou.” Não é melhor se a pessoa que fez errado sofra? Agora, ela
fazer errado e nós sofrermos é o quê? É o cúmulo da burrice! É assim.
Estudamos, mas passamos por cima disso; essas coisas, as pessoas não
sabem. Ele fez errado, ele que sofra! Vamos sofrer ao invés dele para
quê? Esse conhecimento é difícil encontrar onde estudar.

Difícil estudar, mas na verdade é assim. Quem fizer errado que
sofra, para que vamos sofrer? Nós já conversamos, ele disse que ia
fazer, já combinamos. Se ele não fizer, é ele quem está errado, vamos
sofrer para quê? Nós não temos culpa. Que seja! Aonde formos vai ser
assim. Isso é o que se diz: “Em qualquer trabalho que fizer, não vença
as outras pessoas – vença a si mesmo para vencer os outros.” Se vencer
os outros não vence a si mesmo, não vence ninguém. Se vencer a si mesmo
vai vencer os outros. Isso, os estudantes passam por cima. A maioria só
quer vencer os outros e por isso sofrem. Não é? Se não vencer a si
mesmo, quando estará bem? Vencer os outros é só sofrimento. Não serve
para nada. Os estudantes passam por cima disso. Aprendem coisas muito
elevadas, mas passam por cima disso. Não aprendem de forma balanceada,
então passam por cima, não tocam no ponto certo. Não serve.

Bondade é a mesma coisa: as pessoas só querem “bem, bem, bem”, mas
também tem que haver equilíbrio. Se for bom demais, vira maldade, não
é? Nós só queremos “bem, bem”, e vira maldade. Idade é a mesma coisa,
quer viver muito tempo. Bom, não é? “Que tenha longa vida, que tenha
longa vida…” Eu estou doente e as pessoas dizem: 

“Que o senhor viva até os 100, 200 anos!” 

“Hum!? Não venham amaldiçoar o monge! Vêm amaldiçoar o monge para
quê? Já viu gente velha? Não podem andar, ficam deitados, não conseguem
comer. Vocês gostariam de ser assim? Então para que vêm pedir que eu
seja assim?”

Vocês querem? Eu não vejo utilidade nenhuma, não tem utilidade. Só
vêm amaldiçoar o monge. Se fosse para ter vida longa de forma
agradável, então tudo bem, se puder andar, então tudo bem, mas com
idade avançada não consegue sequer cuidar de si mesmo, como isso pode
ser bom? É como as pessoas que amam o pai e a mãe, cuidam com todo
amor, mas se eles estiverem doentes por muito tempo, cinco anos, seis
anos – não morrem nem vivem de verdade e no final todo mundo vai
embora. Às vezes até esquecem de dar comida… Até que ponto nossos
filhos, genros e noras nos amam? Se passar da medida eles não conseguem
amar. Eles jogam fora, não é? Eles pensam consigo mesmos: “Quando será
que ele vai morrer? É muito difícil.” Eles pensam consigo mesmos, ficam
reclamando como se fosse um mantra: “Disseram que ia morrer, já
amanheceu e ainda está vivo. Que dificuldade!”

Entendem? Que as coisas não aconteçam de acordo com os nossos
desejos! Aceite o que vier, não deseje vida longa. Algumas pessoas
querem vida-longa caso estejam felizes; caso estejam com raiva, querem
morrer hoje mesmo, agora mesmo, aqui mesmo!\footnote{Na Tailândia, uma
forma bem-humorada de expressar raiva é dizer “Quero morrer! Quero
morrer!”} Quem vai conseguir agir de acordo com os desejos da mente?
Se quiser viver muito, vai a vários templos estender sua vida, nove
templos,\footnote{Algumas pessoas têm a superstição de que se fizerem
uma certa cerimônia, num certo dia, num número específico de templos,
elas conseguirão estender seu tempo de vida.} vai se benzer para ter
vida longa. Mas quando está com raiva: “Quero morrer agora, hoje!
Amanhã não quero, tem que ser agora!” Às vezes bebe inseticida e morre,
muitos se matam com um tiro. Dá para dizer que pensamos certo? Eu quero
só o suficiente, esta vida em que nascemos não nos pertence, não nos
pertence. É como se fôssemos comprar açúcar no mercado:

“Quanto é um quilo?”

“Um quilo, 20 bahts.”

O vendedor vai colocar no saco e o cliente diz: “Não quero, deixa
que eu coloco sozinho!”, bom, não? O vendedor só coloca no saco, ele
não compra açúcar; se o comprador quer colocar sozinho, é assunto dele,
a gente vende e fim de papo. Isso é pensar na medida certa.
Experimente. Sempre que surge sofrimento é porque passou do limite. Às
vezes é pouco demais, às vezes é muito; quando chega no ponto, é a
medida certa, vem à tona a palavra “suficiente”. Não é?

Por exemplo, estamos limpando uma cumbuca ou um prato, quando está
limpo dizemos: “Já chega, é o suficiente.” Tranquilo, não? Se ainda
estiver sujo, ainda não é o suficiente, ainda não acabou, aí
continuamos limpando. É o que se chama “pessoa insuficiente”. Uma
pessoa suficiente é tranquila: “OK, já chega”, fica tranquilo e não se
preocupa. Isso é uma pessoa suficiente, na medida certa. Já pensou
dessa forma alguma vez? A sabedoria dos monges da floresta vai por esse
lado, ela não voa muito alto.

Vocês funcionários públicos também, se é de baixo escalão não está
satisfeito: “É muito difícil ter que depender dos outros”, reclama que
é difícil. Então vai desejando, desejando, buscando o caminho e no
final vira chefe, vira líder. Quando vira chefe pensa: “Agora está
bom”, mas é bom só naquele pequeno instante, quando conquistou; logo
fica com medo de cair, pois subiu alto demais, tem que ter cuidado. Ter
que depender dos outros não é agradável; quando nós somos chefes, são
eles que dependem de nós. Aí vêm os subalternos: “Conserte aquilo para
mim.” Foram eles que fizeram errado e nós é que temos que consertar! O
que fazer? Se não consertarmos, faz com que reclamem; às vezes causa
muitos problemas, pois são nossos subalternos. Quando vira chefe as
dificuldades de ser chefe surgem.

É por isso que digo: pegue só o suficiente. Conseguiu algo,
satisfaça-se com aquilo. Tranquilo. Dê comida todo dia; se não crescer,
então que fique pequeno! Regue as flores todo dia; se morrerem, que
morram! Coloque adubo todo dia, fazemos nosso esforço e isso é o
suficiente, a planta que cresça sozinha. A gente pensa, mas não pensa
direito. Ou o que é que vocês dizem? Pode falar, o que você diz? Daqui
a pouco vocês já vão embora, se alguém quer dizer alguma coisa, fale. 

Vir ver os monges é bom, é auspicioso. Se temos um estado mental
bom, não precisamos de um palácio, ao pé de uma árvore já estamos
felizes. Mesmo sentados na grama estamos tranquilos. Se a mente está
fervendo, pode deitar num colchão bem macio, mas sente queimar por
dentro. Vira para um lado, vira para o outro, não acha conforto, a não
ser que a mente esteja bem; caso contrário, não acha conforto. Confuso,
não? Nascer como ser humano é confuso, não? Hein? Confuso, não? É assim
o mundo. Aonde quer que vá é assim, há conforto só por um segundo. O
mundo é assim. Eu estava conversando com uma discípula: 

“Trabalhar com os outros é difícil, mas se fico sozinha fico com
medo de fantasma; se fico com os outros, acabo brigando.”

“E onde você vai encontrar bem-estar?”

“Se eu morar sozinha vou ficar bem.”

Aí fica com medo de fantasma. Se tem várias pessoas – brigam. Como
fazer para acertar? Se não pensarmos certo, não fica certo. Onde mora o
bem-estar? Em estar correto. Isso é \textit{Sacca-Dhamma}. Se algo
sumir, que suma, se não puder recuperar, que suma! Vai chorar para quê?
O que venha a acontecer conosco, se for para acontecer, que venha!
Depois que surgir vai desaparecer, não force a permanecer! Esse negócio
é assim. Temos que aceitar, aceitar o Darma do Buda. Aceitar todos os
acontecimentos, não fuja dos acontecimentos! Quando algo acontecer,
temos que conhecer aquele acontecimento, temos que aceitar. As pessoas
não aceitam o Darma do Buda, se recusam a aceitar. 

Eu observo quando vou ensinar onde há um funeral, ensino sobre
\textit{anicca, dukkha, anata}, sobre a incerteza. Algumas pessoas
ouvem, algumas ficam sonolentas, algumas dormem, não entendem o Darma.
Mais um morre e choram de novo. Por que isso? É por não ouvir o Darma,
não aceitar o Darma. Se aceitar o Darma, pensam: “Oh, morreu, isso é
normal”, aceita o Darma. Se conhece, aceita. Se estiver doente, toma os
remédios; se sarar, sarou, se morrer, morreu, se nascer, nasceu, é
assim mesmo. Se aceitar os acontecimentos dessa forma, fica tranquilo.
O problema é que não aceitamos o Darma, só ouvimos a nós mesmos.

Ensino os monges que estão internados no hospital: se sarar, aceite,
se não sarar, aceite. Tranquilo. Caso só aceite se sarar, se não sarar
não aceita – sofre, sofre até chorar. Tem que haver resolução: se
sarar, aceito; se não sarar, aceito. Mas no coração tomamos partido do
sarar, o não-sarar não queremos, aí sofremos. Se sarar tem que aceitar,
se não sarar tem que aceitar, tem que aceitar ambos por completo. Se
sarar, sarou – se não sarar, não sarou. Tranquilo. Continuamos os
mesmos, não perdemos nada. Se deixa influenciar por si mesmo e então
não aceita o Darma do Buda. Eu já vi isso, um parente morre, vai
embora, alguém que eles amam morre, vai embora, os monges ensinam dia
após dia, mas quando alguém morre ainda choram daquele jeito. Choram,
“que pena, que pena…” mesmo quando o monge está ensinando ainda choram
na frente dele: “Que pena…” Por que não refletem: “Chorar para quê? Por
que não ficar bem?” É que eles não aceitam, não praticam, não
contemplam. Só pensam em querer que a pessoa permaneça, que ela não vá
a lugar algum.

A mãe de uma pessoa morreu, ela estava chorando quando veio ouvir o
Darma e eu perguntei “Por que está chorando? A sua dívida já não é
grande o suficiente? Ainda não está satisfeita? Ela cuidou de você
desde que era criança, procurou dinheiro para construir casa para você
com muita dificuldade. Ela morreu e você ainda chora? Que defeito você
está achando no pai e na mãe? Ainda não é o suficiente? Ainda quer se
aproveitar mais deles ou o quê? Ou quer que eles ainda venham ajudar a
pagar as contas? Pense bem, por que está chorando?” Tem que pensar. Eu
digo que já chega, se conseguir aceitar fica tranquilo. Se estiver de
acordo com o Darma, fica tranquilo. Não tem mais problemas. Tem que se
preparar desde antes, meditar…
