
\chapter{Os cinco obstáculos}

{\itshape
Palestra dada por Ajahn Sumedho durante um retiro de meditação.}

Um dos primeiros obstáculos à clareza mental (\textit{nīvaranas)
}é a luxúria, cobiça. Esse é o tipo de problema que surge por focarmos
nos aspectos desejáveis das coisas, você se obceca em desejar algo.
Focar nos aspectos desejáveis, esperando, ansiando prazeres, prazeres
sensuais – isso é chamado luxúria. Uma mente cheia de luxúria é uma que
não desenvolveu a faculdade da discriminação muito bem. A pessoa está
tão obcecada com as qualidades agradáveis de algo ou alguém, que ela só
quer obtê-lo. Ela anseia, antecipa, planeja, trama ou fantasia sobre
aquilo. Mantemos a mente ocupada, temos que criar tudo isso. Se você
parar de pensar, o desejo não surge, desaparecerá. Enquanto você
continuar a pensar naquilo, é claro, a luxúria vai continuar sendo um
problema para você. 

Um modo de lidar com luxúria é apenas estar ciente de como a mente
funciona: se você adicionar algum impulso, se surgir uma atração e você
então focar naquilo, então, é claro, surge luxúria. Luxúria é isso. Mas
se você parar de focar a mente naquilo que você quer, que você deseja,
mantenha sua atenção, não deixe a mente ficar presa em fantasiar,
ansiar, esperar, criar histórias – então o hábito da luxúria irá
diminuir, desaparecer. Se sua mente ainda está ocupada pensando nessas
coisas, então, ao invés de pensar nas qualidades desejáveis de alguém,
pense mais nas qualidades indesejáveis. Por exemplo, se você sente
desejo sexual por alguém, então você foca no cabelo bonito dele, nos
olhos, dentes brancos como pérolas, olhos brilhantes e pele linda.
Vemos a superfície exterior como se fosse algo desejável. A mente com
luxúria não foca, não vê nenhuma das qualidades indesejáveis, quando
você sente luxúria você não nota as verrugas, as cicatrizes, as sardas.


Quando nos ordenamos, um dos ensinamentos que nos é dado, o
\textit{upajjhāya} ensina \textit{kesā}, que significa cabelo,
\textit{lomā}, pelos, \textit{nakhā}, unhas, \textit{dantā},
dentes, \textit{taco}, pele. Cabelo, pelos, unhas, dentes e pele.
Recitar isso faz parte da cerimônia de ordenação. Qual o significado
disso? Por que dizemos isso quando nos ordenamos? Essa é uma reflexão
para a vida monástica. Quando dizemos “cabelo, pelos, unhas, dentes e
pele,” isso é no que tendemos a sentir atração nas outras pessoas,
quando apenas olhamos a beleza de alguém. Mas se nós começarmos a olhar
isso à parte, com mais discriminação, o impulso da luxúria diminui. Se
só vemos cabelo, quando só olhamos cabelo, não sentimos luxúria. Mas se
estiverem todos conectados e arrumados de uma maneira agradável sobre a
pele e os dentes e as unhas e etc., então nós sentimos essa atração. Se
você encontrar o cabelo de alguém na sua sopa, não desperta luxúria.
Nem precisa mencionar se for um dente ou um pedaço de pele! Tende a
estimular aversão, não é? Quando você vai além do cabelo, pelos, unhas,
dentes e pele – fica ainda menos atraente. Se você olhar, digamos,
sangue, linfa, veias, nervos, estômago, baço, bexiga, coração, pulmões,
intestino delgado, intestino grosso, estômago, esse tipo de coisa,
gordura e pus, muco… Isso é o que chamamos “reflexão sobre aquilo que
não é atraente.” Aquilo que não é atraente, que não desperta luxúria. 

Nós todos temos essas coisas, todos temos esses órgãos, coração,
fígado, baço, gordura, muco, pus, sangue e todas essas coisas, mas não
refletimos sobre isso. Quando temos luxúria só vemos que belos olhos,
que belos dentes, belo cabelo, bela pele, não pensamos no fígado, baço,
entranhas, esqueleto. Então essa é uma outra maneira de usar o processo
do pensamento, a capacidade discriminatória, para quebrar a tendência
de focar em quão desejável uma pessoa é, um outro ser humano é. Não é
para criar aversão por outros seres humanos, mas para remover a paixão.
Há uma diferença. Não estamos fazendo isso para pensar em quão nojenta
uma pessoa é, não estamos tentando estimular nojo ou aversão, mas sim a
“não-paixão,” significando um frescor mental, não uma mente presa em
“eu tenho que ter, eu quero, eu preciso, eu devo,” qualidades
excessivas da luxúria, mas sim o frescor da “não-paixão.” É uma forma
hábil de utilizar nosso pensamento: em vez de fantasiar sobre como
maravilhoso será, ansiar, esperar, planejar, usar nossa capacidade de
pensamento para reflexão sem paixão. 

Existem essas duas maneiras de lidar com luxúria, eu penso. Uma é
esta maneira: sendo mais discriminativo, trabalhando com a tendência de
focar na beleza, estar fascinado e obcecado com beleza. Ou apenas
impedindo que a mente crie qualquer coisa que seja, apenas estar
atento, notar a sensação da luxúria, o calor no corpo que surge, mas
não criar nada ao redor daquilo, não seguir com qualquer tipo de ação,
fala ou pensamento, ter apenas atenção pura naquilo.

O segundo obstáculo é o obstáculo da aversão. Nós podemos focar,
podemos sentar em meditação nos sentindo incrivelmente negativos e
aversivos, bravos, raivosos. Quando sua mente está cheia de ódio, raiva
e aversão, é claro, você não tem concentração, sua mente não consegue
se concentrar. Mas, uma vez que tendemos a nos sentirmos culpados sobre
a aversão, no nosso histórico cristão, judaico, nós tendemos a sentir
muita culpa pelo ódio. Então a raiva, nós não sabemos o que fazer com
ela, e nos sentimos culpados por isso. Então desenvolvemos um sistema
muito complexo chamado “culpa,” em que sentimos que odiamos a nós
mesmos por sentirmos ódio – é chamado “culpa.” Nós nos odiamos e temos
aversão por nós mesmos por odiar alguém, ou por estar com raiva. Isso é
complexo, somos seres complexos, não é só o ódio, esse tipo de aversão
a algo, mas é também a aversão à aversão. Por causa disso, investiguem:
aquilo sobre o que você sente aversão, você tende a querer aniquilar,
se livrar. A reação imediata é reprimir, tentar escapar. “Eu sou
terrível,” então tentamos fazer alguma outra coisa, pensar em alguma
outra coisa. Dessa maneira, se você é alguém que está tentando
aniquilar ou se livrar da raiva, ódio, ou suprimindo para fora da
consciência, então você tem que permitir que ela se torne plenamente
consciente.

A raiva reprimida ou o ódio que você guardou na sua mente por toda
sua vida, você deve permitir que se torne um ódio plenamente
consciente. O que significa que você deve trazê-lo à tona porque sua
tendência pode ser de reprimir, empurrar, se livrar daquilo. Ao invés
disso, uma forma hábil é realmente odiar, sentar e realmente odiar,
plenamente, completamente, odiar conscientemente. Mas não é
direcionado, não é um ódio maléfico, não é com o propósito de criar
sofrimento para ninguém, mas é o que chamamos de “soltar o ódio,” uma
purificação da mente em que o ódio só pode acabar quando nós o
compreendemos, quando permitimos que ele cesse ao invés de apenas
reprimir. Nós temos que conscientemente permitir que o ódio se torne
consciente em nossas mentes, não apenas reagir com aversão ou
repressão, aniquilação. Às vezes nos sentamos e nos encontramos ficando
muito aversivos, talvez muitas memórias vindo à tona, sentimentos de
amargura, decepção com parentes, marido, esposa, sociedade, filhos,
seja lá o que for, esse tipo de coisa, e há a tendência de suprimir,
tentar se livrar daquilo. Nós devemos pacientemente trazer à tona para
que se torne plenamente consciente. 

Eu descobri isso no monastério. Querer se livrar do ódio era meu
principal problema, era a culpa e o medo do ódio. E sendo um monge você
pensa que deveria amar a todos, tentando ser um santo, querendo ser
santo e amar a todos e tendo então esse tremendo fardo de culpa por
sentir essa tremenda amargura e ódio. Então eu descobri que um modo
hábil de soltar isso era odiar conscientemente, tentar trazer à tona,
trazer à consciência, tentar pensar, realmente investigar, pensar em
odiar a todos que conseguisse lembrar. Estar disposto a simplesmente
odiar, pensar com aversão e ouvir tudo isso, não acreditar naquilo, mas
ouvir minha mente. “Ele fez isso com ela, ele fez aquilo, ele fez isso
comigo, ela disse aquilo, ele disse isso…” Após um tempo soa ridículo,
se você realmente ouvir, se você levar o ódio ao absurdo, é bastante
divertido. “Minha mãe nunca me amou de verdade, meu pai isso, aquilo…
eu não recebi minha devida parte, ninguém me entende.” 

Ouça, traga à tona, à forma consciente, ouvindo, mas não
acreditando. Acreditar é “ninguém me ama e eu fui mal tratado pela
vida,” e se você começar a acreditar, então é o que realmente acontece.
Mas sim escutar as reclamações, o amargor, a decepção para que você
saiba o que eles são: são condicionamentos da mente que agora você pode
conscientemente aceitar na consciência e deixar que se vá. É um “deixar
ir,” um tipo de limpeza da mente em vez de: “Eu não deveria ser desta
maneira, eu não deveria odiar as pessoas, afinal de contas eu deveria
amar a todos, mas eu tenho esses rancores e não consigo perdoar as
pessoas, eu sou terrível,” e então começamos a odiar a nós mesmos.
Portanto, nós ouvimos nosso ódio por nós mesmos. Ouça, se você tem
muita aversão por si, traga à tona. “Eu não valho nada, eu sou
estúpido, não sirvo para nada, inútil.” Então você consegue ouvir o
condicionamento da mente, o sentimento reprimido de aversão, e pode
deixá-los ir de forma plenamente consciente. Você tem uma perspectiva
deles, você os vê claramente e então pode deixá-los ir.

Você não está tentando simplesmente se livrar deles: “eu não deveria
estar pensando assim, isso é nojento” e então reprimir, mas reconhecer
que muito da nossa vida tem sido raiva reprimida e aversão, porque na
sociedade ocidental, posso falar sobre os Estados Unidos da América, os
americanos são pessoas muito idealistas e portanto eles vêm de um
idealismo muito elevado sobre o que um homem deveria ser – é um ideal,
e então temos medo do modo como somos. Nós sentimos uma tremenda falta
ou uma tremenda inabilidade de fazer jus àquele ideal, então começamos
a pensar sobre nós mesmos de formas muito negativas. Porque é muito
raro conseguirmos viver no padrão em que achamos que devemos estar o
tempo todo, o nível ideal. Tendemos a nos sentir incrivelmente culpados
por nossas fraquezas, por nossos fracassos, nossos medos, covardia,
falta de energia, aversão, tudo isso. Nos sentimos terrivelmente
culpados, aversivos a nós mesmos por não fazermos jus a esses altos
padrões.

Um ideal, nós reconhecemos: um homem ideal, uma mulher ideal – é
apenas um padrão. Não é como alguém consegue viver de forma permanente.
Nossa humanidade, nossa condição humana nos faz ter que nos adaptar a
todo tipo de situações que não podemos prever no nível do ideal, as
quais temos que aguentar e aprender com elas. No ideal, você pode
pensar em como você deveria ser um homem ou mulher ideal, isso é uma
coisa, não é? Mas a vida em si é uma energia em constante mudança em
que não conseguimos prever o que vai acontecer, então temos que
aprender a aguentar e aprender através de tentativa e erro. Mas, por
causa dessa falta de entendimento de como as coisas são, tendemos a
criar um monte de culpa, remorso, aversão para conosco ou o mundo, por
causa da nossa falta de entendimento. Agora poderemos entender.

Esse é o obstáculo da aversão, é aversão porque é algo que tendemos
a não gostar e querer nos livrar. Então, nós temos que trazê-la para
nós, torná-la uma aversão plenamente consciente, ouvi-la. Se você
encontra muita raiva por seu esposo ou esposa, tem a sensação de que
eles não te entendem, sensação de insatisfação – ouça aquilo. Ouça a
aversão em vez de tentar descobrir o que está errado em você ou seu
esposo. Isso é o suficiente, ouvir a aversão para que se saiba
exatamente o que é. Talvez haja fundamento para ela, mas não é
necessário transformar em nada. É melhor deixar passar do que esperar
que o mundo, seu esposo, seus pais, seus amigos, filhos se enquadrem em
todos os seus desejos.

Em Chitrust\footnote{O primeiro monastério da tradição da floresta
criado no ocidente, Ajahn Sumedho foi o fundador e o primeiro abade
deste monastério.} eu era o monge chefe do monastério, às vezes eu
podia ver aversão surgindo com relação a algumas pessoas no monastério,
eles estavam causando problemas, sendo difíceis. Eu dizia “Por que eles
têm que fazer aquilo? Por que eles não podem ser iguais a todo mundo?
Por que eles têm que criar problemas com tudo? Por que eles não podem
ser mais atentos, ter mais consideração? Por que eles não podem ser
sensíveis? Por que eles não podem praticar como eu venho ensinando? Eu
estou ensinando, fazendo meu melhor para ensinar e eles não fazem
nenhum esforço, ficam com preguiça e de repente fogem, nem me dizem que
estão indo, eles simplesmente fogem!\footnote{Quem já viveu num
monastério ou numa comunidade de praticantes já deve ter passado por
essa experiência: muitas pessoas estão procurando por algo que elas
mesmas não sabem o que é, e às vezes elas aparecem no monastério
esperando que o local faça jus aos ideais delas, sem dar importância às
pessoas que vivem ali ou ao modo de agir já estabelecido no local. Esse
tipo de gente costuma criar muitos problemas, reclamam, criticam, criam
divisão, tentam angariar partidários, criar revoluções, e após toda
essa bagunça, como é comum às pessoas de mente confusa, elas
simplesmente se entediam e vão embora, desaparecem, vão procurar novas
“causas nobres” pelas quais lutar em algum outro lugar.} Vivendo de
esmolas! Ingratos, insensíveis.”, etc., e vai dessa forma. E também
sobre qualquer um que lhe cause frustração, ou que seja difícil, você
pode pensar: “Queria que ele fosse embora, queria que ele fugisse.”

Ouvindo isso, ouvindo essa aversão em minha mente – eu usei isso
como meditação. Alguém sendo muito difícil e eu reagindo de maneira
negativa – eu ouvia. E o que saiu disso, o que eu descobri é: “Eu quero
que todos vocês ajam de uma maneira que nunca me incomode, que sempre
me gratifique, me faça feliz, não façam nada que me traga medo, dúvida
ou preocupação, e vocês devem viver suas vidas exclusivamente para o
meu benefício e conduzam a si da forma que eu quero, para que eu não
tenha que me incomodar.” É o que eu percebi estava dizendo, era como eu
estava agindo. “Eu quero que todos os monges, todas as monjas neste
monastério, todos os leigos ajam de forma que não me incomode.” É muito
estúpido, não é? Se é isso que eu quero, é inútil, não é? Esperar que
todos ao meu redor se comportem de forma que não me cause sofrimento,
nem me deixe nervoso ou incomodado por nada. Quando eu realmente
descobri isso comecei a ficar muito mais tolerante e mais disposto a
permitir que mais pessoas fossem do jeito delas. Não me sentia mais
ameaçado por idiossincrasias ou excentricidades nas pessoas ou pela
rebeldia delas ou a teimosia, eu não mais as tomava pessoalmente como
uma ameaça para mim. Comecei a relaxar, a dar bastante espaço para as
pessoas e ser capaz de refletir melhor àquelas pessoas em vez de apenas
forçá-las a se comportar. Os relacionamentos ficaram muito melhores
quando comecei a permitir que as pessoas trabalhassem com os problemas
delas, em vez de forçá-las à conformidade ou me livrando daqueles que
não se conformavam: “Você não é adequado para ser um monge, saia
daqui!” Você pode permitir às pessoas serem como são porque você não
exige que elas sejam de outra forma. Você percebe que é assim que elas
têm que ser naquele momento. Essa é uma boa reflexão para as pessoas
também, não é? Então você pode realmente ajudá-las porque apenas
forçar, castrar, amedrontar as pessoas até elas conformarem é apenas
condicionamento por medo novamente, como animais. Elas talvez se
comportem porque estão com medo, ou porque querem me agradar, mas não
por real sabedoria ou por entenderem o problema. 

Então ficou claro, e ficou muito mais fácil ser um professor quando
eu não estava carregando tudo, quando eu não estava sentindo medo ou
ameaçado pelo que estava acontecendo à minha volta. Na verdade isso
veio através de refletir sobre minha própria mente. O que estava
realmente me incomodando era medo. Medo de que as coisas dessem errado,
medo de fazer algo errado, medo de alguém arruinando algo ou me
causando muito sofrimento ou perturbando a comunidade. Havia um medo,
um protecionismo paternal, do tipo que, eu observo, pais e mães devem
sentir – querer proteger a família de qualquer coisa danosa ou
subversiva. Então se você vê alguém causando muitos problemas ou
desilusão dentro do monastério, quer se livrar deles: “Saia daqui, nós
não queremos você aqui porque você está perturbando tudo.” Um tipo de
desejo de proteger. Como um “pai,” pai tentando proteger sua família de
influências estranhas, incomuns ou subversivas. Isso era se apegar e
tornar-se um pai. 

Refletindo nisso, você consegue ver o tipo de armadilhas em que
podemos entrar, porque soa bastante razoável – se incomodar, proteger
os monges e monjas… Muito admirável. “Ele é um ótimo abade, muito
protetor, ama muito sua família, toma conta dela, isso é admirável.”
Mas também é uma armadilha no sentido de que, se identificando com um
pai, sendo protetor, terá um resultado cármico muito forte. O senso de
dependência e também a exclusão daqueles que você não sente se encaixam
no grupo primário ou no monastério. Você não está criando uma família,
você não quer um grupo privado de amigos íntimos à exclusão dos demais
que você acha não fazem parte, que você pensa podem causar problemas ou
perturbações. Então, você abre o Darma para qualquer um que vier e
pedir, não importando quais são seus sentimentos sobre aquela pessoa.

A forma de medir isso é a quantidade de sofrimento que você tem, não
importa o que esteja fazendo. Machuca. Quanto sofrimento vocês têm em
relação aos seus filhos? Protecionismo, querendo que eles se comportem,
medo de cometer erros, medo de que você faça algo errado, culpando a si
mesmo por não ter sido capaz de sempre ser o melhor e mais sábio.
Sofrimento dos pais, vocês podem refletir sobre isso, esse tipo de
apego, posições às quais você se apega. E todos esses apegos nos trazem
a esse sofrimento. Portanto, a quantidade de sofrimento que você tem
reflete isso. 

O primeiro obstáculo é luxúria ou ganância, o segundo é aversão.
Dois extremos, as grandes paixões. Desejo por algo e desejo de se
livrar de algo. Os três obstáculos seguintes se manifestam quando as
grandes paixões diminuem em sua vida, grandes quantidades de luxúria e
ódio. Quando está livre delas você encontra os estados mentais mais
entediantes, deludidos, eles se tornam mais conscientes, como torpor e
abatimento.

Torpor e abatimento. Antes sua mente tinha uns pensamentos de
luxúria bastante interessantes, ódio verdadeiramente passional e
inveja. De repente é esse abatimento, tédio, cansaço. É aqui que muitos
monges abandonam a vida monástica, quando eles têm que enfrentar
abatimento e torpor, porque daquela luxúria e ódio você tira muita
energia, você ganha muita vitalidade, uma fantasia sexual lhe dá muita
energia. Se você está se sentido abatido e começa a fantasiar sobre
sexo, você… Se você está abatido e deprimido e ódio surge, você pensa:
“Aquele fulano…” Quando estou realmente bravo posso me sentir realmente
vivo, muito afiado, claro. Indignação, eu tenho muita tendência à
indignação, eu fico indignado: “Como ele se atreve a dizer aquilo!? É
revoltante!” E parece tão correto, se alguém faz algo que você sabe é
muito irritante ou estúpido ou horrível, o que quer que seja: “Como ele
se atreve!? Terrível!” É tão correto, tão indigno, tão vivo, e além
disso eu tenho tanta razão! Tem muita energia conectada a isso.

Eu noto que alguns monges em Chitrust tendem a obter muita energia
ficando indignados com coisas. “Você sabe o que aquela pessoa disse?
Sabe o que ele está fazendo? Ele está saindo para fumar cigarros na
floresta!” Se você está num estado abatido e vê alguém fazendo algo
errado, você pode ficar indignado: “Como eles se atrevem!?” Você
assiste àquilo, observe como você obtém energia, porque fantasias de
luxúria, indignação, aversão, raiva passional, ódio e tudo isso lhe dão
muita energia. E então quando você pensa em abatimento e torpor, esses
são estados muito desagradáveis, não gostamos deles porque você acaba
existindo nesse tipo de estado mental abatido, deprimido, e a vida
parece sem esperança como um deserto infinito. Nada naquele deserto a
não ser areia e tédio, nada interessante para olhar, nada tentador. Nem
sequer algo para se indignar, só cansaço, abatimento. Então, nós
podemos afundar nesse abatimento e não colocar nenhuma energia, porque
se nossa energia está sempre vindo da cobiça, da luxúria ou do ódio,
quando essas coisas não estão operando, nós não temos energia alguma.

Temos que começar a trazer energia da nossa própria mente, não é? É
aqui que você começa a trazer energia para o momento. Então você mantém
sua postura, você existe com o abatimento, você investiga o abatimento,
a sonolência – alerta. Às vezes temos tanta aversão ao abatimento que
só queremos nos livrar dele, e por simples força de vontade nos
colocamos eretos para conquistar o torpor, então você vai dessa forma
por um período, muita energia, mas logo vai de um extremo ao outro,
você cria um hábito. Eu vi monges fazendo isso na Tailândia, você os vê
sentados eretos assim, de repente eles simplesmente caem desse jeito e
voltam imediatamente à postura ereta. Não é muito atento, não há
reflexão, você não está trabalhando com o abatimento, você só está
tentando se livrar dele. “Eu odeio este abatimento e eu não vou me
submeter, vou aniquilá-lo.” Há aversão ali, não é? Você ganha a energia
para fazer esse extremo somente por aversão ao abatimento. Quando
passa, você cai. Você precisa usar algo mais sutil, uma reflexão mais
sábia sobre o abatimento e sonolência em vez de apenas ter aversão e
tentar aniquilá-los através de um ato de força de vontade. Portanto,
quando você se sentir abatido, conscientemente note o abatimento e
sonolência. Você aguenta, você trabalha com aquilo, você mantém a
energia, tente manter o corpo ereto e vencer. Continue trabalhando com
isso, com gentileza e paciência. Até você começar a largar o
abatimento, permitir que o abatimento e o torpor se vão, ao invés de
apenas reprimi-los. Se não, você está apenas reprimindo de novo, o que
significa que ele vai continuar voltando.

E o quarto obstáculo é igualmente desagradável: inquietação. É
quando você está inquieto, você se sente inquieto, sua mente não possui
ódio passional, cobiça, você certamente não está abatido, você apenas
quer ir a algum lugar. Uma terrível inquietação, ansiedade, agitação.
Nós também tendemos a tentar reprimir isso, mas quanto mais você tenta
reprimir, mais agitado você fica. Para se concentrar, você não coloca
mais energia na postura, você apenas mantém a postura e aguenta com
paciência, olhando a inquietação, observando com uma atitude bondosa.
Bondade, paciência. Estar disposto a aguentar o que há de desagradável
naquilo sem reagir, sem fugir, estar disposto a aguentar, para sempre
se necessário, com uma paciência incrível.

E o último é “dúvida,” estado mental cheio de dúvidas. Esse é um
grande problema para as pessoas, elas sempre se enroscam em dúvidas.
Dúvidas sobre si mesmas, dúvida sobre o que elas deveriam fazer, dúvida
sobre a prática, dúvidas sobre budismo, dúvida sobre o professor,
dúvida sobre toda e qualquer coisa. “Não sei se estou fazendo certo,
não sei… talvez eu não seja capaz, talvez este não seja o caminho certo
para mim, talvez isto esteja além da minha capacidade, talvez eu
devesse investigar o cristianismo mais cuidadosamente, talvez eu
devesse… talvez \textit{mahāyana} seja mais… não tenho certeza sobre
o Ajahn Sumedho… o que eu faço em seguida? Devo concentrar na minha
respiração ou… Não foi o que ele disse? Quando você ganha um certo
nível de concentração você chega a atenção pura… o que significa isso?”

Dúvidas. Dúvidas. Então, você não é isso. Dúvida, esse estado de não
saber algo. Na nossa mente estamos acostumados a saber coisas. Nós
tendemos a evitar dúvidas ou, assim que uma dúvida surge na mente,
tentamos encontrar a resposta. Você quer tudo decifrado: “Eu faço isso
e depois faço aquilo, então eu sei exatamente o que fazer.” Mas se você
deixar: “Eu não sei o que devo fazer em seguida…” você se sente muito
incerto, inquieto, inseguro, e não gostamos disso. Queremos respostas
definitivas, tudo definido, claro, clarificado, enumerado em detalhes,
este estágio, o próximo, etc. “Onde estou agora como praticante do
Darma? O que já alcancei? Já alcancei alguma coisa? Será que alcancei
algo? Eu vou perguntar ao Venerável Sumedho: ‘já alcancei alguma
coisa?’” Eu costumava tentar espremer informações de Ajahn Chah… era
inútil. Ou tentava ludibriá-lo a dizer algo sobre minhas realizações.
Porque gostamos de saber, nós esperamos saber o que alcançamos.
Ridículo, não é? Então, esse tipo de dúvida, pensar que alguém sabe
algo que você deveria saber e assim por diante, você traz isso à
consciência. 

Esse é um método muito hábil, a mente em dúvida. Você pode usar o
\textit{koan}: “Quem sou eu?” Quem sou eu? A mente fica bem vazia, não
é? Eu sou o Venerável Sumedho, não é? Eu já sei disso. Não estou
fazendo essa pergunta para obter essa resposta, não é? Eu já sei essa
resposta. “Quem sou eu?” Há a pergunta e, naquele momento, se você
estiver realmente atento, no final da pergunta que você fez há um
vazio, a mente fica em branco por um instante, não é? “Quem sou eu?” –
vazio. Naquele instante não há nada na mente. Então você começa “Bom,
eu sou… como assim ‘quem sou eu,’ o que é que eu deveria ser?” Começa
de novo.

Então você pergunta “Quem sou eu?” e ouve aquele vazio, o ponto onde
não há pensamentos. Continue se concentrando naquilo, e trazendo aquilo
à tona, estando plenamente atento àquele ponto em que a mente está
vazia. Quando se fala sobre a mente vazia as pessoas pensam: “O que
isso significa? Eu acho que nunca experienciei vazio, talvez seja algo
sobre \textit{suññata}, o que eles querem dizer com isso?” É claro que
a mente não está vazia quando você pensa: “O que eles querem dizer com
\textit{suññata}?” Você está ocupado procurando uma resposta e então
você não nota o vazio, não é? Você pensa que há uma resposta, algum
tipo de resposta para isso. E você se sente muito desconfortável: “O
que ele quer dizer? Certamente não é isso. Eu tenho que desvendar,
descobrir. Talvez eu deva ler mais livros sobre Zen ou algo do tipo…”
Você lê todos os livros Zen e sua mente ainda não está completamente
vazia e você ainda não entende. “Eu não entendo o que esses budistas do
Zen estão falando. Eu li todos esses livros falando sobre vazio, ‘o som
de uma mão batendo palmas’ e ‘qual é seu rosto original’ e ainda
continuo questionando, para que isso?” É porque ainda estamos
procurando por algo, aquele vazio é… As pessoas que têm esse desejo
desesperado por coisas definitivas, tudo definido e esclarecido e têm
muita fé em respostas e conceitos, acham o vazio ameaçador…

