
\chapter{Conversa com um doutor}

{\itshape
Conversa informal entre Ajahn Chah, um médico e sua esposa.}

\textit{(Dr.)} - Tanto quanto me lembro de ter lido uma vez, havia
uma pessoa de família pobre e esse garoto tinha inteligência, de forma
que os pais e os irmãos achavam que ele seria capaz de estudar e
conseguir se formar. Então os pais e os irmãos todos abriram mão de
estudar para conseguir economizar dinheiro para pagar os estudos deste
garoto. Todos passaram por muitas dificuldades. Esse garoto foi para a
universidade, mas também estudou o Darma e sentiu inspiração profunda.
Quando terminou seus estudos, aconteceu que… o pai, a mãe e todos os
demais tinham grande esperança de que, quando ele estivesse formado,
ele sustentaria a família. Seria capaz de trabalhar e ganhar dinheiro,
ajudando os irmãos mais novos ou ajudando a dividir o fardo do pai e da
mãe, ou seja, o pai e a mãe iam poder descansar um pouco para poder
ajudar os filhos menores, mais adiante. 

Mas aconteceu que esse garoto, quando acabou seus estudos, ficou
muito inspirado pelo Darma, a ponto de pedir licença do pai e da mãe
para não ir trabalhar, queria ordenar-se monge. O pai e a mãe choraram
e tentaram impedir, mas o filho disse ter muita fé no ensinamento do
Buda: “Não me impeçam!” No fim os pais tiveram que dar permissão para o
filho virar monge\footnote{Uma pessoa só pode virar monge caso
autorizado pelos pais.}, mas do meu ponto de vista, tanto quanto eu li
e ouvi dizer, sinto que eles provavelmente não o fizeram de muita boa
vontade. Essa história ficou presa na minha cabeça. 

\textit{(Ajahn Chah)} - Por quê? 

- Peço desculpas, Luang Pó, essa é apenas minha opinião, não sou
\textit{expert} em religião, mas penso que neste mundo tem que haver
duas partes: uma parte é a religião, outra parte são os donos de casa
normais, que necessitam exercer profissão, ganhar seu sustento. Por
exemplo: eu tenho família, tenho filho e esposa. Tendo eu
responsabilidade pela minha família, tenho que cumprir meu dever para
com eles. Estou vinculado a eles e tenho que cumprir meus deveres para
com a minha família, a sociedade, a nação. Agora, tanto quanto li aqui
neste lugar, talvez tenha sido escrito por outra pessoa, não sei… diz
que o objetivo é que todas as pessoas virem monges. Eu penso na minha
cabeça que neste momento sou dono de casa, sustento minha esposa e
filho com meu ofício, dou felicidade a algumas pessoas, dou ajuda a uma
parte da população e faço um pouco de doações ao templo, ou seja,
também ajudo a religião. Mas, se todos virarem monges, incluindo a mim
mesmo, os monges vão ter que arar a terra, vão ter que procurar ofício,
não terão tempo para praticar a regra monástica. Os monges não terão
oportunidade de aconselhar e guiar o povo, dar luz para que haja paz de
espírito, para que haja um caminho a seguir neste mundo. 

Por essa razão, na minha opinião, essa pessoa, se for virar monge
está bem, mas somente quando terminados seus deveres, ou seja, ajudar o
pai, a mãe e os irmãos. Ordenar-se agora, neste momento, se perguntar
para mim, se for para eu decidir dentro da minha ignorância, digo que é
uma maldade, muita maldade. Porque é machucar os pais e é uma maldade
para com os outros também, pois todos ajudaram essa pessoa, e aqui
surge o problema: essa minha opinião está certa ou errada?

- O doutor também tem razão. Vejamos assim, eu vou lhe fazer uma
pergunta: um quilo de ouro e um quilo de chumbo, se vierem dar ao
doutor, o que o senhor escolhe? 

- O ouro.

- Pois é, é assim mesmo. Quando uma opinião é firme dessa forma,
aquilo ganha um valor diferente, tal como ouro e chumbo, tem que
escolher o ouro. Eles trouxeram ambos, porque o doutor escolheu o ouro
e não o chumbo?

- Porque possui valor.

- Pois é, você enxerga que possui mais valor. Da mesma forma nesse
caso, ele enxerga mais valor do que isso a ponto de se decidir dessa
forma. É assim, é a mesma situação. Por isso não devemos pensar assim,
não há nada de errado em pensar, mas tem que refletir até ficar
correto.

Está com medo, hein? O doutor está com medo de que não haja quem
cuide dos pais, não haja quem construa o mundo? Se estiverem
contratando pessoas para tocar música o doutor não precisa se preocupar
pois eles só vão contratar pessoas que souberem tocar música. Se não
souber, como é o seu caso, eles não contratam. Não dá para pegar todas
as pessoas e fazê-las monges, mas proibir para que ninguém vire monge
também não dá. Essas coisas são assim mesmo, não devemos pensar: “que
todos virem monges!”, não dá. “Que ninguém vire monge!”, dá? Não dá,
não é possível, não dá para forçar desse jeito. Aqui temos que enxergar
o que há: quem tiver sabedoria, a que nível, que aja de acordo.

Eu também já passei por isso. Fui dizer que matar animais é demérito
e uma pessoa respondeu: “Luang Pó consegue comer pimenta pura todo
dia?” 

Respondi: “Não, pois não há quem traga pimenta para eu comer todo
dia, tem alguém? Vá buscar, prepare pimenta para eu comer todo dia,
pode ser?” Não concordou. Quem vai ter tempo de preparar pimenta para
monge comer todo dia, sem falta? Esse tipo de conversa é só da boca
para fora, mas neste mundo é assim mesmo. Forçar todas as pessoas a
serem boas ou ruins é impossível. Quem tem sabedoria neste mundo,
escolhe. “Vejam este mundo enfeitado como uma carruagem real, onde
tolos afundam, mas nos sábios não há apego.”\footnote{Verso do
Dhammapada (Lokavagga – 171)} Essa é a garantia que o Buda nos dá. 

Então, nossa intenção em virar monge não é ver nossos pais
perecerem, nossos filhos perecerem, nossa família perecer ou algo do
tipo. Pensamos: “Puxa, minha família está atolada demais na lama, na
sujeira. Se nem eu conseguir sair, estaremos todos presos, assim será
uma perda total…” Então nos esforçamos para sair, por nossa família,
para que nossa família melhore. Mas algumas pessoas chegam a pensar que
não há proveito, que é maldade, como o doutor. Se olharmos por esse
ângulo, é assim. Por isso perguntei: um quilo de chumbo e um quilo de
ouro, se formos dividir, qual você escolhe? 

Isso é igual. Quando ele decidiu virar monge pelo resto da vida
daquele jeito, ele viu que isso tinha muito mais valor, viu o mundo
inteiro como se fosse chumbo, sem valor, portanto não o deseja. Igual
ao doutor que deseja um quilo de ouro mas não quer um quilo de chumbo,
por quê? Porque possui pouco valor, não tem valor, portanto decidiu
pelo ouro dessa forma. Se soubermos refletir, não é o caso que nos
ordenamos para destruir nossos filhos, destruir nossa esposa, nossa
família ou algo do tipo, não é assim. Não é assim.

Isso é muito difícil pessoas conseguirem fazer: alcançar o máximo de
sabedoria até chegar ao ponto que o doutor viu, onde tudo vira do
avesso. É como ser capaz de fazer a palma da mão virar as costas da
mão. Se a pessoa só consegue enxergar da forma que o doutor enxerga,
ainda não será capaz. Ele age daquela forma com boa intenção e está
correto, age por enxergar de verdade os princípios do Darma. 

Se não é assim, então nosso Buda… é só maldade! O Buda é só maldade.
“Não importa quanto sofrimento minha família passe, mas vou ajudar
essas pessoas, ensinar essas pessoas a vir para a luz. Elas já ficaram
na escuridão tempo demais.” – pensando desse jeito ele decidiu
ordenar-se. Como o filho do coronel Pramôt\footnote{Este rapaz
continuou na vida monástica e eventualmente tornou-se um grande mestre.
As pessoas hoje o conhecem pelo nome Ajahn Piak.}, que terminou os
estudos no exterior e veio ordenar-se aqui comigo. Ordenou-se e decidiu
não voltar à vida laica. Inicialmente o pai viu e pensou: “Oh! Quanto
sofrimento!”, mas hoje em dia ele vai ao monastério com frequência,
ouve o Darma com frequência. Hoje em dia não quer mais ficar aqui na
cidade, quer fugir para Wat Pah Pong, não quer ficar aqui. Há uma
mudança quando enxergamos aquilo que antes não enxergávamos e vemos que
não há benefício algum. Existe benefício porque possuímos sabedoria, se
não temos sabedoria não enxergamos benefício naquilo. 

Mas não pense errado: “Todo mundo virou monge, quem vai morar neste
mundo?” Oh! Ainda mais gente, Sr. Doutor! Não é assim! Não pense que o
mundo vai desvanecer – vai ficar ainda mais firme! Não pense que o
mundo vai ficar ralo – vai ficar ainda mais espesso. Quando o senhor
virar monge o senhor vai explicar sobre a maldade para que as pessoas
entendam, fiquem em paz e tenham felicidade – se não muita, então
pouca. Ajudamos. Ajudamos a pegar a bondade e o que há de bom e ensinar
razão: “Não se agridam, afinal todos temos que ganhar a vida, buscar
nosso sustento. Não agridamos uns aos outros!”, todos os pontos de
vista como esse. Se pensarmos: “Eh, se todo mundo virar monge vai ser
uma desgraça”, ninguém vai poder virar monge. 

Veja os dedos da sua mão: Não puxe os dedos para que eles tenham
todos o mesmo tamanho! Esse dedo é assim, aquele é de outro jeito.
Mesmo não sendo todos do mesmo tamanho, ainda são úteis? Têm a
utilidade deles. Pegue esse pequenino aqui, é útil? Se não for do mesmo
tamanho que os demais, corta fora? Do jeito que é ainda serve para
cutucar o nariz! É muito útil. Não podemos pensar desse jeito pois é
uma união. O mundo não pode ser do jeito que o senhor diz, ou então não
é mundo. Desse jeito o mundo não é mundo. Não seria mundo. Mas é
justamente por ser mundo, que é assim. 

Bem onde está certo ensinamos, mas as pessoas não enxergam. Falam
que sim, mas não enxergam. Bem aqui está errado, ensinamos: “bem aqui
está errado”, mas não enxergam. Mas, para alguns, é só verem que ali
está errado e logo entendem, veem que ali está certo e logo entendem.
Algumas pessoas conseguem ver certo e errado. Não fique com medo que o
mundo vá desvanecer. Não pense que não haverá quem construa o mundo no
futuro.

- Eu entendo, Luang Pó. Eu tenho um amigo que é monge, estávamos
conversando sobre esse assunto de fazer doações ao templo. Eu disse a
ele que não faço muitas doações, vez por outra apenas. Estávamos
conversando sobre isso e escutei um fundamento correto. Conversávamos
sobre as pessoas que fazem doações. Tal qual tenho observado já faz
algum tempo, algumas pessoas são pobres a ponto de quase não ter o que
comer, os filhos não têm o que comer, mas tendo dinheiro, os pais o
guardam para comprar durian\footnote{Fruta típica da região sudeste da
ásia, na época era considerada um artigo de luxo.} para dar aos monges.

Eu perguntei se isso estava correto. Meu amigo disse que está
errado, doações têm que ser feitas sem ser um fardo. Não deve ser:
“Puxa, nossos filhos ou nós mesmos queremos comer, temos para comer,
mas não vamos comer! Vamos dar para os monges comerem! Assim vamos
ganhar mérito.” Meu amigo monge disse que isso está errado. Doações
devemos fazer na medida em que é possível e não sendo um fardo para os
demais. É como nessa história que ficou presa na minha cabeça, do rapaz
que se formou e virou monge, eu penso que essa história de fazer mérito
deve fazer nascer felicidade no nosso coração e no das demais pessoas.
Não é como: “OK, vou roubar a carteira dessa pessoa para ir fazer
mérito.” Essa pessoa ia usar o dinheiro para cuidar do filho que está
para morrer no hospital e eu vou lá e roubo, aquilo traz sofrimento
para outras pessoas. Da mesma forma eu insisto, é por isso que não
consigo aceitar que essa pessoa que se formou e virou monge ganhou
mérito. Eu entendi o que o senhor me ensinou agora pouco, mas eu
insisto nisso: se for fazer mérito, não é correto que acabe em maldade.
Quer dizer, um grupo de pessoas ter que sofrer. Caso alguém faça
mérito, tem que gerar felicidade no coração de todos os envolvidos. 

Por essa razão, esse monge que se ordenou, como já disse, mesmo que
ele vá ensinar e propagar o Darma ou o que seja… Mas se for perguntar
para mim, eu digo que esse monge ainda é moleque em primeiro lugar –
ainda é moleque. Já há o senhor Luang Pó aqui, o senhor Ajahn ali, e
quantos mais que já estão cumprindo a tarefa de propagar o Darma. Mas
esse aí vai querer fazer como se fosse o Buda que abandonou o Rahula, a
rainha, e foi se ordenar\footnote{Siddattha Gotama, que eventualmente
tornou-se o Buda, abandonou sua esposa e filho recém-nascido para viver
uma vida ascética e buscar o caminho para a iluminação.}? Aquilo é o
Buda, o líder, o nosso mestre, o pioneiro, os outros ainda não eram
capazes de enxergar, ainda possuíam muitas impurezas, ainda não eram
capazes de enxergar o caminho para que haja felicidade para a maioria
das pessoas. 

Eu não elogio a ordenação desse monge, no momento atual já há o
Luang Pó aqui, aquele Ajahn ali, aquele outro lá. Vários que proclamam
o Darma de forma que possamos estar aqui hoje. Portanto essa pessoa não
é importante tal qual o Luang Pó, ou a ponto de se fazer como o Buda.
Isso me faz censurá-lo. Eu censuro fazer mérito sem refletir sobre o
momento apropriado. Se esperássemos dois, três, cinco anos e tivéssemos
resolução firme, não desistiríamos. Teríamos firmeza em praticar o
Darma até que chegasse o momento apropriado. Quando não gerasse
sofrimento para os demais, não gerasse dificuldades para os demais,
quando surgisse satisfação em todos com nossa fé, então sim, seria o
momento apropriado para ordenar-se. Eu censuro muito ele não ter
escolhido o momento correto, é o que eu considero uma maldade. É por
essa razão que eu vim aqui, com todo respeito, discutir com o senhor.
Eu digo que o momento não era apropriado.

- Me diga, onde você vai encontrar alguém que saiba prever o
futuro desse jeito?

- Essa pessoa, se aguentasse firme mais sete anos e então se
ordenasse – dessa maneira seria uma boa pessoa. Mas supondo que
passados sete anos desistisse, fosse ter esposa, beber cachaça – nesse
caso é uma pessoa ruim, não serve.

- Onde você vai encontrar isso? Todo mundo quer ser assim. Se for
“sete anos para se ordenar”, a morte vai esperar? Ele enxergou assim
antes de ir. Morte: “sete anos e então morre”, há algum acordo? As
pessoas nascem, mas na hora de morrer: “sete anos para virar monge!”
Ninguém sabe. Tem algum acordo? Quando uma pessoa vê dessa forma, o que
ela pode fazer? Ela não tem certeza sobre sua própria vida. Eu digo
que, de todas as pessoas, não há quem tenha um acordo do tipo: “setenta
anos, oitenta anos e então morre”, não existe, não tem como saber. Ele
olha e não enxerga sua própria morte, ele não confia em sua própria
morte. Se ele enxerga dessa forma ele vira monge com certeza. Por que
não? Se aquela pessoa enxerga dessa forma! Não é que nem nós aqui. Nós
aqui temos que obrigar a ser “seis, sete anos” para que chegue a hora.
\textit{Akaliko} – morte tem dia e hora? Se aquela pessoa vê dessa
forma, o que é que você quer que ela faça? 

- Eu digo que essa pessoa é egoísta. 

- Hum?

- Essa pessoa é egoísta.

- É o quê?

- Egoísta. Por quê? Ela toma a felicidade, que é o deleite na
religião, sozinha. 

- Se é assim, tenho que dizer o seguinte: o doutor aprendeu
medicina, portanto é egoísta.

- Correto, correto.

- Pois, é! Sabe por quê? Se ainda existe um ego, ainda há egoísmo.
Mas pessoas que agem desta forma e não são egoístas também existem.
Como o Buda – não foi egoísta, ele destruiu seu próprio ego. Essa
palavra “ego”, “ego”, é apenas uma convenção. Uma pessoa que consegue
atingir aquele ponto não tem mais ego. Nós somos egoístas, então
pensamos que os outros devem ser iguais a nós, pensamos que ele é
egoísta, mas o ego dele já não existe. Há terra, fogo, água e ar, não
há ego. Essa expressão “ser egoísta” pressupõe que terra, água, fogo e
ar são nossa pessoa, são um ser, uma pessoa desse jeito. Esse tipo de
expressão e esse tipo de entendimento estão incorretos e vão juntos
infinitamente. 

O ponto mais alto do ensinamento do Buda é a ausência de pessoa – há
terra, água, fogo, ar, agrupados em uma pilha, só isso. Animal ou
pessoa não existem. Se aquela pessoa atingir esse ponto, como é
possível ele ser egoísta? Pois já não há ego. O que nos faz ser egoísta
é ter um ego dessa forma. Se formos falar sobre não ter ego, ninguém
entende. Conversar sobre coisas mundanas e coisas transcendentes não é
igual. Por exemplo, se eu descrever para o doutor a seguinte situação:
se dissermos “andar para frente” – conhecemos, “andar para trás” –
conhecemos, “ficar parado” – conhecemos. Se vier mais uma frase que
diga: “nem andar para frente, nem andar para trás, nem ficar parado”,
como é que fica? A expressão “andar para frente”, conhecemos. “Andar
para trás”, ouvimos e entendemos. “Ficar parado”, entendemos. Mas vem a
segunda frase: “nem andar para frente, nem andar para trás, nem ficar
parado”, como é que fica? Essa pessoa chegou a um certo ponto, já
aquela chegou a um ponto mais adiante, é assim.

Eu insisto: conversa mundana e conversa sobre o Darma mais elevado –
estendemos a mão, mas não alcançamos. Andar, caminhar – conhecemos.
Andar para trás – conhecemos. Ficar parado – conhecemos. Não andar para
frente, não andar para trás, não ficar parado – não conhecemos, ouvimos
e fica por isso mesmo. Bem aí, aí é que as pessoas não entendem nada a
ponto de haver tantos problemas e confusão demais desse jeito. Mas essa
frase é a forma de expressão daqueles que estão além deste mundo, é a
forma de expressão dos \textit{aryias}.
Crescemos a esse ponto, antes éramos crianças, não
é? Éramos crianças pequeninas, e agora crescemos a esse ponto. Sente
saudades de ser criança? Lamenta não ser criança? Por que foi possível
crescer a esse ponto? Porque é assim: entra comida por aqui, não é
assunto nosso, é assunto do mundo. 

Falamos, mas não acertamos o alvo… Mas todo mundo tem, todo mundo
neste mundo diz ter razão… É verdade, mas quando todo mundo vai
inventando razões, as razões acabam diferentes umas das outras. Gente
burra tem razão, gente esperta tem razão. Gente esperta tem a razão
deles, gente burra tem a razão deles. Significa que essa história de
razão não tem fim. Quando o Buda diz: “Eu ajo além dos motivos, acima
das razões. Além de nascimento, acima da morte. Além da felicidade,
acima do sofrimento”, e aí, como é que fica? É uma história
completamente diferente, completamente diferente. 

Por exemplo, o doutor foi criança, alguma vez brincou com balão
inflável? Via o balão e ficava feliz e alegre por causa do balão:
“oba!, oba!”. Agora cresceu a esse ponto, não é? É diferente de quando
era criança, não pensa mais em brincar com esse tipo de coisa. Por que
não quer brincar? Pois não há utilidade! Já cresceu, vê? Não há
utilidade. Quando éramos crianças, naquela época, víamos o balão como
algo muito valioso. Brincava e se divertia, “oba!, oba!” – sozinho. Não
sabia de nada. Mas quando o balão explodia “pôp!” – chorava… Por que é
assim? Brincar com essas coisas é o que se chama “nossa mente estar
fixa”. 

Então nossa idade se desenvolve, mudamos com o tempo e crescemos a
esse ponto. O doutor ainda vai brincar com balão, como as crianças?
Pois é assim. Como vamos resolver esse problema, quando é o caso que
aquele rapaz pensa dessa forma? O doutor tem que dizer: “Eu não quero
brincar pois não há utilidade.” E as crianças? Isso contradiz a opinião
das crianças. O doutor diz: “Não tem utilidade.”, mas as crianças
discordam do doutor: “É útil sim!” E aí, quem vence? Quem está certo,
quem está errado? As crianças têm as razões delas, os adultos têm as
razões deles, são mundos diferentes. Tem que ser assim. 

Ótimo! As perguntas hoje estão muito boas, quero que faça muitas
perguntas e assim se esclarece. São duas coisas completamente
diferentes, completamente diferentes…

- Estou conseguindo enxergar.

- Pois é. 

- Eu tenho uma opinião, é uma opinião que inventei sozinho, tenho
que admitir. Minha esposa aqui, Pao, cerca de dez anos atrás… Nossa!
Vivia se benzendo\footnote{É comum as pessoas na Tailândia pedirem que
os monges abençoem água para que elas borrifem sobre si para espantar
más energias ou trazer boa sorte.}! Eu perguntava: “Está possuída por
um espírito maligno? Foi fazer algo ruim ou o quê?” Seja lá o que
fosse, tinha que se benzer. Ela chamava para que eu fosse junto mas eu
dizia: “Não fiz nenhum mal, no que diz respeito à minha profissão,
ajudo as demais pessoas. Isso é um tipo de bondade e, em geral, quando
trabalho, trabalho com intenção de fazer direito. Caso cometa um erro,
é sem ter tido intenção, é por não ter tido conhecimento suficiente ou
por ter me equivocado, sem ter intenção de prejudicar ninguém. Eu digo
que não faço maldade.”

Não estou possuído por um espírito maligno, não fiz maldade, então
não me benzo. Pelo contrário, penso que a religião, qualquer que seja,
ensina a ter amor uns pelos outros, a praticar o bem da forma que o
mundo inteiro respeita e a tentar ser uma pessoa pura. Se formos de
acordo com a minha opinião, caso a pessoa chegue a um certo nível,
tendo encerrado seus deveres mundanos, suas responsabilidades, caso
queira ir procurar paz verdadeira – vira monge de vez, vai morar num
local pacífico. Mas só quando não tenha mais preocupações como: “Eh!
Meu filho não vai ter recursos para estudar, como é que fica?” Eu vou
ter que aprender a ser médium, a fazer água benta, a fazer poção
mágica,\footnote{Muitos monges na Tailândia ficam ricos vendendo esse
tipo de “serviço”, embora o Buda tenha proibido aos monges possuir
dinheiro e muito menos realizar esse tipo de atividade, que ele
considerava serem “artes de animais”.} isso não é correto. É por isso
que tenho a opinião de que fazer mérito, e então obter mérito ou não,
depende do coração – se nosso coração for puro, se agirmos sem ter más
intenções, sem maus pensamentos, mesmo que erremos, não acertemos o
alvo, mas tenhamos sensibilidade e procuremos corrigir, essa é uma ação
que penso ser pura o suficiente. Por isso esse tipo de “exibição”, por
exemplo, virar monge – chega a idade costumeira e “pap!”, ordena-se
monge porque mandaram. Quer dizer, ordenou-se por causa das
tradições\footnote{Antigamente era comum na Tailândia que todos os
rapazes se ordenassem monge ao completar vinte anos, por um período
curto de tempo.}, mas no coração talvez não haja paz suficiente ou
pureza suficiente, ou há preocupações e deveres por cumprir. 

Tanto quanto me lembre, temos que receber autorização para
ordenar-se monge e essa permissão deveria ser dada com sinceridade, não
por ter implorado, ter forçado o pai e a mãe dizendo: “Não quero nem
saber, vou virar monge, se me proibirem vou fazer algo ruim…” Isso é
obrigar, e eu digo que é fazer maldade. Por isso digo que, se for
fazer, se for ordenar-se para gerar bondade, depende do coração. Todas
as formas de fazer mérito… como no meu caso: eu tenho preguiça de doar
\textit{pindapāta}, mas minha esposa doa todos os dias pela manhã. Eu
tenho preguiça de tirar os sapatos\footnote{Quando os monges saem de
manhã pelas ruas recolhendo comida para sua refeição, as pessoas
costumam tirar os sapatos em gesto de humildade na hora de doar os
alimentos.}, mas não penso nada de mal, não penso de maneira ofensiva,
de maneira que tenho a opinião de que mesmo que alguém faça mérito até
morrer, na minha cabeça eu penso que mesmo que faça mérito até cair
morto, se no coração houver apenas ganância, confusão, arrogância ou se
fizer que os demais sofram, eu digo é melhor parar de fazer mérito
imediatamente. Vá aprender a fazer o coração ficar bom, a fazer as
outras pessoas felizes. Estou errado?

- Sobre isso aí, escute primeiro, OK? Existem duas coisas: número
um – a dona de casa se benzendo não deve estar certo.

- Mas agora ela já parou.

- O que será que aconteceu?

\textit{(esposa do Dr.)} - Eu sinto que me faz feliz, quando doo
\textit{pindapāta} fico feliz.

\textit{(Ajahn Chah)} - Pode deixar, eu vou explicar, vou resolver
essa questão. Por que age dessa maneira? Imagine uma galinha que temos
em casa: ela cacareja e corre para perto de nós, damos um par de calças
para ela – ela não aceita. Damos uma camisa – ela não aceita. O que
vamos dar para ela? 

\textit{(Doutor)} - Milho.

- Damos milho. Isso é útil para a galinha, não é? Só isso e já é
útil, ela quer aquilo. Camisa é bom só para pessoas, calça é para
pessoas, ela ainda não chegou ao nível humano, o que ela quer é milho.
Ela vem nos procurar, mas nós damos uma camisa e a galinha não aceita,
é melhor dar milho, não é? Desse ponto em diante ela vai melhorando,
melhorando, até virar gente. Quando for gente, se dermos milho para
comer ela não come. No começo é assim, e o doutor mencionou um segundo
item, o que era?

Ah! O senhor disse: “Eu não quero fazer mérito ou, se fizer, pego
minha mente e a transformo em mérito logo de vez.” Mas se o doutor for
uma pessoa esforçada, sendo uma pessoa esforçada, ia aguentar não
trabalhar? Ia aguentar não varrer a casa, ia aguentar não lavar os
pratos ou fazer algo de útil? Se for esforçado, não estou falando de
gente preguiçosa. 

- Trabalharia o tempo todo.

- Pois é, tem que ser assim. Para as pessoas que têm fé, fazer
doações é bom, mas o que o doutor disse sobre passar do limite também
está correto. Tem que saber moderar, tem que saber o suficiente,
conhecer moderação. Fazer demais sem considerar razão, não dá. Mas a
pessoa que tem fé profunda tem que doar \textit{pindapāta} ou fazer
\textit{pūja} e assim por diante, aquela pessoa está cheia de fé. Não
é para fazer como os tolos, mas para fazer como os sábios. 

Então podemos comparar com o doutor aqui, que é uma pessoa esforçada
ao máximo, não é preguiçoso naquilo que gosta. Vê a casa bagunçada,
consegue não varrer? Vê os pratos sujos, consegue não lavar? Vê o
cachorro vindo fazer cocô bem ali, toca ele para fora ou o quê? O
doutor tem que ser incapaz de parar de trabalhar, uma vez que é uma
pessoa esforçada. Aqui é a mesma coisa, tem que pensar dessa forma. Por
que age dessa maneira? Isso se manifesta vindo do coração de uma pessoa
esforçada, tem que ser assim. Se for o caso que ela ofereça
\textit{pindapāta} à toa, sem noção de nada, eu concordo contigo, mas
se ela faz tendo motivo como o doutor – por que limpa titica de
galinha, por que limpa cocô de porco, por que limpa cocô do cachorro,
por quê? Porque o doutor é detalhista da maneira correta, é uma pessoa
esforçada, não pode ver algo desarrumado, tem que limpar e arrumar. É
ali que se manifesta, não é só passar a tarefa adiante. Essa é a razão.


Aquilo, deixa para ela. Que ela vá atrás de água benta, deixa para
ela. Ela está no nível de galinha, que fazer? O nível é de galinha, mas
o doutor vai oferecer camisa – ela é galinha, não aceita. Que fazer?
Tem milho, então joga para ela comer, é melhor assim, para que ela
fique feliz. Tem que separar em níveis desta forma para poder chamar-se
“pessoa de entendimento”. Então o doutor vá contemplar tudo isso.

Ótimo! Hoje foi muito bom, o doutor já veio duas vezes mas ainda não
havia feito nenhuma pergunta. Ótimo, bote todas as perguntas para fora,
para acabar de vez desse jeito, bote para fora até acabar…
