
\chapterAuthor{Ajahn Chah}
\chapterNote{Ajahn Chah conversa com um grupo de monges.}
\chapter[Para quem conhece o Darma não há brigas]{Para quem conhece o Darma\newline não há brigas}
\tocChapterNote{Ajahn Chah conversa com um grupo de monges.}
\markright{\theChapterAuthor}

\ldots{} outras pessoas conseguiram obter fogo daquilo, mas não há um
sinal que diga: “O fogo está aqui neste bambu seco, está bem aqui.”
Como ele tem que fazer para vê-lo? Tem que pegar duas varas de bambu e
esfregá-las sem parar, após um tempo surge o fogo, ele está bem ali.
Acreditamos que há fogo ali e experimentamos, pegamos duas varas de
bambu e esfregamos, fazemos até cansar, quando cansamos vem a preguiça,
não é? O bambu está começando a esquentar e nós: “Humm\ldots{} que preguiça!”
Havia fogo, mas jogamos fora e depois anunciamos a todos: “Não há fogo
aqui, aqui não há fogo.” Nos enganamos. Na verdade há fogo ali, mas se
não criarmos calor suficiente, o fogo não aparece. Isso é porque nos
deixamos influenciar por nossa opinião, não procuramos a verdade. Se
nos esforçarmos em continuar, se tivermos resiliência, energia, esforço
e esfregarmos o bambu até surgir fumaça e fogo, a chama se acende de
verdade. Mas não fazemos o suficiente, então não há fogo, então
decidimos: “Hum, não tem fogo aqui.” Na verdade nosso esforço é que não
foi suficiente. O mesmo ocorre tanto com os monges tailandeses como os
ocidentais, vão indo, indo, mas chega um dia em que pegam um punhado de
flores e vêm dizer:

“Luang Pó, vou deixar a vida monástica\ldots{}” 

“Por quê?” 

“Acabou minha diligência.” 

Falam sem ter vergonha alguma. Como pode? O Buda praticou e a
diligência dele nunca acabou, esses se ordenam um ou dois anos e acabam
com a diligência\ldots{} Como podem ser tão “bons” assim? Falaram errado, na
verdade diligência não acaba, pode praticar à vontade que ela não
acaba, nós é que ficamos com preguiça. Vêm dizer que acabou a
diligência, o que acabou fomos nós, diligência ainda há. É assim, então
falei: “Por que não pensam direito? Estão pisoteando a verdade, vão
pisoteando e depois vêm dizer que acabou a diligência. Se pensar
direito, na verdade a diligência não acaba, não é possível que ela
acabe.”

Por exemplo, se sentamos em meditação aqui e eles batem um tambor
ali, “tum, tum, tum,” continuamos sentados, continuamos sentados
focando a mente, mas ficamos irritados, surge irritação. Por que
ficamos irritados? Pensamos: “Viemos procurar paz aqui e eles vêm bater
tambor na nossa orelha!”, e aí ficamos irritados. Por que ficamos
irritados? Porque pensamos errado, por isso nos irritamos. Como
pensamos errado? Pensamos que aquele barulho vem nos incomodar. Isso é
pensar errado, se pensamos errado surge sofrimento, insatisfação, um
bloco de impurezas composto de desejo, raiva e ignorância. Não é
possível resolver esse problema, pois fomos procurar paz mas não
encontramos porque tinha algo incomodando. Monges com muitos anos de
vida monástica morrem\footnote{Significa que aquele monge abandonou a
vida monástica. O Buda dizia que deixar a vida monástica é como a morte
para um monge, e daí surgiu essa expressão entre os monges.} por causa
disso sem perceber o que ocorreu, mas se a pessoa lutar com a sensação
por um longo período, se surge um objeto mental e ela contempla
continuamente – como esse som de tambor que ela diz incomodar – verá
que na verdade está pensando errado. Por que está errado? Pensamos que
o som vem nos incomodar. Tem que olhar isso por muito tempo para que
nasça sabedoria, conhecimento. Pratique bem ali, o sofrimento está bem
ali. Após muito tempo chega a um ponto, um dia, em que surge o saber:
“Hum, aqui não me sinto bem, ali não me sinto bem, o que está me
incomodando? Mesmo que atravesse o mundo procurando onde morar, onde
acharia um lugar onde não haja incômodos?” Pensa, pensa, pensa, até
conseguir entender: “Oh\ldots{} é assim. Não são eles que vêm me incomodar, eu
é que estou indo incomodá-los!”, ou seja, volta para si. Consegue
voltar desse jeito por ter se irritado, ter sofrido por querer
encontrar paz, então surge o problema naquele lugar, bem onde não
gostamos, onde sofremos, onde nos irrita, e aquilo que é pacífico,
feliz, que não irrita, nasce justamente ali. 

Se temos a opinião: “Não são eles que vêm me incomodar, sou eu que
estou indo incomodá-los,” aquilo se encerra sozinho, é só vermos e
aquilo para sozinho, não precisa forçar nada. Vemos claramente que
somos nós que vamos incomodá-los; se vemos que são eles que vêm nos
incomodar, é por causa da nossa opinião errada, nossa sabedoria ainda
não nasceu. Se vemos: “Oh, somos nós que vamos incomodá-los,” se é
assim, jogamos fora as coisas externas e focamos no interior, não gera
irritação, pois são coisas distintas. Quando a mente sabe desse jeito
ela larga sozinha. Os barulhos podem vir com força total, não importa.
Há bem-estar, há paz. Essa paz não é por não ouvir nada, não saber
nada, essa paz é por ouvir, por saber de acordo com a verdade. Então
podemos soltar a mente, só a utilizamos quando necessário. Mas se
estamos em grupo, viemos procurar paz aqui, os monges estão aqui em
paz, as pessoas não podem vir fazer algazarra aqui. Temos que dizer:
“Pare! Não trabalhe aqui, não fale alto aqui, não venha cantar aqui.”
Temos que usar o método mundano, pois, se deixarmos, eles vão ficar
bagunçando, não gera benefício algum. 

Mas quando chega a hora de verdade, o que podemos fazer? Quando é
hora de largar, tem que largar – neste local faz-se assim, naquele, de
outro jeito. Hoje fazemos assim, amanhã fazemos de outro jeito, depende
da ocasião, de refletirmos quando chegar a hora. Se viermos praticar
aqui e eles vierem cantar e nos irritar, podemos largar. Isso é como
faziam o Buda e os \textit{sāvakas} que já se foram, eles procuravam
evitar as coisas externas que incomodavam. Primeiro faziam como gente
tola e iam construir inteligência em outro lugar, portanto eles fugiam.
O Buda nos ensinou a ser frugal, a viver em reclusão, a satisfazer-nos
com o que temos. Mesmo ele já tendo terminado seu dever em abandonar,
em desenvolver, ele ainda se preocupava conosco. Ele então fugia de um
lugar para outro, mas fugia para lutar, não fugia por ter sido
derrotado, por ser tolo ou por ter defeitos. Ele fugia para treinar,
para vencer. O Buda vivia na floresta, não por odiar a cidade, ele foi
buscar vitória na floresta, buscar paz. Foi treinar na floresta, num
local pacífico. Não fugiu por tolice, ele fugiu por inteligência, fugiu
por sabedoria. Por exemplo, todos nós fugimos do mundo e viemos nos
ordenar, fugimos dos sons, das cores. Alguns odeiam os sons e as cores
exageradamente, fogem continuamente, isso não traz benefício, estão
fugindo por terem sido derrotados. Tem que fugir para vencer, para
fazer conhecimento surgir naquele lugar, conhecer até tornar-se
\textit{lokavidū}, como ensinou o Buda, “conhecer o mundo
claramente.” Enquanto este mundo não estiver claro, tem que treinar,
tem que estudar até vencer. 

Algumas pessoas dizem: “Eh\ldots{} Isso é
\textit{attakilamathānuyogo},\footnote{A prática de atormentar a si
com o objetivo de obter progresso espiritual.} é ter medo. Vive na
floresta, só come uma refeição, vai se sentir fraco.” Essa pessoa fala
à toa, qualquer coisa que for desagradável ela diz que é “atta,” mas é
porque ela não gosta, ela se deixa influenciar por suas próprias
opiniões. Tem gente que é assim. Como é \textit{atta\-kilamathānuyogo}?
Elas não investigam \textit{kāmasukhallikānuyogo},\footnote{A
prática de buscar progresso espiritual através da indulgência em
prazeres sensuais.} não analisam, só analisam
\textit{attakilamathānuyogo}: “Comer uma só refeição não traz
benefício algum. Frugalidade, reclusão, viver na floresta – é tudo
‘atta.’” Então eu digo: “Se é assim, você come – quando mastiga também
é \textit{attakilamathānuyogo.}” Se alguém constrói uma casa, a
madeira não pode ser utilizada enquanto for árvore e estiver na
floresta. Se vamos construir uma casa temos que cortar, plainar,
serrar, tudo coisas difíceis e cansativas. Se isso é
\textit{attakilamathānuyogo}, é melhor não fazer nada de vez! Eles
trabalharam essa madeira, serraram essa madeira, fizeram essa tábua,
com que objetivo? Para construir uma casa e morar ali, fizeram por uma
causa. É a causa, é algo comum, vemos que essa é a causa. Quando há uma
causa como essa, o resultado surge de acordo. O objetivo de construir é
para que tenhamos onde morar, porque é normal as pessoas terem onde
morar, dormir, deitar, ter onde se apoiar. Tem que ter sabedoria desse
jeito. Quando vamos praticar temos que pensar assim. 

Hoje em dia temos muito conhecimento, mas pouca bondade, por quê?
Porque não vemos o segundo nível, vemos o primeiro, só isso, só o
primeiro, não vemos o segundo, o terceiro, etc. Isso é o mesmo que não
saber. Então eu digo que, falando sobre o ensinamento do Buda,
costumamos estudar a teoria, mas essa não é a verdadeira essência.
Então não fazemos progresso no ensinamento do Buda, somos eloquentes
falando, mas não agimos de acordo com aquela verdade, então a coisa
real não surge, é como se não conhecêssemos. Por exemplo, se o
professor na escola ensina e faz junto com as crianças para servir de
exemplo, assim é bom. Mas os adultos não fazem, aqueles que sabem,
aqueles que deveriam fazer, não fazem, então não gera benefício algum,
as crianças ficam ainda mais preguiçosas. 

Eu já falei isso para os monges ouvirem. Quer esteja na cidade ou na
floresta, contemple até entender todas essas coisas, você conhecerá o
ensinamento do Buda através da prática. Tenha poucas opiniões, quem tem
muito conhecimento não consegue morar na floresta, fica com medo de
malária, medo de tigre. Mas a “malária” de Bangkok, os “tigres” em
Bangkok são ainda mais numerosos que os tigres da floresta, não é? Já
levou mordida de tigre? Já foi à floresta e levou mordida de tigre? E
os tigres da cidade, já te atacaram alguma vez? A cidade está cheia de
“tigres” e são muito perigosos. Se vai para a floresta tem medo de
passar fome, ficar fraco, sentir calor, tem medo dos bichos, dos
tigres, mas o calor em Bangkok, ôôôô\ldots{}, queima ainda mais quente, como
um forno. Não se protege dos “tigres,” então a prática vai acabando,
acabando. Não consegue se livrar, então vai para outro lugar, da
Tailândia vai para Índia, da Índia vai para o Sri Lanka e do Sri Lanka
continua viajando até dar a volta ao mundo. Mas não consegue escapar
deste mundo: se não conhece o mundo, não consegue escapar dele. É
assim. Na verdade não procurou desenvolver o que é necessário para
conseguir viver e praticar na floresta.

Atualmente estudam muito a teoria, fabricam todo tipo de diversões
hoje em dia. Os praticantes são assim, quando a mente vê claramente\ldots{}
quando começamos a prática, os ouvidos não ouvirem, os olhos não verem:
isso é a paz. Mas é uma paz falsa, se chama \textit{samathā}. Sente
paz, é pacífico, mas logo volta a confusão novamente. É o berço da
confusão, \textit{samathā} é o berço da confusão. É pacífico porque
os olhos não veem, os ouvidos não ouvem. Esse tipo de paz é capaz de
ser motivo para os monges perderem o caminho. Vivem sozinhos, entram em
\textit{jhāna} e vão lá longe, mas não sabem como são as impurezas
mentais, não conhecem o “tigre.” Quando perdem o \textit{samādhi},
desistem. A paz em que os olhos veem formas, os ouvidos ouvem sons, vê
tudo, mas vê a desvantagem naquelas coisas – essa é a paz verdadeira.
Vê, abre os olhos e vê um perigo bem ali. Nossa mente se pacifica, pois
vemos aquilo que gostamos como um perigo e tomamos cuidado. Se surge
algo que não gostamos, não gera mais confusão. Entendam isso. 

O Buda dizia para não ser descuidado, quem é descuidado está
morto,\footnote{Verso do Dhammapada (Appamadavagga – 21).} quem é
descuidado é como uma pessoa que já morreu. É assim, uma pessoa
descuidada é uma pessoa que não sabe, não tem \textit{sati}, quem não
tem \textit{sati} não sabe de nada, quem não sabe de nada não sabe ter
atenção e cuidado, se não sabe ter atenção e cuidado não sabe o que é
certo e errado, se não sabe o que é certo e errado não sabe como
praticar e fica sonso, surge um objeto mental qualquer e a pessoa
quebra, cai em confusão. Pensa que é por culpa disso, por culpa
daquilo, mas não pensa: “É por minha culpa,” é sempre culpa de alguma
outra coisa. É assim, eles se perdem dessa forma, a maioria dos
praticantes se perdem na prática dessa maneira. 

Portanto, estudem o Darma! Vejam o bem que o Buda nos deixou,
estudem o exemplo dele. Eu penso que temos muito mérito. De que forma é
isso? Hoje em dia, as pessoas que têm sabedoria conseguem encontrar os
ensinamentos verdadeiros do Buda, não precisam procurar muito, basta
pegar e ver. É só pegar o livro, ler e entender, o que estiver errado
jogue fora, o que estiver certo, contemple. Na época em que o Buda
começou a praticar, era como agarrar um peixe no meio do oceano, como
um pequeno pássaro voando no céu, não sabia onde se apoiar, ainda não
conseguia ver. Não havia nada para ajudar, então tinha que se virar
sozinho. Atualmente temos coisas para nos ajudar: se formos ao mar,
temos barcos para utilizar; o que não sabemos, estudamos, todos
estudamos, adquirimos conhecimento. Há todo tipo de coisas que podemos
saber através da nossa prática, mas se conseguirmos olhar do ponto de
vista mais elevado, vemos que na verdade não sabemos nada. Se hoje o
Jagaro\footnote{Nome de um discípulo que estava presente quando esse
ensinamento foi dado.} fica com raiva de mim, pensa: “Humm, é
impermanente, é incerto.” Só isso. Se hoje Jagaro gosta de mim ele tem
que ensinar a si: “Humm, isso é incerto.” Se sabe disso, acabou. 

Mate! Mate continuamente, como uma árvore: quando ela começa a
brotar nós cortamos; passa um tempo, cortamos de novo. Se deixar passar
e não cortar, vira uma floresta enorme. Brota e corta, brota e corta.
Eu observei os cupins numa cabana onde morava, antes não tinha
túnel,\footnote{Há um certo tipo de cupim que costuma construir um túnel
de barro para servir de passagem pelos locais por onde caminha com
frequência.} eles vieram e construíram um e foram subindo, todo dia,
todo dia. Pela manhã eu pegava uma madeira e removia o túnel, eles
construíam de novo, vinham pelo outro lado, eu removia. Eles
construíam, eu removia, não podia descuidar ou eles devoravam a cabana
inteira. Eu fazia assim, queria saber quanta diligência tinham os
cupins. Pois foram vários meses! Às vezes me distraía e eles subiam
pelo outro lado. Eu removia, pela manhã eles vinham por aqui, eu
removia, eles vinham por ali, eu removia. No final os cupins se
irritaram: “Puxa, não importa o quanto construímos, vem alguém e
remove!”, após um tempo eles desistiram e foram embora. Cupins! Eu já
fiz esse teste. Isso também é prática. Conosco é a mesma coisa, quando
as impurezas surgirem tente matá-las continuamente, não faça o que elas
mandam e a força delas diminui dia a dia. Quer seja a preguiça ou o
sofrimento, se matamos, eles se dobram, perdem seu ponto de apoio, não
conseguem grudar em nós. Cortamos fora, cortamos o sofrimento todo dia
até virar hábito. Vira hábito, construímos o hábito. Isso é útil.
Quando eles constroem, nós removemos. Os cupins são assim: se
deixarmos, eles sobem. As impurezas mentais também, elas são como
cupins, podem olhar.

O Buda ensinou a construir sabedoria. Se a sabedoria for hábil, as
impurezas somem; se nossa sabedoria for mais rápida que elas, é ainda
melhor. Construa sabedoria veloz, uma arma para lutar, arma do Darma
dentro do seu coração. É só sabermos que isso é uma impureza, aquilo
não é, e já estamos praticando. Aqueles que não sabem dizem: “Eu gosto,
então faço assim, é agradável,” esses levam tempo, às vezes morrem e
ainda não sabem de nada. “Eu me ordenei porque quero bem-estar, quero
paz, não quero que ninguém me dê ordens,” ficam do jeito que der
vontade, largam e aí fica ainda mais pesado. Naquele monastério ninguém
vai ser capaz de ajudar, dar conselho, eles não escutam, aí afundam de
vez. Vem de várias formas isso, é muito variado esse assunto. Se não
entendemos, não conseguimos captar, fica difícil. 

Jagaro, já viu tigre que mora em casa? Tigre morando em casa, já
viu? Tigre na floresta você já viu, e tigre em casa? Já viu?! A
tigresa\footnote{Nesta ocasião Ajahn Chah ensinava a um grupo de
monges, se o grupo fosse de monjas ele certamente alertaria sobre os
perigos do tigre macho.} é feroz, hein! Muito cuidado! Os dentes são
grandes, as unhas afiadas, tome cuidado. O tigre macho não é muito
perigoso, os dentes não são afiados. A tigresa é perigosa, as unhas e
dentes são afiados, se encontrar tigre que mora em casa, tem que ter
cuidado. E você, já viu tigre? Pois é, se esse tigre pegar, você será
devorado. Ele agarra e devora. 

De agora em diante podem parar de viajar. Já viram vários lugares,
mais que o suficiente, agora nos ordenamos, ficamos no mesmo lugar. De
agora em diante, onde houver um professor que consiga nos ajudar, nos
aconselhar – use aquele ensinamento. Vá praticando de acordo com o que
for apropriado, não precisa procurar muito. Lugar para morar, não
precisa procurar nada muito complicado; comida, só o suficiente para
viver, não precisa procurar coisa boa. Onde conseguir desenvolver visão
correta através da prática, viva ali, contemple ali. Isso é levar a
ordenação a um nível mais alto. Pode procurar a prática onde quiser,
não vai encontrar porque as impurezas não estão do lado de fora, não
estão lá. O que nos leva a ficar andando é impureza tamanho grande! O
que nos leva a ficar viajando é impureza tamanho grande. Ela diz: “Vá
ver ali, lá, província tal, bem ali, bem aqui, vá praticar na montanha,
é bem pacífico na caverna\ldots{}” Se vai à caverna, vê a caverna e pensa:
“Humm, ficar sozinho é bom\ldots{}” Chama um leigo: 

“Ei! Já veio algum monge passar o retiro das monções (\textit{vassa)
}aqui?”

“Sim senhor.” 

“Tranquilo?” 

Se ele responder: “Não senhor, vieram aqui e quase não duraram os
três meses, vários morreram durante o retiro.” 

“Oh! Vou embora, não quero pegar malária!”, e vai embora. Fica com
medo, as impurezas mandam ir onde não dá medo. Outros morreram, então
fica com medo. 

“Os monges que viveram aqui ficaram doentes?” 

“Ôôô! Ficaram doentes o retiro inteiro.”

Um monge pergunta ao outro: “Vai ficar?” 

“Se você quiser ficar, fique! Eu não fico, vou embora!”

Aí penduram a tigela nos ombros e vão todos embora. Medo. Se estão
na cidade dizem: “Vou para a floresta;” se estão na floresta dizem:
“Vou para a caverna;” se estão na floresta, na caverna, ficam com medo
de malária. Aí não têm onde ficar, penduram a tigela nos ombros e andam
o tempo todo. Vão lá, vêm aqui; vêm aqui, vão lá, sem parar. Procuram
onde praticar, mas não praticam. Isso porque não sabem onde é que se
pratica, então são levados a carregar a tigela para cima e para baixo o
tempo todo. Desse monastério para aquele, dessa montanha para aquela,
além disso as cavernas. Então vai para Loei\footnote{Nome de uma
província na região nordeste da Tailândia.} e vê uma caverna, muito
boa: 

“Oh! Ótimo! Vou ficar aqui, gostei daqui,” então vai. 

Estando lá: “Ôôô, é um pouco difícil de achar água, não tem muita
comida, é difícil ir até a vila, é longe\ldots{}”

Passa uma semana e vem uma pessoa: “Ôe! Que está fazendo nessa
caverna? É muito ruim, a caverna na montanha onde moro é ainda melhor
que essa.”

Então pensa “Lá vou eu de novo!”, pendura a tigela nos ombros e vai
até lá. Vai desse jeito, os pensamentos mentem para nós o tempo todo.
No final senta e pensa “Ôôe, na floresta eu já morei, na caverna eu já
morei, já fiz jejum seis, sete dias, já fiz de tudo e não vi benefício
algum, me cansei à toa. É melhor largar a vida monástica e ir plantar
maçãs.” Não é? É melhor deixar de ser monge. Faz isso, aquilo, aquilo
outro\ldots{} cansa: “Chega! Vou deixar de ser monge! Assim não preciso
depender do trabalho alheio, assim não incomodo ninguém. É isso mesmo”
e vai de novo. Aí sofre lavrando a terra e fica irritado de novo:
“Raios! Não tenho tempo livre, passo o tempo todo fazendo isso e
aquilo.” 

Muito bom, não? Ordenou-se monge e achou difícil, muitas regras,
difícil sustentar, incomoda os noviços:\footnote{Por causa da regra
monástica, os monges não podem fazer tarefas simples como cavar o solo
ou cortar plantas; esse tipo de trabalho é em geral delegado aos
noviços do monastério.} “Melhor eu mesmo virar noviço: como noviço
posso plantar para comer, posso cozinhar, assim não dependo dos outros.
Vou deixar de ser monge e me ordenar noviço.” Vira noviço e aí pode
plantar para comer, vive na floresta, na montanha, mas ainda depende
dos leigos, isso é desagradável: “Ainda não posso ter
dinheiro,\footnote{Um noviço observa dez preceitos morais que incluem
renúncia à posse de dinheiro.} melhor deixar de ser noviço, assim
posso ter dinheiro, planto para comer e posso colocar o dinheiro no
banco, é ainda mais confortável. Melhor vestir o manto
branco.”\footnote{Tornar-se um anagārika (ver glossário).} Veste o
manto branco e vai ficando, aí pensa: “Oh! Comer arroz é complicado,
comer grama feito as vacas é o mais confortável! Agora não vou mais
comer arroz, vou comer folhas. Se vierem oferecer arroz, peixe, molho,
não aceito, já estou tranquilo, como pouco, como folhas e raízes da
floresta.” Resultado final: morre. Não pratica para subir, pratica para
recuar, recuar, recuar. Sabedoria naquele momento não funcionou, não
sei onde ela foi parar. 

Se somos discípulos do Buda\ldots{} ele ordenou-se monge, ele observava
essas mesmas regras que nós, ele agia assim. Se somos discípulos, mas
não conseguimos agir como ele, para que continuar? Vamos ter que
recuar, recuar, procurando lugares agradáveis. Pensando errado desse
jeito não consegue enxergar, pois não tem sabedoria. Está errado, tem
uma natureza obstruindo para que não veja, então recua, recua, até o
fim. Quando volta a ser leigo, morre! Não sei onde fica, não sei onde
fica \textit{sati}, não sei onde vai parar sabedoria, ninguém a chama,
não sei\ldots{} É assim, se fosse pobre, seria pobre de tudo, de comida, de
dinheiro, de casa, é pobre de tudo. Se a prática é pobre de sabedoria,
todo o resto é pobre. “Esforço já fiz, melhor largar a vida monástica,
melhor ir atrás de esposa, melhor ir jantar,” fica desse jeito. Esses
vão fácil porque falta a ferramenta para se proteger, falta
\textit{sati}, falta sabedoria, falta esforço, falta tudo. Desse jeito
fica impossível. Quando os monges vêm dizer que vão largar a vida
monástica, eu deixo ir. Pergunto: “Pensou bem? Pensou direito? Vá
pensar.” Eu não posso ajudar a pensar, apenas mando pensar mais uma
vez, alguns mudam de ideia e resolvem ficar, alguns não conseguem
pensar em nada, então vão embora. 

Tudo isso vem das palavras do Darma que conhecemos, não vem de outro
lugar, vem das escrituras que estudamos. Mas hoje em dia eles guardam
os livros, ninguém os utiliza. Utilizam só para reforçar suas opiniões.
Na regra monástica, nos discursos do Buda, está cheio de informações
sobre prática, não desapareceram e foram para lugar algum. Nós
estudamos até ficarmos vesgos, mas tendo estudado não utilizamos o que
estudamos. Eu penso: “De onde o mestre tirou essa prática? Tirou disso
que estudamos.” Estudamos, mas não utilizamos o que aprendemos. É só
isso, temos que utilizar, ora essa! Não tem que buscar nada de mais,
nosso modo de prática não tem nada de mais, basta praticar de verdade,
mas nos deixamos levar por outros interesses. O que aprendemos de
correto, \textit{sīla, samādhi, paññā}, todos falam muito
eloquentes, mas não desenvolvem no coração. Isso não é mais útil do que
falar sobre qualquer outro assunto. 

Hoje em dia no budismo existem várias linhagens. Vai lá, eles fazem
assim, vem aqui, eles fazem de outro jeito. Se entendemos o Darma, pode
criar a confusão que for que não nos confundimos. Se conhecemos maçã de
verdade, pera de verdade, uva de verdade, limão de verdade, conhecemos
de verdade todas as frutas e jogarmos todas numa cesta, todas
misturadas na mesma cesta, se quisermos pegar um limão conseguimos
pegar, se quisermos pegar uma maçã conseguimos pegar, conseguimos pegar
qualquer uma. É \textit{Sacca-Dhamma}, não importa quem ensine o quê,
se o monge entende o Darma, se sabe com certeza, eu digo que ele não
balança. Não precisa ter dúvida. Uma é essa história de linhagem,
\textit{Sacca-Dhamma} é outra coisa. Essa linhagem faz desse jeito, a
outra daquele jeito\ldots{} Ainda mais a molecada hoje em dia – cansativo,
não? Até sobre o objeto de meditação brigam feito cachorros. Esse fala
“inspirando,” aquele “expirando,” esse “\textit{Buddho}”, aquele
“\textit{Dhammo}”, “\textit{sammā-arahant}” e assim por diante,
brigam! Não sabem o que é o quê. Isso é porque não sabem nada sobre a
coisa real. Na verdade todas as coisas se reúnem. Sempre que defendemos
uma linhagem criticamos as demais. Se não defendemos uma linhagem, não
criticamos as outras. O Buda ensinou a estudar todo tipo de coisa, mas
tem que conhecer o fim, onde tudo isso acaba? Aquilo é só a fala de
pessoas, deveríamos ser capazes de ouvir. Por exemplo: a palavra “cão”
não é um termo definitivo, também pode chamar de “cachorro.” É assim,
não precisa se apegar a “cachorro” ou “cão.” É assim. A coisa real
continua no mesmo local que antes. Se entendemos isso, onde formos
entenderemos o significado, eles se unem. 

Tem gente que vai praticar e só foca nos pontos de divergência –
nesse grupo eles aconselham assim, no outro aconselham assim – depois
não sabem o que pegar. Quem pratica “\textit{Buddho}” se apega até
travar, quem pratica \textit{ānāpānasati}, trava;
“\textit{sammā-arahant}”, trava; “inspirando-expirando,” trava. “Eu
estou certo, você está errado.” É sempre assim. Isso não é Darma. Não é
Darma verdadeiro. O Buda disse para memorizar, para estudar os seus
discursos e memorizar – se consegue ler – mas é memorizar para largar
dos apegos. Mesmo em fazer o bem. Eu entendo que ele disse para
praticar o bem, abandonar o mal, fazer o bem, mas esse bem, se nos
apegamos: “É bom de verdade! Eu sou bom!”, já está errado, já ficou
ruim de novo. Se estamos certos podemos dizer, mas se alguém vier e
disser que estamos errados, conseguimos largar. Por exemplo: ele diz
que estamos errados, a gente pensava que estava certo, mas conseguimos
largar e jogar fora. Segure firme apenas a origem do Darma e entenda
que o Darma culmina num único ponto. Não tem outro lugar. Se conhecemos
o Darma do começo ao fim, até o ponto final do caminho, eu digo que
podemos ir a qualquer grupo e não nos confundimos. Jogue maçã, limão,
jogue numa mesma cesta, pode jogar, quando queremos pegar uma maçã,
pegamos só a maçã; se quisermos pegar limão, pegamos só limão, não
pegamos maçã. Se praticamos o Darma, não importa em que grupo estamos:
se conhecemos a verdade do Darma, não erramos. 

É algo parecido com a área da \textit{sīmā}:\footnote{É uma área
designada pela sangha para a realização de cerimônias como a ordenação
de novos monges. Os limites dessa área devem ser designados utilizando
pontos de referência geográficos como pedras, árvores, um lago, um rio,
etc.} em qualquer lugar pode haver \textit{sīmā} se a pessoa
souber enxergar as características da \textit{sīmā}. Pode ser na
água, na floresta, mesmo aqui pode haver: se conhecemos as
características, podemos estabelecer uma. No ponto da verdade, não
importa por onde passe, podemos ter certeza, haverá sempre verdade ali.
Portanto, se entendemos o Darma, não há nada de mais. Mas na hora de
aprender tem que segurar, segurar firme, tem que segurar, segurar
\textit{sīla}, segurar \textit{samādhi}, segurar \textit{paññā}
como fundação. Tem que segurar firme. Na verdade isso são expressões
que falamos no começo, tem que começar segurando aquilo. 

É como se mandassem carregar uma tora de madeira: “Não largue,
hein!”, nós então carregamos. Após um tempo carregando fica pesado, não
é? Como fazer para ficar leve? Largue, jogue fora! Dizem para jogar
fora, mas a pessoa que ouve não quer jogar fora, pensa: “Se eu não
segurar isto, o que vou fazer? Se jogar fora esta tora não vou ter
madeira, que benefício vai ter?” A pessoa que está com o peso, quando
dizem para jogar fora, pensa que não há benefício, que vai desperdiçar
a madeira. Ela não vê o que pode ganhar com isso, então carrega até
quase morrer esmagada. Chega uma hora em que não aguenta mais e joga
fora; quando joga fora não sente mais peso, e se não há peso surge a
leveza. É isso que ganhamos, ganhamos a leveza. Quando vamos pegar
pensamos que não dá para não agarrarmos, não carregarmos, não há
benefício em jogar fora, abandonar, mas o Buda ensinou a jogar fora, a
abandonar para que surja benefício. A pessoa que carrega a madeira
pensa que se jogar fora não vai ganhar nada, carrega até quase morrer,
só quando está a ponto de morrer esmagado é que joga fora, e aí surge a
leveza. É isso que ganhamos quando largamos, abandonamos isso e a
leveza, o bem-estar, surgem bem aqui. 

Quando o desejo nos obriga a entender errado, não vemos o que
poderíamos ganhar, então agarramos daquele jeito. Se conhecemos a
verdade dessa forma, não há muito o que estudar. Nesse mundo só há uma
única coisa, não há várias. Nós somos só isso, só corpo e mente, só
esses dois. Se formos estudar tudo, nunca vai acabar. Se queremos
atravessar essa montanha, não dá para pegar um trator e remover a
montanha. Vai acabar morrendo antes de conseguir. Pegue um facão e
corte uma picada suficiente só para passar. Assim também dá para
atravessar a montanha. Temos essa oportunidade, então fazemos assim.
Praticar o Darma é a mesma coisa. Se vemos o Darma, onde formos não há
nada de mais. Tudo toma parte no Darma. Quando vemos algo, sabemos se
está de acordo com o Buda ou não, se é apropriado ou não. 

Hoje em dia, as pessoas que vão praticar se apegam a isso, se apegam
àquilo: “Esta pessoa está certa, você está errado.” Ficam competindo,
criam ainda mais inimizade, pensando: “Eu estou certo.” O Buda ensinou
a largar. Se eu digo que está certo e você que está errado, brigamos.
Onde vai acabar isso? Onde vai haver fim? Eu digo que está certo, você
diz que está errado. Se pelo menos eu ou você largarmos um lado: “Se
estiver certo está certo, se estiver errado está errado, larga.” Não há
sobre o que brigar, acaba bem aqui. Eu estou certo, você está errado,
eu estou certo, você está errado – onde vai acabar? Não tem como
acabar, por causa do desejo, discutem até morrer. Se pelo menos um
largar, não há briga. A prática é assim. 

O Buda ensinou a ter refinamento, contemplar tudo, pois está bem
ali. Por exemplo, quando fazemos um exame na escola, eles nos colocam
um problema, é um problema e nós temos que resolver o problema. Na
verdade, todos aqueles problemas possuem uma solução, não há exceções;
é um problema, mas a solução está ali, vem junto com o problema. Se a
pessoa não sabe, vê apenas o problema, não há solução, então sofre, é
difícil, complicado. Se a pessoa conhece, todos os problemas têm
solução. Está ali dentro, está bem ali. Se está pesado aqui, fica leve
bem aqui, se está escuro aqui, nasce luz bem aqui, se há prazer aqui,
haverá dor bem aqui, eles se anulam. Não precisa procurar em outro
lugar, tudo é assim, está bem ali. 

O Buda disse para praticarmos e descobrirmos sozinhos. Por que ele
não explicou o caminho da libertação claramente? Por que não explicou
nirvana claramente? Essas coisas não podem ser explicadas claramente. O
problema não está na pessoa que explica: o problema está na pessoa que
ouve, que pergunta. Não se alcança conhecimento no Buda, se alcança em
si mesmo. Entender não vem de outra pessoa, vem de si, tem que praticar
para entender bem ali. Aquele que ajuda só leva até a entrada,
“\textit{Akkhātāro Tathāgatā}” – o Buda apenas aponta o
caminho, ele não pratica no lugar dos discípulos. “Por que é assim?
Mostra que ele não era inteligente!”, pois essa é a inteligência do
Buda. Essa é a inteligência dele. Qualquer coisa, se virmos só
exteriormente, não virmos em si, “qual linhagem está certa,” etc.,
vamos ficar em dúvida para sempre. É assim em tudo. No Darma é assim.
Se explicar, as pessoas começam a brigar e, se brigam, veem o Darma?
Não veem, se perdem ainda mais, só brigam. 

Falando claramente, isso não é o Darma brigando. Não sabem o que é o
quê, se fosse Darma poderiam ir a qualquer grupo sem brigar. Não há
como brigar sobre o Darma, só se pode ouvir – eles falam, nós ouvimos e
contemplamos sozinhos. Se estiver certo, há uma prática naquilo; se
estiver errado, há uma prática naquilo. Não duvide se esse problema tem
solução ou não, não duvide – todos os problemas têm solução. Tudo que
nasce na nossa mente – gostar, não gostar, prazer, dor, etc. – todos os
problemas cessam sozinhos quando olhamos de volta para nós mesmos.
Sabemos sozinho, sabemos por completo, automaticamente. Senhores, este
é nosso verdadeiro lar. Aqui vocês não vão balançar, nunca. Não importa
o que seja, não errarão, não haverá como praticar errado. 

Eu já contei essa história, uma vez vieram me perguntar: “Aqui vocês
praticam \textit{samathā} ou \textit{vipassanā}?” Acho difícil
responder: se pego uma faca, desse jeito, não consigo separar o lado
afiado do lado cego da faca. Ambos têm que estar ali. Eu vejo assim,
essa faca tem tanto afiado e cego. Você pratica \textit{samathā} ou
\textit{vipassanā}? Eu acho difícil responder, porque ambos estão
juntos nisso, se desenvolvem juntos, a raiz cresce, o tronco cresce, as
folhas crescem junto. Só difere a localização de cada. Podemos dizer
\textit{samathā}, quer dizer paz, essa paz é por ter se isolado dos
objetos mentais, não sabe de nada, foge para a floresta, os objetos
diminuem, joga eles fora. Dessa forma a mente consegue se pacificar,
mas por que se pacifica? Porque as impurezas ainda não surgiram naquele
momento, se esconderam – é muita burrice. Desse jeito os praticantes
fogem continuamente, se obcecam pela paz. Em outros lugares exageram,
dizem não que precisa ir buscar paz em lugar algum, paz está aqui,
podemos alcançar paz aqui mesmo. Se consegue fazer, tudo bem, pode
falar, mas é difícil fazer. Quando começamos a aprender temos que fazer
de acordo com o formato dos livros até conseguir. Não é: “Temos que
abandonar tudo isso!”, mas no coração não conseguir largar. “Essa
imagem do Buda – não faça reverência a ela! Largue isso!”, fala como se
fôssemos pegar, alisar, agarrar a imagem, não precisa exagerar a esse
ponto. Temos que praticar olhando a razão: “É correto isso ou não?”
Assim surge de forma harmoniosa, vemos com clareza, então podemos
abandonar aquela prática, isso é automático. Isso é difícil, é uma
parte da prática que está oculta. 

Eu digo: more na floresta, mas não se obceque pela floresta. More na
cidade, mas não se obceque pela cidade. Se gosta de ir em peregrinação,
vá, mas não se obceque nisso. More no monastério, mas não se obceque
pelo monastério. É assim, a verdade é assim. Isso é para entendermos.
Veja por si mesmo, pratique sozinho. Onde quer que consigamos
resultados, temos que praticar ali mesmo. Prática é assim. Às vezes,
pode-se dizer, conhecimento vem de praticar dessa forma. Por exemplo,
somos leigos e temos um amigo, vários amigos, vamos à casa dele e ele
frita um peixe para comermos. 

“Eh, de onde você tirou esse peixe?” 

“Pesquei com vara.” 

Então fritou para nós. Nós entendemos: “Ah, para conseguir peixe
tenho que pescar com vara.” Mais à frente vai visitar um outro amigo,
ele está cozinhando um peixe: 

“De onde tirou esse peixe?” 

“Peguei com uma rede.” 

Se assusta: “Eh, aquele teve que usar vara, esse pegou com rede, por
que não são iguais?” Isso é porque não tem sabedoria, se tem sabedoria,
de onde veio o peixe não é problema. Basta ter peixe para comer, é o
que importa. As pessoas no mundo são assim. Não pense: “Eh, ele usou
vara, todos os peixes do mundo vêm de pescar com vara,” se for rede, o
mundo inteiro pesca com rede, se comprou, todos os peixes vêm do
mercado, pois foi dali que ele obteve aquilo. Desenvolver a mente,
alguns mestres conseguiram graças a \textit{ānāpānasati}, outros
conseguiram usando o mantra “\textit{Buddho}”, alguns contemplaram como
forma e mentalidade e viram claramente, a mente se pacificou. Quando
estudamos com esse, ele nos ensina da mesma forma que ele obteve paz.
Como o que vai buscar peixe, ele nos conta como fez para conseguir
peixe. Não devemos nos apegar: “Eh, para conseguir peixe tem que usar
vara, tem que usar rede, para conseguir peixe tem que ir comprar, tem
que usar um arpão.” Não se apegue a usar arpão, usar vara, usar rede, o
significado verdadeiro é conseguir peixe para comer. Bem ali o
resultado se exprime. 

Prática é a mesma coisa, é muito ampla. A maioria de nós pensa
superficialmente, então é difícil nos entendermos. Eu digo que é
\textit{paccataṁ}, esse Darma é \textit{paccataṁ} de verdade. Todas as
maldades da nossa mente, quem consegue ver se não nós mesmos? Toda a
felicidade que há em nós, a pureza ou as mentiras que há em nós, quem
consegue saber e ver se não nós mesmos? Se nós não vemos, quem virá
resolver nosso problema? Vamos sempre ser ladrões, quem vai nos
corrigir? É assim. Então ele ensinava a ver a si, a conhecer a si,
“\textit{sakkhiputtha}”, seja testemunha de si mesmo, não é dever dos
outros. Isso é ser superficial, pegar os outros como testemunha\ldots{} livros
não servem de testemunha, qualquer um que lê consegue dar sermão, é
assim. A verdadeira testemunha é o Darma. Não é outra coisa. Por
exemplo: fizemos algo errado, deixe o resultado vir. Por exemplo, as
coisas das quais não sabemos a causa, vemos e deixamos estar, não
estamos perdendo nada porque pelo menos estamos vendo o malefício do
nosso carma – isso é mais útil. 

As pessoas vêm até aqui mas eu não tenho nada de mais, vivo quieto
minha vida na floresta, em reclusão. Gosto de viver na floresta, então
vivo aqui, em seguida os demais monges também vieram viver aqui. É algo
bom. É dever de um monge viver desse jeito. A forma ideal é essa, já as
outras formas, depende da sabedoria de cada um. O Buda ensinava a não
se obcecar. Pratique, mas não se obceque, onde quer que viva, não se
obceque. É só isso, assim encontra bem-estar. O que é bem-estar?
Vivendo num certo local, ninguém vem nos incomodar, nos dar bronca, nos
criticar, então temos bem-estar. Não se apegue a isso. Se acontecer de
irmos a um local onde nos criticam, vamos ter bem-estar? As impurezas
mentais estão escondidas. Isso é bem-estar por ter o que deseja, ter o
objeto mental que lhe agrada. Mas em alguns lugares teremos objetos
mentais que não nos agradam, vamos sentir bem-estar? Isso é muito
importante. Hoje não temos doença, então podemos dizer que estamos bem;
amanhã, mês que vem, ficamos doentes, por exemplo, surge dor no corpo
ou na mente, vamos conseguir aguentar? 

Buscamos reclusão para lutar, nos isolamos para lutar; se não
estamos cientes disso, fugimos o tempo todo. Aqui não há paz, então
foge; aqui não há paz, foge de novo; aqui não há paz, foge, pois se
ficar vai sofrer. Acha que sofrimento é inútil. Já eu, penso que
sofrimento é útil, é a Nobre Verdade do sofrimento, se não encontrarmos
sofrimento, não lutarmos com ele, não o entenderemos. Se vê o
sofrimento e foge sempre, sofrerá sempre. É assim que vai ser, vai
continuar apanhando da felicidade e do sofrimento. Mas se encontrar
algo ruim e não conseguir aguentar, não fique ali – fuja, admita
derrota. Fuja, mas fuja para vencer, não fuja para ser derrotado, fuja
para resolver o problema, para aprender e voltar para lutar mais uma
vez. É assim. 

Qualquer coisa que vemos, recolhemos e guardamos, colocamos onde é
garantido: é impermanente, é sofrimento, é \textit{anatta}. Acabou.
Todos os Darmas acabam ali. Acabou o valor, vira um monge sem valor,
sem veneno, sem perigo, não está preso a mais nada. Isso leva à
sabedoria, com certeza, vê a corrente do nirvana de verdade. Quando vê
um problema, a solução surge sozinha, nasce sabedoria ali mesmo, é
\textit{anicca, dukkha, anatta}, sempre. É bem aqui, não precisa
especular: “Como é o nirvana? Será que é felicidade ou sofrimento?”;
tudo se reúne aqui, acaba bem aqui. É assim que os monges têm que
praticar para não se perderem; se usarem isso, não se perdem. Isso
protege contra se perder. Com o tempo vai contemplando até não ver mais
nada, vê só nascer e desaparecer, vê só \textit{anicca, dukkha,
anatta}, então a felicidade e o sofrimento ficam sem valor. Ficam sem
valor. Para que procurar mais? Os apegos se encerraram. Só sobra o
Darma puro, só há nascer e desaparecer, nada de mais. Se falamos assim
temos onde segurar, essa é a fundação.

Isso se chama praticar o Darma. Eu falo tal qual contemplei, não
tenho muito estudo, não sei muito, não tenho nada de mais; em geral
vivi na floresta e sempre tive essa sensação. Mesmo quando ensino meus
discípulos estrangeiros que são formados, eles têm diploma de mestrado
mas não vou correndo atrás do conhecimento deles, eu falo de acordo com
a linguagem que conheço: “É assim, assim\ldots{}”, falo desse jeito, fica a
critério de cada um saber utilizar, eu fico tranquilo. Não precisa de
toda essa confusão com esse ou aquele método, falo de forma resumida
sobre fazer a visão ficar correta. Se estiver correta, acabou a
necessidade de consertar problemas, se não há necessidade de consertar
problemas, não há o que fazer, se não há nada, largamos, é a natureza
do Darma. Se fazendo algo sente preguiça, oponha-se à preguiça e vá
fazer aquilo, vamos conhecer o Darma, a prática, o que fizermos será
Darma, não vamos ficar sem Darma. Se fazer algo é errado, procuramos
evitar, temos medo de errar, então não fazemos: isso ainda não está
puro, ainda há desejo de fazer aquilo. Enquanto não enxergarmos que não
é nosso dever fazer aquilo, ainda haverá desejo. 

Quando não houver mais desejo, já não haverá mais “abandonar,” já
não haverá “praticar” – acabou, só isso. Pode chamar do que quiser,
quem quiser chamar do quê, que chame! O sol é assim, pode chamar do que
quiser, o nome é apenas uma convenção. Quando as pessoas nascem botam
um nome “Senhor X, Senhor Y,” mas na verdade ali não há nome algum, é
assim. É assim a prática, quem praticar seguindo dessa forma não
fracassa! Os monges estrangeiros vêm pedir para morar aqui, eu os
recebo, mesmo dos demais grupos, por exemplo o grupo Dhammayuta, pedem
para viver comigo, eu não proíbo, só pergunto: “Você pratica dessa ou
daquela forma? Seu professor não se opõe? Então pode ficar, mas se vai
ficar aqui, vai ter que praticar do nosso jeito.”
