
\chapter{Nota – Ajahn Vajiro}

Ajahn Mudito pediu-me que escrevesse um prefácio para esta coleção
de ensinamentos de \textit{bhikkus} da floresta da Tailândia. Eu
conheço Ajahn Mudito desde quando ele primeiro aspirou tomar o
treinamento como \textit{bhikkhu} no começo deste século – naquela
ocasião, apenas por e-mails. Desde que comecei a estar mais envolvido
em trazer da tradição da floresta do Budismo para os países de língua
portuguesa, nós tivemos mais contato.

Este trabalho é uma forma de transmitir os ensinamentos dos monges
da floresta da Tailândia. Ajahn Mudito está muito bem qualificado a
preparar esses ensinamentos para as pessoas do mundo que usam o
português como primeira língua. Ele vive a vida de um monge da floresta
na Tailândia e tem dedicado tempo e atenção em aprender destes Ajahns e
discípulos destes Ajahns. Muitos destes ensinamentos não estão
disponíveis em livros de nenhuma outra língua. Os ensinamentos são de
uma tradição oral, e Ajahn Mudito foi o primeiro a transcrevê-los. Eles
foram então utilizados para gerar legendas para vídeos dos Ajahns.
Aquele método foi bom e agora o formato de palavra escrita é um
acréscimo útil.

Estou muito contente que Ajahn Mudito tenha realizado este trabalho
e confio que será uma benção para muitas pessoas.

Ajahn Vajiro

{\itshape
Comunidade Religiosa Budismo Theravada da Floresta}

{\itshape
Ericeira, Portugal}

