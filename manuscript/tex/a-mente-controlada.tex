
\chapter{A mente controlada por ilusões}

{\itshape
Palestra e sessão de perguntas e respostas dada por Ajahn Tan durante um
retiro de meditação.}

A natureza da mente que ainda está envolta em delusão é de
habitualmente se perder, se apegar às emoções e pensamentos de todos os
tipos, como se fossem de fato nossa mente. A correnteza de emoções que
vêm da nossa mente – ela se apega a essa correnteza como se fosse ela
própria. As emoções são manifestações da mente. Se surge desejo, raiva
ou deleite por diversas coisas, nossa mente se apega frequentemente a
essas manifestações ou emoções como se fossem ela própria.

A mente de todos nós deseja a verdadeira felicidade, não há quem
deseje o sofrimento. Já observamos ou não? Por que dentro de nossas
mentes há sofrimento e confusão? Uma vez que desejamos felicidade, por
que às vezes surge sofrimento na nossa mente? Se tivermos um pouco de
atenção, veremos o sofrimento surgindo na nossa mente vindo do desejo,
da raiva, do deleite e da aversão. Por que não procuramos o caminho
para cessar o sofrimento na nossa mente? Visto que temos
inteligência,\footnote{Em tailandês, a palavra comum para “inteligência”
é o composto “sati-paññā.” Às vezes é óbvio que o autor está se
referindo ao significado comum; outras vezes, dá a impressão que ele se
refere ao significado mais profundo, em \textit{páli}. A pessoa que
esteja lendo pode então escolher sozinha como quer interpretar o uso do
termo nas diferentes ocasiões ao longo do texto.} alguma vez já
refletimos sobre porque é que é possível treinar um elefante, treinar
um cavalo ou um cachorro, mas essa mente que nós pensamos ser nossa,
por que não seria possível treinar para que ela melhore? Por que
devemos desenvolver a prática de \textit{samādhi}? Porque nossa
mente, se não possuir \textit{samādhi}, é como o mato e o vento que
sopra das quatro direções. O vento são as emoções que surgem na nossa
mente. Quando o vento vem de qualquer direção, o mato se inclina de
acordo com o fluxo do vento: nossa mente se agita e flui de acordo com
o fluxo das emoções.

Faça sua mente firme como uma montanha que é muito resistente. As
montanhas não se agitam com o vento; mesmo que seja um tufão soprando,
não importa quão forte, ou um tornado, não são capazes de fazer uma
montanha se agitar. A mente de um \textit{anāgāmī} ou
\textit{arahant} também é assim, não se agita com o fluxo das emoções.
Para fazer que a mente tenha \textit{sati}, tenha \textit{samādhi},
nós temos que treinar em focar a mente em \textit{samādhi}, ou seja,
focar \textit{sati} em um objeto de meditação para que a nossa mente
tenha paz. Focar \textit{sati} de maneira contínua numa fundação faz
surgir êxtase, felicidade; surge a equanimidade de \textit{samādhi}.
Se praticar bastante, a mente se unifica, é \textit{ekaggatā arama,
ekaggatā citta}, tem \textit{sati} firme no momento presente. Focar
\textit{sati} em uma fundação de forma contínua faz surgir
\textit{samādhi}; quando \textit{samādhi} surge, gera força para
\textit{sati}, se torna energia para
\textit{sati-paññā}\footnote{Veja nota anterior sobre
“inteligência.”} na nossa mente.

Nossa mente ou \textit{sati}, que vem sendo agitada como o mato, se
perde nas emoções, desejos, raivas, deleites e aversões. Quando
começamos a ter \textit{sati}, a capacidade de lembrar, essa capacidade
de lembrar consegue acompanhar as emoções que surgem na mente. As
emoções não surgem no céu ou na terra, elas surgem na mente das
pessoas. Basta termos \textit{sati} vigiando nossa mente e nós veremos
as emoções, todas as diferentes manifestações da mente surgindo em nós
mesmos. Se perdemos a atenção, se quebramos \textit{sati}, nossa mente
se perde no fluxo das emoções como desejo, raiva e sofrimento que
surgem na mente. Quando perdemos a atenção ou quebramos \textit{sati},
as impurezas – desejo, raiva, delusão – obcecam nossa mente. Tudo que
nós pensarmos, falarmos ou fizermos estará sob o poder do desejo e da
raiva.

Para que \textit{sati-paññā} surja é necessário desenvolver
\textit{samādhi pāramī}. Quando temos a força de \textit{sati},
vemos as emoções e todo tipo de manifestações da mente. É como se nesse
local houvesse uma cerca. Se formos para o andar de baixo ou para a
cozinha, não veremos as pessoas que entram e saem pelo portão. Mas se
nós colocarmos uma cadeira na porta de entrada e nos sentarmos ali,
veremos todas as pessoas que entrarem e saírem. Portanto, se tivermos
\textit{sati}, focarmos \textit{sati} na mente, vamos ver as emoções e
todas as manifestações da mente.

As emoções não nascem nos olhos, ouvidos, nariz, língua ou corpo; as
emoções não nascem nas imagens, sons, odores, sabores ou tato; todas as
emoções, quer seja desejo, aversão ou deleite de todos os tipos, nascem
em nossas mentes. Os olhos são apenas a ponte de contato com as
imagens. Quando os olhos veem uma imagem, surge uma sensação na mente:
surge deleite, aversão ou indiferença. Os ouvidos são a ponte de
contato com os sons; a sensação de deleite, aversão ou indiferença
surge na mente apenas. O nariz contata cheiros, a língua contata
sabores, o corpo com frio, calor, macio e duro, mas todas as sensações
surgem na nossa mente.

Portanto, se nós ficarmos atentos e tivermos \textit{sati} cuidando
da mente o tempo todo, vigiando a mente, veremos os objetos mentais e
as manifestações da mente. Treinar e desenvolver \textit{samādhi
pāramī} é causa para que \textit{sati} nasça e consiga acompanhar
e conhecer os objetos mentais e as impurezas que surgem na mente.
Quando estamos nas posturas comuns, quer seja de pé, andando, sentado
ou realizando alguma atividade, tenhamos \textit{sati} cuidando da
mente, em todas as posturas. Realizando qualquer trabalho, tenha
\textit{sati} focada naquele trabalho. Caso esteja livre de trabalho,
use \textit{sati} para observar sua própria mente, faça sua mente ficar
livre de objetos mentais. Caso haja emoções, pensamentos de diversos
tipos surgindo na mente, teremos inteligência para refletir e ver a
impermanência deles. Quando desejo, ódio, deleite e aversão surgirem,
\textit{sati-paññā} tem que enxergar a impermanência, largar os
objetos mentais, e então a mente fica em equanimidade. Mas se
\textit{sati-paññā} não tiver força suficiente, então focamos
\textit{sati} no objeto de meditação por três ou cinco minutos e o
pensamento ruim some da nossa mente.

Focar \textit{sati} na respiração, ou no mantra “\textit{Buddho}”, é
possível em qualquer postura, quer seja de pé, andando, sentado ou
deitado – é possível focar \textit{sati}, desenvolver
\textit{samādhi} da mesma forma. Mas quando desejamos desenvolver
\textit{sati} e \textit{samādhi} de forma contínua, procuramos a
oportunidade de praticar sentado ou andando em meditação. Por exemplo,
todos nós tivemos a oportunidade de vir praticar neste local. Este
local está afastado de imagens, sons, cheiros, sabores e toques – as
emoções ou pensamentos na nossa mente diminuem. Se nos esforçarmos em
usar \textit{sati} para olhar a mente, veremos todas as formas de
pensamentos surgindo na mente. Quando as distrações de fora diminuem,
os pensamentos em geral vão para assuntos do passado ou do futuro.

Se tivermos \textit{sati} e virmos os pensamentos sobre o passado,
usamos inteligência para refletir: “O passado já se foi, para que
pensar nisso?” Traga \textit{sati} de volta ao momento presente ou de
volta à respiração. Se a mente estiver pensando no futuro: “Amanhã ou
depois de amanhã, o que vou fazer?”, nós usamos inteligência para
refletir: “O futuro ainda não chegou, para que me preocupar?” Traga
\textit{sati} de volta à respiração ou ao mantra “\textit{Buddho,
Buddho}” e a nossa mente se pacifica. Se a mente estiver pensando no
futuro, use inteligência para refletir: “O futuro ainda não chegou,
para que me preocupar?” e traga \textit{sati} de volta ao momento
presente.

Experimentem usar uma técnica de meditação chamada
\textit{maranasati}. Por exemplo, hoje a noite iremos dormir às onze ou
à meia-noite. Use \textit{maranasati} para imaginar que às onze ou à
meia-noite morreremos. Às vezes essa técnica de \textit{maranasati} é
capaz de fazer nossa mente não ser capaz de pensar em nada que esteja
além do horário das onze ou meia-noite, pois sabemos que às onze ou
meia-noite vamos morrer, e a mente volta para o momento presente. Usar
inteligência para contemplar a morte, não para criar medo, mas para
despertar nossa mente para que não seja descuidada com a vida.
Contemplar a morte para cortar a ganância, o ódio e a ignorância da
nossa mente. Ou usem seus próprios métodos para trazer a mente de volta
ao presente, de acordo com a habilidade de cada um.

Portanto, quando estamos num lugar como este, devemos treinar em
focar \textit{sati} e desenvolver \textit{samādhi} continuamente:
desde o momento em que acordar e abrir os olhos, estabeleça
\textit{sati} em sua mente. Quer esteja fazendo um trabalho sozinho ou
em grupo, tenha \textit{sati} naquela atividade. Quando tiver tempo
livre, procure uma oportunidade para praticar sentado em meditação.
Quando cansar, mude de postura e vá praticar meditação andando. 

A primeira vez, na época em que eu era leigo, em que vi a mente e os
objetos mentais, eu tinha um pouco mais de vinte anos. Antes eu tinha o
entendimento incorreto de que os pensamentos e emoções que surgem na
mente são a própria mente. Mas foi na ocasião em que surgiu confusão na
minha mente, após ter voltado da faculdade por volta das sete horas da
noite. Estava voltando para casa por volta das nove ou dez horas e a
mente não parava de pensar. Pensava em todo tipo de coisas, mas por
sorte tive \textit{sati} e consegui me perguntar: “Eh? Esta é a minha
mente, por que não consigo controlá-la?” Eu tive \textit{sati} para ver
os objetos mentais surgindo na mente e tomei a decisão de achar um
caminho para conseguir controlar minha própria mente. Eu não queria
pensar de forma confusa ou naquilo que é danoso.

Na minha casa havia um
altar.\footnote{\thai{ห้องพระ} – Tradução
literal: “quarto do Buda.” Antigamente, na Tailândia, as pessoas
reservavam um cômodo da casa onde moravam para colocar um altar e
utilizar para a prática de pūja e meditação. Hoje em dia ainda é
comum haver um altar dentro de casa mas em geral não é um quarto
separado. Mas ainda assim permaneceu o costume de chamar de “quarto do
Buda.”} Fui até lá e me sentei em meditação. Foi a primeira vez que
pratiquei meditação. É possível que tenha sido por ter visto meu pai.
No \textit{uposatha}, ou quando tinha tempo livre, ele ia até o altar
vestindo uma roupa branca\footnote{Algumas pessoas gostam de vestir
roupas brancas quando observam o uposatha.} e praticava meditação.
Quando minha mente estava pensando sem parar, decidi procurar um
caminho para parar as emoções e pensamentos na mente até encontrar.
Então sentei em meditação e foquei \textit{sati} na ponta do nariz:
quando inspirava e o ar tocava a ponta do nariz, eu pensava “Bud;”
quando expirava, o ar tocava a ponta do nariz e eu pensava “dho.”

Deve ter sido sorte minha que, quando foquei \textit{sati} na
respiração, minha mente parou de pensar em todas aquelas histórias.
Tinha \textit{sati} no momento presente, tinha \textit{sati} na
respiração, tinha \textit{sati} na meditação sobre a respiração de
forma ininterrupta. A mente aos poucos se pacificou, o corpo ficou
leve, a mente ficou leve, surgiu êxtase. Eu percebi que a mente não
estava mais confusa, não estava mais pensando de forma confusa, o corpo
estava leve, a mente estava leve, surgiu felicidade e frescor na mente.
Eu não sentei em meditação por muito tempo, talvez uns dez ou quinze
minutos, mas a mente estava em paz o tempo todo. Quando mudei de
postura e saí da meditação, tinha \textit{sati} firme no momento
presente, a mente estava unificada.

Eu vi o resultado de praticar meditação. Quando saí da postura de
meditação sentada, \textit{sati} era capaz de controlar a mente. Quando
fazia meus deveres em casa, por volta das dez ou onze da noite, minha
mente estava unificada. Então eu descobri que essa prática de
\textit{samādhi} é causa para que haja \textit{sati}, e quando há
\textit{sati} somos capazes de controlar nossa mente. Desde então,
qualquer dia em que sentisse que a minha mente estava pensando de forma
confusa, eu procurava uma oportunidade para sentar em meditação. Mas eu
era uma pessoa acostumada ao conforto, não gostava de dificuldades,
então só praticava meditação nos dias em que a mente estava confusa.
Mas eu tinha essa sorte: todas as vezes que sentava em meditação, minha
mente se pacificava.

Não sentava por muito tempo: no começo eram apenas dez ou quinze
minutos. Então fiquei sabendo que, para ter \textit{sati} controlando a
mente, é necessário praticar \textit{samādhi}. Depois disso eu passei
a procurar controlar a mente para pensar coisas boas, falar coisas boas
e fazer apenas coisas boas o tempo todo. Quando estava em qualquer
posição, de pé, andando, sentado ou fazendo o que quer que fosse, caso
houvesse um pensamento ruim, \textit{sati} via aquele pensamento e eu
usava inteligência para achar alguma forma de expulsá-lo o mais rápido
possível. Se inteligência não encontrasse uma forma para abandonar
aquilo, eu simplesmente trazia \textit{sati} de volta à respiração e o
pensamento desaparecia. Na época em que eu era leigo, usava esse método
para cuidar da minha mente e minha mente tinha paz e frescor o tempo
todo.

Eu observava os cinco preceitos de forma regular e tinha intenção de
só construir bondade no meu coração. Me lembro de que quando tinha
dezessete ou dezoito anos tive uma emoção no coração que dizia:
“Qualquer forma de bondade que exista neste mundo – aspiro a todas
elas. Eu desejo construir apenas coisas boas.” Numa outra época da
minha juventude, quando eu tinha dezenove ou vinte anos, se fosse a um
templo, qualquer que fosse, eu procurava um altar que fosse tranquilo.
Eu me lembro que quando ia ao altar, onde era silencioso, me prostrava
três vezes e o que eu pensava ou determinava não era como as demais
pessoas faziam, na minha mente surgia a frase: “O caminho mais
excelente – eu desejo encontrar esse caminho.” Era assim todas as vezes
que me prostrava, nunca pensei de nenhuma outra forma. Isso eu só
contei para que sirva de reflexão, para despertar o coração. 

Por isso, uma vez que todos nós tivemos a resolução de vir praticar,
observar \textit{sīla} e desenvolver \textit{samādhi}, usem
\textit{sati} para cuidar de sua mente de forma equilibrada, não
tensionem demais para que não surja estresse, nem afrouxem demais para
que não percam \textit{sati}. A prática não é feita para gerar
sofrimento, mas sim para que a mente se desprenda do sofrimento, dos
apegos ao que quer que seja. Hoje eu já ofereci pensamentos o
suficiente, peço para parar por aqui. Em seguida há um pouco de tempo,
quem tiver algum problema relacionado à prática pode perguntar.

-- Eu sou novo em meditação e o que eu gostaria de saber é: como
lidar com a dormência nas pernas e nas costas durante a meditação?

-- Se a mente ainda não se pacificou e sente dor surgindo, se não
consegue aguentar, mude de posição. Mas se focou \textit{sati} e
desenvolveu \textit{samādhi}, se a mente se pacificou primeiro e em
seguida surgiu a dor, por exemplo dor na perna, existindo um pouco de
\textit{sati} e \textit{samādhi}, \textit{sati} vai focar naquela
dor. Use inteligência para contemplar o corpo nesse corpo até enxergar
como elemento terra, água, ar e fogo. Separe a mente e o corpo.
Contemplar o corpo como elementos não é nada difícil. Use inteligência
para refletir que se nós inspirarmos e não expirarmos, ou expirarmos e
não inspirarmos, nosso coração vai parar de funcionar. Quando o
elemento ar cessa, o elemento fogo cessa em seguida: vai surgir frio no
nosso corpo e em três ou cinco dias o elemento água, o sangue, o pus –
nosso corpo não conseguirá retê-los, eles vazarão e se espalharão pelo
chão. A parte do elemento terra, que é a parte dura, o cabelo, os
pelos, as unhas, os dentes, a pele, os ossos, vai se desmanchar e se
espalhar pelo chão.

Procure separar a mente do corpo utilizando \textit{sati-paññā};
contemple o corpo sendo quatro elementos – terra, água, ar e fogo.
Quando nossa mente vir que o corpo é uma coisa e que a mente é outra,
nossa mente se separará do corpo. Quando a mente se separar do corpo,
use \textit{sati-paññā} para contemplar e buscar a verdade sobre a
dor. A dor que surge com este corpo é realmente nossa mente ou não? Dor
é um fenômeno que surge do corpo apenas e é da natureza de surgir e
desaparecer. Procurem usar \textit{sati-paññā} para contemplar o
corpo no corpo e separar a mente do corpo. Usem \textit{sati-paññā
}para contemplar as sensações nas sensações e separar a mente do corpo,
separar a mente das sensações. A mente é uma coisa, o corpo é outra, as
sensações são outra. Esses três não estão relacionados. Mas para que
\textit{sati-paññā} tenha força para contemplar o corpo no corpo, as
sensações nas sensações, tem que haver \textit{samādhi} unido à
\textit{sati-paññā}. Se \textit{samādhi} ainda não surgiu, se nós
não aguentarmos a dor, às vezes temos que mudar de posição.

Na verdade, se nossa mente sair deste corpo, quer dizer, se nós
morrermos, estes elementos, este corpo, não sentem nada, nem calor nem
frio. Levam para cremar ou enterrar sob a terra e ele não sente nada.
As sensações estão somente na nossa mente, que se apega a este corpo
como se fosse “eu.” Deu para entender? Alguém tem mais alguma pergunta?

-- Sim, eu tenho uma pergunta. Tahn Ajahn, você mencionou que
quando tinha vinte anos você descobriu que conseguia encontrar paz na
sua mente através de \textit{sati}, porque você disse estava tendo
muitos pensamentos em sua cabeça, então percebeu, por ter
\textit{sati}. Então você foi a um quarto e então descobriu, ou soube,
que conseguia ter uma mente calma, uma mente muito pacífica. De fato,
para mim, isso já é superar o sofrimento. E então você disse que,
sempre que tinha esse problema, você fazia isso, você fazia essa
meditação. Nos é dito que o budismo serve para superar o sofrimento.
Tahn Ajahn já encontrou, mas ainda assim, apesar disso, ele
foi adiante e tornou-se um monge. A razão pela qual tornou-se monge é
porque está obtendo mais felicidade do que a que ele havia descoberto?
E nesse caso, ele poderia nos dizer onde ele encontrou mais para
superar esse sofrimento? \textit{Sādhu}.

-- Naquela ocasião eu sabia o método para parar o sofrimento vindo
das impurezas mais grosseiras apenas, o que era feito de forma
temporária. Mas, na verdade, na época em que eu era leigo, na minha
vida eu queria: número um, estudar e atingir a graduação mais elevada.
Número dois, encerrados os estudos, ter o melhor emprego; e número
três, eu pensava em ter família. Tinha esses três. Na minha vida laica,
naquela época, eu não tinha visto muito sofrimento, pois eu vivia como
os jovens em geral, tinha muitas diversões e distrações.

Mas o sofrimento que eu vi de verdade… Eu ainda não tinha tido
nenhum sofrimento, mas eu vi na mente. Uma vez, eu tinha dezenove ou
vinte anos, estava indo de ônibus para a faculdade. Eu tinha onde
sentar no ônibus, mas uma moça com um bebê de colo entrou e eu cedi meu
lugar. Aquele bebê ainda não tinha um ano de idade, pois a mãe o
carregava no colo. Eu me levantei e me segurei na barra de ferro perto
de onde ela estava sentada, e meu olhar caiu sobre a criança. Foi só
meu olhar cair sobre a criança e a minha mente começou a pensar, a
pensar em algo que nunca havia pensado antes, foi a primeira vez. Eu
pensei: “Essa criança, até que ela consiga falar, andar e cuidar de si
mesma, até ela ficar do meu tamanho, vai ter que passar por muita
felicidade e sofrimento.” Minha inteligência pensou só isso e surgiu
desencanto na mente. Eu terminei de contemplar sobre a criança, então
olhei para a frente. Eu vi uma pessoa que me pareceu não estar se
sentido bem, estava indo ao hospital, pois esse ônibus passava em
frente ao hospital. Algo que eu nunca havia pensado veio à mente:
“Dentre todas as pessoas, não há quem consiga escapar das doenças,
mesmo eu um dia ficarei doente como ele.” E eu senti desânimo e
desencanto na mente. Então meu olhar saiu daquela pessoa e foi
encontrar uma pessoa idosa, por volta de sessenta ou setenta anos.
Pensei pela primeira vez naquilo que nunca havia pensado: “Dentre todas
as pessoas, não há quem consiga escapar da velhice: cedo ou tarde eu
também serei daquele jeito.” E de novo surgiu desencanto na minha mente
e em seguida eu pensei: “Todas essas pessoas nesse ônibus, essas trinta
ou quarenta pessoas aqui, para onde estão indo?” e surgiu uma sensação
na minha mente que respondeu: “Todas as pessoas nesse ônibus estão indo
em direção à morte.” E surgiu muito desencanto na minha mente. Eu vi
crianças, doentes, idosos no ônibus quase todos os dias, por volta de
vinte ou trinta dias, e minha mente pensava assim todos os dias. E no
último dia eu disse para mim mesmo que quando fosse possível me
ordenaria monge. Se não fosse possível, então faria o máximo para ser
uma boa pessoa. Talvez eu não seja bom na opinião dos outros, mas que
na minha mente eu seja uma boa pessoa.

Esse foi o sentimento que surgiu na minha mente. Mas eu era uma
pessoa acostumada ao prazer e ao conforto e meu coração naquela época
ainda não estava pronto. Eu tinha planos como as pessoas comuns, me
decidi a estudar até a graduação máxima e ainda não tinha alcançado
aquela meta. Número dois, quando acabasse os estudos, queria trabalhar.
Número três, queria ter família. Mas eu disse para meu coração que,
caso não estivesse pronto, eu então seria a melhor pessoa possível e
tentaria observar os preceitos, sempre. Mas se eu estivesse pronto, me
ordenaria. E não me ordenaria temporariamente ou de brincadeira. Se me
ordenasse seria uma única vez e para o resto da vida.

Foi assim, na verdade eu não tinha visto sofrimento ao redor, mas vi
na minha mente daquele jeito. Eu ainda não tinha me encontrado com a
velhice, com a doença, com a morte, mas pensei: “Não importa quantas
vidas eu nasça. Tendo nascido não serei capaz de evitar a velhice, a
doença e a morte.” E surgiu desânimo, desencanto. Deu para entender?

-- Sim, eu acho que entendemos porque Tahn Ajahn decidiu
tornar-se monge. Eu acho que a maioria dos praticantes, incluindo a
mim, ainda estamos esperando para ver os três sinais\footnote{Velhice,
doença e morte. Foi ao ter visto esses mesmos três sinais que Siddattha
Gotama decidiu abandonar seu lar e tornar-me um monge.} em um ônibus,
e se não os virmos, vamos tentar ser o melhor possível.
\textit{Sādhu, sādhu!}

-- Bom. Determinem um objetivo e criem as condições. Gostamos de
ter objetivos na vida, então criem as condições para alcançarem aqueles
objetivos. 

-- Tahn Ajahn, saudações. Eu tenho uma pergunta que diz respeito à
morte, porque não há muito escrito dentro do Budismo Theravada sobre a
morte. Então eu gostaria de saber: em geral, como uma pessoa deve se
preparar para a morte? Quando estamos próximo ao momento, o que devemos
fazer?

-- Quando estiver perto da morte tenha \textit{sati} focada no
momento presente. Ainda estando vivo, foque \textit{sati} e desenvolva
\textit{samādhi} para que a mente se pacifique. Quando a mente
estiver pacificada e ainda estivermos respirando, traga à frente o
corpo como objeto de meditação dos elementos terra, água, ar e fogo,
para que a mente contemple e se desapegue do corpo. Se a mente enxergar
que este corpo não é a mente e que a mente não é o corpo, nossa mente
não vai ter medo de morrer. Nossa mente não vai tremer diante de
qualquer perigo de morte pois vai saber, de forma natural, que é normal
que este corpo surja e se degenere, ele não é a mente. Tendo
contemplado, deixe a mente descansar em \textit{samādhi}.

Aí vai surgir uma imagem do nosso carma, o carma que fizemos desde
que nascemos até o momento da nossa morte. Se o carma bom tiver muita
força, ele dará resultado; se o carma ruim tiver muita força, ele dará
resultado. Haverá uma imagem do carma surgindo na mente. Se fizemos
mérito, seguimos os preceitos, construímos um templo, um monastério, ou
fizemos algum tipo de mérito, às vezes doamos comida aos monges, a
imagem do carma aparece e a mente vai agarrar aquela imagem. A imagem
sendo boa, quando morrermos, a mente vai nascer de acordo com o carma
que acumulamos. Vai de acordo com o resultado do carma. Se acumulamos
carma ruim, surgirá uma imagem ruim. Se a mente agarrar aquilo, surgirá
medo e pavor e quando morrermos vamos para um mundo inferior. Carma
gera os resultados, carma é nossa origem, carma é quem nos acompanha,
carma é o que nos sustenta. Quem fizer carma bom obtém bons resultados,
quem fizer carma ruim obtém resultados ruins.

-- Tahn Ajahn, a maneira como envelhecemos está fora do nosso
controle, e o modo como o cérebro envelhece está fora do nosso
controle. Para uma pessoa que pratica muito o Darma, quando nossa mente
está clara e acontece que a pessoa tem demência ou Alzheimer, haveria
uma boa chance de que a mente permanecesse clara durante a morte para
poder observar e prestar atenção à respiração?

-- Se ela tiver \textit{sati} ela vai conseguir, e o carma que eu
disse, a imagem do carma que tenha realizado, bom ou ruim, irá se
manifestar na forma de imagem. Se for uma imagem boa, nascerá
felicidade e a mente agarrará aquela imagem: nesse caso, a imagem do
carma trará um bom resultado. Se a mente ainda não alcançou um estágio
de iluminação, não é um A\textit{ryia Puggala}, quer seja dos níveis
mais baixos ou mais altos, está exposto à imagem do carma que fez no
passado. Se fez o bem, tem um bom resultado; a imagem surgirá na hora
da morte.

Por exemplo, uma imagem conectada ao budismo, às vezes doamos comida
aos monges e aquela imagem surge, a mente sente felicidade. Se doamos
algo à sangha, e a mente vê aquela imagem. Se construímos uma cabana
para os monges, um templo ou ajudamos a construir, ou oferecemos algo
material para ajudar o budismo, o carma irá se concentrar em uma
imagem. Esse carma é aquele que, vendo, fará a mente sentir felicidade
e ela agarrará aquela imagem. E essa imagem é o que vai nos levar a
nascer entre os seres humanos ou entre os seres celestiais.

-- Tahn Ajahn, eu também tenho uma pergunta relacionada. Digamos
que nosso carma seja bom o suficiente para renascermos num estado
favorável, existe alguma maneira de quando estivermos perto da morte,
se nós ainda não completamos nosso trabalho espiritual, nós ainda não
entramos na corrente,\footnote{Refere-se à realização do primeiro
estágio de iluminação.} existe alguma forma de evitar nascer nos
reinos celestiais para não perder tempo naquele
nascimento?\footnote{Esta pergunta vem do entendimento de que um
nascimento celestial não é favorável à prática do Darma, pois esse tipo
de ser experiencia muita felicidade e tende a perder o ímpeto em buscar
um fim para o ciclo de nascimento e morte.} Obrigada.

-- Tem que determinar o \textit{pāramī}. Se não quiser receber
o resultado do mérito que leva a nascer como anjos ou deuses…
não é? Tem que determinar o \textit{pāramī}. Fizemos mérito
frequentemente e isso é o que gera resultado, mas nós temos que
determinar o \textit{pāramī}: “Eu desejo nascer humano, encontrar
o ensinamento do Buda e viver num país onde o budismo prospera.” Temos
que determinar o \textit{pāramī}. Se acumulamos mérito o
suficiente, determinar o \textit{pāramī} conseguirá que as coisas
aconteçam da forma que queremos. É como o vestibular: tem nota alta e
nota baixa, e cada curso tem um mínimo exigido. Se tivermos nota
suficiente, podemos escolher aquele curso. Se acumulamos mérito
suficiente, podemos escolher.

