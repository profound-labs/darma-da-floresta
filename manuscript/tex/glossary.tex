
\chapter{Glossário}

\begin{quote}
\itshape Nota: alguns dos termos relacionados, apesar de origem sânscrita ou páli, estão sendo assinalados como pertencentes à língua portuguesa, uma vez que já foram incorporados à mesma.
\end{quote}

\begin{description}

  \item[Ajahn]\itemlang{tailandês} Professor. Esta palavra é a transcrição tailandesa do termo em páli “\emph{acārya}”.

  \item[Akāliko]\itemlang{páli} Não limitado pela ação do tempo, não sujeito ao tempo – uma das qualidades atribuídas ao Darma.

  \item[Anāgāmī]\itemlang{páli} O estágio de iluminação anterior ao \emph{arahant}. Nesse estágio, qualquer desejo por existência corporal foi definitivamente abandonado, e portanto essa pessoa não mais toma nascimento entre seres humanos (nascimento como animal já tendo sido abandonado anteriormente no estágio de \emph{sotāpanna}).

  \item[Ānāpānasati]\itemlang{páli} Técnica de meditação que utiliza a respiração como foco.

  \item[Anatta]\itemlang{páli} Ausência de uma entidade, ausência de um “eu”.

  \item[Anicca]\itemlang{páli} Impermanência.

  \item[Arahant]\itemlang{páli} Pessoa que alcançou nirvana.

  \item[Aryia Puggala]\itemlang{páli} Ou “\emph{aryia}”. Aqueles que alcançaram um dos quatro estágios de iluminação (\emph{sotāpanna, sakadāgāmi, anāgāmī} e \emph{arahant}).

  \item[Baht]\itemlang{tailandês} Denominação da moeda tailandesa.

  \item[Bhikkhu]\itemlang{páli} Um monge ordenado dentro da linha monástica iniciada pelo próprio Buda, ou seja, esse termo não engloba todas as pessoas que se declaram “monge budista”.

  \item[Bodisatva]\itemlang{português} Aquele que almeja tornar-se um Buda. (páli: \emph{bodhisatta})

  \item[Buda]\itemlang{português} Uma pessoa que desenvolveu ao máximo suas qualidades espirituais (ver \emph{pāramī}) com o objetivo de tanto alcançar a iluminação como também estabelecer um Buddha Sāsana para ajudar os demais seres a alcançar nirvana. (páli: Buddha)

  \item[Buddha Sāsana]\itemlang{páli} O conjunto de recursos e suportes que o Buda cria para auxiliar as demais pessoas a alcançar a iluminação, ou seja, os textos, os exercícios espirituais, os valores, os costumes, as tradições, etc. De maneira superficial, poderia se traduzir como “o budismo”.

  \item[Carma]\itemlang{português} Resultado de ações boas e ruins. (páli: \emph{kamma})

  \item[Darma]\itemlang{português} A doutrina do Buda. Outro uso comum desta palavra é para se referir a algo como “a verdade transcendental”, a verdadeira natureza de todas as coisas. (páli: Dhamma)

  \item[Dukkha]\itemlang{páli} Sofrimento, desconforto, insatisfação.

  \item[Jhāna]\itemlang{páli} Estados profundos de \emph{samādhi}. No ensinamento do Buda são catalogados oito estágios de \emph{jhāna}.

  \item[Kalyāna Dhamma]\itemlang{páli} Darmas belos e louváveis, boas qualidades do coração.

  \item[Kalyānamitta]\itemlang{páli} Amigo no Darma, pessoa que nos ajuda a trilhar o caminho da libertação.

  \item[Koan]\itemlang{japonês} Um objeto de meditação utilizado na tradição Zen/Chan.

  \item[Lokavidū]\itemlang{páli} Conhecedor do mundo, título atribuído ao Buda.

  \item[Luang Pó]\itemlang{tailandês} Respeitado pai, venerável pai.

  \item[Mahāyana]\itemlang{páli} Uma segunda vertente do budismo cujas expressões mais conhecidas vêm do Tibete, China e Japão.

  \item[Nirvana]\itemlang{português} O objetivo final do caminho budista, iluminação completa, o fim do ciclo de nascimento e morte. (páli: \emph{nibbāna})

  \item[Nobre Caminho Óctuplo]\itemlang{português} O modo de prática ensinado pelo Buda para a realização de nirvana, composto de: visão correta, intenção correta, fala correta, ação correta, modo de vida correto, esforço correto, sati correta e samādhi correto. (páli: \emph{Ariya Atthangika Magga})

  \item[Opanayika-dhamma]\itemlang{páli} Darma que conduz à realização de nirvana.

  \item[Paccataṁ]\itemlang{páli} Algo que só pode ser visto por si mesmo, experienciado por si mesmo.

  \item[Páli]\itemlang{português} Idioma proferido na região norte da Índia na época do Buda. É derivado do magadi, tem semelhança com o sânscrito e os únicos registros encontrados nesse idioma são os ensinamentos do Buda.

  \item[Pāramī]\itemlang{páli} Boas qualidades mentais/espirituais, necessárias no caminho para a iluminação.

  \item[Parinirvana]\itemlang{português} Morte de um arahant, quando ele alcança completa cessação. (páli: \emph{parinibbāna})

  \item[Piṇḍapāta]\itemlang{páli} O ato de os monges saírem pelas ruas recolhendo alimentos como esmola.

  \item[Pūjā]\itemlang{páli} Serviço devocional, gestos de louvor como fazer prostrações, recitar cânticos, oferecer flores, velas, incenso, etc.

  \item[Sacca Dhamma]\itemlang{páli} A Verdade, O Verdadeiro Darma, o aspecto mais sutil e profundo do Darma.

  \item[Sādhu]\itemlang{páli} Em páli, significa “muito bem!”, “excelente!” Já em tailandês, significa algo parecido com “amém!”, não tem um significado muito claro.

  \item[Sakadāgāmi]\itemlang{páli} O segundo estágio de iluminação. A partir deste ponto o praticante nasce no máximo mais uma vida entre seres humanos antes de alcançar nirvana.

  \item[Samathā]\itemlang{páli} O aspecto da prática de meditação que diz respeito a estabilizar, pacificar e descansar a mente.

  \item[Sammāditthi]\itemlang{páli} Visão correta, ponto de vista correto. O primeiro aspecto do Nobre Caminho Óctuplo.

  \item[Sampajañña]\itemlang{páli} Compreensão clara do momento presente.

  \item[Saṁsāra]\itemlang{páli} O ciclo de nascimento e morte onde estão presos todos os que ainda não alcançaram nirvana.

  \item[Sangha]\itemlang{páli} Nome normalmente utilizado para se referir à comunidade de monges budistas. Em geral, quando se referindo à comunidade laica, utiliza-se “Sangha Leiga/Laica”.

  \item[Sati]\itemlang{páli} A capacidade de aplicar a mente ao momento presente, estar ciente, presença mental.

  \item[Sāvaka]\itemlang{páli} Praticantes contemporâneos ao Buda que alcançaram algum estágio de iluminação.

  \item[Sotāpanna]\itemlang{páli} O primeiro estagio de iluminação. A partir deste ponto o praticante nasce no máximo mais sete vezes entre seres humanos antes de se tornar um \emph{arahant}.

  \item[Tahn Ajahn]\itemlang{tailandês} Forma respeitosa de referir-se a um mestre. “Senhor Professor”.

  \item[Theravada]\itemlang{páli} A linhagem budista mais antiga ainda presente nos dias atuais.

  \item[Upajjhāya]\itemlang{páli} Monge que preside a cerimônia de ordenação monástica.

  \item[Uposatha]\itemlang{páli} Dia em que os monges se encontram para confessar suas ofensas e recitar a regra monástica.

  \item[Vāsanā]\itemlang{páli} Tendências de comportamento acumuladas em vidas passadas.

  \item[Vipassanā]\itemlang{páli} O aspecto da prática de meditação que diz respeito a investigar a realidade, enxergar com clareza.

\end{description}

