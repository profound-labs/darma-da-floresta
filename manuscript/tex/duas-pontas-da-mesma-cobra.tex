
\chapterAuthor{Ajahn Chah}
\chapterNote{Discurso dado por Ajahn Chah a um grupo de leigos em Wat Pah Nanachat.}
\chapter{Duas pontas da mesma cobra}
\tocChapterNote{Discurso dado por Ajahn Chah a um grupo de leigos em Wat Pah Nanachat.}
\markright{\theChapterAuthor}

\ldots{} podemos dizer que a fé, a crença firme dos monges estrangeiros, é
muito boa, muito boa. A ponto de nem conhecerem o idioma tailandês, mas
eles aguentam, se esforçam. São pessoas que têm fé. Encontrar fé é
raro, encontrar fé é difícil, encontrar quem leve a sério é difícil.
Mas acho que no nosso país os monges são bons, eles respeitam o modo de
prática, mas levar a sério – quase não levam. Os
leigos em Ubon também, é só verem os monges e pedem os
preceitos,\footnote{Na Tailândia, em quase toda ocasião religiosa, é
costume que os leigos peçam que os monges lhes deem os cinco preceitos.
Quase ninguém tem real intenção de seguir os tais preceitos, é feito
apenas como cerimônia.} parecem famintos. Mas quando damos eles não
levam, ou levam para ir brincar, levam os preceitos para ir brincar. Se
nós estamos por perto vêm pedir como se estivessem famintos, mas quando
damos, jogam no lixo. Sempre foi assim. 

Uns dias atrás as comunidades de várias filiais\footnote{Wat Pah
Pong possui centenas de filiais dentro da Tailândia e dezenas fora do
país.} vieram me visitar, várias centenas de pessoas. Conversamos sobre
isso e perguntei: “O que vocês querem aqui? Eu venho ensinando já faz
quase trinta anos. Várias filiais vieram se reunir aqui hoje, hoje eu
enfim quero perguntar, pois nunca perguntei. Vocês vêm ouvir Darma, vêm
praticar, eu me esforço para ensinar aqui há quase trinta anos. Vocês
alguma vez já se decidiram ou já fizeram, alguma vez observaram os
cinco preceitos de forma regular?” Perguntei: “Quantas pessoas se
decidiram a ser \emph{upāsikās}\footnote{Uma discípula leiga (o
gênero masculino da palavra é “upāsakā”).} e observar os
cinco preceitos pelo resto da vida? Tem alguém?” Todo mundo ficou
olhando para os lados\ldots{} Não tem quem tenha coragem, mas pedem os
preceitos só para ir jogar no lixo. Pedem de novo e de novo. Pobre
gosta de pedir, não é? Pobre gosta de pedir; pede, mas não toma conta,
então continua pobre como antes. É só verem os monges e: “Me dá, me
dá!” Esse negócio é só pedir, não precisa cuidar, não precisa fazer
nada em troca. É como as pessoas que são pobres por
preguiça: pedem sempre. 

Naquele dia eu perguntei mas não tinha quem quisesse os cinco
preceitos. Mostra que não enxergam nada de mais. Eu vi aquilo e: “Oh,
que tristeza!” e falei, naquele dia eu falei: “Eh! Juntando todo mundo
aqui\ldots{} Achei que tinha gente aqui, perguntei e não tem gente, só tem
resto de gente! Neste salão inteiro só tem resto de gente.” Dei uma
bronca para ficarem com vergonha. Se não fizer assim não vão ter
vergonha, ainda vão pedir de novo e de novo. É difícil mudar. Nossa
tradição é fazer assim, normalmente é assim. Pede e fica ainda mais
pobre, pede e não trabalha, pede e não cuida, vê alguém vindo e pede.
Quanto mais pede, mais pobre fica. Não possui nada, não possui nada em
seu próprio corpo e mente, então tem que pedir. É desse jeito.

Quer dizer, estão praticando, mas falta prática. Conseguem estudar,
conseguem saber, mas falta prática. Falta muito. Quando ouvem isso
devem sentir vergonha, não? Somos os verdadeiros donos do budismo, não?
Mas os discípulos que vêm do exterior saem em disparada na nossa
frente, não é? Podem olhar. Eles vêm treinar aqui, o ambiente deles
muda de várias formas – eles aguentam. A comida muda
mais que tudo – eles aguentam. Muda tudo, até mesmo
a língua, muda tudo, muda, mas eles se esforçam. Me lembra um ditado
antigo: “Se não estiver em falta, não deseja.”; não está faltando,
então não deseja. Abrem os olhos e veem monges, mas veem só com os
olhos de fora; com os olhos de dentro não enxergam. Então sentem-se
cheios,\footnote{Em tailandês existe uma palavra específica para quando
se come até ficar satisfeito: \thai{อิ่ม}. Quando alguém
está “\thai{อิ่ม}”, essa pessoa já não sente mais fome. O
“cheio” usado neste trecho se refere a isso.} como as coisas que temos
em casa: se não estiver faltando, nos sentimos cheios, porque temos.
Aqui é a mesma coisa, enquanto há a sangha na Tailândia nós estamos
cheios, cheios de quê eu não sei – cheios de vento
ou o quê, eu não sei. Ficam felizes com isso. 

Outro dia fui até a capital, fui até Bangkok, encontrei com uma
pessoa que é funcionário publico. Enquanto vivia aqui nunca se
interessou pela sangha, mas quando foi morar no exterior ficou com
fome. Ficou com fome – não via monges, queria ver os
monges. Foi visitar um monge discípulo meu que vive em Londres, um
monge ocidental. Ficou impressionado e sentiu que naquela ocasião foi a
primeira vez em que viu o ensinamento do Buda, entendeu o ensinamento
do Buda. Esse tipo de coisa acontece por causa da falta de interesse. 

{\centering\itshape
-- neste ponto houve uma pausa e então\\
Ajahn Chah começa o discurso formal --
\par}

Saudações a todos que tiveram a determinação de vir aqui hoje fazer
mérito, tendo o Sr. Comandante\footnote{Em Ubon há uma base militar.
Provavelmente a pessoa referida é um membro da aeronáutica tailandesa.
} como líder que trouxe todos vocês aqui para fazer mérito. Portanto
hoje Wat Nong Pah Pong e Wat Pah Nanachat possuem um vínculo. Wat Nong
Pah Pong realizou a \emph{Mahā-Pavāranā}\footnote{Cerimônia
realizada ao fim do retiro das monções, onde os monges convidam uns aos
outros a apontar qualquer falta ou mau comportamento que tenham
cometido durante o período.} e hoje realiza a cerimônia de
\emph{kaṭhina}.\footnote{Cerimônia realizada ao fim do retiro das
monções, onde a comunidade monástica deve costurar e tingir um manto e
o oferecer como prêmio a um dos monges que tenha bom comportamento,
como forma de reconhecimento de suas nobres qualidades.} Terminada a
\emph{kaṭhina} esperei um bom tempo, pois o Sr. Comandante me
convidou desde aquela época e não esqueci, mesmo tendo outros deveres,
me esforcei para vir nos reunirmos aqui e trouxe vários discípulos,
tanto monges tailandeses como monges estrangeiros, misturados, todos
vieram se reunir aqui. 

Hoje me senti bastante cansado, mas ainda tive que sair. Para quê?
Para receber os visitantes. Eles vêm de manhã, de noite e ficam
bastante tempo. Eu os ensino. De manhã, desde após a refeição, me sento
ali e os visitantes vêm sem parar. Demorou até eu poder sair de lá e
vir até aqui. O tempo vai passando e a força do velho vai diminuindo,
diminuindo. Mesmo a voz vai diminuindo, o fôlego vai diminuindo.
Portanto hoje estou usando a força do coração, porque o Sr. Comandante
se esforçou em vir aqui e eu tenho que prestigiá-lo. Então peço que
prestem atenção, eu vou ensinar mas por pouco tempo, não muito. Mas eu
trouxe vários discípulos. O pessoal de Korat\footnote{Nome de uma
província tailandesa.} aguenta ouvir se eu mandar os monges ensinarem
hoje? E eu fico apenas de líder, falo só um pouco. 

Terminadas as atividades em Wat Pah Pong, a \emph{kaṭhina}, vim
aqui. Este ano a \emph{kaṭhina} foi muito tranquila, pois as pessoas
só fizeram mérito, não fizeram demérito. Fazer demérito é mais difícil,
mais complicado. Este ano fizemos mérito de verdade. A comida que as
leigas ofereceram aos monges era vegetariana, então foi tranquilo. Mas
não teve muita gente hoje à noite. Juntando todos deve ter sido umas
quatrocentas pessoas. Mas só tinha “gente.” Só tinha gente que veio
ouvir o Darma, não tinha gente que veio à toa. Silencioso e pacífico,
pois não teve nada, não teve música, gincana, teatro, não teve nada.
Começamos convidando os monges novatos que vieram se ordenar comigo
este ano, por volta de vinte monges. Pedi que eles subissem e falassem
sobre como se sentem em terem se ordenado em Wat Nong Pah Pong. Um por
um, todos eles falaram e durou por volta de duas horas. Além disso, nos
reunimos só para ouvir Darma. Eu sinto que este ano nossa prática de
mérito rendeu mérito de verdade. Foi tranquilo. Naquele dia era o
\emph{uposatha}. Normalmente no \emph{uposatha} as pessoas que estão
observando o \emph{uposatha-sīla}\footnote{Entre os oito preceitos
do uposatha-sīla há um que proíbe se alimentar após o meio-dia.} não
jantam, então foi tudo bem, não teve sujeira e bagunça. Bom. Foi bem em
todos os aspectos. 

O Buda gostava que mérito fosse feito de maneira simples. Não
precisa complicar, é simples por natureza, faça com simplicidade, faça
correto, faça com tranquilidade, não precisa se aborrecer. Ontem à
noite eu liderei centenas de pessoas, mas sinto que foi tranquilo, não
houve nenhum problema. Isso é fazer mérito de verdade e faz nos
alegrarmos. Mérito é alegria naquilo que fizemos: as coisas que fizemos
estão livres de maldade, as perversidades não poluem. Mesmo antes de
fazer há alegria, enquanto fazemos há alegria, quando terminamos de
fazer ainda temos alegria. Essa alegria é o que o Buda chamava de
mérito, o que nós encontramos frequentemente é a alegria em fazer
mérito: mérito é a alegria. 

Mérito hoje em dia\ldots{} Existem vários tipos de alegria. Às vezes
ficamos alegres com coisas erradas, por nos enganarmos. Ficamos alegres
por coisas que não sabemos, então ficamos alegres. Um exemplo: hoje as
pessoas de Korat vieram, até mesmo o Sr. Comandante. Quando entrou no
carro a carteira que ele havia posto no bolso caiu. O Comandante não
sabia, então veio com alegria, ele pensou que o dinheiro ainda estava
lá. Na verdade ele deveria já estar triste desde a hora em que a
carteira caiu. Nesse caso, ele está feliz por ignorância. Quando ele
chega ao Wat Pah Nanachat, ele olha o bolso e: “Oh! Me lasquei!” Se
assusta, resmunga, não é? Deveria se assustar desde a hora em que a
carteira caiu do bolso. Aquilo ainda não gerou efeito pois ainda não
sabemos. Às vezes, coisas que nós não sabemos trazem alegria, tem
alegria por não saber, por entender que nosso dinheiro ainda está ali e
então está alegre. Mas quando sabe a verdade, quando chega aqui, a
carteira sumiu e a mente fica perturbada. Essa é uma alegria
equivocada. 

Fazer mérito é a mesma coisa, o Buda ensinou a fazer para que haja
alegria. A mente é algo muito importante. Os sábios então ensinavam que
o coração do ensinamento do Buda não é nada de mais, tem dois ou três
itens apenas. Eles ensinaram: não fazer nenhum tipo de mal através de
corpo, fala e mente, isso é correto. “\emph{Etaṁ
buddhāna-sāsanaṁ}”, esse é o ensinamento do Buda. Esse é o coração
do ensinamento do Buda. Esse é um aspecto. O segundo é fazer que nossa
mente seja meritória, boa, isso também é “\emph{Etaṁ
buddhāna-sāsanam}”; esse é outro aspecto do coração do ensinamento
do Buda. “\emph{Sacitta-pariyodapanaṁ}”, purificar nossa mente é
“\emph{Etaṁ buddhāna-sāsanaṁ}”; esse é o ensinamento do Buda, é o
coração do ensinamento do Buda. Só três aspectos. Não são muitos, só
três frases. Temos que pegá-las e colocar dentro dos nossos corações.
Que esses três aspetos estejam em nossos corações. Nossos corações
serão o coração do ensinamento do Buda. Sentado será o ensinamento do
Buda, andando será o ensinamento do Buda, deitado será o ensinamento do
Buda. O tempo todo esses três aspectos. O Buda queria isso, é o que se
chama \emph{Buddha Sāsanā}. 

A ciência do budismo é a mais elevada de todas. Quer dizer, o saber
no budismo tem que estar livre de sujeira, fé cega, tem que estar
limpo. Enquanto fazemos algo nos sentimos bem, ao terminar nos sentimos
bem. Isso fica firme em nossos corações. Quando abandonamos o mal,
alcançamos o coração do ensinamento do Buda. Quando abandonamos o mal
nossa mente fica tranquila, não há confusão. Aquela é uma mente limpa,
pura, isso é mérito, bondade nascendo bem ali, é paz nascendo bem ali.
Pegamos o coração do ensinamento do Buda e sobrepomos com o nosso
coração. Em seguida nosso coração se purifica, pois não há nada, só há
coisas corretas. É o que se diz “a mente estar limpa.” Quando essas
três coisas se reúnem na mente de uma pessoa, ela terá alcançado o
ensinamento do Buda. Não esquenta, não se agita, isso é o que se chama
mérito, inteligência em fazer mérito. Quando faz mérito fica alegre com
inteligência, livre de sujeiras. Portanto, mérito deve ser feito com
sabedoria, inteligência. A inteligência faz ser limpo, sem sujeira. 

A maioria de nós budistas faz mérito só por fazer, mas não
alcançamos a fundação do ensinamento do Buda, fazemos mérito sem
inteligência. Tudo é assim. Portanto mérito não é muita coisa, mas é
algo correto. Abandonar o mal é que é importante, não é? Abandonar o
mal é mais importante. Abandonar o mal é limpar o local, como quando
vamos construir uma casa. Quando vamos construir uma casa, o terreno é
muito importante. Se a fundação for firme e duradoura, aquela casa vai
ser forte, firme e vai durar muito tempo, esse é um tipo de fundação.
Fazer mérito é a fundação da nossa mente, faz nossa mente se pacificar,
pois se livra de defeitos. Mesmo sentado é mérito, deitado é mérito, o
que quer que faça é mérito. Faz mérito surgir na mente. Mérito não tem
que surgir em outro lugar, tem que surgir na nossa mente por termos
tirado a sujeira para fora, o mal para fora, o que há de errado para
fora. É só isso, o mérito surge ali. 

Quando nós simplesmente fazemos mérito sem remover o mal, surge
confusão, bagunça, de várias formas. Eu já observei, a maioria dos que
vêm fazer oferendas vêm procurar mérito – vêm em vários automóveis –
mas, olhando com atenção, é verdade que vieram fazer mérito, mas não
abandonam o mal, está errado bem aí. Não alcançam a fundação do
ensinamento do Buda; quando fazem, não surge benefício no coração.
Quando não surge benefício no coração, não há sabedoria, nossa prática
está defeituosa. Que isso fique de lição de casa para todos vocês. Não
é longe, não é muito, só isso já é suficiente. Abandonar o mal é o
começo.

Portanto, quando formos fazer mérito, como hoje, o Comandante
levantou as mãos e disse “\emph{Mayaṁ banthe tisaranena saha pañca
sīlāni yācāma}”,\footnote{Versos recitados pelos leigos quando
pedindo os cinco preceitos aos monges.} o que é isso? Isso é tirar a
sujeira para fora do coração – não matar seres, não
roubar, não ter comportamento sexual impróprio, não mentir, não falar
mal dos outros, não beber. Tirando a sujeira é possível não ficar
limpo? A mente fica limpa, com certeza. Os antigos chamavam isso de
\emph{sīla}, colocavam bem ali. Limpar é a fundação, é
\emph{sīla}. Quando é \emph{sīla, samādhi} surge: paz nos
objetos mentais por estar livre de confusão. Quando nasce paz,
sabedoria vem junto. Quando sabedoria vem, entendemos todas as coisas
que surgem. A fundação do ensinamento do Buda é \emph{sīla,
samādhi} e\emph{ paññā}. Essas três coisas são a fundação, o
coração, é o que é importante no ensinamento do Buda. Quando essas três
coisas surgem, já é o caminho, é \emph{sīla, samādhi, paññā},
essas três coisas resumem o ensinamento do Buda. O ensinamento do Buda
se resume em \emph{sīla, samādhi} e\emph{ paññā.} Esse é o
caminho que nós temos que trilhar. 

Quando nós praticamos dessa forma, nossa sensação muda. Nós sabemos
o que é sofrimento, vemos sofrimento, sabemos a causa do sofrimento,
conhecemos o cessar do sofrimento, conhecemos o modo de prática para
cessar o sofrimento. Só isso, saímos por esse caminho. As pessoas
não-iluminadas, quer seja na época do Buda, quer seja nos dias de hoje,
têm que sair por esse caminho. Tem que conhecer o sofrimento.
Sofrimento vocês já devem ter encontrado. Outro dia eu conheci uma
pessoa importante que veio me visitar, foi só ir conversando e surgiu
sofrimento no coração. Ele estava sentado bem na minha frente e as
lágrimas começaram\ldots{} ping, ping, ping. Eu não sabia o que era. Não vi
nenhum machucado, não tinha nada, e as lágrimas correndo. Elas vêm do
sofrimento, o sofrimento as espreme para fora. Eu pensei: “É isso,
enquanto as lágrimas não acabarem, não acaba o sofrimento.” Sentado bem
ali e as lágrimas correndo desse jeito! Eu me perguntei: “Eh, essas
lágrimas saem de onde? Deve ter algo espremendo para elas saírem, as
lágrimas jorrando desse jeito\ldots{}” Eu não perguntei, mas vendo o evento
sabemos – se há lágrimas, tem que haver sofrimento.
Mesmo sendo uma pessoa importante, sofre por dentro, sofre na mente.
Tem dinheiro, bens, \emph{status}, mas ainda sofre. O sofrimento
surge baseado em quê? Baseado em insatisfação. Mesmo tendo coisas muito
boas, ainda não está satisfeito. Falta aquilo que quer
– então sofre. Então ele me perguntou: 

“Eu estou sofrendo, como posso resolver isso?” 

“Você sabe como é esse sofrimento? De onde ele nasce? Você sabe?
Você consegue traçar? Não tenho como saber em seu lugar, você tem que
saber sozinho. Antes de nascer o sofrimento, qual foi a causa? De onde
ele surgiu? Olhe na sua mente agora, você vai ver sozinho. Não precisa
ir procurar em outro lugar.” Eu levantei o
cuspidor\footnote{\thai{กระโถน} – Uma espécie de
cesto de lixo pequeno. Serve para cuspir, jogar lixo e, às vezes,
urinar.} e disse: 

“Desse jeito não é pesado, mas é só levantar que ele começa a ficar
pesado, é só largar que some o peso. Por que é assim? Se levantarmos o
cuspidor ele fica pesado porque fomos levantá-lo, então surge o peso. É
fácil de enxergar. Estando pesado, qual é o método para que o peso
desapareça? Experimente largar. Fica leve, não é? Por que fica leve?
Porque largamos o cuspidor.” Eu peguei o cuspidor de novo e expliquei
mais uma vez: “É assim, o sofrimento é assim. Não está em nenhum outro
lugar, não procure em outro lugar, o sofrimento está bem aqui. Pense
nisso.” Eu expliquei por um período curto, as lágrimas desapareceram e
ele voltou a conversar comigo com voz normal. “Caso no futuro o Sr. não
estiver bem e quiser consultar os monges, procure a causa até
encontrar. Procure até não haver sofrimento.”

Fazer mérito, ouvir o Darma, é para resolver o sofrimento, resolver
o problema. Nesta vida em que nascemos existe sofrimento, mas quando
surge o sofrimento, ficamos com medo. O Buda ensinou para quando o
sofrimento surgir, não ter medo dele, conhecer o sofrimento.
Normalmente vamos atrás do sofrimento, gostamos do que é sofrimento,
mas quando surge o sofrimento ficamos com medo! Na época do Buda era a
mesma coisa. As pessoas sofriam e então iam procurar o Buda. Quando
estão com calor vão para a sombra, se não estiverem com calor ficam
debaixo do sol, vão onde querem. Todo mundo é assim. 

Portanto sofrimento é \emph{Sacca Dhamma}. Todo mundo já passou
por sofrimento, mas não sabem como desfazer o sofrimento, não sabem de
onde ele surge. Então pensam, pensam, e no final choram
– não conhecem a saída. Eu digo que tem que
investigar. Tem que investigar, tem que haver um caminho para resolver
o problema, porque o Buda ensinou que o Darma desfaz o sofrimento das
pessoas. Resolve de verdade o sofrimento que nasce do coração das
pessoas, mas nós não sabemos quando o sofrimento surge. Temos que
traçar de onde é que esse sofrimento vem e descobrir a forma de
resolver. Então ele declarou sofrimento sendo \emph{Sacca Dhamma} –
\emph{Dukkha Sacca}, \emph{Samudaya Sacca, Magga Sacca, Nirodha
Sacca}\footnote{A Nobre Verdade do Sofrimento, a Nobre Verdade da Causa
do Sofrimento, a Nobre Verdade da Cessação do Sofrimento e a Nobre
Verdade do Caminho que Leva à Cessação do Sofrimento.} O Buda saiu por
esse caminho, todos os \emph{sāvakas} saíram por esse caminho,
todos as pessoas têm que sair por esse caminho. 

Isso é uma Nobre Verdade; portanto, quando surge o sofrimento, nós
temos que refletir para entender com inteligência: “isso é só
sofrimento.” Na verdade o sofrimento é contínuo, mas não estamos
cientes disso. Onde estamos agora é só sofrimento. Quando surge o
sofrimento, chamamos de sofrimento; quando o sofrimento desaparece,
chamamos de felicidade. Não enxergamos claramente, pensamos que é
felicidade só porque o sofrimento desapareceu, mas quando desaparece,
reaparece em seguida e sofremos. Sofrimento
surge e desaparece e então é felicidade e agarramos. Agarramos o
sofrimento, agarramos a felicidade, pensamos que são coisas distintas.
O Buda ensinava: só há sofrimento surgindo e desaparecendo, não há nada
além isso. 

Pode investigar, vai que ver que é assim. Sofrimento surge e
sofrimento desaparece. Se nasce, pensa: “Isto é sofrimento;” se
desaparece, pensa: “Isto é felicidade,” mas na verdade é só sofrimento
nascendo e desaparecendo e nós agarrando aqui e ali. Só agarramos
sofrimento, nada mais. Mas nós dizemos que são vários: às vezes é como
felicidade, às vezes é como sofrimento. Nós desejamos felicidade. Tendo
desejado felicidade, quando ela desaparece, o sofrimento surge
novamente. 

A fundação do ensinamento do Buda é a paz. Estar livre dessa
felicidade, desse sofrimento, isso é a paz. Desejamos felicidade, mas
felicidade é um sofrimento refinado que nós ainda não enxergamos. Quer
seja felicidade ou sofrimento, ambos\ldots{} Pensamos: “Eu gosto de
felicidade, quero felicidade, preciso procurar felicidade.” Pensamos
que felicidade não tem um lado ruim, mas não é assim. Felicidade é a
número um, ela mesma já é sofrimento. Quer seja felicidade ou
sofrimento, são como uma mesma cobra. Sofrimento é como a cabeça da
cobra, felicidade é o rabo da cobra. Vemos que essa cobra é longa, o
rabo está aqui, a boca lá longe e pensamos: “Eh, a boca é perigosa, o
rabo não tem perigo, não tem boca desse lado. Não vou chegar perto da
boca senão vou tomar uma mordida, vou tomar uma picada! É melhor pegar
pelo rabo porque a boca está daquele lado.” Mas, quando pegamos, o rabo
ainda é rabo de cobra, não é? A cobra se vira e vem morder, é
sofrimento do mesmo jeito, porque a cabeça da cobra está na cobra, o
rabo da cobra está na mesma cobra. Mas nós não sabemos disso, não
sabemos que felicidade é assim, sofrimento é assim. Na verdade eles
estão na mesma cobra. Nós só construímos sofrimento, só isso. 

Por isso o Buda ensinava: não há muito mais além de sofrimento
surgindo e sofrimento desaparecendo, só isso. Não há nada além disso.
Se procuramos a felicidade é por pensarmos que felicidade é bom, é bom
demais, pois então gera ainda mais sofrimento, porque não vemos que é
sofrimento. Quando se volta e olha o Darma do Buda: “Ah, essa
felicidade não é paz.” Felicidade e sofrimento, se olhar, são iguais,
são o mesmo, mas em uma ocasião se chama felicidade e em outra se chama
sofrimento. Portanto, caso deseje felicidade, quando esta desaparece,
surge sofrimento. Eles são o mesmo dessa forma. Então o Buda ensinou
para termos paz. Tendo paz, quando o sofrimento surge, sabemos: “Eh,
não é isso, larga!” Entendemos o que é.
Quando a felicidade surge é a mesma coisa: não se apegue, é apenas algo
comum. Temos que construir bastante esse carma. Construir, construir
até ficar grande e alto, alto até ultrapassar essa felicidade e
sofrimento. É o que se chama paz. 

Quando a paz já se desvinculou da felicidade e do sofrimento, essa é
a fundação do ensinamento do Buda de verdade. Que isso fique de lição
de casa para todos vocês. Vá refletir se é verdade ou não, porque neste
Darma o Buda não elogiou as pessoas que acreditam nos outros, ele não
elogiou. Ele elogiou ser \emph{paccattaṁ}: acreditar com sabedoria que
nasce em nossa própria mente, por isso ele dizia \emph{paccattaṁ}
– saber por si mesmo. A maioria das pessoas vão
ouvir os monges ensinarem ou vêm me ouvir. Algumas pessoas devem
acreditar em tudo, mas eu peço licença para proibir: não acredite em
tudo, não desacredite, não acredite nem desacredite, deixe de lado até
que sabedoria surja. Em uma ocasião, Sāriputta estava ouvindo o
Darma. O Buda foi explicando e num certo momento ele perguntou:
“Sāriputta, você acredita em mim?” Sāriputta respondeu direto: “Eu
ainda não acredito, Senhor.” O Buda disse “Muito bem Sāriputta, os
sábios não acreditam facilmente. Eles investigam antes de acreditar.”
Não é? Mostra que devemos ver por nós mesmos, o Buda deu permissão. Se
fosse hoje em dia e alguém dissesse: “Eu não acredito,” o
professor deve ficar com raiva, não? Se a mãe perguntar:
“Filho, você acredita em mim?” e ele responder: “Não acredito,” será
que a mãe fica com raiva? Se o pai perguntar: “Filho, você acredita em
mim?” e ele disser: “Não acredito,” será que o pai fica com raiva? Isso
o Buda falou com cordialidade: “Você acredita, Sāriputta?”, e
Sāriputta respondeu: “Ainda não acredito.” Ele concordou: “Muito bem
Sāriputta, um sábio não deve acreditar facilmente, ele investiga com
razão antes de acreditar.” É assim, ele queria que nós conhecêssemos
nossa própria mente, portanto tenham interesse, tenham cuidado. 

Todas as coisas que temos em nós, olhos, ouvidos, nariz, língua,
corpo, estão sob a mente. Por isso o Buda ensinou que essa mente é algo
importante. Tenham cuidado com suas mentes. Você é bom ou ruim por
causa da sua mente, portanto tenha cuidado com a sua mente. Essa mente
é mais venenosa do que todas as cobras venenosas. Mas se o veneno se
acabar, ela se torna mais pacífica do que qualquer outra coisa.
Portanto essa mente é algo muito importante. Quando falamos, a mente
chega primeiro. Quando fazemos algo, a mente chega primeiro. Qualquer
coisa, a mente chega primeiro. Portanto o Buda declarou que o assunto
do ensinamento dele é a mente, é sobre a mente, é o ensinamento da
mente. 

Eu vou contar, eu tenho um discípulo estrangeiro, ele formou-se em
psicologia.\footnote{Em tailandês, a palavra “psicologia” se traduz
“ciência da mente”
(\thai{จิตศาสตร์}). Portanto a
ironia: uma pessoa expert em “ciência da mente” ainda sofre!}
Formou-se em psicologia e veio para a Tailândia, ordenou-se comigo em
Wat Pah Pong. É psicólogo, tem conhecimento, passou no exame, formou-se
em psicologia. Mas\ldots{} sofre o tempo todo. O salva-vidas morreu afogado! A
pessoa estudou sobre a mente, se formou em psicologia, aprendeu apenas
a teoria, mas não teve efeito na mente. Ainda não aprendeu, ou não
sofreria o tempo todo. Tendo se formado em psicologia, deveria conhecer
sua própria mente. Esse é um exemplo: a mente foi criando problemas
constantemente até que desistiu da vida monástica. Isso é se formar em
psicologia de acordo com a teoria: o livro é de um jeito, os textos são
de um jeito, a realidade é de outro. Portanto o Buda ensinou que
aprender o Darma é bom, mas não é o mais elevado. Conhecer o Darma é
bom, mas não é o mais elevado. Praticar o Darma é bom, mas não é o mais
elevado. Ver o Darma é bom, mas não é o mais elevado
– o Buda jogou tudo fora. O que é que faz ser o mais
elevado? Ser Darma. Se for Darma, a mente é Darma, todos os Darmas se
reúnem na mente. \emph{Eko Dhammo}, só existe um Darma: a mente.
Portanto a fundação do ensinamento do Buda é a mente. É o que se chama
\emph{Buddha Sāsanā}. Por isso eu sou capaz de dizer que a ciência
do Buda é a ciência mais elevada, é a ciência que faz as demais
ciências se completarem em seguida. Possui tanto conhecimento quanto
bondade, possui sabedoria. As demais ciências possuem conhecimento, mas
não possuem bondade, é essa a razão. 

Por isso peço a todos que vieram fazer mérito aqui hoje: estejam
cientes de que a mente é a causa. Tenham cuidado com suas mentes,
estejam cientes. Por exemplo: neste momento estamos felizes. Quem nos
fez ficar feliz? As orelhas ou os olhos? Pode olhar: foi a mente. Hoje
estamos com raiva, por quê? Vocês sabem? Os olhos estão com raiva ou as
orelhas? Pode olhar: o coração não está muito bem, por quê? Por causa
da mente. Tal como o Buda ensinou a ensinar a mente, quando tivermos
ensinado nossa mente e ela tiver aprendido, a felicidade nos seguirá.
Olhos, ouvidos, nariz, língua e corpo são secundários; a mente é o que
é importante. Portanto peço que todos vocês que vieram aqui estejam
atentos: Cuidado com a mente. Esta mente é capaz de gerar os piores
males, esta mente é capaz de gerar os maiores benefícios. Se agirmos de
forma correta, gera muito benefício. 

Mente todo mundo deve ter, não é? Mas ela não tem forma. As pessoas
sabem o que é a mente? Todos vocês devem ter trazido suas mentes aqui
hoje, mas se perguntar onde está a mente, fica difícil falar. Olhe por
este ângulo: quem é que prestou atenção no ensinamento que dei hoje?
Quem é que prestou atenção? Quem é essa pessoa que prestou atenção,
onde ela está? Saiba isso, você vai entender que aquilo que prestou
atenção é nossa mente. É o que nos faz correr para cima e para baixo,
nos faz bons ou ruins, nos faz felizes ou tristes, é por causa dela.
Portanto vocês cuidem disso para que possam ver claramente o que se
chama “o coração do ensinamento do Buda.” Se nós entendermos isso, tudo
fica mais leve.

Mas nós aqui, bastar ensinar para respeitar \emph{sīla} e já é
difícil. “Não agridam uns aos outros, não tenham inveja,” só isso e
ninguém consegue enxergar. Se for para ter cuidado com a mente, é ainda
mais difícil. O começo é \emph{sīla}: que nós tenhamos
\emph{sīla}! Como este gravador aqui, o Sr. X deseja muito ter
esse gravador, mas pertence a outra pessoa, então ele devolve, tem medo
que roubar seja maldade – só isso já é suficiente.
Não tem coragem de levar, não tem coragem de roubar
– só isso já é \emph{Sīla-Dhamma}. Não precisa
roubar, pode usar o gravador à vontade. Esse é um exemplo e há outros.
Quem ver isso já viu o suficiente: tem que abandonar, tem que saber
aguentar. 

“Praticar” é praticar o Darma. “Aguentar” é querer fazer o mal, mas
tentar ensinar a mente a não fazer. Isso é prática. Pratique
continuamente até a mente ficar hábil, até que ela conheça certo e
errado, maldade e bondade, benefício e malefício. Eu ensino as pessoas
de um jeito fácil, mas é um pouco difícil. Como é que ensino? Aonde
quer que eu vá eu escuto: “Não sei nada, não consigo aprender, não sei
estudar.” Eu digo: “Não precisa, pode parar. Quando for praticar repita
este mantra fácil para conseguir enxergar: ‘Faz o bem, recebe o bem.
Faz o mal, recebe o mal,’ antes de ir dormir. ‘Faz o bem, recebe o bem.
Faz o mal, recebe o mal’ e foque na respiração. Esteja ciente e repita
‘Faz o bem, recebe o bem. Faz o mal, recebe o mal.’” Até a consciência
levar o bem e o mal para dentro de si, até nossa mente se purificar,
entender “bem” e “mal.” Quando ela entender o mal, ela abandonará o
mal. Quando ela entender o bem, ela praticará o bem. Se for mal ela não
pratica, ela abandona. Constrói o bem naquele momento. A nossa mente
aceita, entende o bem e o mal. 

Na maioria dos casos, os leigos não diferenciam bem e mal. Às vezes
é algo ruim e vão fazer aquilo. Faz com que sofram, chorem, mas ainda
não entendem que aquilo é ruim, não diferenciam bem e mal. Tem que
diferenciar ambos. Quando usamos isso como mantra, entra na corrente da
nossa mente e separamos o bem do mal, entendemos a diferença entre
mérito e demérito, começamos a ser uma pessoa fácil de ensinar.
Pratique com a sua mente. Quando for praticar repita: “Faz o bem,
recebe o bem. Faz o mal, recebe o mal.” Se
conhecemos bem e mal, agimos de acordo, construímos bondade,
abandonamos maldade sozinhos, no final alcançamos a paz. Não deixe a
mente se apegar à bondade ou à maldade, pois caso nos apeguemos por um
longo período, nos apegarmos e não largarmos, não afrouxarmos\ldots{} Por
exemplo: “Só eu estou certo, todo mundo está errado!”, esse pensamento
se apoia na bondade e essa bondade, se nos apegarmos e não largarmos,
pode gerar maldade. Certeza! Pode nascer maldade. Morre por causa da
bondade, não sabe usar a bondade, não sabe moderar. Constrói bondade e
pensa: “Agora ficou bom!” e aí não larga, se apega com firmeza sem
largar. Isso também gera problemas. Em última instância o Buda ensinou
a largar, a conhecer o bem, conhecer o mal. Se conhece, o bem não gera
problemas, o mal não gera problemas, pois nós não nos apegamos. O Buda
ensinava desse jeito. 

De qualquer forma, resumindo, mente é algo muito importante. Possui
muitos malefícios, possui muito benefícios. Choramos por causa da nossa
mente, rimos por causa da nossa mente, ficamos tristes por causa da
nossa mente, ficamos felizes por causa da nossa mente. Se vocês ainda
não veem claro, olhem. Se surgir raiva, olhe na mente: “Quem está com
raiva? Os ouvidos estão com raiva, ou os olhos?” Olhe! Vocês vão saber
de onde nasce a causa. Olhem. Quando ficarem felizes, isso vem dos
ouvidos ou dos olhos? Ou vem da mente? Se vocês acompanharem desse
jeito, estudarem desse jeito, vocês vão conhecer a causa. Se conhecerem
a causa, ela desaparece bem ali onde ela nasce. Não é? Ela desaparece
bem ali. Isso é o que se chama ser uma pessoa que pratica o Darma no
ponto correto. Tudo no mundo vai mudar, vai ver o mundo como
\emph{Lokavidū}. Se enxergamos como \emph{Lokavidū} vemos que
as coisas sempre foram assim, mas somos nós que queremos que sejam de
outro jeito, que não sejam assim, e aí nasce insatisfação. O Buda
nasceu neste mundo e tomou este mundo como objeto de contemplação. Ele
viu nirvana, a saída do sofrimento, neste mesmo mundo. Por isso é
\emph{Lokavidū}: conhece claramente o mundo. Nós não conhecemos
claramente o mundo, nosso mundo tem escuridão. Que vocês reflitam sobre
isso. Que isso fique de lição de casa para vocês. Hoje eu dei
conhecimento, opinião, o suficiente. Por isso peço licença para
encerrar a palestra aqui. 

