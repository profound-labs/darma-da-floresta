
\chapterAuthor{Ajahn Liem}
\chapterNote{Ajahn Liem fala à comunidade leiga de Wat Pah Pong.}
\chapter{Gratidão à natureza}
\tocChapterNote{Ajahn Liem fala à comunidade leiga de Wat Pah Pong.}

Saudações a todos. Hoje é um dia em que temos um dever. É um dia
precioso e esse dever precioso é um adorno para nós. Quando chega um
dia como este, damos importância a reforçar nossas qualidades e
\textit{pāramīs}, o que há de belo e bom, pois entendemos que é
uma parte do que dá valor à vida. Se temos qualidades e bondade, o
valor da nossa vida fará nascer felicidade e progresso. O dia de ouvir
o Darma, o dia do \textit{uposatha}, faz parte das atividades que
devemos realizar. É um dever para nós, seres humanos. Temos que dar
importância a reforçar isso, nos encontrando para tomarmos parte nessas
atividades. Essas atividades que realizamos nos ajudam a ganhar uma boa
disposição, uma mente humilde, a ter o que se chama “algo belo”. Para
obter isso, utilizamos o ato de dar importância. No começo utilizamos
dar importância a essas atividades para criar a fundação. Se
construirmos a fundação, nossa personalidade naturalmente seguirá
naquela direção. 

O dia de ir ao monastério praticar o Darma é condizente com nossa
intenção de reforçar nossas boas qualidades; portanto, viemos dar
importância a isso. Hoje é dia três de agosto e, no budismo, esse é um
dia em que fazemos mérito.\footnote{Naquela ocasião, o dia de três de
agosto era o dia do uposatha.} É o dia da Lua Nova do oitavo mês,
podemos dizer que é o dia de realizarmos nosso dever. É o dia três de
agosto e em agosto tomamos parte em expressar respeito, na forma de
pessoas que possuem gratidão.\footnote{Em agosto se celebra o dia das
mães na Tailândia.} Todos temos que pensar em nossa dívida de
gratidão, temos que ter gratidão. Se não soubermos ter gratidão, é
possível irmos na direção de não mais nos desenvolvermos, porque não
ter gratidão é o caminho da degradação, é o caminho da ruína. Não se
encontra desenvolvimento ou, se quiser, pode-se dizer que não há
desenvolvimento algum. O Buda então comparava, dava o símile na forma
de linguagem, dizia que é um caminho que temos que abandonar, não dar
importância. Nos lembramos da dívida de gratidão para quem? É bom
pensar com gratidão ao pai e à mãe, não é ruim, pois são aqueles que
cumpriram o dever de nos dar suporte. Nos deram nascimento, cuidaram de
nós, nos sustentaram para que pudéssemos nos desenvolver tanto
fisicamente como intelectualmente. Tudo isso graças àqueles que
cumpriram seu dever com bondade e compaixão, que tiveram bem-querer.
Tal qual dizemos que bem-querer é o Darma que sustenta o mundo. Devemos
dar importância a isso. 

Outro item é ter muita gratidão, gratidão completa. E aquilo a que
devemos ter gratidão completa são aos recursos da natureza. Os recursos
naturais geram a oportunidade para nascermos. Não podemos dizer que não
temos envolvimento – temos que nos envolver com os recursos naturais. O
que são os recursos naturais? Podemos entender que a natureza faz parte
das coisas com as quais temos que nos envolver sempre, o tempo todo.
Acho que somos capazes de saber: não está além da nossa capacidade de
pensar e entender com o que estamos envolvidos o tempo todo. É aquilo
que recebe e sustenta diversas coisas, ou seja, o chão, a terra. A
terra é parte dos recursos naturais, temos que ter gratidão a ela,
reconhecer o valor dela. Se reconhecemos seu valor, nós nos adaptamos,
mudamos. Mudamos qualquer comportamento que seja ruim, porque
normalmente já sabemos que ficar de pé, andar, sentar, deitar, tudo,
até mesmo o realizar do nosso ofício, é possível graças à terra, ao
chão, àquilo que chamamos de “mãe”, “mãe terra”. Provavelmente todos já
vimos na forma de imagem na cidade de Warin, onde eles fizeram uma
imagem representando a mãe terra em que ela espreme seu cabelo para que
a água saia.\footnote{Diz a história que, na noite em que o bodisatva
alcançou a iluminação, ele invocou a “mãe terra” para dar testemunho
dos longos anos que passou desenvolvendo os dez pāramīs. Ela,
então, personificada na forma de uma mulher, espremeu a água de seu
longo cabelo, e essa água se transformou em uma grande enxurrada que
arrastou Māra e seu exército, que tentavam obstruir o bodisatva.}
Fizeram uma representação e podemos ver. 

Então isso é algo que devemos entender. Quando entendermos, vamos
prestar reverência, vamos agir de forma a não causar degradação, para
que não desapareça. Se houver degradação haverá consequências, haverá
consequências que podem causar o desaparecimento da nossa vida e também
da espécie humana. Temos que entender assim. Quando entendemos assim,
temos que saber como devemos agir para surgir o sentimento de querer
evitar causar degradação ou destruição. O mesmo que fazemos em relação
a nossos pais: nós os veneramos, honramos nossos pais praticando o que
se chama “bom comportamento”, quer dizer, agir de forma a não fazer o
pai e a mãe sentir preocupação e apreensão. No mínimo temos que nos
comportar de forma a não criar problemas. Se nossos atos criam
problemas, os pais se sentem mal, faz os pais sentirem mal-estar; no
mínimo, sofrem. Por exemplo: se nosso filho ou filha tiver problemas,
fizer coisas que a sociedade não aceita e isso virar um assunto,
disputa na sociedade, a ponto de haver brigas e ir parar num tribunal,
os pais devem ficar incomodados, não têm felicidade, não dormem
tranquilos, não sentam tranquilos, não têm tranquilidade quando estão
de pé. Por que isso? Porque aquele filho não dá importância àqueles que
cumpriram o dever de criá-los. 

Isso é a mesma coisa: nós dependemos dos recursos do mundo, que é o
berço, é onde nascem todas as coisas que têm vida. As coisas vivas e as
coisas que criam condições para que haja vida dependem do mundo,
dependem da plenitude do mundo. Portanto, no que diz respeito à
plenitude do mundo, temos que ser sensíveis e agir com sabedoria e
compreensão. Se nosso saber e compreensão forem completos, agiremos de
forma a nos harmonizarmos. Provavelmente todos já ouvimos falar de
qualidades que entendemos serem aquelas que fazem surgir felicidade por
toda parte. Isso existe no modo correto de agir, conduzir a si,
conduzir a si de forma a ter bem-querer, desejar que haja felicidade
por toda parte. Temos que evitar agir de forma a fazer surgir disputas.
Disputas fazem surgir mal-estar, e mal-estar faz surgir destruição.
Isso é algo sobre o qual devemos refletir. Portanto, nosso
comportamento, especialmente no nosso país, pode-se dizer na nossa
terra, na nossa sociedade… Já estudamos e sabemos isto: conduzir a vida
depende de ações, boas ações, o que chamamos de \textit{supatipanno –
}pessoa que age bem. Ao conduzir a vida de modo a obter os requisitos
básicos para sobreviver, peço que ajam com boas ações. Pensem assim,
pensem de maneira ampla, pensamento do tipo sem barreiras, sem limites,
que é um aspecto do \textit{Appamaññā} \textit{Dhamma},\footnote{“As
quatro qualidades imensuráveis”, ou seja: bem-querer (mettā), alegria
pelo bem-estar alheio (muditā), compaixão (karunā) e equanimidade
(upekkhā).} que é a fundação para viver como um \textit{aryia}. Os
\textit{aryia}s vivem com \textit{Appamaññā} \textit{Dhamma}. 

Todos devemos ser capazes de entender que somos parte da mesma
família. Não é de outra forma. Somos uma espécie e temos um vínculo de
acordo com a verdade da natureza. Não diferimos em absolutamente nada.
Mesmo nossos sentimentos são iguais. Esse sentimento todos conhecem:
todos queremos a felicidade, mas o sofrimento não aceitamos, recusamos.
É assim: não importa em que situação, é sempre igual. Dizemos que
queremos o que é bom. O que não é bom – recusamos. Se vemos assim, se
nos sentimos assim, será característica nossa que sempre que fizermos
algo, terá que ser útil. Ser útil para si e para os demais também. Se
for útil para si e para os demais, trará união. Irá se expressar em
cuidar um do outro, morar juntos ajudando uns aos outros. Isso é algo
necessário. Não é possível evitar. Temos que contar uns com os outros.
O mesmo com relação a nós que estamos vinculados aos recursos naturais,
temos que cuidar. Se somos praticantes do Darma, temos que dar
importância a isso. 

Hoje em dia todos sabemos que a evolução natural, isso quase não
falam mas na verdade ela tem a ver com nossa vida, tem a ver com nossa
prática. Por isso, tem que se dizer e avisar que essa evolução tem
ambos, lado positivo e negativo. O mesmo ocorre com as mudanças na
nossa sociedade. Atualmente elas indicam a aproximação de um colapso,
aproximação do desaparecimento desta sociedade. O que significa colapso
e desaparecimento? Significa o extermínio. Em poucas palavras,
extermínio. A sociedade está de uma tal forma que não gera uma sensação
boa. Temos que pensar, investigar: de onde vem isso? Deve vir de algo
que tem uma causa. Causa do quê? Causa da falta de plenitude da
natureza. Deve ser assim. Quando é assim temos que considerar,
refletir. Eu, por isso, costumo ensinar que nós devemos viver sabendo
ter moderação. Em pāli se chama \textit{bhojane mattaññutā.} Ter
moderação no consumo tem a ver com viver com equilíbrio, com beleza.
Equilíbrio e beleza não têm o que adicionar ou remover. 

É como nossa mente; se for comparar, é como nossa mente. Se não
tivermos a sensação de deleite ou aversão, como fica? Se olharmos,
vamos ver que se não tivermos deleite ou aversão, ou seja, nem preto,
nem branco – se disser que é preto, não é; se disser que é branco, está
errado. É um estado que tem a característica de ser Darma, faz com que
não sinta nem felicidade nem sofrimento. Se não tiver felicidade nem
sofrimento, o que é que há? Há frescor. Há frescor, há bem-estar.
Frescor é bem-estar, no bem-estar há felicidade. O Buda disse que
felicidade maior do que essa não existe, “\textit{natthi santi param
sukham}” – felicidade maior que paz não existe. Paz exterior e
paz mental que fazem parte do Darma, fazem surgir frescor. O modo de
ser que faz surgir aquilo que chamamos de Darma. Darma de quem?
Daqueles que têm bem-querer pelo mundo inteiro, que chamamos de
\textit{aryia}s. Devemos pensar sobre isso. 

Portanto, se queremos nos desenvolver
para que haja plenitude, para que não causemos destruição, devemos
procurar um método, ou reajustar nossas ações. Simplificando: se antes
bebia, para de beber; se antes fumava, não precisa mais fumar; antes
passeava à noite, para de passear,\footnote{No Sigalovada Sutta (Digha
Nikaya, 31), o Buda cita o ato de passear à noite como um mau hábito
que os leigos devem evitar (vale lembrar que antigamente não havia nem
automóveis nem iluminação elétrica nas ruas, e passear à noite era algo
perigoso, que normalmente não era feito por “pessoas de bem”).} e
assim por diante. Como é o resultado? Conseguem ver? Se parar de beber,
na sociedade em geral e dentro da sua família, não haverá
mal-entendidos e confusão, quer dizer, não vai gerar brigas e
desavenças, pois sabemos que a bebida faz o sistema nervoso da pessoa
tornar-se um sistema nervoso de monstros e demônios, de pessoas ruins e
de seres do inferno. E é algo que é fácil de ver, não é difícil de
observar. É fácil de ver esse fenômeno. É algo
gráfico o suficiente para podermos notar. É
algo que conseguimos ver. E parar de fumar – fumar não é algo bom,
queima o dinheiro que ganhamos usando energia e inteligência, aumenta a
quantidade de coisas supérfluas de forma negativa. Temos que
simplesmente abandonar esse hábito, não tem nada de
mais, e ainda traz benefícios para a saúde.
Isso se manifesta dessa forma. Por exemplo, não passear à noite, pois
nossa sociedade é uma sociedade de parceria, sociedade de grupo; se não
passeamos à noite, falando de forma simples, nossa família se sente
acalentada. Nossa sociedade não vai nos olhar com maus olhos. Se
passeamos à noite, nos olham com maus olhos, como uma pessoa ruim,
alguém em que não se pode confiar; não faz surgir a sensação de
confiança. Isso é algo que podemos ver por
nós mesmos. O Buda tinha boa intenção dessa
forma, ele então recomendou um modo de vida que gerasse a sensação de
inocência e segurança, que trouxesse a sensação do Darma para que
tenhamos bem-estar. 

Bem-estar nasce do quê? Nasce de não ter doenças. Nosso corpo é uma
das ferramentas para vivermos. Se essa ferramenta não está em mau
estado nem está danificada, ela não nos gera preocupações. Não ter
preocupações ajuda a termos uma vida que não gera sofrimento. Surge
assim. Pensem nisso, reflitam sobre isso, e conduzam suas ações de
forma a gerar plenitude no nosso viver juntos na sociedade. Quando a
sociedade vive tendo Darma, ela nos traz felicidade. Como eu gosto de
falar com frequência: “Esteja bem aonde quer que vá, tenha Darma sempre
em sua mente.” Darma é a bondade, é o que é correto, bom e belo. Se
nossa mente tem Darma, nosso coração tem Darma, nossas atividades serão
compostas de Darma. Tudo que fizermos será construtivo, fará haver a
sensação de poder contar uns com os outros. Isso é semelhante.
Portanto, devemos praticar usando nosso modo de vida no nosso país, na
nossa sociedade. É verdade que temos essa ou aquela profissão, mas
temos que ter profissão deste tipo: profissão do tipo que traga
felicidade. Essa é nossa profissão básica, que requer nossa energia,
energia do corpo, energia em aguentar as dificuldades, a qualidade da
resiliência – temos que tê-la. Se tivermos essas coisas, elas vêm à
tona. Portanto, em nossas diversas atividades devemos agir de forma a
não gerar disputas e causar problemas. 

A saúde do corpo. Não ter doenças não vem de outro lugar: vem da
alimentação, nossa alimentação, nosso sistema digestivo. Quando comemos
temos que saber: o que há naquela comida que estamos consumindo, trará
malefício ou benefício? Pensem assim. Não é o caso que o Buda não tenha
ensinado sobre isso: ele ensinou, mas nós não nos interessamos em
aprender, então ficamos desse jeito. Portanto, as atividades que
produzem os alimentos – eles vêm da atividade que chamamos de “sol nas
costas, enxada no chão”. É uma atividade que requer esforço,
resiliência, e é necessário que tenhamos parte nessa atividade. Se
pensarmos assim… Quando vejo aqueles que fazem esse trabalho, penso que
temos vida graças a eles; eles se esforçam a ponto de podermos dizer
que devemos muita gratidão a eles. Não olho por outro ângulo,
olho como pessoas que fazem algo muito útil
para que possamos ter vida – é graças a eles. Não devemos
menosprezar essa atividade: em outras
palavras, não devemos ridicularizar, não devemos oprimir, não devemos
pensar que são escravos. Antigamente diziam: “É jeca, é uma pessoa que
não merece respeito.” Mas quando eu olho, vejo uma pessoa de muita
dignidade, muita honra; é uma pessoa de muita beleza e bondade. Se
olharmos assim, sentimos felicidade. Ficamos felizes, felizes em vermos
uma pessoa desse tipo. Quando trabalham é como se nos ensinassem, é
como se fossem nossos professores, aqueles que nos
trazem luz; podemos dizer que fazem e servem
de exemplo para que todos vejam. Isso é uma coisa boa. 

Nós aqui temos que pensar e refletir para dar luz ao Darma. Se
pensarmos e refletirmos para dar luz ao Darma, nossa mente não
conhecerá rancor. Rancor é a mente que ainda não está correta. Temos
que estar atentos a isso. Portanto, esse tipo de coisa hoje em dia
ocorre por se deixar levar pela onda. Caso
se deixe levar pela onda, só há perdas, pois a maioria das pessoas não
usam o cérebro, não utilizam sua própria capacidade. Dependem da
capacidade de pensar dos outros, então em tudo que fazem dependem de
algo. É verdade que dependemos do clima. Na época de chuvas, plantamos;
se não chover não podemos plantar, dependemos disso. Isso é depender da
natureza, ela cria as condições. Temos que pensar nisso. Se
a utilizarmos sem haver paz dentro de nós,
talvez não surja uma intenção bondosa. Temos que treinar nosso cérebro,
treinar o pensamento para enxergar por todos os ângulos. O
mesmo para quando fazemos algo, temos que
fazer de forma plena, fazer de forma útil. Isso é parecido. 

Portanto, podemos observar as diferentes formas de agir, podemos
observar em nossa sociedade. Em alguns lugares, todos os dias, o tempo
todo, vêm pessoas me dizer que hoje em dia o modo de agir para ganhar a
vida faz o tempo de vida de nós, seres humanos, diminuir,
rebaixar-se, encurtar-se. Em tudo é assim. Vocês já devem ter
ouvido falar que antigamente o ato de dar
continuidade à raça humana, o ato da procriação, requeria no mínimo 25
anos de idade. Antigamente era assim, não faz muito tempo – por volta
de sessenta anos atrás. Víamos assim. Mas hoje em dia não é assim. Por
quê? Não sei o que acontece, eu também penso: “O que será isso? O que
está acontecendo na sociedade, no modo de vida de nós, seres humanos?”
Crianças com apenas oito anos! Olhem o que
vem acontecendo. Me faz sentir que nosso modo de agir e de viver está
errado, distorcido. Não está dentro de um sistema alinhado à natureza.
Então esse tipo de coisa ocorre: obsessão sexual, doenças de todo tipo
vêm surgindo. Temos que pensar que dependemos uns dos outros, tudo tem
conexão com nossa vida. 

Quando esse tipo de acontecimento surge, temos que procurar um
método para mudar nosso modo de vida, olhar sob um ponto de vista que
nos leve a entender a natureza. A natureza é pura, como água limpa,
água não contaminada. Como uma região que não oferece perigos, por
exemplo. Dessa forma não haverá doenças, e o sistema que
sustenta a vida se adapta e volta a um
estado apropriado. As doenças diminuem, aquelas que ainda persistem são
as doenças da natureza, como o Buda disse: “Nascer é um tipo de doença,
envelhecer é um tipo de doença, adoecer e morrer
são um tipo de doença.” É uma doença
natural, não é uma doença que gera desarmonia. Se pensarmos dessa
forma, devemos mudar nossa forma de agir, mudar pensando em como
devemos fazer para conseguir pureza, para que haja benefício, não haja
malefício ou perigo. 

Por exemplo, nosso modo de ganhar a vida. Podemos ganhar a vida com
atividades básicas como antigamente, do tipo que antes as pessoas davam
muito valor. Antigamente, mesmo as famílias que conhecemos através dos
livros de história, a família do Siddhattha,\footnote{Siddattha Gotama:
o Buda.} a família do Buda, quando davam nome às pessoas, davam nomes
relacionados a alimentos, cereais, comida. Chamavam-se Sukkodana,
Suddhodana, Amitodana, Dhotodana. Que espécie de nomes são esses? São
nomes de coisas que têm vínculo com o corpo, o que chamamos de
\textit{bhojana} ou \textit{bhojanam} – comida, alimentos. Se traduzir,
chamam-se “alimentos que trazem benefícios
ao corpo”. É assim. Isso é para sabermos que eles expressavam honra
dessa forma. 

Se desejamos isso, temos que agir, temos que praticar sabendo como
manipular os recursos da natureza sem gerar desarmonia. A água não vira
veneno, o solo não vira uma ameaça, não tem substâncias que causam
doenças, perigos à saúde. Hoje em dia não é assim. Por exemplo, se
formos para os lados do distrito Kantararom, distrito Yang Som Nói,
dizem que é uma área contaminada por produtos químicos, substâncias
venenosas à saúde do corpo. Estamos brincando com fogo… Ontem fui ao
distrito Nam Yu, fui a Kantalak e os vi usando aquilo. Ôôô… que fazer?
É difícil as pessoas mudarem de comportamento. Quando olho, vejo que
não tem jeito, eles não têm como evitar, então agem daquele jeito.
Quando fazem, sentem que aquilo não é seguro. Se aquilo acumula no
agregado corporal, surgem anormalidades, vai ao hospital e volta com
uma doença grave, fica desse jeito. Vi quem? Vi um leigo, vi um leigo
que conhecia, conversamos e perguntei: “Por que faz assim?”; “Não tem
importância.”, ele disse. Ele pulverizava inseticida para repelir
insetos, matar ervas daninhas, esse tipo de coisa. Perguntei “Por que
faz assim?” Ele disse que não tem escolha, esse é o único jeito. Eu
então disse: “Ôôô, se eles viverem é bom, os insetos são seres vivos,
mas eles também são úteis para nós. Até os
besouros\footnote{\thai{แมงกีนูน} é um
tipo de besouro que pode ser consumido como alimento.} ainda conseguem
ser úteis.” Se olharmos assim, entendemos. Temos vínculo com tudo, se
dermos o devido respeito não há problema, não há perigo algum, mas
temos que saber o que estamos fazendo. 

Depois recebi a notícia de que ele passou mal, foi levado de
motocicleta ao posto de saúde e voltou com uma doença grave, algo desse
tipo. É realmente assustador. Ontem estava conversando com uma leiga
que veio me procurar e ela disse: “Oh, aquele homem morreu, Luang Pó.”
Pois é, desse jeito morre mesmo, uma pessoa em plena idade produtiva.
Não desligava o trator, pulverizava o dia inteiro, então ocorre desse
jeito mesmo. Por que é assim? É porque ele não tinha outra opção, nunca
teve interesse em mudar, resolver o problema, se não fizesse assim não
conseguiria viver na sociedade, então é assim. Então digo que mesmo que
não tenha oportunidade de fazer direito, procure estudar, procurar
conhecimento complementar.

Por exemplo, investigar o passado, investigar o futuro. Se olharmos
no passado, antigamente a terra não tinha nada, nosso mundo não tinha
nada, era uma matéria sem vida alguma, ali não existia nada que
possuísse vida, mas isso mudou com a evolução da natureza. Antes o
mundo era quente, mas então houve evolução. Se falar sob um certo
aspecto, podemos dizer que era quente e há sinais que indicam o nível
do calor. Dizem que o solo deste mundo explodiu, chamamos de vulcão.
Quando vemos, chamamos de vulcão. Tsunami, quando surge na água, no
meio do oceano. É algo quente, sendo quente há coisas que nascem
daquele calor, temos que pensar assim, também há coisas que nascem do
calor. Não é o caso que aquele calor não tenha nada, ele tem algo. No
mínimo tem uma parte da natureza que chamamos de bactérias ou
micro-organismos, depende de como queremos chamar
seguindo os termos técnicos. 

Se são micro-organismos podemos observar de acordo com a evolução do
mundo, em outras palavras. O mundo esfriou, esta superfície do mundo é
fria, mas abaixo da superfície é quente. A superfície esfriando fez com
que surgissem organismos que chamamos de bactérias ou micro-organismos.
Parte das bactérias são causadoras de doenças; se olharmos assim, mesmo
nós somos uma forma de doença, vemos que também somos uma forma de
bactéria. Se olharmos um pouco mais além, no desenvolver da evolução da
natureza, há lado positivo e lado negativo. É possível que a evolução
seja algo positivo ou negativo. Se pensarmos
assim, temos que pensar em como agir para desenvolver o lado positivo,
temos que pensar; pensar, mas sem ficar loucos. Temos que pensar que é
assim, é natural dessa forma. Então vemos o benefício que surge do lado
positivo, do lado bom, e o utilizamos, como a comida. A comida que
comemos, se olharmos sob um certo ângulo, veremos que comemos coisas
que não são atraentes, comemos coisas que possuem formas muito pequenas
de vida. Muito pequena é pequeno a que ponto? É pequeno a ponto do que
se diz “invisível”, exceto caso tenhamos a capacidade que se chama
visão divina.\footnote{A capacidade de ver que vai além das pessoas
comuns, desenvolvida através da prática de samādhi e jhāna.}
Parece que hoje em dia todo mundo tem, antigamente ninguém tinha, agora
qualquer um fala que tem visão divina.\footnote{Ele está criticando
pessoas têm “visões” e logo se apressam em concluir que aquelas visões
são verdadeiras, que elas desenvolveram “visão divina”, quando na
verdade é apenas a imaginação delas ou, em muitos casos, resultado de
problemas psiquiátricos.} Vemos que aquilo com o qual nos envolvemos
deve ter algo a ver com criar algo composto.


Por exemplo nosso corpo, não pense que é vazio, tem de tudo aqui. Há
micro-organismos no sistema digestivo, podemos dizer que fazem parte
daqueles órgãos. Eles fazem o corpo ser capaz de se adaptar, ajudam o
sistema imunológico, constroem o sistema imunológico. Se há sistema
imunológico, não há doenças. Quando vamos consultar o médico por causa
da degradação dos órgãos e das células, ele então diz que é por não
haver adaptação ou por não haver anticorpos. Se não há anticorpos,
temos doenças, temos que pensar assim. Se pensamos assim, temos que
construir uma composição harmoniosa. Então ingerimos aqueles
micro-organismos. A comida que consumimos está cheia de
micro-organismos. Dependendo do tipo, às vezes gera dano, como quando
comemos algo que faz o corpo se degradar, há dano. Mas se comermos
aquilo que cria uma boa composição, teremos anticorpos. Isso conhecemos
bem, conseguimos entender, mas temos que estudar através de nossa
própria experiência. Isso é algo bom, é útil, e o que é bom é um
mérito. Por exemplo, se nosso corpo não tem doenças dizemos que temos
mérito, temos um tesouro, temos um bem. 

Isso também faz parte da nossa prática, também temos que cuidar da
saúde do corpo, também temos que cuidar da saúde da nossa mente. O
corpo tem que ter sistema imunológico, tem que ter
fortaleza; a mente também tem que ter,
possuindo paz e frescor, possuindo felicidade, permanecendo sem
desarmonia. Isso já é felicidade, já faz surgir felicidade, mas essa
felicidade é uma felicidade individual. Porém, a felicidade que
chamamos de excelente, de \textit{aryia}, nasce dessa felicidade
individual, é dessa forma. Se surgir essa felicidade, é um mérito
nosso, é uma bondade nossa, entendemos assim. 

Portanto, todos nós que chegamos a esse estágio nas nossas vidas
temos que desenvolver boas qualidades
mentais; a bondade, as habilidades que fazem
parte da vida, temos que utilizá-las também. Assim felicidade e
progresso surgirão em nossa sociedade. Nossa
sociedade terá paz e felicidade. Trará paz, seremos como pessoas que
têm familiares por toda parte, seremos capazes de ser como um só. Hoje
eu ofereci vários pensamentos, para que possamos refletir, ponderar e
contemplar, utilizar raciocínio que fará termos felicidade e progresso.
Hoje já ofereci pensamentos o suficiente. Peço licença para encerrar
aqui.

