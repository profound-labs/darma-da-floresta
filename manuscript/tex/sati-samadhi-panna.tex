
\chapterAuthor{Ajahn Anan}
\chapterNote{Ensinamento dado por Ajahn Anan à comunidade leiga e monástica de Wat Marp Jan.}
\chapter{Sati, Samādhi, Paññā}
\tocChapterNote{Ensinamento dado por Ajahn Anan à comunidade leiga e monástica de Wat Marp Jan.}
\markright{\theChapterAuthor}

Sentem-se em meditação e foquem \textit{sati} em um só objeto. Por
exemplo, usem o entrar e sair da respiração como objeto, foco de
\textit{sati}. Quando focamos \textit{sati} em um só objeto de forma
contínua, qual é o resultado? No \textit{Satipaṭṭhāna Sutta}, já
estudamos – usamos o corpo, as sensações, a mente ou o Darma como
fundação para \textit{sati}. Porque, por natureza, \textit{sati} fica
por toda a parte, foca no exterior, não foca na fundação que traz
firmeza. Focamos na fundação que traz firmeza, como por exemplo o
corpo, a primeira fundação é o corpo. A respiração faz parte do corpo,
ou seja, no ato de ir e vir, quando temos atividades e deveres,
estabelecemos \textit{sati} ao mesmo tempo em que trabalhamos, estamos
cientes dos nossos movimentos, ou quando estamos parados temos
\textit{sati}, estamos cientes. Isso é desenvolver \textit{sati},
desenvolver presença mental. 

Quando \textit{sati} aumenta, nós observamos como é a mente, quando
a mente possui apego ou aversão, quando a mente possui desejo, raiva,
ignorância, estamos cientes. Se a mente não tem desejo, raiva ou
ignorância, estamos cientes. Se há raiva, quando pensamos que não
demorará muito e teremos que morrer, é possível abandonar a raiva, o
rancor, a má vontade – se tivermos \textit{sati} olhando a mente. Temos
uma fundação firme, como quando saímos para lutar contra um inimigo:
temos que ter firmeza na nossa fundação, temos que ter uma fundação
para nossa ação. 

Nos esforçamos para focar \textit{sati} nas fundações, como a
respiração ou o mantra\footnote{Na Tailândia é comum o uso de mantras
como objeto de meditação, mas, diferente de outras religiões ou escolas
budistas, não há a crença de que esses mantras possuam propriedades
mágicas sobrenaturais, eles são utilizados apenas como um foco para a
mente.} “\textit{Buddho}”. Pode ser “\textit{Dhammo}”,
“\textit{Sangho}”, pode ser “\textit{Bhagavā}”, pode ser
“\textit{Sammā Arahant}”, ou pode ser “\textit{Araham Sammā}”,
“\textit{Sugato Lokavidū}” ou qualquer trecho do
\textit{Buddhābhitthutim}, \textit{Dhammābhitthutim},
\textit{Sanghābhitthutim}.\footnote{Versos recitados durante a pūja,
que descrevem as qualidades do Buda, Darma e da Sangha.} Às vezes os
mestres usam “\textit{Buddhaṁ Saraṇaṁ Gacchāmi}”, “\textit{Dhammaṁ
Saraṇaṁ Gacchāmi}” – que é o mesmo que “\textit{Buddho, Dhammo,
Sangho}” – como apoio para a prática. Isso é capaz de fazer a mente
parar e ficar em paz. Quando encontramos um objeto de meditação que
pacifica nossa mente, temos que usar esse objeto com frequência. Se
frequentemente ficamos com raiva, temos que usar bem-querer
(\textit{mettā}) com frequência, como “\textit{aham sukhito homi}” -
que eu tenha felicidade, que todos os seres tenham felicidade. Nós
desenvolvemos bem-querer, treinamos a mente. Como se nossa mente
estivesse quente e nós balanceássemos usando algo frio, como a água. Se
o nosso corpo estiver muito quente, nós temos que buscar a ajuda do
frio. Se nossa mente estiver quente, temos que utilizar uma meditação
que esfrie, como o bem-querer, os quatro \textit{brahmavihāras} –
bem-querer (\textit{mettā}), alegria pelo bem-estar alheio
(\textit{muditā}), compaixão (\textit{karunā}) e equanimidade
(\textit{upekkhā}) – para ajudar nossa mente a pacificar-se.
Isso é treinar \textit{sati}; quando \textit{sati} está bem treinada,
surge firmeza mental – \textit{samādhi}. 

\textit{Samādhi} é quando nossa mente se estabelece firmemente,
mas firme de que forma? No Nobre Caminho Óctuplo é citado “pensamento
aplicado/sustentado, êxtase, felicidade, equanimidade”\footnote{Os
fatores do primeiro jhāna.} Quando focamos num objeto de meditação,
\textit{sati} se torna contínua, \textit{samādhi} se firma, há
êxtase, felicidade – é energia para que a mente se desenvolva e surja
sabedoria. Quando nossa mente se firma, contemplamos para gerar
sabedoria. Quando nossa mente pensa, treinamos em contemplar que as
sensações e pensamentos, sensação de felicidade e sofrimento, nascem,
se estabelecem e desaparecem, só isso. Então vemos que este corpo,
nasce, se estabelece e desaparece. Isso é treinar a mente a ter
\textit{sati}. Treinar a mente a se firmar, treinar a mente a ter
sabedoria. No final, quando tivermos sabedoria, não será necessário
treinar, não será necessário focar no objeto de meditação, não será
necessário contemplar para gerar sabedoria pois ela surgirá
automaticamente, porque a mente já largou, já se desapegou. É o caminho
mais rápido, não precisa fazer nada. Essa é a mente cuja sabedoria está
completa. Mas, no começo, na hora em que vamos focar a mente, fazer que
ela fique quieta e em paz, andamos de acordo com o Nobre Caminho
Óctuplo. 

Em alguns lugares dizem que focar a mente não é correto, que
utilizar um objeto para interromper os pensamentos não é correto. Isso,
temos que refletir, não está completando todos os aspectos do Nobre
Caminho Óctuplo que são \textit{sīla, samādhi} e\textit{ paññā}.
\textit{Sīla} educa nosso corpo e fala, isso não há quem contradiga,
todos concordam que é necessário manter o corpo e a fala dentro dos
cinco preceitos. Isso é igual. Desenvolver \textit{sati} é para
entendermos como é nossa atenção. Se nós prestarmos mais atenção, vemos
a movimentação do corpo. Se a mente se unificar em \textit{samādhi} e
nascer sabedoria, vemos: “Oh! Esse andar para cá e para lá\ldots{} Corpo é
apenas corpo, não é um ser, pessoa, um ‘eu’ ou ‘ele.’” Isso é só falar,
mas quando entendemos de verdade, vemos andar para cá e para lá e
percebemos: “Oh\ldots{} não sou eu, não é meu!” A mente e o corpo conseguem se
separar e vemos claramente que o corpo é apenas corpo. Então
conseguimos largar, abandonar o apego, remover o apego, a mente fica à
vontade pois não há “eu.” 

Se enxergar de acordo com a realidade, que não há “eu,” já é
\textit{Sotāpanna}. Não é difícil. Se treinarmos com firmeza,
treinarmos nossa mente, alcançar \textit{Sotāpanna} não é tão difícil
como pensamos. Se enxergarmos que o corpo é apenas corpo, de maneira
clara, enxergamos o Darma naquele mesmo momento. Mas agora nossa mente
está com este corpo e ainda não enxerga, por quê? Há algo escondendo?
Há ignorância, desejo e apegos escondendo a verdade. Mesmo se
ensinarmos à mente: “Isso aqui não somos nós, OK?”, podemos ensinar mas
ela ainda não vê, ela ainda não viu, então ainda entende que são “eu.”
Temos que treinar, treinar a mente a ter \textit{sati, samādhi,
paññā}, ter diligência em praticar, contemplar mais e mais. 

Se nós aqui formos até a Índia, o local onde o Buda nasceu, se
iluminou, ensinou e alcançou parinirvana, e sentirmos fé, surgir êxtase
e felicidade gerando \textit{samādhi}, peço que contemplem que este
corpo e mente são impermanentes e assim verão o Darma, verão o Buda. Ou
se não formos e ficarmos praticando aqui na Tailândia, faça sua mente
entrar em paz, ficar quieta, com êxtase e felicidade, nossa mente em
paz e quieta, e contemple que este corpo e mente não são “eu,” não são
“meu,” claramente. Nasce conhecimento que é \textit{bhāvanā-mayā
paññā};\footnote{Conhecimento que nasce da prática, em oposição ao que
nasce do estudo de textos ou de ouvir ensinamentos.} se virmos o Darma
vemos o Buda, este Buda aqui. Vemos na nossa própria mente, vemos
nirvana na nossa mente. 

Pensem, nesta vida temos uma boa oportunidade de treinar nossa
mente, tivemos a oportunidade de encontrar o Darma do Buda. Ocorrer de
nascermos e encontrarmos o Darma dele não é algo fácil. O Buda está no
Darma: quem enxergar o Darma, enxerga o Buda; quem alcançar o Darma,
alcança ser “Buda,” depende da nossa mente. Se esforcem em praticar ao
máximo: quando tiver oportunidade, faça \textit{pūja} no templo ou em
casa, procure treinar sua mente para que ela se aquiete, se pacifique.
No começo eu também treinei como vocês leigos. Eu era leigo, tinha fé
em ir ao templo, tinha fé em praticar, seguia os cinco preceitos de
forma contínua, tinha fé em fazer \textit{pūja}, praticar meditação,
estudar o Darma, ouvir o Darma, buscar o caminho de prática. Eu
procurei um livro sobre o \textit{Satipaṭṭhāna Sutta} e li\ldots{} Ele
ensina de maneira direta, mas minha mente ainda não entendeu. “No ir e
vir, tenha \textit{sati}. Parado, tenha \textit{sati}. Falando, tenha
\textit{sati}. Em todas as posturas, tenha \textit{sati}, esteja
ciente. Respirando, tenha \textit{sati}.” Eu tentava praticar, mas
ainda não enxergava que o corpo é apenas corpo. 

Que fazer? Vai praticando, praticando. No final, se a mente se
unificar firme em \textit{samādhi}, veremos que os elementos são
apenas elementos, todas as coisas no mundo são só o que são – nascem,
permanecem e desaparecem. O apego por todas as coisas materiais deste
mundo desaparece: não há desejo por elas, o que deseja, é o Darma. Isso
já é ver o Darma: tem fé inabalável no ensinamento do Buda, tem
determinação em praticar, vê que, no mundo inteiro, o agregado do corpo
nasce, permanece e desaparece. Não há refúgio, a morte se aproxima a
todo momento. O conhecimento de \textit{vipassanā} surge claro, se
unifica e nós vemos o Darma, não é muito difícil. 

No começo sentimos como que tateando no escuro, não conseguimos
achar o ponto certo. Como é que se pratica? Ficamos confusos, em
dúvida. Mas mesmo em dúvida, continuamos praticando. Pode ser assim
mesmo, só não desista. Mesmo que esteja em dúvida: “Como devo praticar?
Quero resultados rápidos,” não diminua o esforço, vá praticando
simultaneamente às dúvidas. Na verdade não queremos ter dúvidas, mas as
impurezas mentais nos fazem ficar confusos porque queremos resultados
rápidos. É o comportamento normal que ensinamos à nossa mente, então
fica com dúvidas sem parar na hora de praticar. Surge
\textit{samādhi} e fica em dúvida de novo: “O que é isso? É êxtase ou
felicidade?” e tenta encaixar de acordo com os livros. Mas isso é só
memória: a coisa de verdade já surgiu bem à nossa frente. Êxtase é a
sensação de alegria – nós estamos alegres, está em linha com os textos,
é o que estamos praticando. 

Luang Pó Chah ensinava a praticar até o ponto em que a mente entra
em paz e se aquieta. Contemplando, entendemos que este corpo e mente
são apenas \textit{anicca, dukkha, anatta}. Toda vida que nasce tem que
morrer. Nasce fé no Buda, já viu o Darma, já é \textit{Sotāpanna},
não precisa ter dúvida. Então continua praticando, continua
investigando, enxerga ainda mais claramente que este corpo não é “eu,”
não é “meu.” O entendimento se aprofunda de acordo com o nível, e no
final surgirão \textit{nimittas},\footnote{Imagens que surgem na mente
quando ela se estabelece num nível superficial de samādhi.} mas isso
vem depois. Queremos que desde já apareçam \textit{nimittas} para
vermos claramente. Nisso temos que ter calma: caso não estejamos vendo
\textit{nimitta} alguma, não tem importância. Não é necessário se
preocupar com \textit{nimittas}. Pacifique sua mente, contemple, cuide
para que sua mente não sinta apego ou aversão, já está bom. Não é
necessário ver \textit{nimittas}. Ver \textit{nimittas} é melhor do que
não ver? Não dá para comparar. Às vezes não precisa ver
\textit{nimitta} alguma e a mente consegue remover as impurezas. Então
já está correto, contemplando e comparando o exterior com o interior é
possível alcançar a iluminação, entendam isso. 

Procurem praticar. Acontece que nossa prática ainda não está
pacífica e ficamos com dúvidas. Temos dúvidas e aí ficamos procurando
qual é o método correto. Em alguns lugares eles ensinam: “Não precisa,
não precisa focar em nada, apenas se desapegue. Não precisar fazer
nada, se desapegue\ldots{}”, e nós entendemos? Como Luang Pó Chah ensinava: o
método mais rápido é não fazer nada. Nós entendemos isso? “Não fazer
nada.” Como pode ser possível? Mas vamos praticando, temos que
desenvolver \textit{sati}, desenvolver \textit{samādhi, paññā}, até
estarmos repletos e não precisarmos fazer mais nada. No começo é
necessário praticar, temos que fazer \textit{sati} ser contínua,
\textit{samādhi} ser firme. Não pense “Não precisa \textit{samādhi}
nenhum. Se surgir cobiça, raiva, ignorância, deixa surgir, é natural,
nasce e desaparece sozinho!”, não é assim, tem que treinar. Cobiça,
raiva, ignorância ainda surgem, mas em seguida vão diminuindo, em
seguida permanecem menos tempo, pois \textit{sati} fica mais forte,
sabedoria aumenta e então é possível derrotar as impurezas. Não é: “Não
precisa fazer nada, apenas esteja ciente, isso surge e desaparece
sozinho.” Até está certo, caso permaneça apenas ciente, mas essa
ciência tem força suficiente? Se não treinarmos \textit{sati} para
saber, não treinarmos \textit{samādhi} para estabelecer-se firme, não
vamos conseguir saber de forma clara. 

Eu usei \textit{paññā} para desenvolver \textit{samādhi} –
comecei com \textit{sati}. No começo \textit{samādhi} ainda não se
estabelecia, eu não via muita utilidade em \textit{samādhi}, eu
pensava: “Estou desenvolvendo \textit{sati}”, contemplava e conseguia
me desapegar; desse jeito me sentia à vontade. Alguns monges
praticavam, mas ainda tinham emoções se manifestando. Eu dizia: “Esse
aí ainda não pratica direito, pois ainda tem impurezas, ainda tem
impurezas se mostrando. Se praticamos \textit{sati} desse jeito,
conseguimos acompanhar, mantemos \textit{sīla}. Se não, as impurezas
mentais se mostram no corpo e na fala. Eu consigo controlar minha mente
com \textit{sati}, estou ciente de apego e aversão e consigo
contemplar.” Eu sentia que isso era mais leve, melhor. Outros monges
tinham \textit{samādhi}, mas, quando surgia uma emoção, ela vinha à
tona e eu pensava: “Isso mostra que sua \textit{sati} ainda não é
suficiente.” Aí eu ficava achando defeito naqueles que desenvolviam
\textit{samādhi}: “\textit{Sati} ainda não é suficiente!” Aqueles que
tinham \textit{samādhi} diziam: “Eu já desenvolvi \textit{sati}, seu
\textit{samādhi} é que não é suficiente, você ainda não entende
direito.” E aí discutíamos. Mais à frente o \textit{samādhi} daquele
que tinha \textit{sati} firmou-se e melhorou, a \textit{sati} daquele
que tinha \textit{samādhi} melhorou pouco a pouco. Pouco a pouco
vimos: “Oh, ambos são necessários: tendo \textit{sati},
\textit{samādhi} se firma; tendo \textit{samādhi} tem que ter
\textit{sati}, ter \textit{paññā} junto,
\textit{magga-sāmaggiya}!”\footnote{“Harmonia do Caminho” -- os oito
aspectos do Nobre Caminho Óctuplo devem estar harmonizados entre si.}
Entende que corpo e mente são \textit{anicca, dukkha, anatta}. 

Nesse ponto já não há mais discussão, já não discutem mais sobre o
que é mais importante: termos \textit{sati}, mas \textit{samādhi}
ainda não é suficiente, termos \textit{samādhi}, mas pouca
\textit{sati}; não discutem mais. Entendem que o Caminho Nobre tem oito
aspectos, eles se unem em \textit{sīla, samādhi e paññā –
magga-sāmaggiya}. Isso é algo importante: a pessoa que desenvolve
\textit{sati} tem que firmar a mente – \textit{samādhi}. A pessoa que
desenvolve \textit{samādhi} para vencer as impurezas tem que ter
\textit{sati} – atenção e cuidado – porque é possível separar a mente
dos objetos mentais. \textit{Samādhi} estando firme, quando surgem
emoções é possível separá-las. Nós vemos claramente que a mente é uma
coisa, as emoções são outra. Exteriormente é possível parecer, para as
pessoas que olham, que está com raiva, tem raiva, ódio, mas na mente
não há, a mente é uma coisa, as emoções são outra, elas se separam. Se
nós treinamos de modo comum, observando \textit{sīla}, quando temos
raiva a mente e as emoções vão juntas, isso é no nível de
\textit{sīla}; no nível de \textit{sati} somos capazes de saber
claramente que raiva surgiu e desapareceu, mas quando
\textit{samādhi} se estabelece, \textit{sati} sabe: “Vai surgir
raiva,” mas a raiva e a mente, a emoção e a mente se separam. 

Isso talvez nós ainda não entendamos, mas se continuarmos praticando
veremos claramente que a mente é uma coisa, e as emoções são outra. Por
exemplo, eu vivia com Luang Pó Chah, várias vezes ele mostrava, às
vezes ele mostrava como se estivesse com muita raiva para ensinar os
discípulos. Era só um segundo e aquela raiva desaparecia e ele voltava
a seu estado feliz normal, dava risada, conversava, sorria. Conseguem
ver? Mente e emoções são distintas uma da outra. Peço que treinem nesse
ponto e nós veremos o Darma. Procurem abandonar desejo, aversão e
ignorância, firmem-se em sua prática. Procurem desenvolver
\textit{samādhi}, procurem treinar \textit{sati} para que seja
contínua. Contemplem para que surja sabedoria, enxerguem que este corpo
é apenas corpo, não é um ser, pessoa, “eu” ou “meu” e veremos o Darma.
Possam todos vocês fazerem progresso no Darma. 
