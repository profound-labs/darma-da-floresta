
\chapterAuthor{Ajahn Chah}
\chapterNote{Palestra dada por Ajahn Chah a praticantes leigos na Inglaterra.}
\chapter{Budismo estilo Theravada}
\tocChapterNote{Palestra dada por Ajahn Chah a praticantes leigos na Inglaterra.}
\markright{\theChapterAuthor}

Hoje vim aqui porque meus discípulos, como por exemplo o Sr. Suchin,
convidaram para que viesse me encontrar com as pessoas daqui. Pode-se
dizer que é muita sorte termos nos encontrado aqui. Alguns talvez não
saibam o que é isso que está sentado bem em frente deles. Somos monges
theravada que vêm da Tailândia, fizemos o esforço de vir até aqui hoje.
Viemos nos encontrar aqui hoje para sabermos como é o Budismo
Theravada. Na Tailândia há um local chamado Wat Nong Pah Pong onde há
muitos ocidentais indo estudar a prática da meditação budista. 

O Budismo Theravada que trouxemos aqui é um ensinamento que é sempre
atual, nunca envelheceu, nunca foi danificado e sempre acompanhou os
tempos e as pessoas também. Sempre ensinou todos nós a termos harmonia
e integração como se fôssemos um, com harmonia firme. Por exemplo,
ensina a todos nós de uma mesma maneira: aquilo que é ruim, que
perturba, não é pacífico, é confuso – somos ensinados a tentar
abandonar todas essas coisas. 

Depois disso nos ensina a ter sabedoria, a conhecer o que é errado e
causa perturbação para os indivíduos e para a sociedade como um todo
também. Esses ensinamentos, reunidos, são os pilares do \textit{Buddha
Sāsana} e se chamam \textit{Sīla, Samādhi, Paññā}.

Em seguida somos ensinados a pacificar nossa mente. Esta nossa mente
aqui, mesmo tendo uma casa enorme para que viva bem, ainda não tem
bem-estar, pois ainda está fervendo por causa da confusão, por causa do
modo errado de enxergar as coisas. Estou falando sobre a mente de todos
nós aqui.

Agora vamos tentar concentrar nossas mentes, unificar a mente com um
objeto, o que se chama \textit{samādhi}. Portanto, prestem atenção no
que vou dizer. Fazer nossa mente ter força é diferente de fazer nosso
corpo ter força. Para que nosso corpo tenha força temos que correr, nos
movimentar, de manhã e à noite. Essa movimentação faz nosso corpo ficar
forte. Já a mente, devemos fazer que ela pare e permaneça quieta, fazer
que se pacifique. Isso trará energia, a mente terá força, isso se chama
\textit{samādhi}. Fazer a mente entrar em \textit{samādhi} não
acontece instantaneamente, porque desde que nascemos nunca treinamos
nossa mente a ficar quieta, nunca treinamos nossa mente a ficar em paz.
Nunca fizemos, então não vai ser logo hoje que vamos conseguir
pacificar nossa mente. Por isso temos que ter determinação em praticar,
quer haja paz ou não. Algumas pessoas, que nunca praticaram, têm medo.
Têm medo que fazer a mente se pacificar vá fazê-los ter problemas
mentais. Têm medo de ficar loucos e todo tipo de coisa. Por que isso?
Porque nunca praticaram, então ficam com medo. Não há nada disso. 

Sentem-se da forma que acharem melhor, mas sentem-se em paz. Tenham
uma postura pacífica e confortável. Todos fechem os olhos e foquem a
atenção na respiração, experimentem. Experimentem focar a atenção na
respiração, façam agora. Façam. Quem quiser sentar na cadeira – sente.
Quem quiser sentar no chão – sente. Pratique agora mesmo, quando
inspirar ou expirar esteja ciente. Não force a respiração a ser mais
rápida ou mais devagar. Não force a respiração a ser mais forte ou mais
fraca. Deixe que ela seja natural e tenha apenas ciência da respiração
entrando e saindo à vontade. Vocês veem a respiração? Estão cientes
dela? Como ela é? Como é “aquele que está ciente”? Vocês enxergam nas
suas mentes? Onde está “aquele que está ciente”? Onde está a
respiração? Como é a respiração? Foquem dessa forma, estejam cientes.
Usem a ciência dessa sensação para acompanhar a inspiração e a
expiração, acompanhem a respiração entrando e saindo à vontade. O seu
dever neste momento não é grande, basta estar ciente da respiração
entrando e saindo. Não precisa pensar: “O que vai acontecer se eu focar
a atenção na respiração? Vou saber alguma coisa? Vou ver alguma coisa?”
Não é dever de vocês pensarem nisso. O dever de vocês é estarem cientes
da respiração entrando e saindo e é só isso.

Normalmente não fazem isso. Há muitos anos, muito tempo, que vêm
negligenciando a respiração, não estão cientes de que o ar entra e sai.
Hoje que possuem esta oportunidade, observem como ele entra, como ele
sai, observem bem. A respiração possui muito valor, é a comida mais
importante dentre todas. Se não comermos esse tipo de comida por duas
horas é um problema sério; já comida como o arroz, mesmo se não
comermos por um ou dois dias, ainda conseguimos sobreviver. A comida da
respiração, se faltar por cinco ou dois minutos, não aguentamos.
Portanto, a respiração tem valor. Hoje todos vocês observem em detalhe
a respiração. Observem a respiração, não deixem a mente escapar para
cima, para baixo, para a esquerda ou para a direita, nem para frente ou
para trás. Que a mente esteja ciente somente da respiração na ponta do
nariz, só isso. Não pense que ficará com problemas mentais, que ficará
louco. Quanto melhor praticar, mais são vai ficar. Vai ficar são, vai
se curar de seus problemas mentais. Não pense que será de outra forma.
Pratiquem.

{\itshape -- intervalo para prática de meditação --}

(…) Nós não nos conhecíamos previamente, acabamos de nos conhecer
aqui hoje. Eu que estou ensinando, caso diga algo errado ou
inapropriado, me perdoem, me desculpem. Todos nós, naquilo que ainda
não conhecemos, somos ignorantes. Daquilo que ainda não estudamos,
somos ignorantes. Entendam isso. A mente da maioria das pessoas não se
aquieta. Por exemplo, hoje viemos nos sentar em paz desse jeito, as
pessoas que se acham espertas devem perceber: “Puxa, não tem paz
nenhuma, que barulho é esse vindo me perturbar? Barulho das crianças,
barulho dos barcos, vindo perturbar, todo tipo de barulho… não há paz.”
A pessoa acha que está pesando de forma correta por causa da sua
esperteza, mas ela não enxerga a burrice naquilo. É o barulho que vem
nos incomodar ou nós é que vamos incomodar o barulho? Isso nunca
pensam. Na verdade somos nós que vamos perturbar o barulho, não é o
barulho que vem nos perturbar. É assim ou não é? Nós vamos perturbá-lo
ou ele vem nos perturbar? Olhem. Pois é, somos burros desse jeito. As
pessoas espertas são burras desse jeito. E ainda assim não admitem. Não
sabemos dessas coisas, pois só olhamos o lado de fora, não olhamos o
lado de dentro. Pensamos que somos espertos mas, na verdade, nós é que
vamos perturbar. Mas pensamos que é ele que vem nos incomodar. Isso é
causa para que não tenhamos bem-estar, é causa para nossa mente não ter
paz nem bem-estar, ficar confusa, pois não conhecemos a nós mesmos.

Como o mundo hoje em dia, todos os países, eles são do jeito que
são, eles não têm nada de mais. Somos nós que pensamos: “Esse é bom,
aquele não é”, gostamos desse, não gostamos daquele e só encontramos
confusão. Na verdade o mundo é do jeito que é. Não conhecemos sobre nós
mesmos, pois somos burros. Vários dos meus discípulos vão estudar nas
universidades do ocidente. Estou falando sobre meus discípulos: estudam
e ficam ainda mais burros, sofrem ainda mais, brigam ainda mais, não se
entendem. Disputam entre si, só brigam, não trabalham. Isso é porque
não conhecem a si mesmos. Hoje eu vim falar sobre a ciência do Buda,
hoje vamos estudar ciência do Buda. Hoje em dia tem muitas ciências, é
de perder a conta as ciências que estudamos. Mas essas ciências não se
complementam. Todas as ciências têm que incluir a ciência do Buda. Se
não incluir a ciência do Buda, não serve. Ainda não se harmonizam,
ainda disputam entre si, ainda têm inveja, ainda guardam rancor, ainda
geram confusão o tempo todo.

Hoje eu aproveitei essa ocasião para dizer isso, trouxe esse
presente. Se você é assim, corrija a si mesmo! Na nossa mente há algo
que não é bom? Aos poucos expulse para fora. Não acumule, aos poucos
expulse para fora. Aquilo que não for bom, que oprima a si mesmo,
oprima aos demais, aquilo na mente que não é bom, apenas expulse para
fora, não deixe acumular até ficar pesado. Como ponto inicial da
ciência do Buda, somos ensinados sobre \textit{Sīla}. 

Número um: não oprimir aos demais seres, pequenos ou grandes,
humanos ou animais. Isso é ciência do Buda para não sofrermos.

O segundo aspecto é que não guerreemos pelas coisas, deixem estar.
Não fiquem disputando os objetos, deixem estar. Isso é ciência do Buda.

O terceiro, se formos leigos: é normal termos família, não traia sua
família. Se trair sua família será mais um assunto para gerar
sofrimento.

O quarto é ser honesto: não minta, tenham sinceridade entre si,
todos vocês. Esse é mais um ponto da ciência do Buda.

O número cinco é não beber coisas que embriagam. Já estamos
embriagados o suficiente na nossa mente com todo tipo de coisas. Se
além disso for beber, ultrapassa o limite da embriaguez, não vai mais
saber o que é o quê. 

Se unir tudo isso, chamamos de ciência do Buda. Essa ciência do Buda
engloba todas as ciências. Faz as demais ciências não perderem o rumo,
faz com que elas não gerem sofrimento. A ciência do Buda as engloba
dessa forma. Assim, todas as pessoas viveriam em harmonia, sem
inimizade e rancor pelos demais. Se aprendermos todas essas ciências e
acrescentarmos a ciência do Buda, podemos dizer que somos budistas:
chama-se \textit{Buddha Sāsana}. Onde estiver está bem, está em paz.
Esse é o significado do que vim ensinar hoje.

Vocês que estão sentados aqui, alguma vez sofreram? Sofreram no
coração? Por que esse sofrimento? Qual é o problema? A mente pensa
errado e então sofre. Se pensar correto vai sofrer? Isso é para vocês
refletirem. A prática de \textit{samādhi} que fizemos hoje é
justamente para conhecer isso. É ensinar essa mente a conhecer o
caminho correto, conhecer certo e errado. Mas hoje eu peguei leve, não
havia tempo. Se fosse para sentar de verdade, da forma correta,
teríamos que fazer de novo. Hoje eu não vim ensinar, só vim visitar.
Vim visitar, dar um pouco de opinião. Por isso peço desculpas, mas se
fosse para praticar \textit{samādhi} de verdade eu não ia permitir
sentar em cadeiras, tinha que sentar no chão desse jeito. Se for para
treinar de verdade tem que ser assim, desse jeito aí é confortável
demais, dá sono. A prática de \textit{samādhi} de verdade é feita
sentado no chão. Na próxima ocasião nós fazemos de novo, por hoje peço
desculpas.

Só tem casas grandes por aqui. Aonde quer que vá é tudo grande, mas
não há muito bem-estar, não sei o que acontece. Vivem em casas enormes
mas não têm bem-estar. Às vezes não conseguem dormir a noite inteira,
por que será? Qual é a razão disso? Olhem, este local é ótimo, a
natureza aqui é ótima, mas por que só a natureza aqui é boa e não a
mente das pessoas? Por que razão? Por isso, temos que melhorar nossa
mente. Mesmo num prédio como este, de vários andares, ainda não haverá
bem-estar caso nossa mente não esteja bem. Mesmo se melhorar a natureza
ainda mais, a mente não estará bem, pois não a melhoramos. Hoje eu vim
encorajar vocês a melhorarem suas mentes. Isso é algo muito importante.


Por aqui as casas e os alimentos para o corpo, eu digo, já são
suficientes. Só tem gente gorda! Mas comida para a mente, eu digo,
ainda não é suficiente. Algumas pessoas vão sentar à beira do mar,
sentam e ficam olhando… Elas não estão bem, a atitude delas é de quem
não está bem. Não sei o que se passa. Eu acho que talvez esteja
faltando um tipo de comida. Às vezes vai perguntar: “Senhor, senhor…”,
e nada. “Senhor!”, e eles enfim olham para cima. Eu sinto que estão
famintos, muita falta de comida para a mente. Já viram isso? Olhem,
olhem… Há uma causa? Será que há uma causa? Para ficar desse jeito tem
uma causa, isso não vem do nada, há uma causa. Tem que remover essa
causa. Se não remover a causa vai continuar incomodando, não vai haver
bem-estar. Se deitar num colchão macio, não vai ser macio, se morar
numa casa grande, não vai ser grande. A mente se sente apertada, no
final pensa em morrer. Que peso ela está carregando dentro de si? Mesmo
dormindo ainda carrega. Mesmo dormindo ainda carrega e não larga, às
vezes chora a noite inteira. Não sabem o que carregam, vocês não sabem
pois não praticam \textit{samādhi}. Quando possível, investiguem até
encontrar. Está faltando comida para mente, comida para o corpo já há o
suficiente.

Agora é um bom momento, vou pedir para que eles coloquem um filme
para vocês verem. Esse filme foi feito em Wat Nong Pah Pong, na
Tailândia. A BBC de Londres foi filmar. Eu vim visitar aqui em Londres
e trouxe junto. Eles filmaram sobre como vivem os monges. Eles vivem na
floresta – o que é que eles fazem lá? Eles comem, dormem e ficam à toa,
ou o quê? Assim as pessoas podem conhecer a rotina dos monges. Eles
fizeram esse filme para que possamos ver. 

{\itshape -- intervalo para apresentação do filme --}

(…) É um pouco difícil, é o que chamamos “sentar em meditação”. Não
é o caso de que vamos sentar o tempo todo em meditação. Sentamos só o
suficiente para que a mente tenha uma fundação, para que ela não tenha
confusão. Depois tivemos oportunidade de sentar cerca de trinta ou
vinte minutos, sentamos, treinamos, para a mente ter
\textit{samādhi}, ter frescor. Se temos habilidade em meditar, não é
difícil, é só sentar por dois minutos e já estamos em paz. Mas nunca
cuidamos da mente, há muito tempo que temos deixado ela vagar por todo
lado, aí ela vai de acordo com o hábito dela, pois nunca foi treinada.
Algumas pessoas nunca treinaram a mente, nunca praticaram, pensam que é
trabalhoso, muito difícil. Como na sessão de hoje, algumas pessoas
sofreram muito, trocavam de posição o tempo todo, isso é comum, é
incômodo e desconfortável porque apenas começamos a treinar.

Algumas pessoas se desanimam pois veem que é difícil e trabalhoso
fazer. Aquilo que é difícil é bom fazer. Aquilo que é fácil não vale a
pena. Para que limpar onde já está limpo? Já está limpo! Vá fazer
limpeza onde está sujo, isso sim é útil. Por exemplo esta casa aqui,
antes de virmos praticar meditação aqui, antes de ela ter sido
construída, não era possível ficar aqui. Foi necessário trazer madeira,
ferro, cimento. Havia uma enorme bagunça bem aqui, não era possível
morar aqui. Por quê? Porque a construção ainda não estava concluída.
Agora que já terminou, é confortável? Tem algo bagunçado? Isso é porque
o cimento foi posto em seu lugar, o ferro foi posto em seu lugar, a
madeira foi posta em seu lugar. Não há bagunça alguma. Então é
confortável desse jeito. Mas antes era desconfortável este lugar aqui,
é assim ou não é?

Isso se compara dessa forma. Nós apenas começamos a praticar. Não só
construindo uma casa, mesmo com o estudo, no começo tínhamos preguiça,
não é? Não queríamos ir estudar, tínhamos preguiça, pois era difícil. A
mãe e o pai tinham que obrigar a ir, não é? Então conseguimos estudar e
obtivemos conhecimento e sabemos o valor do estudo. Nós temos que
pensar desse jeito, temos que treinar nossa mente desse jeito,
aconselhar nossa mente assim para que ela tenha força. Quando formos
hábeis, não será mais necessário sentar usando esse método. Poderemos
sentar à mesa, na cadeira do trabalho e estarmos cientes, termos
\textit{sati} acompanhando os estados mentais, e a mente conseguirá se
pacificar. Pratique até a mente ficar esperta, até a mente ficar hábil,
até a mente ser desse jeito. De pé tem essa sensação, deitado tem essa
sensação, indo ou vindo tem que ser assim. Ela já alcançou a paz dela. 

Algumas pessoas pensam que vão ter que se esforçar desse jeito para
sempre, que vai ser trabalhoso e difícil desse jeito para sempre, mas
não é assim. Tendo terminado um trabalho, estamos tranquilos. Se ainda
não terminamos, não estamos bem. Isso também é assim, não é diferente.
Se fosse difícil sem parar desse jeito, eu também já tinha ido embora,
também não ia querer. No começo era sofrido, difícil, trabalhoso, hoje
não é difícil daquele jeito, já deu resultado. Quando é assim, é só
querer e a mente já se pacifica. Onde sentar, fechar os olhos e focar a
mente, ela se pacifica. Ela já foi treinada, é fácil. Quando a mente
está confusa, sentamos tranquilos assim, e ela se concentra: “pup!”
Acabam os problemas. É assim.

Além disso, tendo pacificado a mente, há a contemplação de várias
coisas que surgem e desaparecem. No que surge e desaparece não há nada
de mais. A alegria surge e num instante desaparece, a tristeza surge e
num instante desaparece, não há nada! Mas nós corremos atrás e aí
sofremos, viramos escravos disso, só isso. Isso surge e desaparece,
surge e desaparece. Quando conhecemos o Darma desse jeito, a mente se
pacifica e junto vem a sabedoria. Não há problemas – ou tem, mas são
poucos. Não é preciso resolvê-los, eles logo se resolvem sozinhos. Se
nossa sabedoria ainda não surgiu, sofremos. Por que sofremos? Nós
agimos certo ou errado? Estamos pensando certo ou errado neste momento?
Então somos ensinados a não mirarmos no futuro, a largarmos o passado e
praticarmos neste momento atual. A mente se pacifica. Ou vocês não
concordam? Alguém tem alguma pergunta?

-- Como o Luang Pó pode não se incomodar em visitar o zoológico?
Aqueles animais foram presos contra a vontade deles.

-- Eu não os prendi, quem prendeu é que é o culpado. Eu não tenho
nada com isso. Uma vez fui até o Wat Prabat. Eles prendiam pássaros,
vendiam e soltavam. No dia seguinte eles prendiam e vendiam de
novo.\footnote{Algumas pessoas gostam de fazer mérito libertando um
animal que está preso ou que está para ser sacrificado, como
consequência, espertalhões se aproveitam e prendem animais para
vendê-los em frente a um templo onde as pessoas costumam fazer esse
tipo de coisa.} Um dia eu estava indo visitar e vieram: “Luang Pó,
compre um pássaro para soltar.” Estava numa gaiola. Eu olhei:

“Quem prendeu esse pássaro?”

“Eu prendi.”

“Quem prendeu que solte! Vai trazer para eu soltar para quê? Você
prendeu, você que solte! Eu não prendi, vou soltar para quê? Se eu
soltar o seu pássaro você vai ficar com raiva.”

Ele olhou para minha cara… “De onde saiu esse monge!?” 

Use equanimidade quando não puder ajudar. O pássaro estava preso, se
eu soltasse eles me prendiam. Seria uma grande ofensa. O carma daqueles
animais é aquele, fazer o quê? Eu vejo e fico em \textit{upekkhā} –
deixo a mente equânime, não encorajo, não faço contato, apenas olho.
Penso: “O mundo é assim, se não prendessem, alguém ia querer ver? Então
eles prendem, o mundo é assim.” Se formos soltá-los à força, vai ser
bom? Vai virar um grande tumulto, não é? Se olhamos assim, nós
largamos, ficamos equânimes, em \textit{upekkhā}, não desperdiçamos a
mente. Não é largar demais. Eu também sinto pena, mas deixo minha mente
em equanimidade. Deixo parte do carma para a pessoa que prendeu e a
outra parte para o pássaro que foi preso. Fico quieto. Nada de mais, o
mundo é assim. Se sabemos desse jeito, não há nada de mais. Mais alguma
coisa?

-- A dor, o sofrimento e a felicidade, são iguais em que sentido ou
são similares de que forma?

-- São como as extremidades de um pedaço de pau. Elas são
diferentes? Uma extremidade e a outra, há diferença? Felicidade ou
sofrimento – não pegue, não se apegue a eles. Felicidade e sofrimento
são como uma cobra, vemos e deixamos ela ir embora. Se formos pegar,
ela nos pica, não é? “Minha felicidade” pica, “meu sofrimento” pica,
como uma cobra. Felicidade é como uma cobra venenosa, sofrimento é como
uma cobra venenosa. Se vemos que ela é venenosa, deixamos ela ir.
Felicidade e sofrimento são iguais a uma cobra, se formos pegar: “Puxa,
que felicidade!”, quando ela desaparece surge sofrimento, eles têm o
mesmo valor. Tem que pensar de acordo com o Darma para entender. Mas as
pessoas não gostam de sofrimento, só querem felicidade – não dá! Pegar
a felicidade é o mesmo que pegar o sofrimento, pegar o sofrimento é o
mesmo que pegar a felicidade. É assim. Mande mais perguntas! Tem mais
alguma?

-- Pergunta sobre praticar no momento presente: Se ficarmos somente
no momento presente, não pensarmos no que aprendemos e fizemos no
passado e não fizermos planos para o futuro, como vamos progredir?

-- Ainda não entendemos o Darma. O presente é resultado do passado,
não é? O presente é causa para o futuro, não é? Quando contemplamos o
presente, é como contemplar o passado e o futuro. Não passa disso. O
presente é resultado do passado. O presente é causa para o futuro que
vem em seguida. Se contemplarmos o presente, é como se contemplássemos
o passado e o futuro, não precisa ficar repetindo muito.

\textit{(Tradutor)} -- Ele está dizendo: a prática dos monges pode
estar de acordo com as tradições do oriente, da Tailândia, mas ele acha
que esse tipo de prática não é completa porque os leigos fornecem todos
os recursos, essa deve ser a opinião de várias pessoas no ocidente. Ele
não disse abertamente, mas acho que é algo do tipo “vocês são
egoístas”. E se é assim, como essa prática pode estar correta?

-- Se você não roubar e eles disserem que você roubou, é verdade?
Pois é, não precisa complicar muito. Praticamos por bondade àqueles
quem têm sofrimento, têm visão incorreta, ensinamos para que saibam o
que é o quê, para que haja harmonia entre as pessoas. Ensinamos as
pessoas a conhecerem o Darma para que haja harmonia, para que não se
agridam. Mas se enxergarmos desse jeito, é como se fossemos egoístas,
mas os monges não querem nada, comem só uma refeição por dia. Dão
suporte aos monges para que eles ensinem as pessoas a conhecer
\textit{Sīla}, a não se agredirem. Se você quer pensar que é egoísmo
fique à vontade. Mande mais uma, mande mais uma! Amanhã vou embora,
quem quiser perguntar, pergunte!

-- Pergunta sobre o Nobre Óctuplo Caminho,
\textit{sammā-ājivā} (modo de vida correto). Ele é advogado e às
vezes tem que brigar e discutir durante o trabalho. Como deveria
praticar na vida dele para que esteja de acordo com o modo de vida
correto?

-- É advogado? É honesto?

-- Ele tenta.

-- Não são só os advogados. Mesmo o rei, se não agir certo, está
errado. Se agir errado não é certo, é errado. Não são só os advogados,
mesmo o rei, se agir mal, não está certo, está errado. O Darma é assim.
Resolva as questões para que fiquem corretas, não aja mal e isso é o
suficiente. Fora disso, tem que procurar. Que método utilizar depende
da nossa sabedoria. Nós temos que refletir sozinhos: onde trabalhamos,
está certo ou errado? Temos que refletir sozinhos. Para ser advogado,
aprender as leis, tem que gastar muito dinheiro. É melhor não agir mal,
já investiu muito nisso, se agir mal, não vai ser bom. Se perguntar
para os monges, eles têm que responder assim. O Darma é assim, e você
então leve e vá praticar de acordo com a sua sabedoria.

-- E na hora de exercer sua função, o que poderia servir como ponto
de referência? Como ele deveria olhar?

-- Olhe para sua própria mente, olhe se a mente está sendo honesta
ou não. Detesta aquela pessoa, mas gosta dessa outra? Ajuda aquela
pessoa, mas não ajuda esta aqui? Olhe para a própria mente, não olhe em
outro lugar. Só há isso, só há a mente, faça a mente estar correta.
Olhe para a própria mente. Visão correta (\textit{sammāditthi}) nasce
bem ali, bem ali na mente.

-- Os budistas acreditam em vida futura?

-- Hã? O que você disse? 

-- Quando morrer nesta vida, vai haver uma nova vida em seguida?

-- À frente existe? Amanhã existe ou não? O futuro existe ou não? O
passado existiu ou não? O presente existe ou não? Vá pensar sozinho. Se
você perguntar: “Vida futura existe ou não?” e eu disser: “Existe!”, o
que você vai fazer? Você vai acreditar ou não? Se você acreditar você é
burro, porque estamos sentados aqui, eu digo: “Existe” e você
simplesmente acredita sem razão alguma. Sabendo disso, essa não é uma
resposta a ser dada. O que dá para responder é: “Amanhã existe, futuro
existe, passado existiu, presente existe?”, é a mesma coisa. Sabendo
disso, esse problema você tem que resolver sozinho. Se eu disser:
“Existe vida futura” e você acreditar… você não tem sabedoria, pois eu
estou apenas sentado aqui, digo: “Existe” e você logo acredita sem
razão alguma. É assim, essa não é uma resposta que se deva dar, é uma
questão que você deve procurar dentro de si mesmo. Tendo feito uma
pergunta, você deve ir praticar. Não pergunte à toa. Você tem que usar
isso como combustível para fazer surgir sabedoria, tem que contemplar
como um aspecto da prática.

Amanhã não haverá atividades. Todos vocês, hoje eu me despeço,
amanhã eu já viajo de volta. Estou muito contente com todos, mesmo que
vocês perguntem todo tipo de questões, estou contente, estou contente
em dar opinião, contente em responder de forma direta. Dessa vez fica
por isso mesmo, eu ensinei e praticamos. 

Vou falar honestamente, a pessoa que vá ensinar meditação, na
verdade, deveria ser um monge. Esse ensinamento teve origem no Buda,
uma pessoa pura. Falando de forma direta, é verdade, os leigos também
conseguem, mas de forma indireta, não vão direto. Tenham cuidado, hoje
em dia existem vários métodos de meditação, muitos métodos hoje em dia.
Muitos professores ensinando meditação. Cuidado para não serem vítimas
de um professor falso…
