
\chapterAuthor{Ajahn Chah}
\chapterNote{Ajahn Chah ensina um grupo de pessoas que veio experimentar a vida monástica por um período curto de tempo.}
\chapter{Prática correta}
\tocChapterNote{Ajahn Chah ensina um grupo de pessoas que veio experimentar a vida monástica por um período curto de tempo.}
\markright{\theChapterAuthor}

\ldots{} quando êxtase surge, há felicidade, não dá para descrever, mas
quem chegar nesse ponto vai saber. Quando surge a felicidade,
unificação mental surge. Então há pensamento aplicado, pensamento
sustentado, êxtase, felicidade e unificação mental.\footnote{Quatro
fatores do primeiro jhāna.} Essas cinco coisas se unem no mesmo
ponto, não importa que elas tenham características diferentes; elas se
unem no mesmo ponto. Nós enxergamos todas elas. Como frutas que estejam
no mesmo cesto: não importa que elas sejam diferentes, vemos todas
naquele cesto. Quer seja pensamento aplicado, pensamento sustentado,
êxtase, felicidade ou unificação mental, se olharmos na mente, vemos
todos. É dessa característica, é desse jeito, existe assim. É
felicidade de que jeito? Pensa de que jeito? É êxtase de que jeito? Não
dá para falar. Quando elas se unem naquele lugar, nós vemos, acontece
dentro do nosso coração. A partir daqui fica diferente, a prática fica
diferente do que era antes.

Tem que ter \textit{sati}, \textit{sampajañña}. Não se engane! Não
pense: “O que é isso?” Isso é só uma característica da mente, uma
capacidade da mente, não fique com dúvidas na hora de praticar. Se
afundar no chão – não importa. Se flutuar no espaço – não importa. Se
for morrer sentado aqui – não importa, não fique duvidando, prática é
assim. Olhe: quando ela é assim, qual é a característica da nossa
mente? Fique com isso. É só isso. Vá praticando. Isso já é dar
resultado. Já deu resultado. Há \textit{sati}, \textit{sampajañña},
está ciente de si quando está de pé, andando, sentado ou deitado. Desse
ponto em diante, quando vemos alguma coisa não nos prendemos como
antes, não nos apegamos. Quando vemos aquilo surgindo continuamente –
gostar, não gostar, felicidade, sofrimento, ter dúvida, não ter dúvida
– há investigação, reflexão, investigação dos resultados dessas
atividades. Não aponte: “Isso é aquilo!”, não faça isso em hipótese
alguma. Apenas esteja ciente. Veja que todas as coisas que surgem na
mente são apenas sensações. Isso é impermanente, nasce e cessa. Nasce e
então permanece, permanece e então cessa, só vai até aí. Não há “eu,”
“meu,” “nós,” “ele.” Não devemos nos apegar a nenhuma dessas coisas. 

Quando vê que corpo e mente são assim por natureza, quando sabedoria
vê dessa forma, viu por completo. Vê a impermanência da mente, vê a
impermanência do corpo, vê a impermanência da felicidade e do
sofrimento. Amor e ódio não são permanentes. Só vai até aí, a mente
vira e: “Oh\ldots{}”, se desencanta.\footnote{A palavra tailandesa
“\thai{เบื่อ}” agrega uma conjunção de significados
difícil de reproduzir em português. Ela expressa o tipo de cansaço
mental que é uma mistura de tédio, irritação, falta de interesse e
desânimo.} Se desencanta com este corpo e mente, se desencanta com as
coisas que surgem e desaparecem, coisas que são incertas. Só isso. Onde
sentar-se, verá. Quando a mente se desencanta ela procura o caminho de
saída, procura o caminho de saída de todas essas coisas. Não quer ser
assim, não quer ficar assim, vê a desvantagem. Viu a desvantagem neste
mundo, vê desvantagem na vida em que nasceu. É assim. Quando a mente é
assim, onde quer que nos sentemos, vemos \textit{anicca, dukkha,
anatta}, e não há mais onde segurar nada. 

Aí, caso sente sob uma árvore, ouve o Buda ensinando. No topo de uma
montanha, ouve o Buda ensinando. Numa área plana, ouve o Buda
ensinando, ouve. Vê todas as árvores como uma só, vê seres de todos os
tipos como de um só tipo. Não há nada além disso, nasce e então
permanece, permanece e então muda, cessa; só isso. Então vemos o mundo
mais claramente, vemos o corpo e a mente mais claramente, vemos o mundo
material e imaterial mais claramente. Fica mais claro em
termos de \textit{anicca, dukkha, anatta}. Se as pessoas se apegam
pensando: “É permanente, é verdadeiro,” imediatamente surge sofrimento.
É assim que ele surge, mas se virmos que corpo e mente são assim, ele
não surge. Vemos que corpo e mente são assim mesmo, não nos apegamos a
eles. Onde quer que nos sentemos teremos sabedoria. Mesmo vendo uma
árvore nasce sabedoria, vê os seres nascendo nesta terra e surge
sabedoria, vê os insetos e surge sabedoria, no final todos se unem no
mesmo ponto, são Darma. 

O ponto é: são incertos,\footnote{Em tailandês
“\thai{ไม่แน่}”. Essa expressão é recorrente no
ensinamento de Ajahn Chah, ela é um aspecto de “anicca,” expressa
impermanência, incerteza, imprevisibilidade, a característica de não
ser garantido, não confiável e, portanto, não digno de apego.} essa é
a verdade, esse é o \textit{Sacca-Dhamma}, é algo impermanente. Onde há
permanência? Só há permanência no fato de que surge e é assim, não é de
outra forma. Só isso, não há nada de mais além disso. Se virmos dessa
forma, terminou a viagem. No budismo, pensar: “Eu sou mais ignorante
que ele” ainda não está certo. Pensar: “Eu sou igual a ele” ainda não
está certo. Pensar: “Eu sou melhor que ele” ainda não está certo – pois
não há “eu.” É assim. Remove o apego à ideia de “eu.” É o que se chama
\textit{lokavidū}, sabe claramente de acordo com a verdade. Se
enxerga de verdade desse jeito, a mente é verdadeira, ela conhece a si
mesma, sabe até o fim. Ela já cortou a causa e, não havendo causa, não
surge mais nada, a chuva não consegue mais chover.

Eu estou falando sobre a prática, ela decorre dessa forma. As
fundações que devem que praticar, novatos iguais a vocês, são: número
um – ser honesto, sincero; número dois – ter vergonha, medo, vergonha
de fazer o mal; número três – ter humildade na mente, satisfazer-se com
pouco, ser frugal. Se a pessoa é frugal ao falar e nas demais coisas,
ela vê a si mesma, ela não se envolve em confusão. Essa é a causa, essa
é a fundação. Naquela mente não há nada. Só há \textit{sīla}, só há
\textit{samādhi} na nossa mente, está cheio de \textit{paññā} na
nossa mente, não há outra coisa. A mente naquela hora está andando
dentro de \textit{sīla, samādhi, paññā}.

Por isso, praticantes, não sejam descuidados. A instrução “não ser
descuidado” nós quase não ouvimos – pouco, ouvimos pouco. Descuidado\ldots{}
Descuidado significa: com todo tipo de coisa, não seja descuidado.
Mesmo que esteja correto – não se descuide. Errado – não se descuide.
Bom – não se descuide. Tem felicidade – não se descuide. Todas as
coisas – não se descuide. Por que não devemos nos descuidar? Porque
tudo isso é incerto – segure desse jeito. Nossa mente é a mesma coisa:
caso haja paz, largue a paz. Também queremos ter alegria, é bom, mas
entendemos o assunto? Ruim – entenda; bom – entenda. 

Por isso treinar a mente é, na verdade, assunto de nós mesmos. O
mestre ensina o método para treinar a mente. Se vamos treinar a mente,
tem que ser nós mesmos, pois a mente está em nós. Conhecemos tudo que
está em nós, ninguém é capaz de saber tão bem quanto nós essa história
de prática. Se fizer de forma correta desse jeito, ficamos numa boa.
Andando estamos bem, sentados estamos bem. Se quer ver por si mesmo,
tem que fazer de verdade, não faça de mentira. Alguns dizem: “Se fizer
de verdade não vou ficar cansado?” Não fica, pois faz na mente, pratica
na mente, treina na mente. Caso haja \textit{sati} e
\textit{sampajañña}, tem que ser capaz de saber o que é certo e errado.
Se sabe isso, conhece o modo de prática. Não precisa muito. E, tendo
conhecido todas as atividades, todos os modos de prática, traga tudo de
volta a si mesmo.

Como com os monges e noviços que moram juntos, existem duas coisas:
emular e imitar.\footnote{Mesmo em tailandês o significado dessa
expressão não é óbvio, o ponto chave aqui é o jogo de palavras
\thai{เอียง - อย่าง}
(“íeng” - “yáng”). O que ele quer dizer é que não devemos só imitar a
expressão externa do comportamento de alguém, mas sim olhar mais
profundamente e procurar aprender com o que há de bom naquela pessoa.}
Emular e imitar – não se deve imitar, deve-se emular. Por exemplo, Tahn
Ajahn Tongrat que é um grande mestre do passado. As pessoas que não
eram inteligentes não conseguiam receber o Darma dele, pois o imitavam.
Não o emulavam, mas o imitavam. O hábito do Tahn Ajahn Tongrat era
falar de forma descuidada, era o comportamento dele. Volta e meia ele
pedia coisas.\footnote{Dentro do círculo de monges da floresta é
considerado deselegante um monge pedir algo aos leigos, principalmente
se isso os estiver incomodando. Conta a história que no vilarejo onde
Luang Pó Tongrat morava havia uma família muita avarenta, que não dava
nada de esmola aos monges quando eles saíam em piṇḍapāta. Luang Pó
Tongrat, famoso por sua personalidade excêntrica, usava o seguinte
recurso para “ajudar” aquela família a deixar a avareza e fazer mérito:
ele ficava de pé bem em frente à casa gritando “E aí? Cadê a comida?” e
não ia embora enquanto eles não trouxessem um pouco de arroz para
oferecer. A família então tentou outra estratégia para não doar: no dia
seguinte, quando Luang Pó chegou, a dona de casa se desculpou dizendo
que não havia cozinhado porque havia acabado a lenha. Na manhã seguinte
Luang Pó apareceu na porta da casa com um feixe de lenha nos ombros:
“Tá aqui a lenha, eu fico aqui e espero você cozinhar o arroz.”
Detalhe: é óbvio que uma colher de arroz tem muito menos valor que um
faixo de lenha – ele realmente estava fazendo aquilo por compaixão
àquelas pessoas.} Quando ele dava bronca em público, eram broncas
ferozes. O comportamento dele era assim. Mas na mente dele não havia
nada, no fundo, ele já não tinha nada. Aquilo era apenas falar, o
objetivo era o Darma. Na verdade, no que quer que dissesse ou fizesse,
o objetivo dele era o Darma, mas eramos nós que não entendíamos nada. O
objetivo dele era o Darma, não causar o mal – e não causava. Andava
para cima e para baixo, não era circunspecto. Onde quer que fosse o
comportamento dele era assim. Alguns monges o imitavam e se
prejudicavam. Era assim.

Portando todos nós, como eu já contei, \textit{dukkhā patipadā
dandhabhiññā, dukkhā patipadā khippabhiññā, sukhā patipadā
dandhabhiññā, sukhā patipadā khippabhiññā}. Não podemos
competir para ver quem tem mais \textit{pāramī} que o outro. Dá
para comparar quem tem mais força no corpo, mas não dá para competir
sobre \textit{pāramī}. É como aquele monge disse: “Eu tenho boa
intenção, quero progredir, tenho boa intenção, mas ainda não é
possível.” Mas tem que tentar, vá dando chibatadas. \textit{Dukkhā
patipadā dandhabhiññā} é a pessoa cuja prática é difícil, há
atrito, tudo é difícil, pratica com dificuldade e alcança o Darma
devagar. Significa que acumulou pouco \textit{pāramī}. Tem que
usar esta oportunidade aqui para acumular muito, não desista. Como um
pobre, se ele pensa: “Eu sou pobre, não vou trabalhar,” aí piora de
vez. Se nos achamos pobres, temos que ter esforço no mínimo igual à
demais pessoas e aí poderemos comer. Isso é igual, nós temos que
praticar muito, treinar muito, desenvolver muito, e também será
possível para nós.

Esse é o \textit{dukkhā patipadā dandhabhiññā}. O
\textit{dukkhā patipadā khippabhiññā} pratica com dificuldade mas
alcança rápido, diferente do anterior. Alcança rápido, quase morre, vai
aos trancos e barrancos mas alcança rápido. Se machucar não importa, se
doer não importa, alcança rápido. Esse é o segundo tipo. O terceiro é
\textit{sukhā patipadā dandhabhiññā}: esse pratica fácil, mas
alcança devagar. Não há muito atrito, não há muita confusão, vai aos
poucos, mas usa muito tempo. Às vezes alcança quando já está quase
morrendo. Na época do Buda tinha gente assim. \textit{Sukhā
patipadā khippabhiññā}, o quarto tipo aqui, pode-se dizer, pratica
com felicidade e facilidade, alcança rápido, esse tipo também existe.
Isso o Buda declarou. É assim.

Isso é igual\ldots{} É igual ao \textit{ugghatitaññu},
\textit{vipacitaññu}, \textit{neyya}, \textit{padaparama}. O
\textit{ugghatitaññu} não precisa de muito, é só ele olhar e já sabe o
que deve fazer e como fazer. É só a sensação do sofrimento surgir:
“pup!” e ele resolve: “pup!”, muito rápido o \textit{ugghatitaññu}.
\textit{Vipacitaññu} é preciso aconselhar, dar símiles para ele ouvir,
ele não consegue abandonar o sofrimento em uma ou duas noites, mas ele
consegue corrigir seus erros. É possível dar conselhos, dá para
resolver o problema. Esse é o \textit{vipacitaññu}. \textit{Neyya\ldots{}}
Estou falando da mente aqui, o nome da mente é \textit{ugghatitaññu},
\textit{vipacitaññu}, \textit{neyya}, \textit{padaparama}. Estamos
falando sobre a mente. Existe mente que alcança muito rápido, se fosse
um cavalo não precisava bater, era só levantar o chicote e ele já
começaria a correr, era só ver a sombra do chicote e já começaria a
correr. O \textit{vipacitaññu} é do tipo que dá para aconselhar,
levanta o chicote e ele não corre – então bate! Só então esse cavalo
corre. O terceiro tem que bater sem parar – ainda não corre. Então bate
com frequência, caso contrário ele não corre – esse é o \textit{neyya.}
O \textit{Padaparama}, pode bater à vontade – não vai, não corre, não
anda. Se deixa levar por qualquer emoção: ensinamos o suficiente para
que possa enxergar, mas ele não anda, fica quieto ali mesmo, bate e ele
ainda fica parado. Essa mente não enxerga, há muita escuridão. Esse é
difícil, o cavalo desse quarto tipo aqui, o Buda joga fora. Não está
jogando fora nada de mais, é algo que só oprime, não dá para ensinar,
não entende – então joga fora. Por que isso? Carma do passado. Carma do
passado vem interferir a ponto de o carma novo não conseguir se
manifestar. Como água numa garrafa que já está cheia e não está
vazando; se quiser colocar água nova, não consegue. Tem que jogar fora
a água velha antes. Tem que usar a água velha até acabar antes de
colocar a nova. Isso é igual, se tiver carma interferindo é difícil. Se
tiver carma antigo interferindo, é muito difícil. 

Então, de qualquer forma, para nós aqui, se vai ser rápido ou
devagar, ou o que quer que seja, é fazer o bem e isso é bom. Quando
faz, nada se perde, tendo feito é bom. Fazer o bem desfaz o mal, fazer
correto expulsa o que é errado. Assim só traz benefícios; se nesse
momento ainda não deu resultados, numa ocasião futura dará. Carma,
quando o carma antigo está dando resultado, o carma novo não consegue
se manifestar. É assim. Alguém aí já acumulou carma? Se esse carma está
dando resultado, o carma que nós estamos fazendo agora não consegue
dar, mas não é desperdiçado. Quando esse carma se exaurir, aquele carma
surge em seguida.

Por isso eu disse que não há desperdício: faça como reserva para
colher benefícios. Todas as coisas, caso ainda não tenha chegado a hora
delas, elas permanecem. Todos vocês que vieram praticar no Wat Nong Pah
Pong durante o retiro das monções, tenham sempre a atitude de “deixar
estar” para com os demais monges e noviços. Tenham consideração pelos
outros, ajudem-se uns aos outros. Ajudem-se a acumular
\textit{pāramī}, tenham firmeza e seriedade. Entendam que todos
nós que viemos nos reunir aqui hoje somos filhos do Buda. Não somos
filhos de pai e mãe, somos filhos através do Darma. Se não agirmos de
acordo com as instruções do pai e da mãe, só os difamamos. 

Tal qual declaramos: “\textit{Buddhaṁ saraṇaṁ gacchāmi, Dhammaṁ
saraṇaṁ gacchāmi, Sanghaṁ saraṇaṁ gacchāmi}” e assim por diante,
essa é uma fala muito importante. Nós tomamos o Buda como refúgio, nós
o estimamos. \textit{Dhammaṁ saraṇaṁ gacchāmi}, nós temos o Darma
como refúgio, nós o estimamos. \textit{Sanghaṁ saraṇaṁ gacchāmi}, nós
tomamos a Sangha como refúgio, nós a estimamos e assim por diante. É
fácil falar, mas difícil fazer; é fácil falar, mas difícil agir de
acordo. Se não tivermos sinceridade, não diminuirmos nossa arrogância,
não alcançaremos o Buda, o Darma e a Sangha. Portanto não discuta e
brigue sobre o modo de prática. Viemos construir \textit{pāramī},
então acerte o alvo. É verdade que o tempo é curto, é verdade que a
quantidade de tempo é pequena, pois que seja pequena como uma cobra que
é pequena, mas tem veneno. Não seja pequena de outra forma, seja útil.
Pense assim: “Fazer maldade 100 ou 1000 anos é ainda pior. Vir fazer
bondade, que seja cinco ou sete dias, é muito melhor.” Tem esse
potencial, dá resultado. 

Portanto, a maioria das pessoas, antes mesmo de chegar na metade do
retiro\ldots{} na verdade, a característica das pessoas é com o passar do
tempo ficarem descuidadas. Quebram as regras de conduta que foram
estabelecidas. Começam de um jeito e terminam de outro. Mostra que
nossa prática ainda está defeituosa. Portanto, aquilo que no começo do
retiro nos decidimos a fazer, façamos até que esteja completo. Que
nesses três meses pratiquemos de maneira uniforme, do começo ao fim.
Todos têm que se esforçar, todos. Antes de começar o retiro nós
decidimos praticar de que forma, vamos fazer o que, como vamos
praticar, o que foi que decidimos? Lembrem-se. Lembre-se, se estiver
ficando desleixado tem que se reajustar continuamente.

Como praticar \textit{ānāpānasati}: a respiração entra e sai
de maneira contínua; quando a mente se perde nos pensamentos nós a
reestabelecemos, firmamos de novo. Quando ela vai seguindo os
pensamentos nós a reestabelecemos, firmamos de novo. Isso é igual,
firmar a mente é assim, firmar o corpo é assim, tem que haver esforço.
Os outros tipos de coisas que fazemos desde que nascemos até os dias de
hoje, vários anos, já conhecemos. Não dá para fazer ainda mais maldade,
coisas ruins, já fizemos muito. Agora viemos decididos a construir
bondade durante três meses. Todos nós neste retiro, pensem que estão
nascendo de novo, comecem de novo, abandonem os velhos hábitos. Quer
vocês continuem como monges ou vão embora, que isso passe a fazer parte
da sua personalidade. Fazer o bem não é desperdício, fazer o que é
correto não vira desperdício. Só não é bom para a maldade, só não é
correto para o errado. Não vai para lugar algum, fica na sua própria
mente. 

A maldade, quando fazemos, é só pensar e nos sentimos mal, quer
tenha sido ontem, anteontem, dois ou três meses atrás – quando
lembramos nos sentimos mal. Por exemplo, na época em que nós éramos
leigos destruímos a cabeça da estátua do Buda aqui, isso é um exemplo.
É só entrarmos no monastério e vermos a imagem do Buda que sentimos:
“Ôôô! Que maldade!” Mesmo estando fora do monastério, é só pensarmos e
sentimos a maldade, não nos sentimos bem. Mesmo que venhamos nos
ordenar dessa forma, ainda pior: vamos fazer reverência à imagem do
Buda todo dia e não tem cabeça porque nós a destruímos. Isso se chama
fazer o mal, é maldade, não nos sentimos bem. Quando fazemos maldade,
mais tarde sofremos. A bondade quando feita não gera sofrimento, isso é
igual.

Agora que viemos virar monges; se nos comportarmos direito, vamos
nos sentir muito bem. Depois que largarem a vida monástica, quando
pensarem em Wat Nong Pah Pong, vão sentir felicidade, sentir que
fizeram algo de bom para que o pai e a mãe possam se orgulhar. Vai ser
nosso refúgio, vai ser o refúgio dos nossos filhos e netos. Este local
é onde construiremos bondade nesta vida. Tenham isso bem firme no
coração de vocês. Muito bom. Fazer as coisas é assim, se fizer o mal
encontra o mal mais tarde\ldots{}
