
\chapter{Nascimento e Renascimento}

{\itshape
Ensinamento dado por Ajahn Chah à comunidade monástica de Wat Pah Pong.}

(…) se há algo a que não nos apegamos, aquilo não vem a existir,
sofrimento não nasce. Sofrimento nasce de existência, tem que haver
existência para que nasça. Nascimento, nasce na
existência\footnote{\thai{ภพ-ชาติ}. Essas duas
palavras tailandesas têm um significado difícil de traduzir ao ocidente
pois existem dentro do sistema de pensamento que enxerga a vida como
uma em uma longa sequência de vidas. Em geral, essas duas palavras são
utilizadas nesse contexto. A diferença para nós é semelhante à
diferença entre “a vida” e “uma vida” - o significado “uma dentre
várias” já vem embutido na palavra. Não há tradução direta para o
português, o melhor que conseguimos arrumar foi “existência” para
\thai{ภพ} e “vida” para \thai{ชาติ}, embora “vida” no
sentido de “estar vivo, possuir vida” é melhor traduzida como
\thai{ชีวิต}. Mesmo para alguém que fale o idioma
tailandês, o jogo de palavras feito por Ajahn Chah aqui é um pouco
inusitado e requer um certo esforço para entender.}. O próprio apego é
a existência. Agarrar-se, apegar-se é existência para que sofrimento
nasça, para sofrimento nascer – pode olhar. Não olhe longe, olhe o
presente, olhe o nosso corpo e mente neste momento em que estamos
praticando. Quando surge sofrimento, por que é sofrimento? Olhe bem
naquele momento! Quando surge felicidade, por que é felicidade? Olhe
bem naquele momento: onde quer que ela nasça, investigue bem ali.
Sofrimento nasce do apego, felicidade nasce do apego, só isso.
Nascimento necessita de existência, vida. Se há vida ela tem que vir da
existência, então há vida. Na vida há velhice, na velhice há doença,
morte, sofrimento, angústia, etc. Não precisa olhar longe, olhe bem
aqui. Apego faz nascer felicidade, faz nascer sofrimento, sem se dar
conta. Se há algo que goste, se apega: “Isto é meu, disto eu gosto”, e
fica feliz. Vê? Nasceu. Nasceu por quê? Apego. É fácil de ver, não
precisa olhar muito. Naquilo onde houver existência, será mais um lugar
onde nascerá sofrimento. Sofrimento nasce da existência, sofrimento é
vida; quando há vida, há existência – sofre. Nasce bem ali. 

Esta vida já sabemos: ela nasce da existência; seja felicidade ou
sofrimento, nasce da existência. Só isso já vem a ser uma vida. Nasce
felicidade, nasce uma vida, nasce sofrimento, nasce mais uma vida. Não
é o caso que as pessoas nasçam e morram num dia ou noutro, não é assim.
São os objetos mentais que morrem e nascem, morrem e nascem. Isso é o
que Buda chama “pessoa nascendo uma vida”, mil vidas. Nisso ele não
está interessado, no que se chama pessoa morrendo e nascendo o Buda não
estava muito interessado. No que ele estava interessado é o momento
mental que nasce e morre – quantas vidas nisso? 

O nosso corpo nasceu e nós ainda não morremos, não é? Estamos
sentados aqui, ainda não morremos. Mas morte na mente não é uma ou
duas, são dez mil, cem mil. Às vezes tem felicidade e gosta, às vezes
tem sofrimento e não gosta – estão vendo? Mas nós já nascemos quantas
vezes? Quantas vidas, existências? Isso é que é importante. O corpo não
tem importância; enquanto a mente estiver se apegando, não precisa
duvidar – vai continuar sendo escravo dos desejos, se falarmos em
termos da realidade última da mente. Sendo assim, sabemos na nossa
mente: uma pessoa sem sabedoria, um tolo, nasce continuamente. Está
desejando aquilo que vem a nascer, deseja aquilo que nasce, ou seja,
vida, e não quer morrer. Está vendo? Nascer e não morrer, isso existe?
O Buda ensinou assim, mas nós queremos nascer, contudo não queremos
morrer – é digno de se ouvir? É para dar risada? É bobagem? Tem
sabedoria? 

Vida. Se há vida, ela nasce da existência e tem que morrer, mas as
pessoas desejam muito, desejam nascer mas não querem morrer. Olhe, em
nós mesmos é assim. Como querer que não haja sofrimento? Tem que ser
assim. Isso é a pessoa que se deixou iludir pelo mundo, perdeu-se no
mundo, não vai a lugar algum, deseja estar aqui. Na verdade o mundo
real nasce, envelhece e morre o tempo todo. Os antigos mestres,
praticantes de verdade, viam assim. Eles viam que é assim – nasce,
morre, nasce, morre, não há nada garantido em lugar algum. Se
contemplar, verá por todos os aspectos. É assim mesmo, não há nada
garantido em lugar algum. Nasce, morre, nasce, morre, não há real
substância. Andando ele sente que é assim. Sentado ele sente que é
assim. E aí, como é que vai ser? Só há sofrimento. No mundo como um
todo, só há sofrimento. Como uma grande barra de ferro que foi posta na
fornalha e está quente por inteira. Pegue, ponha a mão para ver: em
cima está quente, embaixo está quente, dos lados está quente, só há
quente, não há onde não esteja quente, por quê? Ela saiu da fornalha, a
barra inteira está vermelha de calor, tem onde esteja fria? 

Por isso o Buda ensinou a contemplar até que não haja mais
nascimento. Se falamos “que não haja nascimento” as pessoas se
assustam, não é? Morrer e não nascer é a pior coisa possível. As
pessoas no mundo entendem: morrer e não nascer é a pior coisa no mundo
– a pior! Não tem quem queira morrer, não é? Na verdade, se morrer e
não nascer, está encerrado. Não há nada nascendo, não há mais nada, não
há nada. “Vai ser felicidade?” Não há felicidade, e aí? “Então é
sofrimento.” Não há sofrimento. “Então não é agradável.” Naquele lugar
o pensamento de que não é agradável não existe. Como é que fica? Se
formos pensar “hum… eu não estou gostando…” o pensamento “não estou
gostando” não existe ali, acabou. Se formos falar é assim; se for falar
de acordo com a verdade, as pessoas se incomodam, não é? “É doido
varrido, oh…” Então tem que ensinar de acordo com o mundo:
“\textit{\=ayu, vanno, sukham, balam}”\footnote{Palavras finais dos
versos que são recitados por monges como forma de benção ou
agradecimento (anumodan\=a).}. Eles ficam felizes e dizem
“\textit{saaaaaaadhu}!”, não é? Idade: que chegue ao máximo. Feição: a
mais bela possível. Felicidade: que tenha ao máximo. Que tenha energia
ao máximo. Bom, não é? Essa é a benção que se dá às pessoas mundanas,
elas gostam assim. 

Isso aqui, se não contemplarmos de acordo com a verdade, não
entenderemos nada. Tem que conhecer a causa; se não houver causa, não
dá em nada, não há efeito. Hoje em dia também é assim. Temos que ver
com clareza, temos que não nascer, tem que não nascer. Experimente,
para não nascer tem que entender o que é nascimento. Mesmo o pensamento
“Puxa, essa pessoa me desagradou, me fez ficar com muita raiva”, já não
existe. “Essa pessoa me agradou, eu gostei muito”, já não existe, há
apenas a expressão mundana que fala “gosto muito”, “odeio muito”. Falam
de um jeito, pensam de outro – são diferentes. Tem que usar as
convenções do mundo para poder se entender com os demais, só isso. Eles
já não têm mais nada, eles estão acima. Tem que estar acima desse
jeito, é lá que os \textit{aryias} moram. Conosco é a mesma coisa,
temos que praticar desse jeito, fazer um esforço. Não fique em dúvida,
vai duvidar do quê? 

Quando eu ainda não praticava, pensava: “Estando o ensinamento do
Buda presente no mundo, por que algumas pessoas praticam e algumas não
praticam? Ou praticam um pouco e desistem, esse tipo de coisa, ou
aqueles que não desistem não praticam com dedicação total, por quê?
Eles não sabem…” Eu então fiz uma promessa no meu coração: “Pois bem!
Nesta vida eu ofereço este corpo e mente. Até a morte, vou praticar de
acordo com o ensinamento do Buda em todos os aspectos, então vou saber,
vou praticar para saber nesta vida. Ficar sem saber é doloroso, então
vou me entregar por completo, vou me esforçar! Não importa quão sofrido
ou difícil, tenho que praticar. Se não for assim, vou ficar para sempre
na dúvida.” Eu pensava dessa maneira, então me decidi a praticar: “Não
importa quão sofrido ou agradável, tenho que praticar. Esta vida vai
ser como um dia e uma noite, só isso. Vou jogá-la fora! Jogar fora! Vou
praticar de acordo com o ensinamento do Buda, com o Darma, até saber
porque toda essa confusão e dificuldade neste \textit{sams\=ara}. Eu
quero saber isso, eu quero ser assim.” Eu pensava em praticar e sentia
alegria naquilo. 

Neste mundo os monges renunciam a alguma coisa? Se é monge e ainda
não desistiu, ainda pratica – joga fora tudo. Não há nada a que ele não
renuncie: formas, sons, odores, sabores, toques – joga fora tudo. Todos
os contatos, joga fora tudo. Renuncia ao máximo; aquilo que o mundo
deseja, ele joga fora por completo. Por isso nós, sendo praticantes,
temos que nos contentar com pouco. Temos que possuir frugalidade,
contentamento. Mesmo em falar, conversar, comer, beber, tem que haver
simplicidade ao máximo. Come com simplicidade, dorme com simplicidade,
seja o que for, faz com simplicidade. Da forma que se diz “é nosso modo
de ser normal”. Do tipo “simplicidade” – desse tipo. 

Vá praticando, quanto mais pratica… Quanto mais pratica, sente mais
alegria. Vemos nossa mente, coisas estranhas e incríveis. Queremos que
as pessoas também vejam, mas está fora da nossa capacidade fazê-las
ver. Por isso esse Darma é \textit{paccattam}, sabemos por nós mesmos.
Se é sabendo por nós mesmos, então temos que praticar sozinhos. Na
nossa prática contamos com o Ajahn em parte apenas. Por exemplo, hoje
eu estou ensinando, vocês estão ouvindo? Estão. Isso ainda não é útil,
falando claramente, mas o que estou dizendo ainda é algo que vale a
pena ouvir. Ainda não é algo útil para nós mesmos; mesmo que haja
alguém que acredite só porque eu falei, ainda não gera benefício por
completo. Se alguém acredita piamente no que eu digo, aquela pessoa
ainda é burra. Se ouvir e for investigar com razão, for contemplar até
ver claramente na sua própria mente, até conseguir largar sozinho,
fazer sozinho – isso sim é algo que já deu muito fruto, já conhece o
gosto dele e a pessoa sabe por si mesma dessa forma, de verdade. 

Por isso o Buda não falava diretamente, não dá para falar com
clareza. É como explicar a luz para uma pessoa cega, dizer: “Puxa, é
branco de verdade, é amarelo mesmo, é preto de verdade!”, não dá para
dizer. Explica, mas é como não ter explicado, pois explicou para um
cego. Explica, mas não traz benefício, pois aquela pessoa é cega. Por
isso ele voltava e dizia para fazer e ser \textit{paccattam}, a ver
claramente em si mesmo. 

Quando vê claramente em si mesmo, torna-se \textit{sakkhiputtha},
viramos testemunha de nós mesmos, de verdade. Não temos dúvida, quer
estejamos de pé, sentados ou deitados. Se alguém vier dizer: “Sua
prática está completamente errada!”, ainda não nos preocupamos, ainda
temos nossa fundação, não esquentamos a cabeça. Tem alguma fundação?
Por exemplo, estamos sentados aqui, sentados, estamos confortáveis. Se
não houver algo fora do comum, estamos confortáveis. Se houvesse um
jarro e surgisse vontade de pegar, ele pesa um ou dois quilos, é o peso
dele. Eu estendo a mão para levantar aquele jarro, é algo fora do
comum, eu sei: “Hã… isso aumentou de peso, dois quilos.” Se estamos
apenas sentados, esse peso não existe. O peso vem de estendermos a mão
e levantar o jarro, aí sentimos que há peso por volta de dois quilos
aqui. Isso só nós sabemos. Mas se alguém disser: “Não há peso, você não
está sentindo peso algum”, acreditamos nele? Nós não temos que pensar:
“Eu não sinto peso – eu sinto peso”, não precisa falar. Por isso o Buda
dizia para saber por si mesmo, não dá para explicar de verdade esse
caminho. Só há “botar a mão na massa” e praticar. A sangha também é
assim, eles praticam.

Eu vejo que aqui a maioria das pessoas têm essa atitude: vão indo,
olham aqui, olham ali, perguntam isso, perguntam aquilo, pensam: “Se eu
falar algo que lhe agrade, ele vai dizer que é bom e ficar feliz.” Mas
não é assim, isso não resolve o sofrimento, não corrige as opiniões.
Essas explicações – não é que não se deva ouvir – é bom ouvir. Tendo
ouvido, vá contemplar até ver claramente em si mesmo, só isso. É bom
ouvir, não seja desrespeitoso, é digno de ouvir, digno de respeito, vá
contemplar. Se for pensar, um praticante tem que ser assim. Aonde quer
que formos, não é possível que alguém explique para que possamos saber
com clareza, exceto nossa própria sensação, nossa própria visão – ela
vira o elemento de \textit{samm\=aditthi}. Prática é assim, conosco é a
mesma coisa. Nessa prática, mesmo se formos ordenados monges por cinco
anos, dez anos, mas não tivermos praticado mais que um mês – fica
difícil. A verdade é essa. 

Na verdade temos que lutar com as sensações o tempo todo. Feliz,
confortável – tem que saber. Triste – tem que saber. Gosta – tem que
saber. Não gosta – tem que saber. Tem que conhecer convenção, tem que
conhecer libertação; convenção e libertação vêm juntos. Se conhece o
bem conhece o mal, conhece junto, eles nascem juntos. Isso é o
resultado que nasce do trabalho do praticante. Portanto, o que quer que
seja que gere benefício para si e gere benefício para as demais pessoas
– de acordo com o ensinamento do Buda que diz: “Tendo beneficiado a si
mesmo, beneficie os demais” – isso é agir de acordo como o Buda. 

E eu sempre ensino: aquilo que deveriam fazer, não querem fazer.
Como a rotina do monastério que eu ensino com frequência. As pessoas
falam que não querem se dedicar porque não sabem fazer, mas na verdade
é preguiça, é irritação, é confusão, essa sim é a causa. Se formos para
algum lugar e não houver envolvimento, vai haver alguma coisa? Comida é
a mesma coisa. Se comermos e não houver envolvimento, como é que é?
Gostoso? Experimente para ver. O ouvido – se alguém falar e não nos
envolvermos, entendemos alguma coisa? Se não entendemos nada, vemos o
problema? Não havendo problema, há razão? Onde corrigir? Tem onde
corrigir? Tem que entender assim. 

Antigamente eu vivia no norte, fui morar com uns monges velhos.
Tinham se ordenado já com idade avançada, tinham dois ou três anos de
vida monástica. Eu já era monge havia dez anos. Fui ficar com os
velhos, mas eu havia me decidido a praticar, então recebia a tigela
deles, recebia o manto deles, limpava o cuspidor, todo tipo de
coisa\footnote{Estes são deveres normalmente realizados pelos monges
recém-ordenados.}. Eu não pensava “Estou fazendo isso para essa ou
aquela pessoa”, não pensava desse jeito, eu fazia como uma prática. Se
os outros não fizessem, eu fazia. Era lucro para mim, era algo que me
dava bem-estar, alegria. Quando vinha o dia do \textit{uposatha} eu já
conhecia os protocolos\footnote{Deveres dos monges no preparar do local
para a cerimônia do uposatha.}, mesmo sendo um monge jovem. Eu ia
preparar o salão do \textit{uposatha}, varria, colocava água para
lavar, água para beber, todo tipo de coisa – tranquilo. Aqueles que não
conheciam os protocolos ficavam na deles; eu não dizia nada porque eles
não sabiam – isso eu tomei como prática. Eu fazia e me sentia alegre,
tranquilo. Preparava os assentos e tudo mais e, quando dava a hora,
vestia o manto e praticava meditação andando numa boa, alegre. A
prática ia bem e tranquila, tinha energia. Minha prática tinha energia,
muita energia. 

Onde quer que seja necessário trabalho no nosso monastério, quer
seja na nossa habitação ou na dos outros, onde estiver sujo – pode
limpar. Não precisa fazer para os outros verem, para se exibir, faça
por sua própria prática. Se fazemos assim é como se varrêssemos a
sujeira para fora dos nossos corações. Varrer as cabanas é como varrer
a sujeira para fora dos nossos corações pois somos praticantes. Que
isso esteja nos corações de todos nós. Não precisa muito, não precisa
pedir harmonia – surge sozinha. Que seja Darma dessa forma. Tranquilo,
tranquilo, pacífico, quieto. Ajudem uns aos outros. Procurem fazer seus
corações serem desse jeito e nada será capaz de perturbá-los. Quando
houver algum trabalho pesado que requeira que todos ajudem – ajudem.
Quando nos ajudamos é tranquilo, não leva muito tempo, ajudem e fica
fácil, logo está feito – essa é a melhor forma. O manto, a comida, as
cabanas, os medicamentos são os nossos suportes.

Eu também já passei por isso, mas saí lucrando. Fui morar onde havia
muitos monges, alguns deles diziam: “Ei, no dia tal vamos tingir os
mantos, OK? Ferva a tintura.” Eu fervia bem a tintura, alguns monges,
amigos meus, apareciam, lavavam o manto e iam embora em seguida,
penduravam para secar e iam dormir numa boa. Não ajudavam a ferver a
tintura, não ajudavam a lavar a panela, não ajudavam a arrumar nada;
eles pensavam que eles desse jeito estavam levando vantagem. É o cúmulo
da burrice, acumula ainda mais burrice em si mesmo porque pensa: “Não
trabalho e os amigos arrumam tudo, vou só quando estiver pronto.” Isso
é o cúmulo da burrice, aumenta a burrice ainda mais. Pode ver: aquilo
não gerava benefício algum para eles, olhem bem isso. Isso é o
pensamento ignorante das pessoas. Elas pensam: “Nós somos bons, limpos,
onde der para se esquivar, ficamos ali mesmo. Não precisamos trabalhar;
o trabalho que tem que fazer, não fazemos. Se der para se esquivar, é o
que há de melhor.” Isso é o cúmulo da burrice. Temos que ter essa
opinião nos nossos corações, e aí não vai ser possível ser assim. Eu já
fui assim, já me deparei com isso, já vi isso, tendo visto, entendi: “É
assim.” Quando era hora de ficar com esse grupo, eu ficava. Quando era
hora de ir embora, eu ia, não tinha problema algum. Portanto as coisas
erradas e certas são os assuntos da nossa prática. 

Numa época vieram alguns amigos de prática, eu já contei essa
história várias vezes. Naquela época o Venerável Sumedho veio e o
doutor… como é o nome? Esses europeus vindo me procurar e eu nunca os
tinha visto antes. Apenas tinha ouvido falar que o país deles era muito
divertido, mas nunca havia visto. Eu pensei: “Agora eles terão
oportunidade de ficar conosco”, quando ouvi a notícia… Dr. Burns! O
nome era Dr. Burns. Eu fiquei incomodado, com medo, fiquei com medo:
“Xiii, eles vão morar comigo, eles só têm prazer, tudo confortável,
como eles vão conseguir ficar aqui? Comida quase não tem, não é boa,
molho de pimenta com peixe azedado… Verduras e legumes comuns da roça…”
Fiquei incomodado, pensando desse jeito. Então eles vieram e: “Hummm…”,
consegui refletir: “Hummm… tente pensar desse jeito: Eu só tenho dez
dedos na mão, cinco em cada mão. Se eles pedirem para ver minha mão com
dez dedos cada, eu tenho para mostrar? Cada mão tem cinco dedos; se
eles quiserem ver uma mão com dez dedos, sete dedos, oito dedos, eu vou
ter para mostrar? Eles vieram para a Tailândia, por quê? Porque vieram
estudar o budismo, estudar a cultura tailandesa da Tailândia: ‘Como é
que os tailandeses moram, como é que eles comem, o que eles fazem.’
Eles querem conhecer. Se eu arrumar tudo para eles, eles vão ficar é
burros. Se ficarem comendo pão\footnote{Originalmente não havia
produção de trigo na Tailândia. Pão era considerado um artigo de luxo.
Atualmente, com o processo de ocidentalização, pão é um produto comum e
pode ser encontrado em qualquer lugar.}, vão ter ganho algo para quando
voltarem para casa? ‘Como é na Tailândia? Como eles praticam?’, eles
não vão saber nada. Vai ser causa para que sejam burros desse jeito.
Portanto eles podem ficar.” Então ficaram. Foi só pensar assim e fiquei
feliz, tranquilo, mas para eles foi difícil. Como o Venerável Jagaro,
esse pessoal aqui, vem para cá e tem problemas de digestão. Ficam
doentes um ano, dois anos – eu também fico com pena, mas no final eles
ficam mais espertos, dá resultado. Eles aprendem várias coisas, eles
praticam, eles conhecem, passam a conhecer o método de prática do
budismo na Tailândia – muito bom. 

Tem que investir… Tem que investir. Portanto, quando for falar ou
fazer algo, tem que haver a sensação: “O que é que vim fazer aqui?”
Queremos comer bem, dormir bem, esse tipo de coisa, mas “O que é que
vim fazer aqui? Qual o propósito de ter vindo até aqui?”, temos que
saber. Se pensarmos assim sempre, vai despertar nosso coração o tempo
todo. Não perde atenção, fica desperto o tempo todo. Quer esteja de pé,
você estará praticando. Andando, estará praticando. Deitado, estará
praticando. Se estiver praticando é assim. Aqueles que não praticam não
são assim. Sentados é como se estivessem em casa. De pé, estão em casa.
Não demora, vão brincar em casa, vão brincar com as pessoas em
casa\footnote{Vão deixar o monastério e voltar à vida laica.}, ficam lá
arrumando confusão com elas. O coração fica desse jeito, não pratica.
Ficam assim, sem treinar e controlar a própria mente, largam a mente ao
vento, às emoções. É o que se chama “seguindo as emoções”. Como uma
criança em casa, se fizer tudo o que ela quer, é bom? O pai e a mãe
fazer tudo que a criança quer, é bom? Se fizer tudo seguindo suas
emoções, é como uma criança. Se tem inteligência, tem que se irritar
com isso, pois é burrice. 

Tem que ser assim, treinar a mente tem que ser assim, tem que se
sentir assim, tem que conhecer, tem que saber treinar sua própria
mente. Se nós não treinarmos nossa mente, esperar que os outros o façam
por nós é difícil, muito difícil. Por isso, agora eu já me decidi:
quando vier até Wat Pah Pong aqui, de agora em diante, vou me esforçar,
já me decidi. Por quê? Minha vida já não vai durar muito, vou me
esforçar para ensinar aqui. De manhã, na hora da \textit{p\=uja}, deixo
a cargo de Ajahn Chu e Ajahn Liem ajudarem. De noite, deixe para mim,
eu lidero os monges e noviços aqui. Vou tentar fazer com regularidade.
Já me decidi assim, já refleti. Vou tentar fazer mesmo que ninguém
venha, vou tentar fazer assim. A oportunidade é essa, é uma boa
oportunidade. Quando penso no Buda lá atrás, vejo que é correto. Velho
daquele jeito, abriu mão da vida, do agregado corporal e do mental –
largou. Se for para morrer, que morra dentro dessa nossa prática. Isso
é suficiente.

Tem que se esforçar, todos os monges e noviços, tem que refletir,
tem que se esforçar na prática de \textit{sam\=adhi}, tem que haver
continuidade. Se vocês praticarem \textit{sam\=adhi} com frequência, a
\textit{sīla} de vocês vai melhorar, vocês terão atenção e cuidado.
Se vocês tiverem atenção e cuidado, a \textit{paññ\=a} de vocês vai
melhorar. \textit{Paññ\=a} melhorando, \textit{sīla} melhora.
\textit{Sīla} melhorando, \textit{sam\=adhi} melhora.
\textit{Sam\=adhi} melhorando, \textit{paññ\=a} melhora. É um ciclo
desse jeito. Esse é o dever que temos que cumprir, temos que nos
esforçar. Se houver alguma dúvida no modo de prática, que não sejam
muitas, não fique duvidando muito. Ao invés disso, pratique muito. Na
minha prática já tive muitas dúvidas, mas não tinha muita coragem de
pedir respostas para as dúvidas. O mestre já havia ensinado, mas eu
ainda não sabia por mim mesmo. Ele falava de acordo com a opinião dele.
Ainda não resolvia, mas era bom, era um ponto de referência que eu
tinha para investigar. Não me deixava influenciar pelas minhas opiniões
ou as dos outros, ia estudando até ver por mim mesmo.

Não pense que, estando aqui, não está praticando: prática não possui
obstáculos. Dá para praticar de pé, sentado ou deitado; mesmo varrendo
e limpando o monastério é possível alcançar a iluminação. Mesmo olhando
um raio de sol é possível alcançar a iluminação. Não precisa ir sentar
de olhos fechados o tempo todo, cuidado! Quer estejamos de pé,
sentados, deitados, onde quer que seja, temos que estar prontos. Por
que tem que ser assim? Porque assim há a oportunidade de alcançar a
iluminação a qualquer momento, em qualquer lugar em que tenhamos
determinação, quando estivermos contemplando. Por isso, não sejam
descuidados, não sejam descuidados. Cuidado, estejam cientes. 

Nos sentamos, saímos em \textit{pindap\=ata}, surgem várias
sensações antes de voltar para o monastério. Bem ali tem vários Darmas
ótimos surgindo. Quando voltamos para o monastério e nos sentamos para
comer, tem vários Darmas ótimos surgindo para conhecermos e estudarmos.
Conhecer que é assim que isso nasce. Se tivermos dedicação o tempo
inteiro, não vai ser de outro jeito. Vai haver ideias, problemas,
Darmas, vai haver \textit{dhammavicaya}, investigação do Darma, o tempo
todo. Será \textit{bojjhanga}, “\textit{Bojjhango sati-sankh\=ato
dhamm\=anam vicayo tath\=a. Viriyam-pīti-passaddhi-bojjhang\=a ca
tath\=apare. Sam\=adh’upekkha-bojjhang\=a satt’ ete sabba-dassin\=a.
Munin\=a sammadakkh\=at\=a bh\=avit\=a bahulīkat\=a.
Sanvattanti…}”\footnote{Trecho da Bojjhanga P\=arita.} Na medida que
estudamos, somos eruditos. Estamos estudando alguma coisa? Estamos
estudando esses objetos mentais. Darma vai nascer nesta mente, não
precisa ir estudar com ninguém, em lugar algum, estude bem aqui. Sempre
que tivermos \textit{sati}, temos o que estudar. 

“\textit{Bojjhango sati-sankh\=ato dhamm\=anam vicayo}” – se tiver
\textit{sati} tem \textit{dhammavicaya.} Eles estão em contato direto,
são fatores da iluminação. Se temos \textit{sati}, temos
\textit{dhammavicaya,} não ficamos à toa. É fator da iluminação: se
estiver nesse sistema, alcança a iluminação. Está bem aqui, eu acredito
firmemente nisso: aqui dentro da nossa mente, a prática, a contemplação
não tem dia ou noite, não há horário. Nos esforçamos ao máximo, é
assim. Outros assuntos não atrapalham. Se estiverem atrapalhando,
notamos, corrigimos. Tem que virar \textit{bojjhanga}, o fator da
iluminação, ter \textit{dhammavicaya} no coração o tempo todo,
investigar o Darma o tempo todo. Por isso \textit{bojjhanga} são os
fatores da iluminação. “\textit{Bojjhango sati}” – tem \textit{sati},
“\textit{dhamm\=anam vicayo}” – surge investigação do Darma o tempo
todo. 

Quando a mente é assim, caso ela já tenha caído na correnteza, ela
não investiga outras coisas: “Vou passear por lá, vou passear por aqui,
por ali, aquele município, aquela província”, não vai desse jeito. Isso
é se perder no mundo, então morre ali mesmo. Não vai desse jeito. Se
for, morre e é enterrado bem ali. Há lucro e prejuízo o tempo todo, tem
que pensar assim. Por isso se esforcem. Não é só sentar e fechar os
olhos e vai haver sabedoria. Olhos, ouvidos, nariz, língua, corpo,
mente estão sempre presentes; estamos sempre acordados, estamos sempre
cientes, estudando o tempo todo. Do lado de fora vemos uma folha de
árvore, vemos os animais e estudamos o tempo todo. Trazemos para dentro
– é \textit{opanayika-dhamma. }Vemos claramente em nós mesmos – é
\textit{paccatam}. Mesmo que haja objetos de fora entrando em contato
conosco, é \textit{paccatam}, não há desperdício. 

Falando de maneira simples, é como um forno de produzir carvão, já
viram? Forno de produzir tijolo, já viram? Se fizerem direito, acendem
o fogo à frente do forno, cerca de um metro, e toda a fumaça é sugada
para dentro do forno. Podem olhar, de onde vem isso, por que falo
assim? Vemos claramente assim, é a forma de comparar. Se fizer direito,
um forno de produzir tijolo vai ser assim: acendemos o fogo na frente
do forno, cerca de dois ou três metros; quando surge a fumaça ela é
sugada para dentro do forno por completo, não sobra nada. O calor se
acumula no forno, nada escapa, tem que ser assim. Como a fornalha de
cremar defunto: quando eles acendem, o fogo é sugado todo para dentro,
o calor vai atuar o mais rápido possível. É assim a sensação do
praticante, é a mesma coisa. Quando é assim, qualquer sensação é sugada
para dentro de \textit{samm\=aditthi}. É só os olhos verem, os ouvidos
ouvirem, o nariz sentir um cheiro, a língua sentir um gosto, tudo é
sugado para dentro e vira \textit{samm\=aditthi}, é assim que é. Isso é
um símile, tenho que explicar desse jeito.

Que vocês levem a sério, quer seja monge residente, quer seja
visitante recém-chegado, quer seja monge recém-ordenado, quer seja
candidato à ordenação – tem que entender dessa forma. É o modo de
prática aqui em Wat Pah Pong. Tem que conhecer o modo de se comportar
aqui em Wat Pah Pong. Eu estando aqui, não me envolvo muito com
ninguém, com os discípulos. Algumas pessoas podem não entender,
pensando: “Ele não pergunta nada, ele não conversa nada”, não sejam
burros a esse ponto. Tome Pó Kao Pân\footnote{“Pó Kao” (pai branco) é o
nome utilizado na região nordeste da Tailândia para se referir a homens
que moram no monastério e observam oito preceitos (anag\=arika).} como
exemplo. Nunca perguntei, esse ano fui perguntar: 

“Pó Pân, de onde você é?” 

“De Nong Pen Nok.”

“E há quanto tempo mora aqui?”

“Oito anos.” 

“Já mora há oito anos, ainda não está entediado?” 

“Ainda não estou entediado.”

“Como pode ser que ainda não esteja entediado!?”

“É porque decidi vir morrer aqui.” Isso é muito bom de ouvir:
“Decidi vir morrer aqui.” 

“E até que ponto você vai praticar? Como é que vai ser?”

“Vou até encerrar.” Este ano perguntei a Pó Kao Pân:

“E já está chegando perto?”

“Já está chegando.” 

Isso eu ouvi sair da mente de Pó Kao Pân. Ele veio morar aqui, nunca
reclamou: “Luang Pó, minha prática está confusa, eu quero virar monge,
eu quero virar noviço…” Nunca. Nunca ouvi: “Eu estou desse, daquele
jeito…” Nunca ouvi. Mesmo doente, eu só pergunto:

“Tem dinheiro para usar?” 

“Tenho um pouco.” 

Não sei se tem ou não tem, sei lá. Os filhos vêm visitar ou não –
nunca me interessei. Oito anos até ter perguntado, perguntei este ano.
Isso é o que se chama “árvore na floresta”, não precisa regar, não é?
Ela suga a energia do sol, usa o sol como adubo, não morre – natureza é
assim.
