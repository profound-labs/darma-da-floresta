
\chapter{Nota do tradutor}
\markright{Mudito Bhikkhu}

Em Abril de 2012 comecei um trabalho de tradução, inicialmente
somente de textos de Ajahn Chah e seus discípulos e mais à frente
também de outros mestres tailandeses, discípulos diretos ou indiretos
de Tahn Ajahn Man Bhuridatto. O resultado pode ser encontrado no site
\href{http://www.dhammadafloresta.blogspot.com/}{www.dhammadafloresta.blogspot.com},
onde é possível ouvir a gravação original em áudio de todos os
ensinamentos contidos neste livro, assim como ver fotos dos mestres que
os proferiram. 

Com o tempo o número de palestras traduzidas no site foi aumentando
até que, seguindo sugestão de uma pessoa do site
\href{http://www.ouvindoodhamma.com}{www.ouvindoodhamma.com}, foi iniciado o trabalho de seleção de um
pequeno grupo de textos para publicação em forma de livro. Foram
selecionados oito textos de Ajahn Chah e um de cada um de seus
discípulos cujos ensinamentos constavam no site. A proposta do projeto
era transformar o conteúdo do site, que havia sido traduzido com ênfase
em ser o mais fiel possível às palavras dos mestres, num texto mais
amigável ao formato escrito e a um público iniciante. 

Neste livro, em vez do idioma páli, optamos por fazer uso máximo da
grafia portuguesa de palavras relacionadas ao ensinamento do Buda que
já foram incluídas oficialmente na nossa língua. Nominalmente: Buda
(\emph{Buddha}), Darma (\emph{Dhamma}), carma (\emph{kamma}),
nirvana (\emph{nibbāna}), parinirvana (\emph{parinibbāna}),
bodisatva (\emph{bodhisatta}) e páli (\emph{pāli}). Todos os
demais termos ou foram traduzidos para o português, ou foram utilizados
em suas grafias originais, em páli. Apesar do esforço em traduzir o
máximo possível de palavras em páli para o português, uma grande
quantidade delas permaneceu em seu formato original. Em parte isso se
deveu à dificuldade em encontrar correlatos satisfatórios em português
e em parte pelo contexto em que os ensinamentos foram dados, como vai
ser explicado mais adiante.

Vale a pena mencionar que todos os ensinamentos de Ajahn Chah
publicados aqui são inéditos: jamais foram traduzidos ou publicados em
qualquer outra língua ocidental (a maioria deles nem mesmo em
tailandês) e o mesmo é verdade para a maior parte dos textos de seus
discípulos incluídos no livro. 

No que diz respeito à grafia de termos em tailandês, procurei manter
o padrão mais aceito para evitar confusões desnecessárias. Não existe
um só padrão para romanização de palavras tailandesas que realmente
funcione e por isso diferentes pessoas utilizam diferentes métodos. Por
exemplo, alguns escrevem “ajaan,” outros “ajahn,” outros “atchan” e
assim por diante. Não quero eu também inventar ainda mais um sistema de
romanização, mas uma vez que os sistemas mais populares foram
produzidos para um público anglófono, algumas distorções se apresentam
para nós que utilizamos a língua portuguesa e por isso eu abandonei a
grafia mais usual de algumas palavras: “Luang Por” (\thai{หลวงพ่อ}) foi escrito “Luang Pó,” “Ajahn Dtun” (\thai{อาจารย์ตั๋น})
foi escrito como “Ajahn Tan” e “Ajahn Mun” (\thai{อาจารย์มั่น})
como “Ajahn Man.” 

Um aspecto muito importante do conteúdo deste livro que precisa ser
mencionado é “contexto” – quem disse, para quem foi dito, quando, por
quê, etc. Em especial, quando falamos sobre mestres da tradição da
floresta tailandesa, discípulos de Ajahn Man, entender o contexto se
torna ainda mais importante, pois esses eram mestres que
verdadeiramente ensinavam de coração. A prática deles não ocorreu
dentro de um mundo fechado de livros, de conceitos e teorias: eles
usavam suas próprias vidas como combustível na busca do Darma e por
vezes levavam a prática às suas últimas consequências – muitos deles
chegaram à beira da morte (incluindo Ajahn Chah). 

E não é só essa característica que vem à tona quando ensinam; também
o fato de terem verdadeiramente experienciado em si mesmos níveis
elevados de realizações espirituais faz que seus ensinamentos tenham,
além de espontaneidade, um tom de autoridade que alguns podem achar
incômodo. Por essa razão, acho importante lembrar que os ensinamentos
contidos aqui, em sua maioria, ocorreram num ambiente onde havia um
real clima de amizade e confiança entre mestre e discípulos, onde o
mestre se sentia à vontade para falar abertamente sobre o assunto e em
utilizar uma linguagem mais sofisticada (pois sabia que a audiência
estava familiarizada com termos em páli). O resultado final é
um tanto paradoxal: ao mesmo tempo em que o ensinamento é extremamente
simples e direto ao assunto, também apresenta uma sofisticação de
linguagem que pode exigir do leitor um pouco de esforço em compreensão.
Para ajudar, muitas notas ao longo do texto – e também um glossário ao
final do livro – foram incluídas.

Outro aspecto relevante, principalmente quando o ensinamento é
destinado à comunidade monástica, é o nível de desenvolvimento
espiritual da audiência. Por exemplo: para uma plateia de iniciantes,
Ajahn Chah diz que as pessoas sofrem e não conhecem a si mesmas por não
possuirem \emph{samādhi.} Em outra ocasião, onde os presentes
talvez fossem já bastante hábeis na prática de \emph{samādhi}, ele
critica duramente e diz que \emph{samādhi} “é o cúmulo da burrice.”
Isso deve ser entendido dentro do contexto: um grupo ainda não chegou
àquele nível, outro já chegou e agora precisa ser encorajado a
continuar seguindo adiante e não apenas contentar-se com o estágio
alcançado. A incompreensão desse fator já levou algumas pessoas (em
especial alguns professores leigos nos Estados Unidos) a defender que
Ajahn Chah tanto não ensinava como sequer elogiava a prática de
\emph{samādhi} – nada poderia estar mais longe da verdade.

Por fim, devo mencionar que a compilação deste livro só foi possível
graças à ajuda de Angelo De Vita, com um trabalho importantíssimo de
revisão e consulta, e também Venerável Dhammiko, Gabriel Laera,
Jaqueline Szili, Jair Seixas e Carla Schiavetto, que ajudaram a revisar
e dar sugestões para a melhoria do texto final. Ainda assim, peço que
quaisquer omissões ou erros sejam de minha inteira responsabilidade.
Que o mérito dessa realização seja causa para que todos progridam no
Dhamma, em direção à paz e libertação – nibbāna.

Anumodanā.
\bigskip

{\raggedleft
Mudito Bhikkhu

Wat Khao Sam Lan\\
Nakhon Nayok, Tailândia
\par}

