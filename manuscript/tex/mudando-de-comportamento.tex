
\chapterAuthor{Ajahn Chah}
\chapterNote{Ensinamento dado à comunidade monástica de Wat Pah Pong.}
\chapter{Mudando de Comportamento}
\tocChapterNote{Ensinamento dado à comunidade monástica de Wat Pah Pong.}
\markright{\theChapterAuthor}

\ldots{} \emph{upajjhāya}. Tendo pedido tutela ao
\emph{upajjhāya}, se não houver oportunidade de viver com ele, o
\emph{upajjhāya} então o delega a um professor, um que seja
bem-comportado, ensine bem e pratique bem. O aluno pede para viver
junto a ele e pede tutela àquele professor, vive sob a tutela dele.
Essa é uma fundação muito importante. Podemos comparar, quer seja no
mundo ou no Darma. Atualmente é assim: o aluno que quer estudar deve
permanecer na escola por cinco anos; após cinco anos ele vai estudar em
outro local. É igual no mundo. Todas as pessoas têm seu comportamento,
carregam seu próprio comportamento. Pois esse comportamento existe
graças ao poder dos desejos, da ganância. Ganância surge tendo
ignorância como fundação e então ela tem a oportunidade de se apoiar
naquele comportamento. 

Quando nos ordenamos monges temos que mudar de comportamento, temos
que pedir tutela ao \emph{upajjhāya}, ao professor, para que ele
nos supervisione sempre. Não deve se ausentar do professor. Caso se
ausente por dois dias, deve pedir tutela novamente, pois a anterior se
quebrou. Os monges e noviços devem agir assim. Caso se ausentem, mesmo
que só dois ou três dias, a tutela se quebra. Quando o mestre, o
\emph{upajjhāya} ou o professor vier, tem que renovar a tutela, tem
que pedir tutela novamente.\footnote{Se o discípulo se ausenta sob
permissão do professor, não é necessário pedir tutela novamente, pois
esta ainda está vigente.} É feito dessa forma para que fique mais
atento, mais cuidadoso. Não seja descuidado em pedir tutela, não seja
descuidado em seguir tutela, porque ainda está sob tutela. Temos que
contar com a supervisão do \emph{upajjhāya} e do professor até o
quinto ano de vida monástica.\footnote{A regra monástica sugere um
período de tutela de cinco anos, mas esse período pode ser estendido
caso o professor entenda que o aluno ainda não é maduro o suficiente
para cuidar de si mesmo.} 

Após cinco anos já saberá o suficiente; caso tenha morado com um
\emph{upajjhāya} ou um professor que pratica de forma correta,
conhecerá o modo de prática. Conhece e então sente vergonha e
medo,\footnote{Esse é um aspecto do treinamento frequentemente
enfatizado pelo Buda. Em \emph{páli}: “hiri” e “ottappa.”} sabe o
que é uma ofensa à regra monástica com profundidade. Quer esteja de pé,
andando, sentado, deitado, onde quer que vá, não age como bobo, não
pensa à toa, não faz à toa. Mas ainda há ganância, raiva, ainda há de
tudo, mas está ciente, não tem coragem de fazer aquilo, não age sob a
força dos desejos. Não importa o quão doloroso e difícil seja, não tem
coragem de fazer. Por quê? Medo. Medo e vergonha. Tem vergonha do que é
errado, não tem coragem de fazer, não tem coragem de dizer. Tem
vergonha, medo. Vergonha e medo transformam-se em \emph{sīla}
impecável. Tornam-se hábito e no final a mente se pacifica. 

Tem que se tornar uma pessoa de hábitos de comportamento e fala
diferentes do que eram antes. Muda do comportamento velho para um
comportamento novo. Comportamento de monge, de noviço, comportamento de
asceta. Muda de comportamento. As pessoas que praticam também, pode
olhar nossos monges e noviços, se já possuem um comportamento
estabelecido, têm que mudar aquele comportamento. Para mudar de
comportamento, a mente tem que mudar. Se tentar mudar sem mudar a
mente, a mente não toma conhecimento e não consegue mudar. Pode ouvir
todo dia sem falta, mas não consegue mudar. Por quê? Porque a mente
ainda não tem medo, não tem vergonha, ainda não adquiriu novo
comportamento. Se adquiriu o novo comportamento ela sente vergonha e
medo, não tem coragem de falar, muda todo o sistema de corpo e fala
daquela pessoa. Muda tudo, a não ser o que é \emph{vāsanā}. 

Como por exemplo o Sāriputta. Esse \emph{vāsanā} os
\emph{sāvakas} largam com facilidade, mas só em parte. Não como o
Buda, só ele consegue largar por completo, consegue abandonar qualquer
forma de comportamento. Esse \emph{vāsanā} é o comportamento que
as impurezas mentais condicionaram em vidas passadas, elas
condicionaram por várias vidas. Viemos agindo dessa forma até ficarmos
habituados, falamos até ficarmos habituados, agimos até ficarmos
habituados, até virar nossa personalidade, até entrar nos ossos, até
fixar-se firme nos ossos. Como um bêbado. Bebe todo tipo de bebida, é
um boêmio de longa data, mesmo quando não bebe, anda como se estivesse
bêbado. Sempre que olham pensam que está bêbado: fala como bêbado. Isso
é porque já se fixou firme na mente, nos ossos, já virou parte da
personalidade dele. Mesmo quando não bebe é um bêbado, mas na verdade
não há nada, não está embriagado. Isso não é possível mudar.
\emph{Vāsanā} é a mesma coisa. 

Como o Sāriputta. No passado ela era um macaco. Ele nasceu como
macaco, e os macacos gostam de correr, balançar, pular para frente e
para trás. O Sāriputta também não conseguia largar esse hábito, esse
\emph{vāsanā}. Às vezes andava até uma poça de água e pulava,
“pum!”, como um macaco. Mas era só um pulo, não havia nada de mais. Não
tinha a ver com nada. Tinha a ver com o \emph{vāsanā} dele ser
assim; quando fazia, não havia nada de mais. Não gerava impureza, não
gerava ganância, maldade ou demérito. É o que se diz, a mente se
acostumou àquele hábito. 

Isso é a mesma coisa. Se viermos praticar querendo encontrar paz,
querendo praticar, temos que abandonar nossa forma anterior de ser.
Mudar nosso corpo, nossa fala, nossa postura, nosso modo de falar,
nosso modo de agir. Muda tudo. Vai mudando de pouco em pouco. Isso é o
que se chama pedir tutela e ganhar um novo hábito. Quer dizer, jogamos
fora o hábito antigo, a forma antiga de falar, a forma antiga de
brincar, a forma antiga de fazer piadas, jogamos fora. Ao ver isso ele
joga fora. Ao jogar fora, ele muda. Abandona o demérito, pratica o
mérito, abandona a maldade, pratica a bondade, abandona o comportamento
antigo e toma um novo. Quando o novo comportamento vem, ele muda a
mente por completo. Sabe sentir medo, sabe sentir vergonha, sabe todo
tipo de coisa. Isso já é uma nova vida. É como nascer em uma nova vida.
Morre daquela vida e nasce nessa outra. Morre daquele comportamento e
nasce nesse outro. São duas pessoas diferentes, mas são a mesma pessoa:
morreu e nasceu de novo. Muda, muda, muda. 

De acordo com o nível de prática, chamam-se \emph{Sotāpanna,
Sakadāgāmi, Anāgāmī, Arahant.} Esses são nomes de pessoas
que conseguiram remover impurezas mentais. São a mesma pessoa, não são
várias pessoas, são a mesma pessoa. Existem vários nomes porque eles
mudam de acordo com a característica da mente, tomando a capacidade de
abandonar e remover impurezas mentais como ponto de referência. Quando
remove aquelas impurezas, aquele comportamento, não há diploma como há
no mundo. Quando faz contato com aquele conhecimento, muda, abandona,
fica diferente. É o que dizemos “nasce novamente,” muda. Pode chamar de
“nascimento \emph{aryia}” se quiser. Ele já escapou do inimigo
anterior, aquilo que passou era um inimigo. Aquilo que nos fazia
degenerar se acaba. Não há mais inimigos, já os ultrapassamos. Os
\emph{aryias} ultrapassaram os inimigos, escaparam dos inimigos, o
coração escapou dos inimigos, a fala escapou dos inimigos, o corpo
escapou dos inimigos do ensinamento do Buda, do que é belo e bom.
Escapa. É um inimigo e nós escapamos dele. Matamos o que é inimigo à
nossa sinceridade. 

No começo é como nascer de novo – ganha um novo comportamento, adota
um novo comportamento. Os monges também. Pode olhar aqui mesmo: eles
praticam, praticam, até virar hábito. Podem comparar como são agora com
o que eram no passado – é muito diferente. Mudam de sensação,
pensamento, são completamente diferentes. Isso se chama “comportamento
interior.” Quando o Buda diz que uma pessoa já não precisa de tutela, é
porque a mente já não pratica mais o mal. Por isso se diz que o
\emph{Sotāpanna} é uma pessoa que não volta. É uma mente que não
volta, é uma mente que caiu na correnteza do nirvana. Não chega a oito
vidas.\footnote{De acordo com as escrituras, um sotāppana alcança
iluminação em no máximo 7 vidas.} Aqui podemos observar que o que
mudou foi o comportamento. Quando surge sabedoria, é capaz de remover
essa impureza. Quando tem ainda mais sabedoria do que isso, não
consegue mais se apegar, acontece sozinho. Aquele comportamento muda. 

Vê a coisa velha e joga fora, como se fôssemos à floresta e
entrássemos para colher frutas. Vemos as frutas que ainda estão verdes
e colhemos, enchemos a sacola. Em seguida vemos frutas que estão
melhores que aquelas, então temos que jogar fora as frutas verdes.
Quando vemos assim, jogamos fora, mudamos. Nossa mente é igual.
Mudamos, vemos a desvantagem do antigo e abandonamos ainda mais,
jogamos fora. Quanto mais vemos, mais abandonamos. O que estamos
praticando aqui – se pensamos que está correto, que é bom – é um
aprendizado contínuo, vamos abandonando continuamente. Por isso o nome
vai de acordo com a prática, com o nível da mente que vai mudando,
mudando. Chama-se comportamento. 

Então, todos nós também temos que mudar nosso comportamento antigo.
Quando mudando de comportamento, no começo temos que obrigar, mas em
seguida não é preciso obrigar, a mudança ocorre sozinha. No começo
temos que nos adaptar, ensinar, aconselhar a nós mesmos, aguentar.
Vamos praticando até conseguirmos ficar de pé sozinhos, então não há
mais problemas. Vira nossa própria natureza, nossa personalidade. Nasce
uma sensação firme na mente, não admitimos mais, não admitimos fazer
errado, não admitimos fazer o mal. 

Então, o \emph{Sotāpanna} é uma pessoa que não volta,
\emph{sotā patipanno} – pessoa que segue adiante, não volta, segue
adiante, vai continuamente. Como ir a Ubon ou Warin, quanto mais
formos, mais próximos ficamos, mais próximos ficamos até chegar a Ubon.
É uma pessoa que flui adiante, \emph{sotā} significa fluir. Como os
rios e pequenos riachos, que fluem em direção ao oceano. É a mesma
coisa. Isso sabemos no aqui e agora, não dependemos da aprovação de
ninguém, basta praticarmos e saberemos. Aquilo que é pesado fica leve,
aquilo que é difícil fica fácil. Aquilo de que gostamos, mas que quando
fazemos gera malefício, não temos coragem de fazer. Portanto, aqueles
que praticam o Darma, se forem leigos, são diferentes dos demais
leigos. Se forem monges, são diferentes dos demais monges, graças à sua
prática. Em seus modos, seu ir e vir, comer, em tudo que fazem, não há
bobagem, não brincam. A mente fica hábil. No começo o mestre ajuda a
obrigar, nós nos obrigamos a treinar. Com o passar do tempo a mente
enxerga e então mudamos. Quando ela muda, não quer mais voltar a ser
como antes. Não quer voltar ao que era antes, pois é errado, não está
certo. Portanto, o \emph{Sotāpanna} consegue remover impurezas
desse jeito e é uma pessoa que não volta. Não volta a fazer maldade,
não volta a causar destruição. Mesmo que ele queira fazer, queira
falar, queira o que seja, não segue aquele impulso. Se chama “uma
pessoa que não volta.” 

Todos nós também, podem olhar, nos ordenamos e, se praticarmos,
vamos mudar. Mudamos nosso comportamento antigo, nossa sensação antiga,
nosso modo de falar antigo. Tem que mudar, muda continuamente, tem que
mudar. Por quê? Porque estamos atentos, temos cuidado, chama-se
“\emph{sīla samvara}”. \emph{Sīla} é aquilo que controla,
para que tenha medo, vergonha, até se desenvolver e não querer mais
fazer o que é ruim, o que não é bom, graças à \emph{sīla. Sīla}
nos faz mudar de comportamento, nos faz ver claramente: “Isso está
errado, aquilo está certo,” nos faz ter vergonha e medo. Esses dois,
medo e vergonha, são a totalidade de \emph{sīla}. Quer seja
\emph{sīla} de monge, noviço, leigo, todas as formas de
\emph{sīla}. Se não tiver vergonha e medo, fica tudo sujo. 

Portanto, aqueles que são sábios não sentem dificuldade em observar
\emph{sīla}. Se não entendemos \emph{sīla}, sentimos
dificuldade em observar. Não precisa ler a regra monástica do começo ao
fim: “Se fizer isso é uma ofensa, se fizer aquilo é uma ofensa,” tentar
fazer dessa forma é muito difícil, não tem como não ser. Na verdade os
sábios observam \emph{sīla} resguardando suas mentes para que
sejam dessa forma normalmente e pronto, isso já é \emph{sīla} por
completo. Já é \emph{sīla}. Portanto, quando a prática chega ao
nível mais alto, a prática de \emph{sīla} se refina ainda mais, o
cuidado, a atenção, pois a mente fica hábil, a mente vê. Não precisamos
forçá-la, ela é correta naturalmente, ela vê por si claramente, então
se distancia ainda mais do que é ruim. Ela vê que a prática, que todas
essas observâncias, têm muito valor. Vemos o mal naquilo que fazíamos e
que não é bom. Quando vemos o mal naquilo, podemos praticar. Qualquer
coisa em que ainda não vimos o mal, não conseguimos abandonar. Qualquer
coisa em que ainda não vimos o benefício, não queremos praticar. Na
hora de abandonar vemos o malefício, na hora de praticar vemos o
benefício, assim fica fácil. É como se fôssemos puxados. Vemos o
malefício e fugimos dele, vemos o benefício e vamos em busca dele.
Quando vê o malefício dessa forma, a sensação, o pensamento que surge
naqueles sábios que se dedicam a essa prática é: “Quando conseguirei
escapar disso?” Surge desencanto, desilusão, procuram um caminho para
não voltar a nascer, não buscam mais nascimento, buscam aquilo que não
nasce, tentam encontrar aquilo que não nasce mais. Procuram abandonar,
procuram remover, observam e verificam seus corpos, fala e mente, onde
estiver errado – praticam bem ali, se esforçam bem ali. É assim mesmo. 

Portanto, esse comportamento que o Buda ensinou, deixou para nós,
está muito correto. Quem não tem esse comportamento tem que mudar o
antigo. Ainda tem o comportamento antigo, portanto tem que mudar de
comportamento. Se o praticante quiser seguir de acordo com o Darma,
deve mudar. Quando sabe e vê, ele muda. Aquele monge fica diferente dos
demais monges, o noviço fica diferente dos demais noviços, o leigo fica
diferente dos demais leigos, pois é algo diferente. 

Aqueles que praticam o Darma têm que conhecer moderação. Mesmo em
casa, por exemplo, tendo uma esposa ou esposo, ainda consegue tornar-se
um \emph{Sotāpanna}, mas tem que conhecer moderação. O
\emph{Sotāpanna} conhece a moderação dele, não é uma pessoa que não
conhece moderação, ele conhece moderação. Conhece moderação de que
forma? Tendo sede, bebe água e chega. Tendo fome, come e chega. Isso é
conhecer moderação. No nosso grupo é assim: quando comemos, chega.
Beber álcool, drogas, cerveja, o \emph{Sotāpanna} não vai nessa
direção. Se tem uma família, toma deleite somente em sua família, quer
dizer, não admite buscar diversão nas diversas formas de sensualidade.
Não busca esse tipo de diversão. É como se ele caminhasse até cansar
parasse neste mundo só um minuto para descansar e depois fosse embora.
Só isso. Não vê como algo bom, algo elevado, não vê como algo especial,
ele vê como algo normal. Viver, comer, para aquele que pratica, é
tranquilo. Vivendo em casa ele é diferente das demais pessoas, o modo
de viver é diferente dos demais, vive com felicidade. Se vive no
monastério é a mesma coisa. Se é monge também é diferente dos demais
monges. O modo de ser dessa pessoa é diferente das demais. Ela possui
um coração saudável. Não tem coragem de fazer o mal, respeita o Buda, o
Darma e a Sangha ao máximo. 

Prática é a mesma coisa: se formos desenvolver nossa mente, temos
que saber o que se passa na nossa mente, o que estamos sentindo. Tem
que ter medo, vergonha, não ter coragem de fazer o que não é nosso
dever. Nosso dever é apenas abandonar. Essa sensação nasce graças à
prática: estando ciente, abandona. Não é para estar ciente mas não
abandonar, isso é ficar só na teoria, não alcança o coração. É a mesma
coisa, o que estamos estudando juntos, desenvolvendo juntos, requer que
pratiquemos muito. Tem que ser muito. Não é só vir aqui no dia do
\emph{uposatha} e ouvir o sermão. Se praticarmos vamos ver, se
praticarmos vamos saber, até mudarmos de comportamento. Se mudamos não
precisamos mais ficar direcionando muito, só um pouco e a mente vai
sozinha. Por exemplo, se for \emph{Sotāpanna} não precisa mais
direcionar. Ele já caiu na correnteza, ele segue sozinho, ele consegue
ir sozinho. É uma pessoa que já não precisa de tutela. 

Hoje em dia quase não existe quem siga essa regra. Se ordenam e em
um ou dois dias já vão embora, seis ou sete dias e vão embora sem
avisar ninguém, de acordo com o que der vontade. Ainda não terminou a
tutela, ainda toma o comportamento antigo, o jeito antigo, a sensação
antiga e vai praticar. Ainda é um leigo, um dono de casa, não tem
comportamento de monge, pois ainda não mudou. Portanto o Buda disse que
se ainda não tem vergonha, não tem medo, não tem disciplina, mesmo que
tenha sessenta anos de vida monástica, ainda deveria estar sob tutela.
Mesmo que tenha setenta anos de vida monástica, ainda deveria estar sob
tutela. Deixamos a tutela quando conseguimos cuidar de nós mesmos, é
isso. No mínimo cinco anos. É verdade, se durante os cinco anos teve
determinação em praticar, se esforçou em fazer, estará bom. Quer seja
monge ou noviço, onde for viver será bom. Após cinco anos pode-se
dizer: “Dá pro gasto.” Sabe executar tarefas, é hábil em remover
impurezas mentais. Portanto o Buda determinou que houvesse a tutela.
Enquanto estamos sob tutela dependemos de outra pessoa, dependemos do
\emph{upajjhāya}, dependemos do professor. Ou, caso deixe aquele
professor, tem que buscar um monge de bom comportamento,
\emph{sīla} impecável, e pedir tutela a ele. Esse é o modo de
prática do Buda. Assim não degenera. Se agir assim, não se degenera, só
encontra progresso à frente. 

Bom, ajudem uns aos outros. Se ajudarmos alguém a ficar dentro da
ordem monástica, ele se beneficia ao máximo. É fácil de ensinar, não é
um fardo. Não é um fardo para o professor nem para os demais monges,
pois já sabe o que é errado, o que é perigoso e abandona ambos – caso a
mente seja desse jeito. Isso consegue acontecer graças à prática, a
saber aguentar, a fazer esforço. Olhe num único ponto: nossa mente.
Olhe aqui porque “saber” está na nossa mente, nosso comportamento está
na nossa mente. A mente é um ponto no nosso corpo, este amontoado
chamado forma, esta forma que usamos para sentar, há um único ponto em
que sabemos, vemos, é o ponto do coração. Os olhos, ouvidos, nariz,
língua e corpo passam pelo ponto do coração. Portanto, o Buda dizia
para saber cuidar do coração. Ter habilidade em cuidar do coração. Ter
\emph{sati} cuidando do coração, porque quando guardamos ali,
cuidamos ali, naquele ponto, qualquer objeto mental que passe vai
chegar àquele ponto. Quando chegar naquele ponto nós saberemos. Quando
sabemos, é como um visitante que chegou, temos que recebê-lo. Se somos
uma pessoa esperta, recebemos com sabedoria, conseguimos lidar com
aquele objeto mental. Se não somos espertos, temos que recebê-lo com
ignorância e ele vai nos puxar pelo nariz.\footnote{Na Tailândia rural
era comum o uso de argolas no nariz para domar búfalos.} A mesma coisa
com os objetos mentais. 

Portanto, ele dizia para aprender, mudar de comportamento, conhecer
todas essas coisas. Quando chega o \emph{uposatha} estudamos
\emph{sīla}. A mesma coisa para nós praticando o \emph{uposatha}
hoje, é para podermos ser dessa forma. Para podermos ver dessa forma,
ser dessa forma. A prática do ensinamento do Buda não requer muito,
basta observar nossa mente, viver com o mestre e observar. Quando um
objeto mental surgir, não se apresse em ir atrás dele, não vá correndo
atrás dele: “Eu gosto muito disso,” quando a mente estiver assim, não
vá atrás dela. “Disso não gosto nem um pouco!”, não diga ainda,
observe. Observar é uma forma de enriquecer a sabedoria. Faça, largue
aquilo, largue isso. Dessa forma sabedoria consegue surgir. Se formos
correndo atrás dos objetos mentais, sabedoria não consegue surgir. Se
corrermos para aquele lugar, sabedoria não consegue surgir. Por quê?
Porque aquele lugar não é onde nasce sabedoria. Sabedoria nasce onde
vemos essas duas coisas. 

Corra, veja e volte. Depois de segui-la, voltamos para nosso lugar,
é aqui que nasce sabedoria. O que segue os objetos mentais é a
ignorância, é ela que segue. O Buda dizia para ficar firme: quando os
objetos mentais fazem contato, aqui continua firme. Firme significa
normal. Se algo desagradável fizer contato – continuamos firmes. Essa
firmeza que não se inclina à esquerda ou à direita, é bem aqui que
sabedoria nasce. Talvez perguntem: “Por que não mais vamos atrás dos
objetos mentais?” Porque a mente vê que aquilo é incerto, que aquilo
não é de verdade. Aquilo é incerto, não é preciso ficar correndo atrás,
não precisa se apegar. A mente age sob essa razão, ela entende todas
essas coisas de acordo com a verdade dessa forma. 

Por isso observe sua mente continuamente, contemple continuamente.
Sabedoria nasce da mente, sabedoria nasce na mente. Fazer a mente ficar
esperta é fazer sabedoria nascer ali. Se vamos correndo atrás, não
temos esperteza, somos burros. Se corremos atrás, não somos espertos,
não somos sábios, então somos burros, ficamos correndo de um lado para
o outro. Corre, corre, sem saber para onde está indo. Quando está
feliz, corre até o limite da felicidade; quando está triste, corre até
o limite da tristeza; quando gosta, corre até o limite do gostar;
quando não gosta, corre até o limite do desgostar. Só sofrimento. O
Buda ensinava a ficar firme, a ser unificado. 

Falando de forma simples, pratique “\emph{Buddho}”\footnote{Na
Tailândia é comum o uso da palavra “Buddho” como mantra para ajudar a
desenvolver samādhi.} até a mente ficar unificada o tempo todo. Mesmo
o Buda e os \emph{sāvakas} tinham felicidade e sofrimento, tinham
coisas de que gostavam e não gostavam, mas eles conheciam aquilo e não
corriam atrás. Largavam, pois viam o mal daquilo. Esse tipo de Darma
não é um Darma para desenvolver, é um Darma para abandonar. Quando
surge deleite, estamos cientes; quando insatisfação surgir é a mesma
coisa, eles têm o mesmo valor. Ele ensinava a largar, largar esses dois
objetos mentais. Não é para largar por ignorância, é largar por
conhecer todas essas coisas. Eles surgem rápido. Quando é felicidade,
sabemos que é a mente feliz. Surge sofrimento e sabemos, mas não nos
apegamos ao sofrimento. Não há dono daquele sofrimento nem daquela
felicidade. Assim que está ciente, solta. Surge e desaparece, surge e
desaparece. Os elementos surgem e desaparecem, só isso. 

A mente é livre. A mente não se mistura com os objetos mentais:
mente é mente e objetos mentais são objetos mentais. Quando separamos a
mente dos objetos mentais, vemos que são distintos. Um objeto ruim
surge, mas a mente ainda está boa, ela não fica ruim atrás do objeto. O
que é ruim é o objeto, a mente não corre atrás. Objetos bons, ruins, de
que gostamos, de que não gostamos, são apenas objetos mentais. É apenas
o Darma surgindo e desaparecendo. Se sentarmos e olharmos, vemos apenas
surgir e desaparecer, surgir e desaparecer. Quando ouve ou vê, está
ciente, o mesmo quando de pé, andando, sentado ou deitado. Vê que
aquilo é do jeito que é, não sofre por causa daquilo, não fica feliz
por causa daquilo. Não se apega à dor ou prazer que vem daquilo. É da
natureza daquilo surgir e desaparecer em seguida, vê o surgir e
desaparecer, surgir e desaparecer. Naquela hora ele sabe que aquilo que
surge tem que desaparecer. Vê o Darma surgindo e desaparecendo. Vê o
Darma surgindo e desaparecendo, o que mais haveria de ser? 

É como uma onda no mar que cresce, cresce, e nós observamos. Quando
ela chega à margem ela: “Ahhh!”, quebra e se dispersa, não sobra nada.
Essa onda acabou, vem a seguinte, vem até a margem; quando chega, se
dispersa novamente. É assim o ano inteiro, a vida inteira, é desse
mesmo jeito. É como uma onda atingindo a margem. Esses Darmas são a
mesma coisa: surgem e desaparecem, surgem e desaparecem, desaparecem e
surgem, surgem e desaparecem. Eles são assim mesmo. A mente não nasce,
não desaparece junto com aquilo. A mente não nasce ou desaparece junto.
Por que não nasce nem desaparece junto? Porque vê que tudo isso é
existência, vida. Não é possível ficar correndo atrás disso. Quando a
mente sabe de tudo isso, quando se torna uma pessoa esperta – larga, a
mente larga. 

Largar não é o mesmo que não saber, tem que saber para poder largar.
Se usar a linguagem do Darma, podemos dizer que a mente ficou vazia. Se
disser que a mente fica vazia, é como o Venerável
Mogharaja.\footnote{Sutta Nipata 5.15 (Mogharaja-manava-puccha)} Māra
não conseguia alcançá-lo, a morte não conseguia alcançar, ele já tinha
uma proteção. Māra é a morte, ela não conseguia alcançar. Não
conseguia alcançar o Venerável Mogharaja. Por que não conseguia
alcançar? Porque ele sabia de acordo com a verdade, que não existem
seres ou pessoas. O que nasce e morre é só “nascer e desaparecer,”
mente e corpo surgem e desaparecem de acordo com a natureza delas. Ele
não morre – Māra não conseguia alcançá-lo. Terra, fogo, água, ar se
dispersam de acordo com a natureza deles, não há um ser ou pessoa ali.
É vazio dessa forma, não é vazio de outra forma. Não é vazio sem saber
de nada, sem conhecer prazer, dor, não saber de nada – vazio.
Normalmente a mente de uma pessoa não consegue ser vazia dessa forma.
Se ficar vazia dessa forma, ainda vão existir dúvidas. 

Temos que procurar o que é vazio naquilo que não é vazio; temos que
procurar libertação nas convenções.\footnote{Trocadilho entre as
palavras “somut” (\thai{สมุด}) e “vimut”
(\thai{วิมุต}), pode ser interpretado como “o
incondicionado no que é condicionado.”} Temos que procurar frescor no
calor, procurar paz na confusão, procurar felicidade no sofrimento. Tem
que ser assim mesmo. A mente vazia não é vazia sem saber de nada. Ela
larga dos apegos, larga desejo, larga raiva, larga ignorância. Esses
objetos mentais não conseguem mais criar desejo ou raiva, pois não há
mais dono. Surge prazer, mas não há dono daquele prazer. Surge dor, mas
não há dono. Quando forma, sensações, percepções, formações
mentais\footnote{Os cinco khandhas (o quinto sendo “consciência”).}
morrem e se dispersam, não há alguém para morrer. Māra não consegue
alcançar, não consegue alcançar essa pessoa. Portanto Māra não
conseguia alcançar o Venerável Mogharaja. Māra só consegue alcançar
aqueles que se apegam a “eu,” “meu,” “nós,” “ele.” Só alcança aqueles
que têm impurezas mentais, que não sabem de acordo com a verdade. Por
isso se diz “mente vazia,” “mente vazia.” Se só pensarmos a respeito,
não veremos nada de mais. Pensa, fala isso, fala aquilo mas não sabe
nada. Não sabe nada sobre o que é vazio, pois ele é justamente o
pacificar de todas essas coisas, não é outra coisa. 

É o que se chama comportamento. No começo isso nasce do
comportamento e vai mudando continuamente, \emph{Sotāpanna,
Sakadāgāmi, Anāgāmī, Arahant}, etc. Essas são manifestações
da mente, vêm da prática e recebem esses nomes por abandonarem essas
impurezas mentais, não é outra coisa. Não pergunte: “Eu pratiquei esse
tanto, minha mente chegou a esse ponto, que nível é esse, o que já
atingi?”, não pergunte pois só cria confusão. A mente e esses termos
são diferentes, não são iguais. Já leram? “Desejo, raiva, ignorância,”
já leram? O que surge no seu coração e a palavra nos livros, são
diferentes? Se só ler o livro é fácil: “Eh, desejo é assim, raiva é
assim” – fácil. Mas quando nasce na nossa mente é daquele jeito? Eles
vêm com violência, a coisa real é assim. Portanto, é característica da
prática enxergar mais fundo do que isso. O conhecimento teórico não
leva a abandonar. Com o conhecimento que vem da prática, se conhercemos
de acordo com a verdade, conseguiremos abandonar. Se aquilo é ruim,
vemos de verdade; se é útil, é útil de verdade. Gera malefícios ou gera
benefícios na mente de verdade. É assim, e por isso o comportamento vai
mudando continuamente. Vai nascendo continuamente, continuamente nossa
mente, por isso temos que procurar sabedoria na nossa mente. 

Na maioria dos casos as pessoas pensam que essas são unidades
separadas. Por que o Buda separou em \emph{sīla, samādhi,
paññā}? Separou para que os que estudam a teoria aprendam fácil,
entendam fácil. Essa característica se chama \emph{sīla}, essa
\emph{samādhi} e essa \emph{paññā}. Mas na verdade
\emph{sīla, samādhi, paññā} são a mesma coisa. São o mesmo,
mas separamos, por exemplo nosso corpo: estes são os olhos, estas as
orelhas, esta a boca, este o braço, todas as coisas no corpo de alguém,
são a mesma pessoa. É a mesma pessoa. Separamos para saber “ali são os
olhos, orelhas, nariz, pés, corpo” – separamos. Pode separar o quanto
quiser, eles não vão a lugar algum, continuam sendo a mesma pessoa. O
praticante tem que ser assim, vai unindo, unindo. Unindo
\emph{sīla, samādhi, paññā. Sīla, samādhi, paññā} são
um mesmo corpo. Se são um mesmo corpo, fica difícil estudar a teoria,
então separam: corpo e fala são \emph{sīla}, mente é
\emph{samādhi}, \emph{paññā} é \emph{paññā}. Separam. A
característica de sentir, pensar, saber com clareza chama-se
\emph{paññā}, a característica que é pacífica se chama
\emph{samādhi}, a característica que tem cuidado e atenção se chama
\emph{sīla}. 

O nariz inala e sente fedor, sente perfume. Os olhos veem uma forma,
os ouvidos ouvem um som, tudo se reúne no saber da mente. Por que
separamos? Para saber que aqui é o nariz, conhecer o nariz. Este é o
nariz, estes são os olhos, esta é a boca, esta é a carne, a pele, os
pelos, separa tudo mas ainda fica no mesmo lugar. Quando juntamos tudo
isso e contemplamos, vemos que este corpo é algo incerto; então a mente
consegue escapar, consegue largar, ver a desvantagem daquilo. Entende e
vê todos os fios de pelo, todos os fios de cabelo, todos os ossos,
todos os tendões, vê tudo e surge \emph{nibbidā} – desencanto com
este corpo. Separamos para podermos entender mais facilmente, só isso,
mas no final tudo se reúne. 

A prática é a mesma coisa, \emph{upacāra samādhi, appanā
samādhi}. Como é \emph{appanā samādhi}?\emph{ Upacāra
samādhi, khanika samādhi, appanā samādhi}, tudo isso diz
respeito à mesma mente. \emph{Khanika samādhi} é ter paz por um
período curto. \emph{Upacāra samādhi} vai indo aos poucos, senta
um período maior. São períodos de tempo, esse período é curto, aquele é
longo. \emph{Appanā samādhi} fica direto várias horas, é firme;
quando a mente está assim se chama \emph{appanā samādhi,
upacāra samādhi, khanika samādhi}. É a mesma mente. Quando tem
pouco \emph{samādhi} tem um nome, quando tem mais
\emph{samādhi} que aquilo, tem um outro nome, quando chega ao
máximo tem um outro nome. Esse assunto temos que estudar para poder
conhecer. Por exemplo, estude como é \emph{khanika samādhi}, como é
\emph{upacāra samādhi}, como é \emph{appanā samādhi}, pode
estudar antes, mas ainda não vai saber. Tem que fazer, aí vai ver que
existe paz por um período curto – \emph{upacāra samādhi} anda,
\emph{upacāra samādhi} passeia, reflete. Passeia dentro daquela
paz por um período. \emph{Appanā samādhi} é firme, tem ainda mais
força que os dois tipos anteriores de \emph{samādhi}. Esses três só
podem ser encontrados na mente daquele que pratica. Se perguntar antes,
não vai entender: “O que é isto?” Não vai entender porque quando chega
em \emph{upacāra samādhi} não surge nenhuma placa com os dizeres:
“Aqui é \emph{upacāra samādhi}”, não aparece desse jeito. Só as
características são visíveis. Se temos sabedoria vemos: “Este nível é
leve desse jeito, aquele é leve daquele jeito, aquele outro é assim,”
enxergamos a mente desse jeito, de acordo com os textos. Essas são
manifestações da mente. 

Tem que ver por si mesmo, não precisa pensar nem complicar muito. No
começo da prática, peço que tenha atenção e cuidado com
\emph{sīla}. Esse é o começo de tudo, tem que virar comportamento.
Se estiver em grupo é como se estivesse sozinho. Se está sozinho, está
sozinho. Não há diferença, está ciente. Tem diligência o tempo todo
dessa forma. Isso é o que se chama prática: quando a mente enxerga, é
assim. Ela não abandona seu dever. Quando chegar a esse nível, cuide
para que a mente continue assim, procure cuidar dessa mente
continuamente. Então saberá, verá tudo o que surgir naquela mente. Como
uma tela de cinema: não olhe para outro lugar, olhe para a tela.
Qualquer evento que ocorrer vai aparecer na tela; a mente é a mesma
coisa. Não precisa dar nó na cabeça, não precisa pensar demais. Tenha
cabeça fria e cuide para que nada consiga lhe enganar. Conhecemos
aquilo que gostamos e aquilo que não gostamos de acordo com a verdade,
largamos aquilo até isso virar comportamento. Quando largamos ficamos
tranquilos. Quando carregamos é pesado, quando largamos fica leve,
vemos de acordo com a verdade na nossa prática. Não é de outra forma,
não precisa ter dúvidas. Peço apenas que pratiquem, cada um por si,
pouco ou muito, mas se comportem de acordo, criem as condições,
respeitem o treinamento. 

Quando a mente se distrair, um pouco que seja, esteja ciente e
largue aquilo. Quando quiser dizer algo que é ruim, largue e tente
unificar a mente, isso se chama \emph{samvara-sīla}:\emph{
}cuidado, atenção, unificar a mente. Vá diminuindo até chegar ao ponto
em que é fácil de olhar. Quando sobrar somente corpo e mente, já
acabou. Declarar possuir poderes psíquicos, quebrar o
celibato,\footnote{Ambos são ofensas à regra monástica.} isso está
longe, mas acontece bem aqui, temos que praticar bem aqui. É fácil ver,
temos que verificar a regra monástica bem aqui. Temos que treinar, pois
esse treinamento é a nossa prática. Não precisa olhar isso ou aquilo:
veja que nossa mente sofre e será capaz de largar esse sofrimento. Veja
a mente se apegando a desejo, raiva e ignorância, pouco a pouco. Não
precisa ver \emph{garudās, nāgas, devatas,
brahmas},\footnote{Diversas classes de seres celestiais.} pratique bem
aqui, não precisa olhar para essas coisas, já é suficiente olhar para
sua própria mente. O Buda queria que víssemos isso, soubéssemos isso de
acordo com a verdade, nada mais.
